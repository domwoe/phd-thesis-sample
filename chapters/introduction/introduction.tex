\chapter{Introduction}
\label{sec:introduction}

\epigraph{The most profound technologies are those that disappear. They weave themselves into the
fabric of everyday life until they are indistinguishable from it.}{--- \textup{Mark Weiser}}

\section{Context and Motivation}

25 years ago Mark Weiser envisioned the computer of the 21st Century. Rather than interacting with a personal computer, computing technology is seamlessly integrated into the fabric of everyday life. Computing is to become ubiquitous \parencite{weiser:13609162}, embedded in every object, and equipped with communication and sensing technologies to form an \ac{IOT}.
Besides all the hype \parencite{Gartner:2015,manyika2015unlocking}, today, we still experience only glimpses of true ubiquitous computing. Start ups and established hardware companies are embedding computing chips and network stacks in ever more products. From light bulbs to cars, everything is getting infused with parts from the smartphone supply chain \parencite{Evans:2015,Evans:2016}. But instead of seamless interconnectivity and interoperability, i.e. objects playing in concert, we are getting ever more soloists, remotely controlled by proprietary smartphone apps -- or the manufacturer's backend. In exchange for this convenience we pay by giving up ever more privacy. Data about our daily habits, our health and our finances are constantly collected, and stored in large corporate databases. Data is the new oil \parencite{schwab2011personal}, but only few know how to refine it \parencite{manyika2015unlocking}. Thus, a lot of the collected data lies dormant, awaiting the next security breach\footnote{See https://www.privacyrights.org/data-breach for public data breaches.}. Exchange of consumer data between companies is opaque and bears liability risks for the companies \parencite{ISI:000330253700010}, instead of being available to create societal value.

This is not the only inefficient allocation of resources. Most of the time your computer uses a few percent of its computing power. Your disc is only filled by 30\%, and your Internet connectivity is only exhausted when streaming a movie. The cloud computing paradigm \parencite{Hayes:2008:CC:1364782.1364786,Armbrust:2010:VCC:1721654.1721672} with the technology of virtualization \parencite{barham2003xen} has brought efficient resource allocation to data centers, but with the emergence of ubiquitous computing and the Internet of Things, it can be assumed that most of the computing power will soon be outside of corporate data centers. 

New paradigms like Edge or Fog computing \parencite{Bonomi:2012:FCR:2342509.2342513,Yi:2015:SFC:2757384.2757397} attempt to extend the cloud paradigm to end devices, but centralized control is always a bottleneck. Grid computing has only been moderately successful in scientific contexts \parencite{Anderson:2002:SEP:581571.581573,Beberg2009Folding}, and peer-to-peer systems \parencite{Rodrigues:2010:PS:1831407.1831427} suffer from free-riders \parencite{10.2307/3003400} and sybil attacks \parencite{douceur2002sybil}. Nevertheless, there are examples of successful peer-to-peer systems such as BitTorrent \parencite{cohen2003incentives}, that have been resisting malicious attacks and governmental interference.

If transaction costs are low, the market and price mechanisms allow for division of labor and efficient resource allocation \parencite{smith1887inquiry}. With increasing transaction costs, however, hierarchical structures and centralized control become more efficient \parencite{ECCA:ECCA386}. 

Information technology is decreasing transaction costs. The Internet, based on an open protocol stack, diminished global communication costs. Databases can store and query huge amounts of data. The \ac{IOT} captures real-time information of the physical world, and advances in machine learning and artificial intelligence begin to replace costly human decision making, or at least narrow down choices and consequences such that humans with bounded rationality \parencite{simon1982models} are able to make a quick informed decision. 

In a truly interconnected world automation has to bridge trust boundaries. Smart objects have to work in unison to unlock the greatest and economic and societal benefit \parencite{manyika2015unlocking}.
Many smart objects are mobile and thus the topology of the network is changing constantly. While at one point in time an object is surrounded by trusted objects, i.e. the neighbors provide credentials that are attested by a trusted source, e.g. the manufacturer. At some later time the situation might be different and only untrusted devices are around. 

Let us make an analogy to our daily lives. If you transact with friends and family you trust in reciprocity, and therefore you have little hesitation to cooperate. Transactions are implicit -- stored in the shared mind of the participants -- and will eventually cancel each other out. However, if you transact with a stranger, the transaction is made explicit by exchanging money. Thus, reducing the need for trust. However, if you do not pay by cash, but by credit card, the direct participants in the transactions have to place trust in intermediaries. The payee has to trust that she eventually receives the money, and the payer has to trust that the payee does not charge more than agreed upon. The eventual enforcement of this implicit contract is executed by the trusted intermediary. This manifests in literal transaction costs. Trusted third parties are costly \parencite{szabo2005trusted}.

Until 2009 there was no digital equivalent of physical cash allowing for trust-less pseudonymous payments over the Internet. Prior schemes always involved a trusted third party, and all implemented solutions have failed. Bitcoin is peer-to-peer electronic cash \parencite{nakamoto2008bitcoin} that solves the trusted third party issue by cleverly combining a variety of cryptographic techniques and mechanism design, i.e. aligning economic incentives to create a completely open self-sustaining system without requiring any kind of identification. The underlying technology is usually termed \emph{blockchain technology}. Although the system is not perfect in theory \parencite{Eyal2014}, it is surprisingly successful in practice. The market capitalization of Bitcoin the currency oscillates around \$10 billion USD. More than 210000 transactions are recorded daily with a volume of more than \$150 million USD, neglecting the \emph{off-chain} trades on the numerous online exchanges. Bitcoin has often been described as digital gold because of the implemented monetary policy. The maximal number of bitcoins\footnote{Bitcoin with capital \emph{B} refers to the system, whereas bitcoin refers to the unit of currency.} that will be ever created is limited to 21 million, and the resource-intensive process of \emph{minting new coins} that ensures security is termed \emph{mining}. In effect, the number of bitcoins and their rate of creation is transparent, and independent of any governmental monetary policy.

As will be clear later on, Bitcoin is far more than an ordinary currency. Bitcoin is the first programmable money and the first digital bearer asset. Ownership and control of bitcoins is non-custodial and augmented by possession of cryptographic keys. As Richard Gendal Brown, former IBM executive, put it: "On the blockchain, nobody knows you're a fridge" \parencite{brown2013}. Machines are fist-class citizens. Giving machines control over money and enabling true micropayments orders of magnitude below the traditional interpretation of micropayments, i.e. a few dollars, will stimulate new types of autonomous transactions without human involvement. The beginning of an economy of things, and an additional driver towards ubiquitous computing.

Bitcoin itself is a role model of how a trusted third party can be replaced and automated by a network of untrusted computers by defining rules through code and the market mechanism \parencite{lessig2009code}. Indeed, there are already hundreds or thousands of alternative cryptocurrencies and blockchains exploring the space of possible implementations. However, most of them do not add much value. The creation and maintenance of such a decentralized manifestation of a trusted third party, however, does not come without cost either. Currencies are prime examples of network effects in action: A currency is only useful if a trading party is accepting it, and the party will only accept it if she is convinced that other parties, who she might want to trade with at a later time, accept it as well. Transaction networks are comparable to communication networks, and the derived value from a growing user base should thus be superlinear (c.f. Metcalfe's law \parencite{10.1109/MC.2013.374} and Reed's law \parencite{reed1999sneaky}). Since a cryptocurrency ecosystem involves more parties than merchants and customers, namely, miners, developers and speculators, network effects are multi-sided and feedback loops can be particularly strong \parencite{Giaglis2014}. Only time will tell if Bitcoin's dominance will be disputed, and if there will be a consolidation or a rich ecosystem of cryptocurrencies and blockchain-based networks.

In terms of  market capitalization the next cryptocurrency after Bitcoin is Ethereum \parencite{ethereumWhite}, which is valued collectively at approximately a tenth of Bitcoin. Ethereum has a different mission statement than Bitcoin. It aims to be \emph{the world computer}, a decentralized always-on trusted backend for applications - a multi-purpose trusted third party. In this vision, the concept of programmability becomes the focus, and the currency becomes a means to this end. 

Besides coordination between smart objects and their resources in an economy of things, there is another economy that is fueled by the \ac{IOT}: The \emph{sharing economy} \parencite{sundararajan2016sharing}.

The most prominent representatives of this economy are Uber and AirBnB. These are platform businesses that coordinate underutilized, mostly private, assets such as cars and apartments to increase their economic value. Only a few years ago it was inconceivable for most people to rent their homes to some stranger over the Internet, or to enter a private car of some stranger. But these norms have been updated. Apartments and cars can be seen as the tip of an iceberg. Shared usage of assets starts with high value assets, but may continue to lower value assets if transaction costs are decreasing. Furthermore, the value of an asset is relative. A \$250 USD solar home system that provides electricity to charge a few phones and batteries is not a high value asset to our standards, but in east Africa or southeast Asia the situation is quite different. In these regions products are already equipped with \ac{IOT}-technology in combination with mobile payment technologies to enable innovative business models like \emph{pay-as-you-go} and \emph{lease-to-own} in countries with low-levels of legal enforcement to make products affordable for the people at the bottom of the pyramid \parencite{ISI:000351842100012}. As software is eating the word \parencite{andreessen2011software}, the \emph{as-a-Service paradigm} originating from cloud computing, takes hold in physical world. 

Cryptocurrencies in unison with the concept of \emph{smart contracts} and \emph{smart property} \parencite{szabo1997,smartproperty2011} can support these business models by decreasing the need of trust in counterparties, and attaching financial services to the product itself.

% This thesis discusses the novel concepts of cryptocurrencies and blockchain technology from the vantage point of the Internet of Things and ubiquitous computing. Smart objects may become autonomous economic objects participating in a microscopic economy by exchanging services with other objects near and far. Cryptocurrencies will allow for frictionless value exchange and a basis for trust-minimizing interactions. We will discuss the suitability of Bitcoin for machine-to-machine micropayments and discuss various applications. In particular, we discuss the Sensing-as-a-Service model in detail, and present prototypes. A related concept, is mobile crowdsensing, where the sensing device is the human-operated smartphone. We present a system (with implementation) for incentivizing mobile crowdsensing with bitcoin micropayments based on unidirectional payment channels.
% Moreover, the concept of smart property allows smart objects a seamless integration into the macroscopic sharing or access economy. Shared ownership, low-trust trading and renting, as well as elaborate financing and revenue-sharing models can be centered at the product itself. 


%\newpage
\section{Objective and Approach}

This thesis aims to provide initial insights into the impact of cryptocurrencies on the \ac{IOT}. Due to the novelty of cryptocurrencies, the research approach is based on design and engineering of prototypical applications. Thus, the research is explorative and iterative. One promising application of cryptocurrencies in \ac{IOT} is investigated - the trading of digital goods and services between connected devices exemplified by the Sensing-as-a-Service scheme. Characteristics of cryptocurrencies in general, and Bitcoin in particular, to aid this scheme are identified, and two prototypes are developed. The first prototype illustrates the concept of trading data for electronic cash based entirely on the Bitcoin network and protocol. The evaluation of the concept and prototype reveals issues concerning confidentiality, latency, and the inappropriateness to perform micropayments. By leveraging the programmability of Bitcoin transactions, smart contracts are introduced to enable low-latency micropayments between a large number of data requesters and data providers mediated by a trust-less hub. The concept is implemented as a mobile crowdsensing application allowing users of a smartphone application to offer data on a global market. 
Further investigation of the concept of smart contracts and the related concept of smart property leads to the notion of economic devices. The concept of economic devices is explained and illustrated with a prototype of an Ethereum-enabled public display. The display offers the service to show user-selected content in exchange for cryptocurrency payments and issues tokens that entitle the bearer to receive a share of revenue in real-time. 

As a foundation to understand cryptocurrencies and their significance, this thesis provides a technological introduction to the two most important cryptocurrency platform, Bitcoin and Ethereum, and an investigation of the broader cryptocurrency and blockchain ecosystem.


\section{Outline}

Chapter \ref{sec:iot} provides a brief introduction to \ac{IOT}. The historical evolution from \ac{RFID} to the consumer \ac{IOT} is provided. Furthermore, opportunities and challenges for individuals, the industry, and the society as a whole are discussed. Many of these challenges can be addressed by creating a more decentralized \ac{IOT} with increasing autonomy of devices and greater user control. Thus, the chapter concludes with an overview of developments towards decentralized architectures, and a motivation for the need of a digital equivalent to physical cash. Chapter \ref{sec:crypto} introduces the technologies behind digital cash. The chapter starts with an indepth presentation of Bitcoin. Focus is on the programmability of Bitcoin transactions, which provides the basis for smart contracts used in later chapters. In addition, the current challenges of Bitcoin, and decentralized cryptocurrencies in general, are discussed based on a literature review. The challenges are complemented with current approaches to tackle them. 
Thereafter, ways to extend Bitcoin's functionality are discussed. This leads to another, more general cryptocurrency design with the focus on a decentralized application and smart contract platform embodied in Ethereum. After presenting Ethereum, a brief comparison with Bitcoin is given. The chapter ends with the introduction of permissioned and private blockchain platforms, which have become increasingly popular throughout industries in recent months. Chapter \ref{sec:ecoystem} provides a complementary perspective by reviewing the Bitcoin and blockchain ecosystem. After reviewing the Bitcoin start-up ecosystem, the broader ecosystem entailing altcoins, metacoins, and permissioned blockchains is investigated. The chapter concludes with a discussion of the economic relevance of cryptocurrencies. Having laid out the foundations, the first application of cryptocurrencies in \ac{IOT} is discussed. Chapter \ref{sec:s2aas} investigates the concept of S\textsuperscript{2}aaS with Bitcoin. The most important example of the exchange of a digital good or service against payment in the \ac{IOT}. Thus, relevant characteristics of Bitcoin are presented, and a first prototype is discussed. Chapter \ref{sec:goingoffchain} extends the initial prototype in order to enable trust-less instant micropayments between a large number of data requesters and data providers. Therefore, unidirectional hub and spoke payment channels based on Bitcoin smart contracts are developed and illustrated by a mobile crowdsourcing application allowing anybody to sell smartphone sensor data. Chapter \ref{sec:economicobjects} introduces the concept of economic devices which is enabled by cryptocurrencies. The concept is illustrated with a prototype of an Ethereum-enabled public display, and its significance for developing countries with underdeveloped financial services and legal systems is discussed. Chapter \ref{sec:discussion} concludes the thesis by recapitulation the key findings and its implications for research and practice. Furthermore, an outlook and ideas for future work is provided.

% \begin{figure}
% \centering
% \includegraphics[width=\textwidth]{./externalized/structure.pdf}
% \caption{Structure of this thesis.}
% \label{fig:structure}
% \end{figure}

\section{Credits}

The work presented in this thesis is based on collaboration with colleagues at the Chairs of Prof. Elgar Fleisch at ETH Zurich and the University of St. Gallen, the Chair of Distributed Computing at ETH Zurich, and the Digital Currency Initiative at the MIT Media Lab. 

The discussion of the Bitcoin start-up ecosystem in Chapter \ref{sec:ecoystem} and the case studies in Section \ref{appendix:cases} is based on joint work with Thomas von Bomhard, Yan-Peter Schreier, and Dominik Bilgeri \parencite[c.f.]{Worner2016ecis}.

The concept of S\textsuperscript{2}aaS with Bitcoin and the extraction and discussion of relevant characteristics as presented in Chapter \ref{sec:s2aas} is based on joint work with Kay Noyen, Dirk Volland, and Elgar Fleisch \parencite[c.f.]{DBLP:journals/corr/NoyenVWF14}. The initial prototypical implementation was developed together with Thomas von Bomhard \parencite[c.f.]{Worner:2014:YSE:2638728.2638786}.
The construction of mediated payment channels to enable a scalable solution to incentivize mobile crowdsensing is joint work with Christian Decker. The prototypical implementation of the system was mainly done by Francisc Bungiu\footnote{See \url{https://github.com/domwoe/datamarket}}. 

The idea to use the concept of economic devices to finance and capitalize productive assets in developing countries emerged in discussions with Michael Casey and Anders Brownworth at the MIT Media Lab. The Ethereum-enabled display was developed with the help of Andrew Koh. 
