\chapter{Introduction}
\label{sec:introduction}

\epigraph{The most profound technologies are those that disappear. They weave themselves into the
fabric of everyday life until they are indistinguishable from it.}{--- \textup{Mark Weiser}}

\section{Motivation}

25 years ago Mark Weiser envisioned the computer of the 21st Century. Rather than interacting with a personal computer, computing technology is seamlessly integrated into the fabric of everyday life. Computing is to become ubiquitous \parencite{weiser:13609162}, embedded in every object, and interconnected -- the Internet of Things.
Besides all the hype \parencite{Gartner:2015,manyika2015unlocking}, today, we still experience only glimpses of true ubiquitous computing. Start ups and established hardware companies are embedding computing chips and network stacks in ever more products. From light bulbs to cars, everything is getting infused with parts from the smartphone supply chain \parencite{Evans:2015,Evans:2016}. But instead of seamless interconnectivity and interoperability, i.e. objects playing in concert, we are getting ever more soloist, remotely controlled by a proprietary smartphone app -- or the manufacturer's backend. In exchange for this convenience we pay by giving up ever more parts of our privacy. Data about our daily habits, our health and our finances is constantly collected, and stored in large corporate databases, so called silos. Data is apparently the new oil \parencite{schwab2011personal}, but only few know how to refine it \parencite{manyika2015unlocking}. Thus, a lot of the collected data lies dormant, awaiting the next security breach\footnote{See https://www.privacyrights.org/data-breach for public data breaches.}. Exchange of consumer data between companies is opaque and bears liability risks for the companies \parencite{ISI:000330253700010}. Thus, allocation of data is not efficient. 

This is not the only inefficient allocation of resources. Most of the time your computer uses a few percent of its computing power. Your disc is only filled by 30\%, and your Internet connectivity is only exhausted when streaming a movie. The cloud computing paradigm \parencite{Hayes:2008:CC:1364782.1364786,Armbrust:2010:VCC:1721654.1721672} with the technology of virtualization \parencite{barham2003xen} has brought efficient resource allocation to data centers, but with the emergence of ubiquitous computing and the Internet of Things, we can assume that most of the computing power will soon be outside of corporate data centers. 

New paradigms like Edge or Fog computing \parencite{Bonomi:2012:FCR:2342509.2342513,Yi:2015:SFC:2757384.2757397} attempt to extend the cloud paradigm and virtualization \emph{to the last mile}, i.e. to end devices, but centralized control is always a bottleneck. Grid computing has only been moderately successful in scientific contexts \parencite{Anderson:2002:SEP:581571.581573,Beberg2009Folding}, and peer-to-peer systems \parencite{Rodrigues:2010:PS:1831407.1831427} suffer from freeriders \parencite{10.2307/3003400} and sybil attacks \parencite{douceur2002sybil}. Nevertheless, there are examples of successful peer-to-peer systems such as BitTorrent \parencite{cohen2003incentives}, that have been resisting malicious attacks and governmental interference.

If transactions are low the market and price mechanisms allow for division of labor and efficient resource allocation \parencite{smith1887inquiry}. With increasing transaction costs, however, hierarchical structures and centralized control become more efficient \parencite{ECCA:ECCA386}. 

Information Technology is decreasing transaction costs. The Internet, based on an open protocol stack, diminished global communication costs. Databases can store and query huge amounts of data. The Internet of Things captures real-time information of the physical world, and advances in machine learning and artificial intelligence begin to replace costly human decision making, or at least narrow down choices and consequences such that a human with bounded rationality \parencite{simon1982models} is able to make a quick informed decision. 

In a truly interconnected world automation has to bridge trust boundaries. Smart objects have to work in unison to unlock the greatest and economic and societal benefit \parencite{manyika2015unlocking}.
Many smart objects are mobile and thus the topology of the network is changing constantly. While at one point in time an object is surrounded by trusted objects, i.e. the neighbors provide credentials that are attested by a trusted source, e.g. the manufacturer, at some later time the situation might be different, and only untrusted devices are around. 

Let us make an analogy to our daily lives. If you transact with friends and family you trust in reciprocity, and therefore you have little hesitation in helping them out or trusting information presented you. Transactions are implicitly stored in the shared mind of the participants and will eventually cancel each other out. However, if you transact with someone else the transaction is made explicit by exchanging money. Thus, reducing the need for trust. However, if you do not pay cash, but by credit card, the direct participants in the transactions have to place trust in intermediaries. The payee has to trust that she eventually receives the money, and the payer has to trust that the payee does not charge more than agreed upon. The eventual enforcement of this implicit contract is executed by the trusted intermediary. This manifests in literal transaction costs. Trusted third parties are costly \parencite{szabo2005trusted}.

Until 2009 there was no digital equivalent of physical cash, that allows for trustless pseudonymous payments over the Internet. Prior schemes always involved a trusted third party, and all implemented solutions have failed because of governmental intervention. Bitcoin is peer-to-peer electronic cash \parencite{nakamoto2008bitcoin} that solves the trusted third party issue by cleverly combining a variety of cryptographic techniques and mechanism design, i.e. aligning economic incentives to create a completely open self-sustaining system without requiring any kind of identification. The underlying technology is termed \emph{Blockchain} or \emph{Distributed Ledger} but the terminology and taxonomy is still in flux. Although the system is not perfect in theory \parencite{Eyal2014}, it is incredible successful in practice. The market capitalization of bitcoin the currency oscillates around \$10 billion USD. More than 210000 transactions are recorded daily with a volume of more than \$150 million USD, neglecting the \emph{off-chain} trades on the numerous online exchanges. Bitcoin has often been described as digital gold because of the implemented monetary policy. The maximal number of bitcoins that will be ever created is limited to 21 million, and the resource-intensive process of \emph{minting new coins} is termed \emph{mining}. In effect, the number of bitcoins and their rate of creation is transparent, and independent of any governmental monetary policies. Hence, the price of bitcoin is often a barometer for the trust in fiat money and the banking system.

As will be clear later on, Bitcoin is far more than an ordinary currency. Bitcoin is the first programmable money and the first digital bearer asset. Ownership and control of bitcoins is non-custodial and augmented by possession of cryptographic keys. As Richard Gendal Brown, former IBM executive, put it: \emph{On the blockchain, nobody knows you're a fridge} \parencite{brown2013}. Machines are fist-class citizens. Giving machines control over money and enabling true micropayments orders of magnitude below the traditional interpretation of micropayments, i.e. a few dollars, will stimulate new types of autonomous transactions without human involvement. The beginning of a (micro) economy of things, and an additional driver towards ubiquitous computing.

Bitcoin itself is a role model of how a trusted third party can be replaced and automated by a network of untrusted computers by defining rules through code and the market mechanism \parencite{lessig2009code}. Indeed, there are already hundreds or thousands of alternative cryptocurrencies and blockchains exploring the space of possible implementations. However, most of them do not add much value. The creation and maintenance of such a decentralized manifestation of a trusted third party, however, does not come without cost either. Currencies are prime examples of network effects in action: A currency is only useful if a trading party is accepting it, and the party will only accept it if she is convinced that other parties, who she might want to trade with at a later time, accept it as well. Transaction networks are comparable to communication networks, and the derived value from a growing user base should thus be superlinear (c.f. Metcalfe's law \parencite{10.1109/MC.2013.374} and Reed's law \parencite{reed1999sneaky}). Since a cryptocurrency ecosystem involves more parties than merchants and customers, namely, miners, developers and speculators network effects are multi-sided and feedback loops can be particularly strong \parencite{Giaglis2014}. Only time will tell if Bitcoin's dominance will be disputed, and if there will be a consolidation or a rich ecosystem of cryptocurrencies and blockchain-based networks.

In terms of  market capitalization the next cryptocurrency after Bitcoin is Ethereum \parencite{ethereumWhite} which is valued at approximately a tenth of Bitcoin. Ethereum has a different mission statement than Bitcoin. It aims to be \emph{the world computer}, a decentralized always-on backend for applications, a multi-purpose trusted third party. In this vision, the concept of programmability becomes the focus, and the currency becomes a means to this end.

Besides coordination between smart objects and their resources in a microscopic economy of things, there is also a growing macroscopic economy that is fueled by the Internet of Things: The \emph{sharing economy} \parencite{sundararajan2016sharing} or currently, more precisely, the \emph{access economy} \parencite{eckhardt2015sharing}.

The most prominent representatives of this economy are Uber and AirBnB. These are platform businesses that coordinate underutilized, mostly private, assets such as cars and apartments to increase their economic value. Only a few years ago it was inconceivable for most people to rent their homes to some stranger over the Internet, or to enter a private car of some stranger. But these norms have been updated. Apartments and cars can be seen as the tip of an iceberg. Shared usage of assets starts with high value assets, but may continue to lower value assets if transaction costs are decreasing. Furthermore, the value of an asset is relative. A \$250 USD solar home system that provides electricity to charge a few phones and batteries is not a high value asset to our standards, but in east Africa or southeast Asia the situation is quite different. In these regions products are already equipped with IoT-technology in combination with mobile payment technologies to enable innovative business models like \emph{pay-as-you-go} and \emph{lease-to-own} in countries with low-levels of legal enforcement to make products affordable for the people at the bottom of the pyramid \parencite{ISI:000351842100012}. As software is eating the word \parencite{andreessen2011software}, the \emph{as-a-Service paradigm} originating from cloud computing, takes hold in physical world. 

Cryptocurrencies in unison with the concept of \emph{smart property} \parencite{szabo1997idea,smartproperty2011} can support these business models by decreasing the need of trust in counterparties, and attaching financial services to the product itself.

This thesis discusses the novel concepts of cryptocurrencies and blockchain technology from the vantage point of the Internet of Things and ubiquitous computing. Smart objects may become autonomous economic objects participating in a microscopic economy by exchanging services with other objects near and far. Cryptocurrencies will allow for frictionless value exchange and a basis for trust-minimizing interactions. We will discuss the suitability of Bitcoin for machine-to-machine micropayments and discuss various applications. In particular, we discuss the Sensing-as-a-Service model in detail, and present prototypes. A related concept, is mobile crowdsensing, where the sensing device is the human-operated smartphone. We present a system (with implementation) for incentivizing mobile crowdsensing with bitcoin micropayments based on unidirectional payment channels.
Moreover, the concept of smart property allows smart objects a seamless integration into the macroscopic sharing or access economy. Shared ownership, low-trust trading and renting, as well as elaborate financing and revenue-sharing models can be centered at the product itself. 


\newpage
\section{Contributions}

\begin{itemize}
\item{Contrast Bitcoin and Ethereum from the vantage point of IoT applications.}
\item{Overview of the Bitcoin startup ecosystem with focus on applications beyond financial services. Presenting six case studies of companies. Joint work with Thomas von Bomhard, Yan-Peter Schreier and Dominik Bilgeri.}
\item{Overview of technologies that enable machine-to-machine micropayments with Bitcoin, and discussion of possible applications.}
\item{Presenting the Sensing-as-a-Service application in detail together with prototypical implementations. Joint work with Kay Noyen, Dirk Volland, Elgar Fleisch and Thomas von Bomhard}
\item{First Implementation of unidirectional mediated payment channels to allow micropayments between a large number of parties. Joint work with Francisc Bungiu and Christian Decker}
\item{Presenting concepts based on smart property. Discussion of their implementations in Bitcoin and Ethereum.}
\item{Survey of companies and projects.}
\end{itemize}

\newpage
\section{Structure}

\begin{figure}
\centering
\includegraphics[width=\textwidth]{./externalized/structure.png}
\caption{Structure of this thesis.}
\label{fig:structure}
\end{figure}
