\chapter{Bitcoin and Beyond: Economic Perspective}
\label{sec:ecoystem}

\epigraph{One general law, leading to the advancement of all organic beings, namely, multiply, vary, let the strongest live and the weakest die.}{--- \textup{Charles Darwin}}


The last chapter introduced Bitcoin and its underlying mechanics such as the blockchain from a technological perspective. This chapter provides a complementary perspective by investigating the Bitcoin, cryptocurrency and blockchain ecosystem from an economic perspective. Only rarely have software projects originating from a single developer, or a small group of developers, such a profound economic impact spinning up entire ecosystems of start-ups, open source projects, a new age of non-governmental monies, novel business models, and cross-industry consortia involving some of the world's largest corporations. This chapter retraces the evolution of the narrow and extended Bitcoin ecosystem, and provides empirical evidence that Bitcoin is still the dominant cryptocurrency.
Venture capital investment is almost non-existent for companies focusing on public blockchains besides Bitcoin. However, Bitcoin implicitly introduced and proved a novel business model, the \emph{appcoin} or \emph{Nakamoto} business model, that may allow to fund and sustain decentralized organizations, applications and software protocols. Although the biggest experiment, The DAO - a form of democratized and decentralized venture capital fund implemented as Ethereum smart contracts, has failed, the combined market capitalization of all blockchain-based coins, excluding Bitcoin, is almost \$2 bn USD, and almost \$117 million USD have been collected in \ac{ICO}s since 2013. Although the legal status of these experiments is still undefined and heavily debated, the number of these experiments is likely to increase in the near future. 

After the presentation of the blockchain ecosystem based on empirical data, the chapter concludes with a discussion of the economic relevance of cryptocurrencies based of five characteristics, i.e. inclusion, innovation, disintermediation, automation, and new business models.

\section{Data and Method}

\paragraph{Bitcoin Ecosystem}

To attain a wide dataset that adequately 
represents the presently existing Bitcoin start-up ecosystem, a variety of publicly available online sources was utilized: The start-up platform AngelList\footnote{\url{https://angel.co/}}, 
searched with the broad keyword \emph{cryptocurrency}, a compiled list of crypto technology 
companies by industry expert \cite{Mougayar2015} and \cite{Coindesk2015}, a reputable 
and well known bitcoin news site, which maintains a comprehensive list of venture 
capital invested into Bitcoin start-ups. To validate and add additional data to 
the compiled set we used the websites of the individual companies as well as press 
releases and the Crunchbase database\footnote{\url{https://crunchbase.com}}. We identified a total of 704 start-ups 
and projects in the cryptocurrency and blockchain space, which are existent today. 
Of these, 599 belong to the Bitcoin ecosystem, and 65 received venture capital. 
Following an iterative process, we identified representative categories that help 
understanding the Bitcoin ecosystem. 

\paragraph{Blockchain Ecosystem beyond Bitcoin}
A list of publicly traded blockchain-based tokens together with market price and market capitalization was acquired from Coinmarketcap\footnote{\url{https://coinmarketcap.com/currencies/}}. The description was added by the author based on consultation of the relevant project websites. 

Venture capital investment numbers are taken from Coindesk Venture Capital\footnote{\url{http://www.coindesk.com/bitcoin-venture-capital/}}, and provides only publicly disclosed funding. Classification in different categories (altcoin, appcoin, bitcoin, ethereum, permissioned) was done based on consultation of the relevant project websites. Data concerning \ac{ICO}s was collected from Cyber Fund\footnote{\url{https://cyber.fund/radar}}. Only completed and successful \ac{ICO}s were considered. Funding amounts were given in bitcoins, and converted to USD based on the exchange rate at the end data of the respective \ac{ICO}. The exchange rate was taken from Coindesk. The underlying blockchain of the coins was added by the author based on the project websites.

Data concerning blockchain consortia is based on \parencite{Mougayar2016} and the consortia websites. Consortia with no information about members were excluded.

All data was acquired on 2016-12-11.

\section{Bitcoin Ecosystem}

\label{sec:sec:eco}

The open and permissionless nature of Bitcoin has led to a Cambrian explosion of 
projects and start-up companies. Table \ref{tbl:ecosystem} provides an overview of the diversity 
of the narrow ecosystem, i.e. only Bitcoin companies are considered. 

\begin{figure}[htbp]
\centering
\includegraphics[width=0.9\textwidth]{./externalized/ecis-fig002.png}
\caption{Distinction between challengers and the Bitcoin ecosystem without 
challengers.}
\label{fig:challengers}
\end{figure}

The ecosystem is divided in two main categories (see Figure \ref{fig:challengers} for illustration). 
The first category consists of start-up companies that act mainly inside the Bitcoin 
ecosystem itself. Examples are wallet providers, mining operations and Bitcoin 
exchanges. We refer to this category as \emph{Bitcoin ecosystem without challengers}. 
Hence, the second category is termed challengers. Those are companies that use 
Bitcoin technology to attack traditional companies and business models outside 
the Bitcoin ecosystem. Examples of this category are payment processors like BitPay\footnote{https://bitpay.com/}, challenging traditional online payment processors like PayPal, and remittance 
services like Abra\footnote{https://www.goabra.com/}, challenging incumbents like Western Union by cutting 
down transaction fees through disintermediation and decreased vulnerability to 
fraud. The classification of companies is not always clear-cut and may change as 
the ecosystem develops, but provides a lens to identify sectors, which might get 
disrupted first.

Interestingly, the challenger category is not limited to the financial service 
industry. We identified three main sectors beyond financial services where start-ups 
use Bitcoin technology to innovate and thereby challenge incumbents: (1) notary 
services, (2) marketplaces, and (3) digital assets. Notary services use the Bitcoin 
blockchain as an immutable public database and time-stamping service. Applications 
are records management, by providing verifiable audit trails and provable data 
integrity, as well as identity registries, which are not tied to a particular identity 
provider. Marketplaces provide a decentralized infrastructure where physical as 
well as digital goods and services can be traded for bitcoins. The digital assets 
sector is concerned with the management of \emph{anything that exists in a binary format 
and comes with the right to use} \parencite{wikipedia2016da}. This entails digital art, 
photographs, music, but also coupons and tickets. Furthermore, we extend the 
concept of digital assets to incorporate \ac{IOT}), since \ac{IOT} 
is concerned with the digital representation of physical devices. Concise case studies of representative companies are provided in Appendix \ref{appendix:cases} and have been published in \parencite{Worner2016ecis}. 

\begin{figure}
\centering
\includegraphics[width=0.9\textwidth]{./externalized/projects.pdf}
\caption{Number of new Bitcoin projects over time.}
\label{fig:eco:projects}
\end{figure}

\begin{sidewaysfigure}
\centering
\includegraphics[width=\textwidth]{./externalized/scatter.pdf}
\caption{Evolution of the venture-backed Bitcoin start-up ecosystem.}
\label{fig:eco:evolution}
\end{sidewaysfigure}

Figures \ref{fig:eco:projects} and \ref{fig:eco:evolution} give an overview of the temporal evolution of the ecosystem. Figure 
\ref{fig:eco:projects} shows the evolution of all Bitcoin projects whereas 
Figures \ref{fig:eco:projects} and \ref{fig:eco:evolution} is restricted to 
the venture-backed Bitcoin start-up ecosystem. Notably, there are no companies 
that were founded in the first two years of Bitcoin's existence. Projects that 
started earlier have not survived (e.g. Mt. Gox, 2015) or have not attracted venture 
capital. The following three years (2011-2013) are characterized by companies building 
the infrastructure for the Bitcoin ecosystem. From 2014 on, the number of challengers 
has grown, and the sectors beyond the financial sector have gained traction. Hence, 
these companies are very young and are just in the process of entering the market.

\begin{table}
\resizebox{\textwidth}{!}{
\tiny
\begin{tabular}{|>{\centering\arraybackslash}m{17pt}|>{\centering\arraybackslash}m{46pt}|>{\centering\arraybackslash}m{68pt}|>{\centering\arraybackslash}m{115pt}|>{\centering\arraybackslash}m{53pt}|>{\centering}m{85pt}|}
\hline
 & \textbf{Sector} & \textbf{Subsector} & \textbf{Description} & \textbf{Representatives} & \textbf{Revenue 
Mechanics}\tabularnewline
\cline{2-6}
\multirow{12}{*}{\rotatebox[origin=c]{90}{\textbf{Challengers}}} &  \multirow{3}{*}{Digital Assets} & Internet of Things & Service to register 
and manage connected devices & \textbf{Filament} & Product as a Service\tabularnewline
 \cline{3-6}
 &  & Intellectual property & Service to register and manage IP like music and 
art & A\textbf{scribe}, Monegraph & Transaction fee (\%)\tabularnewline
\cline{3-6}
 &  & Generic Platform & Generic platform to register and manage all kinds of digital 
assets & Colu, CoinSpark & Not applicable\tabularnewline
\cline{2-6}
 & \multirow{2}{*}{Marketplace} & E-Commerce & Buying and selling of physical or digital products & O\textbf{penBazaar} & Not 
applicable \tabularnewline
\cline{3-6}
 &  & Digital (micro) Commerce & Buying and selling digital micro services like 
individual API usage & 2\textbf{1} & Device sales (currently)\tabularnewline
\cline{2-6}
 & \multirow{2}{*}{Notary} & Records Management & Service to provide data integrity and auditability & F\textbf{actom} & Token 
sale\tabularnewline
\cline{3-6}
 &  & Identity & Service to provide an identity to authenticate across the Intern & O\textbf{nename}, 
Shocard & Not applicable \tabularnewline
\cline{2-6}
 & \multirow{3}{*}{Financial} & Payment processor & Payment services for merchants & BitPay, Coinbase & Transaction 
fee (\%)\tabularnewline
\cline{3-6}
 &  & Remittance & Global transfer of money across boarders & Abra & Flat fee for 
depositing/withdrawing funds\tabularnewline
\cline{3-6}
 &  & OTC Infrastructure & Infrastructure to trade commodities, financial instruments, 
and derivatives & Symbiont, Mirror & Project based\tabularnewline
\cline{3-6}
 &  & Lending & Service to facilitate peer-to-peer lending & Bitbond & Transaction 
fee (\%)\tabularnewline
\cline{3-6}
 &  & Crowdfunding & Service to facilitate crowdfunding for projects & Koinify, 
Swarm & Transaction fee (\%)\tabularnewline
\hline
\multirow{5}{*}{\rotatebox[origin=rB]{90}{\parbox{2cm}{\textbf{Bitcoin Ecosystem} \\\textbf{without Challengers}}}}
& Blockchain Technology & Sidechains & Technology for custom 
blockchains compatible with bitcoin & Blockstream & Project based\tabularnewline
\cline{2-6}
 & Analytics & Analytics & Service for browsing and analyzing blockchains & Coinalytics, 
Blockchain & Project based\tabularnewline
\cline{2-6}
 & Mining & Mining & Manufacture and/or operating mining infrastructure & BitFury, 
KnC Miner & Bitcoin mining, device sales\tabularnewline
\cline{2-6}
 & \multirow{3}{*}{Financial} & Wallet / Vault & Product or service to store private keys, and optionally 
initiate transactions & Case, Xapo, Coinbase & Ancillary services\tabularnewline
\cline{3-6}
 &  & Exchange & Service to trade bitcoins for other currencies & Bitstamp, Kraken, 
Gem & Transaction fee (\%)\tabularnewline
\cline{3-6}
 &  & Compliance & Service to provide KYC and AML compliance\footnote{ KYC: Know 
your customer. AML: Anti money laundering. Both are bank regulations} & Elliptic, 
BlockScore & Transaction fee (\%)\tabularnewline
\hline
\end{tabular}}
\label{tbl:ecosystem}

\caption{A categorization of the (venture-backed) Bitcoin start-up ecosystem.}
\end{table}

\section{Blockchain Ecosystem beyond Bitcoin}

\subsection{Altcoins and Metacoins}
\label{sec:eco:altmeta}

Bitcoin is an open source project with a public code base. Since a blockchain is defined by its history starting from the genesis block, new cryptocurrencies can be created with the same code base starting from a new genesis block. However, more interesting than cryptocurrencies based on the same code base are variations thereof. An early example of a Bitcoin-derived, and reasonably successful, cryptocurrency is \emph{Litecoin}. Litecoin has different technical parameters (e.g. block generation rate is targeted at 2.5 min instead of 10 min) and parameters concerning monetary policy (84 million litecoins instead of 21 million bitcoin). Other \emph{coins} started from the idea of Bitcoin but are not based on a mere fork of the code base, but are developed on a new base to build improved technical features such as increased privacy. For example \emph{Monero} uses one-time ring signatures and stealth address to create an opaque blockchain in order to resist analysis and to provide unlinkable transactions. Similarly, Ethereum was started from scratch, but adopted many ideas from Bitcoin. Table \ref{tbl:eco:coin} displays the ten \emph{crypto coins} with the highest market capitalization. The market capitalization of a given coin is calculated by its current available supply multiplied by its exchange rate or price in a base currency. The term \emph{coin} is used deliberately in contrast to currency, since three of the listed coins have special roles in particular decentralized applications.

\begin{table}\footnotesize
  \centering
  \begin{tabularx}{\textwidth}{ p{1.5cm} p{1.8cm} p{1cm} X }
    \toprule
    \textbf{Coin} & \textbf{Market Cap.} \newline \textbf{[M USD]} & \textbf{Price} \newline \textbf{[USD]} & \textbf{Description} \\
    \midrule
    Bitcoin & 12,330 & 768.87 & First decentralized cryptocurrency \\
    Ethereum & 704 & 8.11 & Decentralized app and smart contract platform \\
    Ripple & 243 & 0.001 & Interbank value transfer \\
    Litecoin & 178 & 3.65 & Early bitcoin clone with slight changes \\
    Monero & 106 & 7.81 & Cryptocurrency with improved privacy \\
    Ethereum \newline Classic & 82 & 0.95 & Pre-hard-fork Ethereum \\
    Dash & 63 & 9.16 & Cryptocurrency with improved privacy  \\
    Steem & 47 & 0.21 & Appcoin for social media platform\\
    Augur & 35 & 3.17 & Equity token for prediction markets platform \\
    MaidSafe & 30 & 0.07 & Appcoin for decentralized app and storage platform \\
    \bottomrule
  \end{tabularx}
  \caption{Overview of the ten coins with highest market capitalization (11/12/2016).}
  \label{tbl:eco:coin}
\end{table}

\emph{Steem} is a token to reward participation on a social media platform. 

\emph{Augur} is a prediction market platform built on top of Ethereum, and the corresponding token is a form of equity, which entitles the bearer to a share of the market fees on the prediction market platform. In addition, the token obliges the bearer to provide oracle services to the platform, i.e provide truthful information to clear particular prediction markets. Neglecting the obligation or providing erroneous information, as determined by a contest, leads to automated confiscation of tokens by the platform. 

\emph{MaidSafe} denotes the token in the SAFE network. These tokens can be earned by providing storage or computation resources to the network, and have to be spent to access and consume resources by the network.

\emph{Ripple} is a partly permissioned network targeting financial institutions such as banks to lower costs of cross-boarder payments and to replace the traditional correspondent banking system. Partly permissioned, because nodes can in principle individually decide with whom they interact, in practice, however, Ripple Labs, the company behind the software,  provides the majority of nodes, and suggests a list of trusted nodes per default. According to Ripple Labs, the tokens (XRPs) serve two main functionalities. First, \ac{DoS} prevention by requiring to \emph{burn} a certain amount of XRP with transaction. Second, a counterparty-free, network-native, currency to provide liquidity between entities aiming to facilitate cross-currency transactions. 

\emph{Ethereum Classic} is also worth noting. After \emph{The DAO} contract got drained by an anonymous attacker capturing about \$ 50 million USD worth of ether, parts of the Ethereum community, supported by Vitalik Buterin and the Ethereum Foundation, decided to return the funds by means of a software upgrade leading to a hard fork. This decision was controversial, and parts of the community decided to continue with the \emph{classic} software. Hence, there are two Ethereum blockchains and currencies sharing the same history up to block no. 1920000. 

These coins are only the tip of the iceberg. In December 2016, Coinmarketcap lists 705 different coins of which 565 are actively traded, and thus have a market price. Figure \ref{fig:eco:mcap} shows the market capitalization of all 565 coins ordered from highest to lowest on a logarithmic scale. There is a broad regime of exponential decay enclosed by two regimes of super-exponential decay at the edges. This shows that the value of all currencies and tokens is mainly centered on a few with Bitcoin at the top. Figure \ref{fig:eco:bitcoindominance} shows the share of Bitcoin of the total market capitalization of all coins over time. The long term trend is slightly downwards. However, recently Bitcoin's dominance is strongly increasing from 80\% to more than 85\%. Figure \ref{fig:eco:googlesearch} shows the Google Trends query index of the search terms \emph{Bitcoin}, \emph{Blockchain}, and \emph{Ethereum} from 2012 to December 2016. "The query index is based on query share: the total query volume for the search term in question within a particular geographic region divided by the total number of queries in that region during the time period being examined. The maximum query share in the time period specified is normalised to be 100, and the query share at the initial date being examined is normalised to be zero" \parencite{ECOR:ECOR809}. \emph{Bitcoin} clearly dominates the Google search queries. The term \emph{blockchain} appreciates almost linearly since fall 2015. The interest in \emph{Ethereum} appreciated in the beginning of 2016 and maxed in the summer with the rise and fall of \emph{The DAO}.

Of course there are more metrics than the ones presented here. In particular metrics that might have a predictive element such as number of develops working on core and ancillary software, number and size of meetups and conferences around the world. Furthermore, the combined market price of an underlying coin together with meta coins could be interesting to consider. 


\begin{figure}[ht]
  \centering
  \begin{subfigure}[t]{0.5\linewidth}
    \centering
	\includegraphics[width=0.95\textwidth]{./externalized/mcapcoins.pdf}
	\caption{Coins listed on Coinmarketcap sorted by market capitalization.}
	\label{fig:eco:mcap}
  \end{subfigure}
  \begin{subfigure}[t]{0.5\linewidth}
  	\centering
	\includegraphics[width=0.95\textwidth]{./externalized/googlesearch.pdf}
	\caption{Google Trends Search Volume Index comparison.}
	\label{fig:eco:googlesearch}
  \end{subfigure}%
  \\
  \begin{subfigure}[t]{\linewidth}
  	\centering
	\includegraphics[width=0.7\textwidth]{./externalized/bitcoindominance.png}
	\caption{Share of Bitcoin of the total market cap of all cryptocurrencies.}
	\label{fig:eco:bitcoindominance}
  \end{subfigure}%
  \caption{Illustration of Bitcoin's dominance in the cryptocurrency space}
\end{figure}


\subsection{Venture Capital Investments}

Figure \ref{fig:eco:vcfund} shows the monthly venture capital investments in the entire blockchain ecosystem. The date denotes the actual investment. Companies with multiple investment rounds show up repeatedly. In the figure, the blockchain ecosystem is divided into five distinct categories. Companies are allocated to a given category if thy focus on a specific ecosystem or technology. Exchanges are classified as Bitcoin, although trading of additional coins is generally possible. 

In most months the investments in Bitcoin start-ups vastly outweighs investments into the other categories. The only category with increasing investments is that of permissioned blockchains.
\begin{figure}[ht]
\centering
\includegraphics[width=0.9\textwidth]{./externalized/monthlyVC.pdf}
\caption{Monthly venture capital investments in blockchain start-ups.}
\label{fig:eco:vcfund}
\end{figure}

Table \ref{tbl:eco:mostfunded} lists the ten start-ups with highest venture capital funding. Only three of them can be clearly allocated to the field of permissioned blockchains. \emph{Ripple Labs} was already briefly discussed in the last section. \emph{Digital Asset Holdings} builds a distributed and cryptographically-secured network of known entities to improve efficiency, security, compliance and settlement speed in capital markets with applications in post trade and repo clearing. \emph{Chain} started as a Bitcoin company, providing developer \ac{API}s to the Bitcoin network, but pivoted towards permissioned blockchain technology, and recently released their open source blockchain solution \emph{Chain Core}. According to their website, "Chain works with leading companies including Visa, Nasdaq, Fiserv, Citigroup, Capital One, Orange, State Street, MUFG, and many more"\footnote{\url{https://chain.com/faq/} accessed 12/15/2016}.

\begin{table}\footnotesize
  \centering
  \begin{tabularx}{\textwidth}{ l p{1.8cm} X l }
    \toprule
    \textbf{Company} & \textbf{Funding \newline [M USD]} & \textbf{Sector} & \textbf{Blockchain} \\
    \midrule
    Circle Internet Financial & 136 & Financial Services & Bitcoin \\
    21 Inc & 121.05 & Infrastructure / Marketplace & Bitcoin \\
    Coinbase & 116.5 & Wallet / Vault / Exchange & Bitcoin \\
    Ripple Labs & 96 & Infrastructure & Permissioned \\
    Blockstream & 76 & Infrastructure & Bitcoin \\
    Digital Asset Holdings & 60 & Infrastructure & Permissioned \\
    BitFury & 60 & Mining / Technology & Bitcoin \\
    Chain & 43.7 & Infrastructure & Permissioned \\
    Xapo & 40 & Wallet / Vault / Exchange & Bitcoin \\
    BitFlyer & 33.94 & Exchange & Bitcoin \\
    \bottomrule
  \end{tabularx}
  \caption{Overview of the ten most venture-backed start-up companies in the blockchain ecosystem.}
  \label{tbl:eco:mostfunded}
\end{table}


Thus, venture capital is focused on Bitcoin companies and on companies that build blockchain-based infrastructure with applications to financial services and capital market.


\subsection{Nakamoto Business Model, Crowdsales and Initial Coin Offerings}
\label{sec:eco:crowdsale}

Bitcoin essentially proved a new type of business model that can also be used as a financing mechanism. \cite{Brener2016} introduced the term \emph{Nakamoto Business Model} for a business model that allows a company or a team of developers behind a decentralized network or a software protocol to generate income. The basis of this business model is that the network or the application is tied to a particular coin. \cite{Brener2016} explains the standard procedure of the Nakamoto Business Model as follows (\cite{Brener2016} uses the term user token instead of appcoin):
\begin{enumerate}
	\item Publish a white paper defining the specifications of the network and a road map for its future development.
	\item Publicly announce the token and release the source code prior to creating the first token.
	\item Deploy the network and secure user tokens via mining. Alternatively, allocate a portion of the pre-sale tokens to the founding team as a reward for ideating and developing the network.
	\item Advertise the network and sell user tokens to anyone, anywhere.
	\item Work to grow the number of people using. building apps on top and maintaining the network.
\end{enumerate}

In the case of Nakamoto and Bitcoin this model created an organization with a market capitalization of more than \$12 bn USD and estimations are that Satoshi Nakamoto owns around 1 million bitcoins \parencite{Lerner2016}. A related concept is that of a crowdsale or \ac{ICO}. The most prominent \ac{ICO} was that of Ethereum. During an online crowdsale during July to August 2014 global investors bought ether with a value of more than 30 000 bitcoins, a year before the actual network went live. The bitcoins were held by the Swiss non-profit foundation \emph{Stiftung Ethereum} and used to fund the development. The investor accounts were then included into the genesis block of the Ethereum blockchain. Ethereum itself allows to implement coins, mostly termed tokens, and crowdsales in form of smart contracts. The most prominent was \emph{The DAO} which collected more than \$150 million USD worth of ether in April and May 2016. Although this project ended early and badly because of a software bug in the contract code that allowed hackers to steal the collected ether funds, the crowdsale model on top of Ethereum is still used. Figure \ref{fig:eco:crowdfunding_detailed} gives an overview of successful crowdsales classified by the underlying platform to implement the coins or tokens. In particular since 2016, most of the crowdsales with a sum above \$5 million USD are on the Ethereum network. Besides that, the Bitcoin or Ethereum crowdsale model with native network coins is still popular. Figure \ref{fig:eco:crowdfunding_all} shows the combined quarterly of value of the crowdsales. It can be seen that the importance of this model has significantly increased over the last year. 

\begin{figure}[ht]
  \centering
  \begin{subfigure}[t]{0.5\linewidth}
    \centering\includegraphics[width=0.95\textwidth]{./externalized/crowdfunding_detailed.pdf}
    \caption{Individual projects classified by underlying blockchain.}
    \label{fig:eco:crowdfunding_detailed}
  \end{subfigure}%
  \begin{subfigure}[t]{0.5\linewidth}
    \centering\includegraphics[width=0.95\textwidth]{./externalized/crowdfunding_all.pdf}
    \caption{Total quaterly crowdfunding results}
    \label{fig:eco:crowdfunding_all}
  \end{subfigure}
  \caption{Successful crowdsales and \ac{ICO}s}
\end{figure}

\cite{Ehrsam2016} argues that this new business model will lead to "that businesses that are based on network effects will start to be built \emph{decentralized first}" (emphasize in the original). The appcoin model allows to offset the network effect. In the beginning of new application dependent on network effects or a software protocol the value to a user is low. The value increases with every new user, but the inventive to join in the beginning is low, resulting in a chicken-and-egg problem. Appcoins can provide this incentive. In the beginning of Bitcoin, every user could mine with her personal computer with a reasonable chance to find blocks, and thus receive newly minted bitcoins. Although bitcoins had no price in the beginning because there was no market, users with some confidence in the product were incentivized to provide service to the system in two ways. First, to provide security with mining, and second, by advertising the system and recruit new users. Thus, growing the usefulness and value of the network organically. In other words the intrinsic speculative element of a tradable token required to use the network can help to bridge the gap until network effects alone provide enough value to users. Bitcoin not only proved the that such a model could work, but also provided the first infrastructure, namely pseudonymous electronic cash, to provide value and funding for the next generation of decentralized infrastructure and applications. 

However, the same model can be misused as a \emph{pump and dump pyramid scheme}, where the developers and early users are only interested in a quick financial gain instead of building a sustainable product. 


\subsection{Permissioned Blockchains and Consortia}
\label{sec:eco:consortia}

Permissioned blockchains are blockchains with known participants that do not necessarily fully trust each other. Thus, the likely use of permissioned blockchains is predominately in inter-business relationships with static memberships. Therefore, over the course of the last year a number of consortia around the development, standardization and use of blockchain technology have been established. Figure \ref{fig:eco:consortia} shows 21 blockchain consortia based on their founding date and their country of incorporation. Furthermore, the size and sector is illustrated. 

\begin{figure}[ht]
\centering
\includegraphics[width=0.9\textwidth]{./externalized/consortia.pdf}
\caption{Overview of blockchain consortia.}
\label{fig:eco:consortia}
\end{figure}

The biggest consortia are around standards and the technological infrastructure. The \emph{Hyperledger}  project is an open source collaborative effort to advance cross-industry blockchain technologies. It is hosted by the Linux Foundation and has 100 member companies such as Accenture, IBM, Intel, J. P. Morgan, and the Deutsche B\"orse Group. Currently the Hyperledger project entails four distinct open source projcts:
\begin{itemize}
	\item \textbf{Blockchain Explorer:} User-friendly web application to view and query Hyperledger blockchains.
	\item \textbf{Fabric:} A modular blockchain architecture originating from Digital Asset Holdings and IBM.
	\item \textbf{Iroha:} A blockchain infrastructure contributed by Soramitsu, Hitachi, NTT Data, and Colu.
	\item \textbf{Sawtooth Lake:} Blockchain with novel consensus protocol (proof of elapsed time) using Intel's trusted computing platform SGX \parencite{cryptoeprint:2016:086}.
\end{itemize}

The largest domain specific consortium is R3 with about 70 of the world's biggest financial institutions. In collaboration with development teams of member companies \emph{Corda} was developed and open sourced on November 30. 2016 \parencite{Brown2016corda}. The Corda distributed ledger is explicitly designed to record the state of deals and obligations between institutions and people. In contrast to the general design of blockchains, Corda keeps most of the data private between the interacting parties and potential notaries.

All projects of these consortia are still in the exploration or proof of concept phase. To the authors knowledge non is in production.


\section{Economic Relevance of Cryptocurrencies}

Although permissioned blockchains and related consortia have gotten a lot of attention during the last year, applications of these systems are still in the ideation or proof of concept phase. Bitcoin, in contrast, is used productively by thousands of companies and at least hundreds of thousands of individuals. Hence, this thesis is concerned with the potential of cryptocurrencies. In the following, five important characteristics provided by cryptocurrencies with potential economic relevance are presented. These characteristics are accompanied with the open and permissionless nature of cryptocurrency blockchains, and can hardly be replicated on private or permissioned systems.

\subsection{Inclusion}

Public cryptocurrencies such as Bitcoin or Ethereum aim to be inclusive and censorship resistant. They provide a financial and payment infrastructure reaching as far as the Internet. This provides a novel alternative for inhabitants in developing countries who are often excluded from financial services \parencite{chaia2010counting} or are dependent on a monopoly (e.g. M-Pesa). Cryptocurrencies are built solely on cryptography and mechanism design cast in executable computer code, instead of human institutions, trust or reputation. This provides the basis for anonymity and the equal participation machines and software agents. 

\subsection{Innovation}

Financial institutions have grown over centuries and the financial system is large, complex, and opaque. In contrast, cryptocurrencies are open source, and all code and transactions are publicly verifiable. Individuals and companies are able to create products and services interfacing or building on top of cryptocurrencies. Everyone is able to review code, and to suggest changes and advancements. Effective governance structures are still missing \parencite{Narayanan2015}, but there is always the possibility to fork the code, implement the changes, and deploy a new network. Since everything happens in public, the community can learn effectively from each other, and an environment of cumulative innovation can evolve. 

\subsection{Disintermediation}

Non-duplicable transfer of value over the Internet has always involved a number of trusted and regulated intermediaries such as banks, credit card companies, payment service providers (e.g. PayPal or Stripe). These intermediaries provide valuable service based on the current financial infrastructure, but are also gatekeepers extracting fees. Cryptocurrencies allow to circumvent these intermediaries in many cases, and decentralized application platforms allow to implement certain functions of formerly trusted third parties completely in code. 

\subsection{Automation}

Cryptocurrencies are programmable money in two senses. First, cryptocurrency transactions can include their own verifiable rules under which the transaction is valid or invalid. This allows trustless atomic transactions in some cases (e.g. in the case of trade of blockchain-based tokens, trade involving smart property \parencite{smartcontr}, or zero knowledge contingent payments \parencite{maxwell2016zk}), or high granularity (micropayments) to minimize losses and build up trust, or the inclusion of a competitive market of trustless arbitrators, who can be consulted in case of a dispute. Second, cryptocurrencies provide a native interface for machines. No legal identity is needed to participate, and machines can reason about the finality of payment due to proof-of-work. Traditional payments, except physical coins, are essentially reversible if a human decides. With a cryptocurrency a whole ecosystem of miners, individuals, and companies would have to decide to rollback the blockchain. These characteristics allows automation at edges with machines acting as autonomous economic actors.

\subsection{Novel Business Models}

The Nakamoto business model and variations thereof provide a novel mechanism to finance and create infrastructure and applications based on network effects. This business model does not need the formal creation of a company and includes users naturally as shareholders, thus offsetting network effects and naturally incentivizing users to recruit others.

In addition, automation and disintermediation decrease literal transactions costs. Thus, micropayments become technically viable. Although human micropayments are psychologically discouraged because of mental transaction costs \parencite{szabo1999micropayments}, machines and software agents can make use of direct micropayments to enable new kinds of economic interactions and share resources securely. 


\section{Conclusion}

This chapter provided an overview of the Bitcoin and related blockchain ecosystem that originated with the unexpected success, the first decentralized cryptocurrency. First, the narrow Bitcoin ecosystem is considered, and divided into challengers and the rest. Challengers are companies with the potential to disrupt traditional companies and business models. Non-challengers are in particular companies that exist inside the ecosystem and provide necessary services to sustain and grow the ecosystem. Further, the ecosystem is classified in different sectors and restricted to companies with venture-capital funding. Besides financial services three further important sectors are identified: (1) digital assets, (2) marketplaces, and (3) notary services. Thereafter, the broader blockchain ecosystem is considered. Numerous alternative coins with own blockchains, but also meta coins existing on host chains have emerged. Still, Bitcoin is dominant according to a variety of metrics. Venture capital in the broader ecosystem is almost non-existent, except investments into permissioned blockchains. The emergence of permissioned blockchains without cryptocurrencies goes hand in hand with the emergence of industry and cross-industry consortia exploring the applicability of blockchain technology in a variety of sectors. However, although little venture funding flows into companies behind and around alternative public blockchains, the sector blooms because of new forms of cryptocurrency-powered crowdfunding and variations of the Nakamoto business model. The chapter closes with a discussion of five characteristics with potential economic relevance which are implicitly derived from the investigation of the ecosystem: (1) inclusion, (2) innovation, (3) disintermediation, (4) automation, and (5) novel business models. These characteristics further motivate the investigation of the application of cryptocurrencies in the Internet of Things. 
