\chapter{The Bitcoin Ecosystem: Disruption Beyond Financial Services?}

\section{Introduction}

Incepting from a white paper (Nakamoto, 2008) published under the pseudonym Satoshi 
Nakamoto on a cypherpunk mailing list in late 2008 the first peer-to-peer electronic 
cash system - bitcoin - has developed into a global network of thousands of computers 
(with more than 5000 full nodes), a cryptocurrency with a market capitalization 
of more than \$5B (Nov. 2015), and a vivid ecosystem of open source projects, start-up 
companies, as well as industry endeavors. According to recent data, almost \$1B 
of Venture Capital has been invested in the bitcoin ecosystem since 2011 (CoinDesk, 
2015). 

Bitcoin is a particular implementation of a novel information technology, called 
blockchain, combining cryptography, peer-to-peer computing, and economic incentives 
to enable systems with networked trust. Instead of trusting a single custodian, 
system-wide consensus is reached by an ever-growing proof of computational work. 
Arguably, this technology has the potential for disintermediation and disruption 
(Baiyere et al., 2015), which is not limited to the financial service industry. 

As shown by the literature review of Morisse (2015), the recent phenomena of cryptocurrencies 
has not yet reached mainstream IS research. Although there are several interesting 
links to IS research domains. For example, the social context of Bitcoin as a currency 
has already been investigated by IS scholars: Glaser et al. (2014) and Hur et al. 
(2015) analyzed the participation and intentions of Bitcoin users. Ingram et al. 
(2015) explored the resilience of the Bitcoin entrepreneurs facing the early crashes 
of the bitcoin currency like the bankruptcy of Mt. Gox (Dougherty and Huang, 2014). 

However, as Giaglis and Kypriotaki  (2014) have highlighted: an interesting area 
for IS scholars could be \emph{to assist the transition from the first era of applications 
[...] (i.e. Bitcoin as currency) to more disruptive uses of theBbitcoin protocol 
as an enabler of decentralized trusted peer-to-peer transaction ledger systems 
and applications.} (p. 3) We've followed the development of the Bitcoin ecosystem 
closely over the last years and acknowledge that practitioners are far ahead of 
IS scholars. A first technical taxonomy for decentralized consensus systems has 
been developed (Glaser et al., 2015). Moreover, digital business models of Bitcoin 
start-ups that depend mainly on the currency aspect of Bitcoin have been inspected 
in form of case studies (Kazan et al., 2015). We aim to advance the discussion 
beyond the cryptocurrency aspect and the financial service industry. Therefore 
our contribution is twofold. (1) We categorize the venture-backed Bitcoin start-up 
ecosystem along two dimensions, i.e. the potential for disruption and the particular 
sector, and present the evolution of the ecosystem since its inception. (2) Based 
on the categories, we investigate six companies in case studies with the aim to 
extract the core innovations and the features of Bitcoin that fuel those innovations, 
and critically discuss their potential for disruption.

The structure of the paper is as follows. In Section 2 we give a slightly technical 
introduction to Bitcoin as an implementation of a novel information technology. 
The focus thereby is on explaining the features that are utilized by the representative 
cases. In Section 3 we present the methodology by which we collected data, how 
we selected the cases, and how we analyzed them. Section 4 is dedicated to the 
introduction and categorization of the Bitcoin start-up ecosystem, and we discuss 
its evolution over time. Thereafter, we present the case studies in Section 5, 
followed by key findings in Section 6. We conclude with a critical discussion of 
the disruptive potential and avenues for future research.

\section{Innovation, Disruption and Disruptive Innovation}

In a broader definition, innovation can be described as \emph{the generation, acceptance, 
and implementation of new ideas, processes, products or services.} (Thompson, 
1965, p. 2). Depending on the context, the term \emph{innovation} incorporates divergent 
attributes, such as different process stages or underlying constructs (Baregheh 
et al., 2009). Hence, the purpose of an innovation process could be to change or 
improve products as well as entire business models (Bucherer et al., 2012). Business 
models can be described as \emph{the rationale of how an organization creates, delivers 
and captures value} (Osterwalder and Pigneur, 2010). Describing the logic of a 
company's business activities (Linder and Cantrell 2000), the business model comprises 
four high-level components including value proposition, operational model, financial 
model and customer relations. (Bucherer et al., 2012). 

Depending on whether a company introduces a new service or changes its current 
business model, innovations can be characterized by their \emph{degree of newness}, 
a metric, which Garcia and Calantone (2002) describe as \emph{Innovativeness} and 
\emph{the degree of discontinuity in marketing and/or technological factors} (Garcia 
and Calantone, 2002, p. 112). In recent years the two notions \emph{radical} and \emph{disruptive}
have received increased attention from scholars and practitioners alike to describe 
innovations with a high degree of innovativeness (Christensen et al., 2015; Latzer, 
2009). 

Radical innovations refer to highly discontinuous (technological) changes (Latzer, 
2009) and are characterized by totally new features, high uncertainty and the necessity 
for companies to acquire new capabilities to fully exploit emerging opportunities 
(Latzer, 2009). Additionally, these often technology-driven innovations can have 
the potential to shift existing paradigms and might possibly disrupt industries 
(Latzer, 2009). Following Christensen's theory, disruptive innovation can be described 
as a \emph{new product [or service] encroach[ing] on the low end of the existing market} 
(Schmidt and Druehl, 2008, p. 348) with the potential to move upward to satisfy 
higher customer expectations (Christensen, 1997, 2006). This is possible because 
incumbents focus on improving their offerings along performance dimensions that 
are valued specifically by the majority of demanding customers, while entrants 
focus on different performance dimensions that are (initially) valued only by niche 
segments, which are usually neglected by incumbents. At the same time, Christensen 
et al. (2015) define \emph{disruption} as \emph{a process whereby a smaller company with 
fewer resources is able to successfully challenge established incumbent businesses}. 
Hence, a disruptive innovation does not inevitably leads to the disruption of a 
market, nor does a disruption of a market necessarily has to be triggered by a 
disruptive innovation. 

The literature on disruptive innovation is currently still in an early stage (Markides, 
2006) and it remains controversial, how to define the concept (Danneels, 2004; 
Yu and Hang, 2010). While some researchers support Christensen's understanding, 
contributing own thoughts to the theory, others generally \emph{criticize the vagueness 
of the concept} (Yu and Hang, 2010, p. 438). According to its critics Christensen's 
theorization does not allow for a clear differentiation between underperforming 
technologies and potentially disruptive technologies with initially inferior performance 
(Tellis, 2006) Finally, a main criticism focuses on the lack of measurability of 
disruptive innovation (Govindarajan and Kopalle, 2006). To overcome this gap, Govindarajan 
and Kopalle (2006), define the following criteria to identify disruptive innovations. 
The innovation should (1) possess inferior attributes regarding what mainstream 
customers would value; (2) attract new customer segments by offering new value 
propositions; (3) be offered at lower costs; and (4) start from niche markets (Yu 
and Hang, 2006; Govindarajan and Kopalle, 2006).

\section{Methodology}

\textbf{Mapping the Bitcoin ecosystem}: To attain a wide dataset that adequately 
represents the presently existing Bitcoin start-up ecosystem, we utilized a variety 
of publically available online sources: The start-up platform AngelList (2015), 
searched with the broad keyword \emph{cryptocurrency}, a compiled list of crypto technology 
companies by industry expert William Mougayar (2015) and Coindesk (2015), a reputable 
and well known bitcoin news site, which maintains a comprehensive list of venture 
capital invested into Bitcoin start-ups. To validate and add additional data to 
the compiled set we used the websites of the individual companies as well as press 
releases and the Crunchbase database (2015). We identified a total of 704 start-ups 
and projects in the cryptocurrency and blockchain space, which are existent today. 
Of these, 599 belong to the Bitcoin ecosystem, and 65 received venture capital. 

Following an iterative process, we identified representative categories that help 
understanding the Bitcoin ecosystem. 

\textbf{Selecting and analyzing the representative companies: }Based on this classification 
and our focus beyond financial services, we identified six application-specific 
subsectors for investigation by representative case studies. The particular companies 
were selected based on funding and attention by the community. The case studies 
are grounded in a rich set of secondary data sources. Besides company websites, 
white papers and press releases, there are detailed founder interviews on Zapchain 
(2015) and Epicenter Bitcoin (2015). 

Although our research approach was inherently exploratory we had two guiding questions 
in mind: (1) What is the core innovation of the company, and (2) on which Bitcoin 
features is this innovation based.

\section{The Bitcoin Start-up Ecosystem}

The open and permissionless nature of Bitcoin has led to a Cambrian explosion of 
projects and start-up companies. Table 1 provides an overview of the diversity 
of the ecosystem. Notably, the table is restricted to the venture-backed Bitcoin 
ecosystem. Projects and companies without funding as well as the growing extended 
ecosystem, consisting of Bitcoin-inspired alternative coins and blockchains, were 
not taken into account. 

\begin{figure}[htbp]
\centering
\includegraphics[width=449pt, height=118pt, keepaspectratio=true]{./externalized/ecis-fig002.png}
\caption{Distinction between challengers and the Bitcoin ecosystem without 
challengers.}
\ref{fig:challengers}
\end{figure}

We divide the ecosystem in two main categories (see Figure 2 for illustration). 
The first category consists of start-up companies that act mainly inside the Bitcoin 
ecosystem itself. Examples are wallet providers, mining operations and Bitcoin 
exchanges. We refer to this category as \emph{Bitcoin ecosystem without challengers}. 
Hence, the second category is termed challengers. Those are companies that use 
Bitcoin technology to attack traditional companies and business models outside 
the Bitcoin ecosystem. Examples of this category are payment processors like BitPay 
(2015), challenging traditional online payment processors like PayPal, and remittance 
services like Abra (2015), challenging incumbents like Western Union by cutting 
down transaction fees through disintermediation and decreased vulnerability to 
fraud. The classification of companies is not always clear-cut and may change as 
the ecosystem develops, but provides a lens to identify sectors, which might get 
disrupted first.

Interestingly, the challenger category is not limited to the financial service 
industry. We identified three main sectors beyond financial services where start-ups 
use Bitcoin technology to innovate and thereby challenge incumbents: (1) notary 
services, (2) marketplaces, and (3) digital assets. Notary services use the Bitcoin 
blockchain as an immutable public database and time-stamping service. Applications 
are records management, by providing verifiable audit trails and provable data 
integrity, as well as identity registries, which are not tied to a particular identity 
provider. Marketplaces provide a decentralized infrastructure where physical as 
well as digital goods and services can be traded for bitcoins. The digital assets 
sector is concerned with the management of \emph{anything that exists in a binary format 
and comes with the right to use} (Wikipedia, 2015). This entails digital art, 
photographs, music, but also coupons and tickets. Furthermore, we extended the 
concept of digital assets to incorporate the Internet of Things (IoT), since it 
is concerned with the digital representation of physical devices. We will discuss 
these sectors in more detail by looking at representative cases in Section 5.

\begin{figure}
\centering
\includegraphics[width=\textwidth]{./externalized/projects.pdf}
\caption{Number of new Bitcoin projects over time.}
\end{figure}

\begin{figure}
\centering
\includegraphics[width=\textwidth]{./externalized/scatter.pdf}
\caption{Evolution of the venture-backed Bitcoin start-up ecosystem.}
\end{figure}

Figures 3 and 4 give an overview of the temporal evolution of the ecosystem. Figure 
3 shows the evolution of all Bitcoin projects whereas Figure 4 is restricted to 
the venture-backed Bitcoin start-up ecosystem. Notably, there are no companies 
that were founded in the first two years of Bitcoin's existence. Projects that 
started earlier have not survived (e.g. Mt. Gox, 2015) or have not attracted venture 
capital. The following three years (2011-2013) are characterized by companies building 
the infrastructure for the Bitcoin ecosystem. From 2014 on, the number of challengers 
has grown, and the sectors beyond the financial sector have gained traction. Hence, 
these companies are very young and are just in the process of entering the market.

\resizebox{\textwidth}{!}{
\tiny
\begin{tabular}{|>{\raggedright}p{17pt}|>{\raggedright}p{46pt}|>{\raggedright}p{68pt}|>{\raggedright}p{115pt}|>{\raggedright}p{53pt}|>{\raggedright}p{85pt}|}
\hline
 & S\textbf{ector} & S\textbf{ubsector} & D\textbf{escription} & R\textbf{epresentatives} & R\textbf{evenue 
Mechanics}\tabularnewline
\hline
C\textbf{hallengers} & Digital Assets & Internet of Things & Service to register 
and manage connected devices & F\textbf{ilament} & Product as a Service\tabularnewline
\hline
 &  & Intellectual property & Service to register and manage IP like music and 
art & A\textbf{scribe}, Monegraph & Transaction fee (\%)\tabularnewline
\hline
 &  & Generic Platform & Generic platform to register and manage all kinds of digital 
assets & Colu, CoinSpark & Not applicable\tabularnewline
\hline
 & Marketplace & E-Commerce & Buying and selling of physical or digital products & O\textbf{penBazaar} & Not 
applicable \tabularnewline
\hline
 &  & Digital (micro) Commerce & Buying and selling digital micro services like 
individual API usage & 2\textbf{1} & Device sales (currently)\tabularnewline
\hline
 & Notary & Records Management & Service to provide data integrity and auditability & F\textbf{actom} & Token 
sale\tabularnewline
\hline
 &  & Identity & Service to provide an identity to authenticate across the Intern & O\textbf{nename}, 
Shocard & Not applicable \tabularnewline
\hline
 & Financial & Payment processor & Payment services for merchants & BitPay, Coinbase & Transaction 
fee (\%)\tabularnewline
\hline
 &  & Remittance & Global transfer of money across boarders & Abra & Flat fee for 
depositing/withdrawing funds\tabularnewline
\hline
 &  & OTC Infrastructure & Infrastructure to trade commodities, financial instruments, 
and derivatives & Symbiont, Mirror & Project based\tabularnewline
\hline
 &  & Lending & Service to facilitate peer-to-peer lending & Bitbond & Transaction 
fee (\%)\tabularnewline
\hline
 &  & Crowdfunding & Service to facilitate crowdfunding for projects & Koinify, 
Swarm & Transaction fee (\%)\tabularnewline
\hline
 & S\textbf{ector} & S\textbf{ubsector} & D\textbf{escription} & R\textbf{epresentatives} & R\textbf{evenue 
Mechanics}\tabularnewline
\hline
B\textbf{itcoin Ecosystem without }\linebreak{}
\textbf{Challengers} & Blockchain Technology & Sidechains & Technology for custom 
blockchains compatible with bitcoin & Blockstream & Project based\tabularnewline
\hline
 & Analytics & Analytics & Service for browsing and analyzing blockchains & Coinalytics, 
Blockchain & Project based\tabularnewline
\hline
 & Mining & Mining & Manufacture and/or operating mining infrastructure & BitFury, 
KnC Miner & Bitcoin mining, device sales\tabularnewline
\hline
 & Financial & Wallet / Vault & Product or service to store private keys, and optionally 
initiate transactions & Case, Xapo, Coinbase & Ancillary services\tabularnewline
\hline
 &  & Exchange & Service to trade bitcoins for other currencies & Bitstamp, Kraken, 
Gem & Transaction fee (\%)\tabularnewline
\hline
 &  & Compliance & Service to provide KYC and AML compliance\footnote{ KYC: Know 
your customer. AML: Anti money laundering. Both are bank regulations} & Elliptic, 
BlockScore & Transaction fee (\%)\tabularnewline
\hline
\end{tabular}}

\textit{Table 1.  A categorization of the (venture-backed) Bitcoin start-up ecosystem.}

\section{Case Studies}


In the following, we investigate the challengers beyond financial services in more 
detail. Therefore, we present six comprehensive case studies (marked bold in Table 
1).

\subsection{Filament (Internet of Things)}
\label{sec:ecofilament}

Filament (2015) provides wireless sensor networks for the (industrial) IoT, e.g. 
for smart cities or smart agriculture use case. Most IoT platform providers follow 
a centralized approach by connecting all devices to their respective cloud-infrastructure. 
This has the major disadvantage that devices depend on a central infrastructure 
in order to operate. Moreover, it can be argued that this approach cannot keep 
up technically and economically with the increasing number connected devices.

Filament is one of the first companies that develop a fully decentralized IoT infrastructure, 
which encompasses three blockchain-related aspects: (1) Each device is registered 
on the blockchain providing a verifiable and immutable identity. This enables discovering 
of and authenticating with other devices/services without the need of a dedicated 
backend infrastructure. Therefore, devices are technically autonomous and are able 
to operate independently of Filament. (2)~Each device is governed by a ``smart'' 
contract, which manages agreements of device control/ownership, data access and 
financial agreements concerning the device. Ownership can be transferred permanently 
or temporarily by a simple transaction on the blockchain. Filament implements the 
financial agreements as a Product-as-a-Service, which means that the owner gets 
paid directly for the ongoing use of the device. (3) Furthermore, each device is 
able to transfer value in form of bitcoins to other devices in order to get access 
to data or request some service.

As described, devices can be operated and governed by using only the blockchain 
as a backend and therefore without any technical dependence on the platform creator 
(Filament) or other third parties. This might bare great benefits for customers 
needing to deploy large Industrial IoT applications with a lifetime of 5-10 years. 
Because they want to minimize the risk of a lock-in with a specific company. Moreover, 
according to Filament, customers prefer paying continuously on a real-time basis 
instead of an upfront investment, which can be solved efficiently by Bitcoin micropayments. 
Ownership is decoupled from usage and both are independent of the manufacturer 
or a platform provider.

Filament itself is a venture capital-backed company formerly known as Pinocc.io. 
They claim to have their first deployments with Fortune 500 companies in 2016. 
They will get paid for the ongoing use of their devices (by owning the smart contract). 
Moreover, they work on a licensing model, i.e. Customers can attach a module version 
to their own devices, which will give them all the described benefits of Filament 
in return to a small fraction of the payments for the ongoing use of the devices. 
However, since all protocols will be open and there is no dependence to Filament 
by design, other companies could use that and build their own hardware without 
Filament.

\subsection{Ascribe (Intellectual Property)}
\label{sec:ecoascribe}

Ascribe (2015) aims to provide provenance of intellectual property. Digital work, 
like art, photos, and music, can be registered publicly on the bitcoin blockchain 
together with its accompanying terms and conditions. The technological basis is 
the spool protocol (De Jonghe and McConghy, 2015), an overlay protocol that uses 
Bitcoin transactions to represent unforgeable ownership transfers and licensing 
agreements for digital work. Thus, authenticity of ownership and usage rights can 
always be proven. Ascribe focuses on digital art in particular. Although art is 
in principle copyrighted by the time of creation, it is currently cumbersome and 
expensive to officially register work and prove ownership (see e.g. http://copyright.gov/docs/fees.html). 
Therefore, hardly any digital art gets registered. Ascribe aims to change the status 
quo by providing a virtually free and automatable registration process. Moreover, 
the unique representation of digital work based on cryptography and the blockchain 
is the basis for the creation of a secondary market for digital work. For example 
the spool protocol enables creating limited editions of digital work. So far, this 
has not been possible without relying on some central institution.

The spool protocol is open source and can be used by anybody. In principle, the 
only costs are bitcoin transaction fees. However, Ascribe wraps the protocol in 
convenient web services and provides tools adapted to particular customer groups, 
e.g. individual creators, museums and marketplaces. Ascribe's revenue model is 
then based on a share of the rentals and sales of registered digital work that 
are facilitated by their APIs.

The core innovation is the application of the bitcoin blockchain to provide commoditized 
provenance of intellectual property. Provenance is demonstrated by relying on the 
immutability of the bitcoin blockchain, instead of an authority. 

\subsection{OpenBazaar (E-Commerce Marketplace)}
\label{sec:ecobazaar}

OpenBazaar (2015) is an open source project consisting of a protocol and a reference 
implementation that enables a decentralized peer-to-peer e-commerce marketplace. 
In comparison to traditional e-commerce market places like eBay and Amazon, there 
is no central server or authority that is running the market place. Thus, there 
is no middleman who is able to charge fees or to restrict offered products and 
services. Everyone with Internet access is able to set up a shop by running a network 
client. Payments are facilitated using Bitcoin transactions. Therefore, no payment 
provider or banking account is needed. This lowers payment transaction fees and 
increases the global reach. Besides sellers and buyers there are notaries and arbiters 
for dispute resolution participating in the marketplace. Those latter participants 
are involved by using bitcoin multi-signature transactions. Thus, OpenBazaar unbundles 
the functions of traditional marketplaces. Trades on the OpenBazaar network are 
based on Ricardian Contracts (Grigg, 2004), i.e. an electronic document that defines 
the terms of a trade such that it is readable by computers and humans, and is cryptographically 
signed. Apart from selling physical and digital products, OpenBazaar can also be 
used to trade speculative contracts, which can be readily represented by Ricardian 
Contracts. 

The main value proposition of OpenBazaar to sellers as well as to buyers is the 
elimination of fees and restrictions. Since marketplaces are subject to network 
effects most users will most probably not switch immediately from traditional marketplaces 
to OpenBazaar. However, OpenBazaar could set foot in under-served niche markets. 
Examples could be digital goods, developing countries with limited access to traditional 
payment services, and prohibited goods. 

OpenBazaar itself is not a company, but its main developers founded the venture 
capital-backed company OB1. Their current focus is on developing the OpenBazaar 
protocol and its reference implementation. As outlined above, OB1 is not able to 
profit directly from transactions on the marketplace in the way traditional revenue 
mechanics on centralized marketplaces work. 

\subsection{21 (Digital Micro Commerce Marketplace}
\label{sec:eco21}

At first sight, the categorization of 21 Inc. (2015) as a challenger seems odd, 
since 21 is an infrastructure and platform provider for the bitcoin ecosystem. 
Indeed, it is often categorized as a mining company. However, we argue that 21 
is better classified as a marketplace for digital micro services, which has the 
potential to challenge traditional Internet business models.

With a funding of \$121M, 21 supersedes every other start up in the bitcoin ecosystem. 
They have developed an embeddable ASIC\footnote{ Application-specific integrated 
circuit. A hardware chip customized for a particular task in contrast to a general 
CPU. In this case task is Bitcoin mining, which involves the extensive execution 
of SHA265 operations.} mining chip that they are using in their own mining operations, 
but which is also embeddable into arbitrary connected devices. In November 2015 
they released their first product, the 21 Bitcoin computer. Essentially the 21 
Bitcoin computer is a full-stack development platform to build bitcoin-payable 
digital services, which can be published and discovered on 21's digital marketplace. 
Individual service consumptions, like an API call, can be billed at as little as 
1 satoshi\footnote{ Satoshi is the smallest denominator of a bitcoin. 100,000,000 
satoshi correspond to 1 bitcoin. Thus, 1 satoshi is currently worth approximately 
USD 0.000003.}. The embedded mining chip, which is currently coupled to a mining 
pool (Antonopoulos, 2014, pp. 209-211) operated by 21, supplies the device with 
a continuous stream of satoshis.

21 aims to embed their chips into any connected device (e.g. smart phones) to establish 
bitcoin as a system resource like CPU, bandwidth or disk space, but for the purpose 
of buying and selling digital goods and services (Srinivasan, 2015). It is crucial 
to understand that it does not make sense to sell the small amounts of mined bitcoins 
for Fiat currency on an exchange. Instead, the idea is to supply every device with 
a continuous stream of bitcoin from the point of commissioning on so that it can 
directly operate on the marketplace. 

Having such an infrastructure of bitcoin-enabled devices in place at scale, could 
offer compelling new opportunities and even disrupt traditional business models 
of the Internet. For example, it has been difficult for news sites to directly 
monetize their content on the Internet. It is still tedious for users who want 
to read just one article to signup, enter their credit card information and buy 
a subscription. Thus, most news sites still depend on indirect revenue by advertisements, 
which becomes more problematic with the increasing spread of ad-blockers. These 
problems could be eliminated with the diffusion of a bitcoin-enabled infrastructure 
for frictionless micropayments. Similarly, a bitcoin-enabled IoT device, e.g. for 
automatic irrigation of farms, could pay a weather service API in return for accurate 
weather prediction data without the need for signup. Moreover, the device could 
search automatically for the cheapest (and best in terms of reputation) weather 
service API on the marketplace.

The 21 platform is a mixture of a centralized and a decentralized model. As of 
today, the mining power is bound to 21, and returns get allocated to a wallet owned 
by 21. However, it is announced that this will change in the future. The marketplace 
for digital services on the other hand is in principle decentralized and trades 
can be conducted peer-to-peer. Currently, 21 generates revenue by private mining 
operations and by sales of the bitcoin computer. In the future, however, all kinds 
of interesting revenue models are imaginable: selling and licensing of mining chips, 
revenue sharing of embedded mining operations or tiny transaction fees for off-chain 
transactions to name just a few. 

In conclusion, 21 could change how resources in the Internet are paid for, and 
thereby also contribute to making new resources available, which have not been 
available yet because of missing incentives

\subsection{Factom (Records Management)}
\label{sec:ecofactom}

Factom (2015) is an open source software project that provides businesses with 
the ability to prove data integrity and to create verifiable and immutable audit 
trails. 

While data integrity could be achieved by directly adding a hash of the data to 
a bitcoin transaction and thereby time-stamp the data on the bitcoin blockchain, 
this method does not scale. On one hand, this is because of inherent scalability 
issues of bitcoin, and on the other hand because of transaction fees. Therefore, 
Factom consists of a peer-to-peer network that is independent of the bitcoin network. 
Customers of Factom generate hashes of their data and send them for recordkeeping 
to the Factom network. There, all hashes are compressed to a single hash by building 
a Merkle tree (Merkle, 1980) and taking the root of the tree. This single hash 
value is then stored in the bitcoin blockchain. This provably time-stamps all individual 
records without having to write all records individually into the bitcoin blockchain. 
The network maintains its own crypto-currency, ``factoids'', which is used to incentivize 
participants of the peer-to-peer network to provide their resources. A factoid 
can be transformed into entry credits, which can be used to submit new records. 
The price of a factoid depends on the market value, but the price of an entry credit 
is fixed to 1/10\textsuperscript{th} of a cent. 

In summary, Factom provides a decentralized platform for data provenance with a 
permanent, time-stamped record of an unforgeable reference to the data anchored 
in the blockchain. This offers an efficient and cheap alternative for businesses, 
institutions and governments to have a proof of existence, proof of process or 
proof of audit for their data. Their first publicly announced project is using 
Factom for an official land title registry in partnership with the government of 
Honduras (Chavez-Dreyfuss, 2015). In fact, Factom is an interesting option for 
governments in developing countries. They often face mismanagement and corruption, 
but cannot afford or enforce infrastructure and processes to guarantee compliance 
of their administration. Moreover, the Factom solution offers also an opportunity 
for small businesses/start-ups to have more auditing and to prove compliance with 
regulations without hiring expensive professional companies for that.

\subsection{Onename (Identity)}
\label{sec:ecoonename}

Onename (2015) allows registering identities on the Bitcoin blockchain. This blockchain 
identity can be connected to various online identities like Facebook, Twitter, 
PGP keys and a Bitcoin address. Thus, it provides a probabilistic identity which 
reliability grows with the number of verified connected accounts. Onename is based 
on an open source overlay protocol called Blockchain ID. Everyone is able to register 
his identity without having to rely on Onename services. Blockchain identities 
are independent of the company Onename, are referenced on the Bitcoin blockchain 
and are therefore owned by the holder of the respective private key.

Blockchain IDs will allow signing up on third party websites comparable to Facebook 
Login or Google Sign-In. There is no need for passwords since authentication is 
done using digital signatures. Blockchain ID is also used as an identity provider 
for OpenBazaar. 

Moreover, it is possible to add additional namespaces. In this sense Blockchain 
IDs could represent not only humans, but also machines. Thus, Blockchain IDs might 
be the basis of a decentralized DNS system or an IoT registry. 

Essentially, the technology allows individuals to own their online identities, 
rather than being dependent central institutions. Holding, i.e. registering and 
prolonging, a Blockchain ID requires fees that directly support the Bitcoin ecosystem. 
One part of the fee is a typical Bitcoin transaction fee that will be collected 
by a miner. The other part of the fee is particular to the protocol and leads to 
\emph{burning} of bitcoins. Since the number of bitcoins is constrained, the elimination 
of bitcoins theoretically increases the value of existing bitcoins. While it might 
seem that our current identities are free, we actually pay with our personal data. 
Every time we use Facebook to login into a third party website, we give away more 
information about us. Identity is also subject to network effects. Institutions 
will only start to accept Blockchain IDs if there are enough people using them 
and demand acceptance. 

Onename basically follows the same strategy as OB1. The initial focus is on developing 
open source protocols and advocating their adoption, instead of having a revenue 
model in place.

\section{Key Findings}

Table 2 presents the key findings concerning our main research questions. We state 
the core innovation of the company in the respective sector and list the features 
of Bitcoin that underlay those innovations. The innovations in the digital asset 
sector and in the notary sector are enabled mainly because of the usage of the 
Bitcoin blockchain as an immutable public database. In contrast, the decentralized 
marketplaces profit from the value transfer features Bitcoin provides, in particular 
multi-signature escrow and micropayments. Both features are ultimately based on 
the scriptability of Bitcoin transactions, i.e. the nature of Bitcoin as programmable 
money. Another important aspect that encompasses all sectors is the inclusivity 
and permissionless nature of Bitcoin. This aspect has two implications. First, 
it allows companies to build protocols on top of Bitcoin and the blockchain without 
having to ask anyone for permission. All of the presented companies are based on 
this. Second, everyone with Internet-connectivity is able to participate in the 
system. In particular, people from developing countries without proper access to 
financial services can be active on the marketplaces. 

Rather surprising is the appearance of mining. The business model of traditional 
mining operations is to generate revenue by selling newly minted bitcoins. In contrast, 
21 aims to leverage mining as a tool to supply users directly with bitcoins circumventing 
the cumbersome process of acquiring those on an exchange. Thus fuelling the emerging 
ecosystem of bitcoin-payable digital micro services. 

\resizebox{\textwidth}{!}{
\tiny
\begin{tabular}{|>{\raggedright}p{12pt}|>{\raggedright}p{44pt}|>{\raggedright}p{37pt}|>{\raggedright}p{80pt}|>{\raggedright}p{38pt}|>{\raggedright}p{47pt}|>{\raggedright}p{20pt}|>{\raggedright}p{30pt}|>{\raggedright}p{23pt}|}
\hline
S\textbf{ector} & S\textbf{ubsector} & C\textbf{ompany/ Project} & C\textbf{ore 
innovation} & \multicolumn{5}{p{159pt}|}{B\textbf{itcoin features}}\tabularnewline
\hline
 &  &  &  & I\textbf{mmutable public database} & I\textbf{nclusive (global and 
permissionless)} & M\textbf{ining} & M\textbf{icropayments} & M\textbf{ultisig}\tabularnewline
\hline
Digital Assets & Internet of Things & Filament & Decentralized IoT infrastructure 
based on autonomous devices & x & x &  & x & \tabularnewline
\hline
 & Intellectual Property & Ascribe & Commoditization of registering IP; trade and 
license digital work & x & x &  &  & \tabularnewline
\hline
Marketplace & E-Commerce Marketplace & OpenBazaar & Decentralized market place 
for physical and digital goods without restrictions and fees &  & x &  &  & x\tabularnewline
\hline
 & Digital Micro Services Marketplace & 21 & Ecosystem for bitcoin-payable digital 
(micro) services &  & x & x & x & \tabularnewline
\hline
Notary & Records Management & Factom & Commoditizing data integrity and auditability 
 & x & x &  &  & \tabularnewline
\hline
 & Identity & Onename & Individually-owned identities & x & x &  &  & \tabularnewline
\hline
\end{tabular}}





\section{Discussion}

In the title of this work we stated the question on whether the Bitcoin ecosystem 
may disrupt sectors beyond the financial service industry. We would argue that 
the companies we investigated in this paper are leveraging Bitcoin as global programmable 
money and as an immutable public database to foster radical innovations in their 
respective sectors. Central authorities are eliminated and new markets are created. 
But as shown in Section 2.2 radical innovations are not necessarily disruptive 
innovations and disruptive innovations may or may not disrupt a market eventually. 
Thus \emph{identifying disruptive innovation is a complex process} (Kaltenecker et 
al., 2015). The current literature still lacks a thorough theoretical foundation, 
on how to define disruptive innovations (Yu and Hang, 2010). Additionally it remains 
unclear to what degree disruptive innovations can be predicted ex ante (Barney, 
1997; Tellis, 2006). This article aims to offer an overview on the fast developing 
blockchain ecosystem highlighting mainly the potential inherent to these technologies. 
Most sectors that we considered are underdeveloped. The commoditization of registration 
of digital work and intellectual property in general could be counted as a new-market 
disruptive innovation. Current registration of IP is expensive and cumbersome, 
but is backed directly by jurisdiction. Therefore, in terms of legal certainty, 
the traditional process is clearly superior. Decentralized marketplaces exhibit 
also the characteristics of a disruptive innovation. Today, OpenBazaar has an inferior 
user experience in comparison to Amazon or eBay. Moreover, it obviously does not 
have the same strong user base. However, the OpenBazaar platform is free to use 
and there are no restrictions on what can be sold. These characteristics will most 
probably attract users in niche segments and could grow from there on. In cases 
like Filament and Onename the notion of disruptive innovation does only apply to 
a lesser extend. Interestingly, most of the core innovations that the companies 
provide are in form of open source protocols. Thus, the disruptive potential is 
not entirely dependent on the future of the particular companies. From the point 
of the start-ups, open source protocols and decentralization can help entering 
a market that is dominated by a central institution. However, it becomes much harder 
to capture value and thus to find a sustainable business model.

As shown in Figure 4, most of the ecosystem beyond financial services is younger 
than two years. The companies are in a very early stage and the focus is clearly 
on creating value and on advocating the use of their protocols. Some of the companies 
are experimenting with ways to capture value. But it is too early to investigate 
the business model in detail. Noteworthy, all of the six start-ups under consideration 
build platforms. In fact, these platforms are rather different. For example Filament 
builds a decentralized platform for the Internet of Things, Onename builds a decentralized 
identity layer and so on and so forth. Hence, it will be worthwhile to study the 
emergent value networks as soon as the platforms get populated.

Despite these first insights, more research will be required to gain a better understanding 
of the theory of disruptive innovation in general, as well as in the context of 
blockchain technologies in particular (cf. Danneels, 2004). We encourage other 
scholars to take this article as a starting point to investigate more precise measurements 
(cf. Govindarajan and Kopalle, 2006) and clearer definitions (cf. Markides, 2006) 
of disruptive innovation in a Fintech context as well as to draw clear distinctions 
to related research fields such as radical or open innovation (cf. Yu and Hang, 
2010). 

In this work, we deliberately restricted our scope on start-up companies building 
on Bitcoin technology. However, there are numerous implementations of cryptocurrencies 
and blockchains. While most of them are mere copies with different parameter choices, 
there are also developments with the potential to advance the technology and applications 
significantly. Most notably there is Ethereum, which e.g. replaces the restricted 
scripting language of Bitcoin with a Turing-complete programming language that 
allows programs \emph{living} on the Ethereum blockchain, so called contracts, to 
govern over funds autonomously. This empowers developers to build richer decentralized 
applications on top of the (Ethereum) blockchain. The Ethereum network launched 
as recently as July 2015 and has still to prove its resilience. Thus far, there 
are hardly any venture-backed companies in sectors beyond financial services that 
use alternative cryptocurrencies and blockchains. However, this might change soon.