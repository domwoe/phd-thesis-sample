\chapter{Towards Economic Devices}
\label{sec:economicobjects}

\section{Context and Motivation}

Cryptocurrencies enable connected devices to receive and control money. Off-chain payment systems based on payment channels may eventually enable trustless and instant micropayments at scale. This provides the basis for reliable and autonomous economic interactions between connected devices and between connected devices and humans. In this chapter, we zoom out a bit and introduce the notion of \emph{economic devices}. These are connected devices with economic capabilities such as buying and selling (digital) goods and services, as well as the issuance of cryptographic tokens, i.e. specific digital bearer assets living on a cryptocurrency blockchain. These tokens can represent e.g. concrete IOUs, claims on revenue or access/control rights. Economic devices may play an important role in developing countries where Internet access and smartphones are becoming ubiquitous, but access to capital, financial institutions, and law enforcement lacks behind. We present the prototypical 

We draw from the concepts of smart property and distributed autonomous organizations 


\section{Background}

\subsection{Smart Contracts}

\subsection{Smart Property}

The term smart property is closely related to the original definition of smart contracts. 

The term smart property was introduced in \cite{szabo1997} as a manifestation of smart contracts. Szabo gives the example of a car:
\begin{quote}
Consider a hypothetical digital security system for automobiles. The smart contract design strategy suggests that we successively refine security protocols to more fully embed in a property the contractual terms which deal with it. These protocols would give control of the cryptographic keys for operating the property to the person who rightfully owns that property, based on the terms of the contract. In the most straightforward implementation, the car can be rendered inoperable unless the proper challenge-response protocol is completed with its rightful owner, preventing theft.
\end{quote}

And Indeed, modern car keys use smart card technology and cryptographic primitives to authenticate the driver and provide the functionality described by Szabo. However legal ownership is independent of possession of the cryptographic key. Noteworthy, currently the possession of a physical document authenticates the legal ownership. 

With the inception of Bitcoin and blockchain technology, the term smart property usually refers to property whose ownership is controlled by a blockchain.
The Bitcoin Wiki notes that smart property may also include non-physical property like shares in a company or access rights to a remote computer \parencite{smartproperty2011}. 

\section{Economic Devices}

Cryptocurrencies are digital bearer assets existing on public blockchains. Decentralization, cryptography and incentive-compatibility provide the basis for permissionless access independent of legal status. This allows software agents to act as autonomous economic agents which are able to control monetary value and enter contractual agreements with other economic agents, either humans or machines. These software agents can either be instantiated on a decentralized application platform with built-in cryptocurrency such as Ethereum or on tamper-proof hardware that is able to store and use cryptographic keys without interference. We define connected devices interfacing cryptocurrency blockchains as economic devices due to their ability to contribute directly to an economy by facilitating economic interactions. These economic interactions can either be rigid and predefined in order to automatically execute transactions as programmed by humans, or the result of economic decisions made by intelligent economic devices. Economic devices could be the building blocks of a decentralized \ac{IOT} by interacting with cloud services on the basis of market economics instead of vertical integration.


In the limit of complete autonomy, economic devices would \emph{own themselves}. However, ownership (in contrast to possession) is a legal concept. Thus, economic devices would need to be recognized as legal entities. Property rights theory assigns the owner of an asset the residual control rights over that asset \parencite{10.2307/2937753}. Thus, in principle, ownership is only of concern if contractual relationships are incomplete, i.e. not all actions and payments for all possible contingencies are specified \parencite{GKlein:2015en}. \cite{lessig2009code} has argued that there are more modes of regulation than law. Also the market, architecture, and norms regulate the actions of the individual and their interactions within the society as a whole. Norms are less important when interactions are anonymous and arguably even less so for software agents and economic devices. Architecture, in form of capabilities and constraints by hardware, software and cryptography, and the market, in form of financial incentives or disincentives, i.e. deposits and penalties, are able to provide a basic regulatory framework for economic devices and their interactions with humans. Cryptocurrencies and blockchains are merging the concepts of architecture and the market and enable the rise of a \emph{Lex Cryptographia} which is "characterized by a set of rules administered through self-executing smart contracts and decentralized organizations" \parencite{wright2015decentralized}.
These forms of regulation are particularly important in nations with weak law enforcement and when considering global, i.e. supranational, economic interactions via the Internet.

\subsection{Basic Capabilities}

In the following, we discuss basic capabilities of economic devices that can be implemented with the dominant cryptocurrency platforms Bitcoin and Ethereum.

Cryptocurrencies allow a machine to reason about the finality of a payment. Thus, an economic device is able to autonomously \emph{sell services or (digital) goods}. It is worth to stress this point. Only because a machine can be sure that a payment can not be reverted by some authority, it can safely sell anything. A machine itself is not able to enter a legal arbitration process which is also expensive due to the involvement of humans. Sensing as a Service as presented in the last two chapters is one example of this capability to autonomously offer a sensing service or sell data as a digital good. However the most fundamental example is mining which provides security to the cryptocurrency infrastructure itself in return of a probabilistic payment. 

Once in control of the cryptocurrency, an economic device is able to \emph{buy services or (digital) goods} from other economic devices or from humans (see e.g. Amazon Mechanical Turk\footnote{https://www.mturk.com/mturk/} or participatory mobile crowdsensing \parencite{burke2006participatory}). In addition to these basic economic capabilities, more complex agreements can be implemented by \emph{issuing tokens representing (transferable) rights} concerning the economic device. These tokens can represent e.g. concrete IOUs, claims on revenue or access/control rights. IOUs are like vouchers or coupons and authorize the bearer to redeem the token for a specified service or good. Claims on revenue allow to implement revenue-sharing schemes between economic devices and humans. We will discuss an application of this in the case study in Section \ref{sec:caseDisplay}.
Transferable, i.e. tradable, access/control rights allow to own\footnote{Not necessarily in the legal sense.}, as well as to trade, and rent an economic device without having to trust the counterparty. Hence, economic devices entail and extend the notion of smart property.

\begin{table}
  \centering
  \begin{tabularx}{\textwidth}{ l  X  }
    \toprule
    Capability  & Example \\
    \midrule
    Selling & S\textsuperscript{2}aaS, mining \\ 
    Buying & Data, energy, Internet access \\ 
    Token issuance & IOUs, claims on revenue, smart property \\  
    \bottomrule
  \end{tabularx}
  \caption{Basic capabilities of economic devices.}
  \label{tbl:capabilities}
\end{table}


\subsection{Requirements}

In order to implement economic devices a number of requirements have to be met.
First, a digital bearer asset is needed to enable dis-intermediated and non-repudiable payments between machines. Thus far, cryptocurrencies based on public blockchains are the only practical implementation of this concept. Second, interactions with economic devices should be trustless. This can either be achieved with financial deposits and penalties or with verifiable open source hardware and software. Deposits and penalties may require arbitration to decide if a counterparty did not fulfill its obligations. Programmable cryptocurrencies enable competitive markets for trustless arbitration. Open source hardware and software, and trusted computing platforms allow the counterparty to verify the integrity of an economic device and thus reason about its behavior. Third, counterparties should not be able to modify hardware or software for their benefit, i.e. the effort of modification or tampering should outweigh the benefit. In addition, there are the basic requirements of having access to energy and to the Internet. 

% \begin{table}
%   \centering
%   \begin{tabularx}{\textwidth}{ l  X  }
%     \toprule
%     Requirement  & Description \\
%     \midrule
%     Programmable digital bearer asset & \\ 
%     Energy & S\textsuperscript{2}aaS, mining \\ 
%     Internet access & Data, energy, Internet access \\ 
%     Open Source & IOUs, claims on revenue, smart property \\
%     Tamper-proof hardware & \\

%     \bottomrule
%   \end{tabularx}
%   \caption{Requirements of economic devices.}
%   \label{tbl:requirements}
% \end{table}

\section{Case Study: Public Display for Developing Countries}
\label{sec:caseDisplay}

In the following we present a prototypical implementation of an economic device. The prototype aims to represent a large public display that can be installed in public areas. The display provides the service of showing customer defined content in exchange for a time-based fee. The display is in many ways able to act as an autonomous business entity based on predefined rules instantiated in form of an Ethereum smart contract. The smart contract defines the economic capabilities of the economic device over its entire lifetime, even before it physically exists. The economic device issues tokens representing shares, or more precisely claims on revenue. The initial sale of these tokens can provide the capital to bring the economic device in physical existence, and can be used to implement revenue-sharing with a local human entrepreneur to take care of maintenance and revenue-maximizing usage of the economic device. This case study serves to exemplify the value of an economic device even in a rather simple instantiation. Furthermore, the case study provides some initial observations and lessons learned concerning the actual implementation of an economic device.

\subsection{Motivation}

Outside high-income \ac{OECD} countries large parts of the world population is essentially unserved by traditional financial institutions such as banks. Nearly 2.2 bn adults in Africa, Asia, Latin America, and the Middle East are \emph{unbanked}. In sub-saharan Africa this culminates to 80\% of the adult population \parencite{chaia2010counting}. On the other hand, Internet access and smartphone ownership is rising. Connected devices can serve as an anchor for new forms of financial services. A prominent, and early example is M-Kopa which offers pay-as-you-go models to lease \ac{SHS} to rural customers in Africa, where most customers do not have the capital to purchase such a device. Since 2011, M-Kopa has provided 375 000 households access to solar energy in Kenya, Tanzania, and Uganda \parencite{adabou2016}. Each \ac{SHS}, consisting of a solar panel, a battery and some appliances, is equipped with a SIM card which allows M-Kopa to remotely disable the system if payments have stalled. Furthermore, customers are able to remortgage their \ac{SHS} in exchange for consumer finance loans to procure productive assets such as cooking stoves and smartphones. Thus, M-Kopa is essentially a finance company based on a connected product and mobile payments. The model is tightly vertically integrated. Economic devices allow to unbundle this model by creating a transparent and competitive market. Although we were motivated by the prospect of financing rural electrification by instantiating \ac{SHS}s, solar micro utilities and solar micro grids with economic devices, we implemented the public display because it is conceptually similar to a solar micro utility and it can be demonstrated more easily and with more effect.


\subsection{Approach}

We assume a manufacturer and a local entrepreneur with limited capital to begin with. 


Thus, a crowdsale is initiated to bridge the financing gap. In a crowdsale, special tokens, i.e. digital bearer assets on a cryptocurrency blockchain with defined rights, are offered publicly in exchange for cryptocurrency. Due to the global and inclusive nature of cryptocurrencies, investors from all over the world can participate in the crowdsale with tiniest amounts. In this case, the tokens are claims on future revenue of the economic device. By buying a token, an individual becomes an investor and shareholder of the economic device. Although we are using legal terms here, we do not consider the implications of securities laws, but use these terms only to provide a familiar analogy. All rights and obligations, we consider, are expressed in the smart contracts discussed later on. Revenue generated by displaying content of customers is transparently and automatically dispersed to the shareholders. 

Note, that this has similarities to an Initial Public Offering of a private company, but instead of a traditional company, the entity of concern is the individual product - the economic device.
This allows the local entrepreneur to bridge the financing gap to put a revenue-generating asset into use. Instead of taking out a loan to purchase the asset, or a bilateral financing agreement with the manufacturer, a competitive and transparent market can emerge - a prerequisite for efficient capital allocation.
The local entrepreneur is incentivized to maximize revenue by increasing her share as a function of generated revenue. In other words, payments trigger token issuance. The newly minted tokens are issued to the entrepreneur, hence, increasing her share, and diluting investors. Over time \emph{ownership} of the display is centralized with the entrepreneur. 

\subsection{System Overview}

In the following we present the system, the stakeholders and their primary roles. Figure \ref{fig:smartproperty} provides a simplified illustration of the stakeholders and their roles. 

\begin{figure}[!t]
    \centering
    \includegraphics[width=0.7\linewidth]{./externalized/publicdisplay.pdf}
    \caption{Stakeholders and their primary roles in the model.}
    \label{fig:smartproperty}
  \end{figure}

\subsubsection{Stakeholders}

\paragraph{Manufacturer}

The manufacturer produces the economic object, and is the beneficiary of the initial sale of shares.

\paragraph{Entrepreneur / Operator}

The entrepreneur identifies a potentially profitable location for a public display and takes care of proper operation such that maximal revenue is generated. The entrepreneur has to have a double role as investor to have financial skin in the game and is incentivized by being awarded additional shares depending on generated revenue.

\paragraph{Investors}

Investors seek a profitable investment. Hence, they provide capital to close the entrepreneur's financial gap in return of a revenue share.

\paragraph{Customers}

Customers are individuals or software agents that are willing to a pay for the service the economic device provides.

\subsubsection{Economic Device}

The economic device is the center of the system. It has a physical representation in form of a public display augmented with an embedded computer and connectivity, and a digital representation as a (partly) decentralized application hosted partly on the device and partly on the Ethereum network in form of a smart contract. The display can be rented in the sense that customers can pay for the ability to cast content on the display for a specified time. Revenue generated by providing this service is dispersed automatically to shareholders and additional shares are issued and awarded to the entrepreneur.

\subsection{Implementation}

As mentioned in the former paragraph, the economic device has a physical instantiation and a digital instantiation. Although the physical instantiation is important to provide security against physical tamper, we focus on the digital instantiation. Ethereum allows to implement application logic as stateful programs which are stored and executed on the Ethereum network (c.f. Section \ref{sec:ethereum:tech:tx}). These programs are called (smart) contracts. By linking a connected device with a smart contract, we are able to implement an economic interface. In order to interface with the Ethereum network, a Geth client \parencite{Geth} is running on the device. The Geth client provides an \ac{RPC} interface that can be accessed conveniently from Node.js via the web3.js library \parencite{Web3}. Figure \ref{fig:displayImp} provides an illustration of the system as implemented. The aspects of the initial token sale are not shown. 

\begin{figure}
 \centering
 \includegraphics[width=0.7\textwidth]{./externalized/displayImp.pdf}
 \caption{Illustration of the implemented system after deployment by the entrepreneur and the manufacturer.}
 \label{fig:displayImp}
 \end{figure}


\subsubsection{Economic Device Contract}

The contract is implemented using Solidity, the JavaScript-like higher-level smart contract language that can be compiled to Ethereum Virtual Machine byte code. In the following we describe the functions of the economic device contract that provides the economic interface.

\paragraph{Shares as Tokens}

Shares are represented as tokens. We base the smart contract on the standard token contract\footnote{https://github.com/ConsenSys/Tokens/blob/master/Token\_Contracts/contracts/StandardToken.sol}. The token contract is essentially a simple accounting system that maps account numbers to values, i.e. the number of \emph{tokens} an account owns. A \emph{transfer function} then allows an account owner to debit its balance and credit another. This provides the basis for investors to trade shares. Since the entrepreneur's share is increased over time, selling shares is comparable to remortgaging.

\paragraph{Initial Financing based on Token Sale}

Initial share allocation can be done with an adapted crowdsale contract\footnote{https://www.ethereum.org/crowdsale}. The amount of initial shares is fixed and corresponds to the price the manufacturer demands. A certain amount of shares need to be bought by the entrepreneur as a down payment. Revenue of the crowdsale is credited to the manufacturer after a successful crowdsale. The crowdsale has a specific duration. If not enough shares are sold during this period, the contract automatically refunds the investors, and a new contract with different parameters can be deployed.

Instead of having a fixed price per share or a defined goal, an auction-based crowdsale could be implemented.

\paragraph{Payments, dividends and dilution}

The contract exposes a public \emph{pay function} (see Procedure \ref{alg:payforservice}). The function checks if the device is currently rented. If so, the customer is refunded. If not, internal functions to pay dividends and to increase the entrepreneur's share are called. Finally, a \emph{payment event} is created. The payment event carries the data tuple (\emph{payer}, \emph{paidUntil}), where payer denotes the Ethereum account address of the customer invoking the pay function, and paidUntil denotes the timestamp until the customer has paid for renting the display. 

Since share issuance is handled within the contract handling payments, the overhead is low. There is no need for an additional transaction, but only an update of the token balance. Thus, there is no need to aggregate payments until a certain threshold before the entrepreneur gets credited with additional shares. 

Noteworthy, there are two possibilities to pay dividends to shareholders. The obvious approach would be to let the contract actively send dividends to the shareholders Ethereum accounts. However, sending value from a contract is expensive in comparison to updating an internal balance, and the cost would have to be provided by the customer invoking the pay function. To avoid this, the contract keeps track of the dividend balances and we add a \emph{withdraw function}. The withdraw function allows shareholders to collect their dividends whenever they wish. Thereby, costs for withdrawing have to be borne by the withdrawing shareholder, and shareholders may decide for themselves when collecting dividends is appropriate. 

\begin{algorithm}[!t]
  \floatname{algorithm}{Procedure}
   \caption{Pay for service}
    \begin{algorithmic}[1]
      \If{now $\geq$ paidUntil}
      \State \Call{creditDividends}{$value$}
      \State \Call{dilute}{$value$}
      \State paidUntil = now + value/pricePerTimeUnit
      \State \textbf{Event} Payment(customer,paidUntil)
      \Else 
      \State return money to customer
      \EndIf
      \Statex
      \Function{creditDividends}{$value$}
      \For{sh in shareholders}
      \State balanceEther[sh] += balanceShares[sh]/totalShares*value
      \EndFor
      \EndFunction
      \Statex
      \Function{dilute}{$value$}
      \State newShares = value * supplyIncreaseRate
      \State balanceShares[indexOfEntrepreneur] += newShares
      \State totalShares += newShares
      \EndFunction
\end{algorithmic}
\label{alg:payforservice}
\end{algorithm}

\paragraph{Contract - Device Communication}

The application running locally on the device has essentially only one job: watching for payment events executed by the respective contract instantiation. Solidity events provide an interface to the logging facilities of the Ethereum Virtual Machine, and can be used to trigger client application logic with data payload. The payment event triggers a state transition of the device (c.f. Figure \ref{fig:displayStates}).

Logs are committed to the block header and are thus accessible for light clients (c.f. Sections \ref{sec:eth_blockchain} and \ref{sec:eth_lightclient}). Thus, a future implementation can be based on an Ethereum light client with much lower computation, storage and bandwidth requirements - an important point for possible applications in developing countries.

\begin{figure}
 \centering
 \includegraphics[width=0.7\textwidth]{./externalized/displayStates.pdf}
 \caption{State machine of the device. There are two main states: available and rented. If available, the pay function can be invoked and the payment event notifies the device that customer $C$ has paid to cast content until time $T$.}
 \label{fig:displayStates}
 \end{figure}


\subsubsection{Interacting with the Device}

In principle, all interactions could be facilitated via the Ethereum network. In particular with the availability of the \ac{P2P} messaging protocol Whisper \parencite{whisper} and the Mist browser\footnote{https://github.com/ethereum/mist}. However, these tools are at a very early stage. Instead, the economic device provides a web server to serve graphical user interfaces in form of (mobile) websites. The websites are essentially static single page web applications. 

The client does even not necessarily need an Ethereum client which is important as long as fully functioning light clients are not available. We use the Lightwallet library\footnote{https://github.com/ConsenSys/eth-lightwallet} to keep the wallet and private keys locally on the client, and interact with the Ethereum network via an Ethereum client with a public \ac{RPC} interface.

In the following we briefly present the steps a customer needs to take in order to use the display. An illustration of one view of the customer interface is shown in Figure \ref{fig:payview}. More views are shown in APPENDIX!!!.

\begin{enumerate}
\item Open \emph{personal} website of the display.
\item Open an existing wallet by providing a seed and a password or create a new wallet. Make sure that the wallet has sufficient funds.
\item Make a payment.
\item Send links to content directly to the device via a websocket connection. The messages are authenticated, such that the device can verify if the sender is the same account as the one that paid to the contract. You can change the content as long the period you paid for is not expired.
\end{enumerate}  



\begin{figure}
 \centering
 \includegraphics[width=0.7\textwidth]{./externalized/displayUserPay.pdf}
 \caption{An example view of the customer interface. In this view the customer can make the payment.}
 \label{fig:payview}
 \end{figure}




 \subsubsection{Parameter Selection}

\subsection{Could We Implement the Concept with Bitcoin?}

\section{Further Applications}




\section{Lessons Learned and Discussion}


\begin{itemize}
\item role of tokens

\end{itemize}

%\subsection{Requirements}

\subsubsection{Trustworthy hardware and software}

The smart property itself has to be trustworthy. A potential seller has to be sure that the software running the smart property follows the protocol. In particular, after a sale the smart property has to be controlled by the buyers ownership key, whereas the sellers ownership key is disabled. Hard- and software of the smart property must be able to ensure that is has not been tampered with. Ideally, software is open source and independently verifiable. Smart property may prove its integrity to a third party using trusted computing. Thereby the smart property may also prove its current state, i.e mileage and services in the case of a car. A certificate by the manufacturer may be used to ascertain provenance. Alternatively, provenance could be inferred from the signature of the original registrar.

\subsubsection{Kill switch}

Liquidification requires that the smart property can be rendered temporarily unusable. Otherwise the contract between lender and borrower is not remotely enforceable, and the borrower is able to use the property without paying back the loan. Enforcement of the contract would have to happen via the judicature system which is expensive and cumbersome if borrower and lender are in different jurisdictions. 

If the use of a kill switch is possible is dependent of the actual smart property. Cars, for example, are already equipped with immobilizers which are reasonably hard to bypass. Popular solar home systems, as deployed in Africa and South-East Asia, can also be remotely deactivated. Smart property where the value of the object itself or its components is high independent of a function or service that can be switched off, however, is not suitable for this scheme. 

\subsubsection{Native on-chain currency}
Selling, renting, and liquidification all involve currency. Only a fungible native token of value allows these processes without additional counterparty risk. 

\subsubsection{Light-client proofs}

In many cases smart property will not be able to interact with the blockchain network directly, but has to rely on proofs provided by untrusted parties. For example, a buyer has to be able to prove to the car that it now belongs to him. The underlying blockchain and the protocols should account for such a scenario, since e.g. a car should be able to be started even if there is no Internet connectivity.



%\subsection{Why has smart property to live on a blockchain?}

Let us stay with the example of a car. Alice owns the car and the ownership is augmented by the possession of a cryptographic key. Assume Alice wants to transfer the ownership to Bob. Of course, she can not send him the keys, since the key is nothing more than a digital file, and copying a digital file has zero marginal cost. In other words, we face the classic double spending problem. However, we can think of a more complex protocol. Instead of sending a key to Bob, Bob is creating a new private-public ownership key pair. Bob sends the public part to Bob, and Alice uses her current ownership key to initiate an ownership transfer to Bob's key. The car itself acts as a deterministic trusted party. We will later discuss avenues to give Bob more certainty about the trustworthiness of the smart property.
So far, we only used public key cryptography, and there was no need for a blockchain. However, typically someone does not just transfer ownership to someone else. Alice wants something in return. In most cases this something is money. If Alice and Bob meet in person they are able to do the ownership transfer protocol and the cash payment concurrently. However, if the trade happens via the Internet, one party has always be the first and may be defrauded by the counterparty. Traditional payments can be undone. Thus, inhibiting risk for the seller. Simple bitcoin payments mitigate this risk, but expose the buyer to the risk of not getting the car. Traditionally, third party escrow has been used to solve this dilemma.
However, if ownership of the car is represented on a blockchain, Alice and Bob can perform an atomic trade. Ownership and money is exchanged in one single process. If one part fails, the other part will fail as well. If one part succeeds, the other part will succeed as well. A similar logic can be applied to all of the capabilities smart property provides.

\subsubsection{Contract Immutability}

\subsubsection{Permissioned Blockchains}

A permissioned blockchain is a blockchain, in which transaction processing is performed
by a predefined list of subjects with known identities \cite{BitFuryPermissioned2015}. In context, a consortium of device manufacturers could operate a permissioned blockchain collaboratively. Permissioned blockchains do not need a native token of value, and typically do not have one since fair distribution of tokens, such that trust and value can emerge, is non-trivial. However, there are ways to bring external value in and out of permissioned blockchains in order to facilitate economic interactions. 
\paragraph{IOU Issuance}
One approach is to allow specific parties to issue IOUs on the blockchain. These parties could be the device manufacturers themselves but more probably banks and financial service providers. In this case a buyer interested in a smart property would send money off-chain to a third party which then issues tokens on the blockchain. In addition, the seller has to accept those tokens. 

\paragraph{Cross-chain/ledger protocols}
A better approach that requires less trust is the application of a cross-chain/ledger protocol. We discussed sidechains already, which allow the immobilization of a token in one chain and subsequent unlocking in another chain. In addition, there is the Interledger protocol \cite{hope2016interledger} and the atomic cross-chain trading protocol \cite{atomiccrosschaintrading}



\section{Related Work}
% \subsection{Shared ownership}

% Multi-party ownership of smart property. Complex scenarios are possible. For example selling of a smart property may need approval from all parties and revenue is distributed evenly.

\section{Conclusion}

