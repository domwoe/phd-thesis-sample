\chapter{Towards Economic Devices: Insights from an Blockchain-enabled Display}
\label{sec:economicobjects}

%\epigraph{What I cannot create, I do not understand.}{--- \textup{Richard P. Feynman}}


\section{Context and Motivation}

In the last two chapters the application of Bitcoin as a platform, and as medium of exchange between connected devices was discussed, and explored by building prototypical systems. Bitcoin contracts were introduced to enable instant micropayments between a large number of data requesters and data providers. These \emph{smart contracts} allow to automate rules for which a trusted third party had been needed. The direct exchange of digital goods and services is but one economic interaction enabled by cryptocurrency and smart contract platforms. In this chapter, the term \emph{smart contract} is explored in a broader sense. Furthermore, the related concept of \emph{smart property} is introduced. Based on these two foundations, the concept of \emph{economic devices} is introduced. Economic devices extend the capabilities of connected devices with economic capabilities that are independent of the manufacturer and third party services. These capabilities can be divided into active and passive capabilities. Active capabilities entail the trade and exchange of digital goods and services, such as the buying and selling of sensor data, but also indirect contractual relationships. These relationships can be encoded as \emph{tokens}, and thus be tradable itself. Passive capabilities are based on the notion of smart property. In this regard, the economic device is the passive object of economic interactions and not the active subject. These capabilities allow the trust-less trade of an economic object as well as its capitalization, i.e. using the economic object as collateral for loans. These capabilities are of particular significance for developing countries where large parts of the population are not served by traditional financial institutions, and private property rights are underdeveloped and not enforceable. In addition, economic devices could provide the building blocks for future sharing economy applications, and an increasingly autonomous and interoperable \ac{IOT}.

The concept of economic devices is illustrated and investigated with a prototype of a Blockchain-enabled display. The prototype implements two active economic capabilities. First, it offers the service to show user-select content on a pay-per-time basis payable with cryptocurrency. Second, it issues tradable tokens which can be compared to shares in a company. However, Token-holders receive a share of revenue in real-time instead of yearly dividends. The prototype is based on an Ethereum contract. This allows to encode the entire business logic on the blockchain and minimizes the required trust in the device itself. An alternative implementation based on Bitcoin is discussed briefly. 

% With the advent of decentralized cryptocurrencies, connected devices are able to immediately and autonomously participate in economic interactions. Arguably many of those interactions will be tiny payments between machines to provide efficient resource allocation and utilization without human involvement. S\textsuperscript{2}aS is a scheme that targets efficient utilization of sensor data produced by all kind of connected devices all over the world. Assuming that \emph{a sensor earns money} and is acting as an autonomous business entity paying for electricity and Internet access. How would the sensor be built and maintained? We argue that an approach towards answering this question provides a basis for financing productive assets in general. The idea is 




% . Off-chain payment systems based on payment channels may eventually enable trust-less and instant micropayments at scale. However, c



% This provides the basis for reliable and autonomous economic interactions between connected devices and between connected devices and humans. In this chapter, we zoom out a bit and introduce the notion of \emph{economic devices}. These are connected devices with economic capabilities such as buying and selling (digital) goods and services, as well as the issuance of cryptographic tokens, i.e. specific digital bearer assets living on a cryptocurrency blockchain. These tokens can represent e.g. IOUs, claims on revenue or access/control rights. Economic devices may play an important role in developing countries where Internet access and smartphones are becoming ubiquitous, but access to capital, financial institutions, and law enforcement lacks behind. We draw from the concepts of smart property and distributed autonomous corporations 



% \section{Smart Contracts and Smart Property}

% \subsection{The need for a new type of contracts}

% According to Marriam-Webster a contract is "a binding agreement between two or more persons or parties". In particular, an agreement that is legally enforceable. Although often the actual document describing the terms of a contract is identified with the contract, the actual agreement is the contract, and the document itself is only evidence. Contracts enable transactions that can not be completed on the spot. A simple example is a loan agreement between a borrower and a lender. The borrower and the lender might agree that the borrower lends \$ 10 000 USD to the lender now, under the obligation that the lender repays \$ 11 000 USD one year later. Contract law thereby offsets the trust the borrower needs to have in the lender, with trust in the authority of the legal system, which will eventually enforce the terms of a contract in case of dispute between the parties. Often a loan is secured with collateral. Collateral is property of the lender of which the borrower is able to take possession if repayments are not made as specified. Legal enforcement and dispute resolution is then provided by the judicial and executive system of a sovereign. However, this is a manual and costly procedure which makes only sense if the disputed value justifies the costs. In many cases the deterrence of legal enforcement is sufficient to incentivize the compliance to legal terms. However regulation by law is not always appropriate. There are at least three general scenarios where this is the case.

% \begin{itemize}
%   \item Transaction values are low in comparison to dispute resolution and enforcement costs.
%   \item Legal system of sovereign lacks power to enforce contracts.
%   \item Contracting parties are immune against enforcement.
% \end{itemize}

% The first two scenarios are straightforward. First, going to court because of \$ 10 USD is unreasonable. Second, developing countries often do not have a reliable judicial and executive system. The third case concerns artificial autonomous economic agents, i.e. software or machines capable of being parties in economic transactions. 

% \cite{lessig2009code} argues that there are more modes of regulation than law. In particular, there are three additional modes of regulation besides law: (1) the market, (2) norms, and (3) architecture. The market enables and prohibits certain kinds of transactions. 
% The market is able to regulate economic interactions with supply and demand and the price mechanism. Even more generally, financial incentives or disincentives in form of penalties can be used to regulate interactions and behavior. Norms are a psychological form of regulation. Norms are s evolving informal shared understanding of groups. As such they incentivize certain behaviors of group members, and provide a basis to predict the behavior of a group member. Another form of regulation is architecture. In the world around us, the laws of physics constrain our behavior. Human-made architecture can constrain transactions as well, e.g. by building a wall. However, the most powerful form of regulation by architecture is computer code. Computer code is regulating cyberspace. Although social norms and the market influence what code is written, code eventually defines the transactions. Cryptoeconomic systems, consisting of code, cryptography, and mechanism design, provide novel tools for regulation independent of legal identities, nation states and traditional judicial systems.
% Cryptocurrencies allow entities instantiated by code, i.e. software agents, to have autonomous control over funds. Before cryptocurrencies digital monetary transactions have always dependent on intermediaries and custodians who are able to revert transactions. This enables a new class of contracts, that do not depend on being legally enforceable, but are either enforced ex-ante by code or compliance is strongly incentivized based on economic incentives and disincentives. These contracts are suitable for autonomous machine-to-machine transactions, for low value transactions, for transactions between anonymous parties, and for transactions with parties living in countries with a weak legal system.

% \subsection{Smart Contracts}

% \cite{} coined the term smart contract and defined it as "a computerized transaction protocol that executes the terms of a contract." A crude and simple example of a smart contract is the vending machine. The customer and the owner of the vending machine agree on the exchange of \$ 1.50 USD and a can of soda. This agreement is then enforced automatically by the vending machine. Bitcoin allows to implement smart contracts of two different kinds. First, contracts enforced by the Bitcoin protocol itself. An Example are payment channels which enforce that the locking of funds, and the temporal validity of the refund transactions. Another example is that of an assurance contract instantiated by a Bitcoin transaction that defines to allocate a certain amount of bitcoins to a receiver, without specifying the inputs. Various parties can then add inputs, and be sure that the transactions becomes only valid if sufficient funds have been collected. 
% Second, trusted software running on trusted hardware can interact natively with money, and thus take part in economic interactions. In this sense trusted computing can be an automated mediator for contracts and economic interactions between humans. An instantiation of this idea will be discussed in Sec. \ref{sec:caseDisplay}.

% Smart contracts enforced by the protocol itself that do not depend on trust in a single machine are advantageous. The implementation of scripting in Bitcoin limits the types of contracts that can be enforced by the protocol. In contrast, Ethereum allows in principle to instantiate arbitrary programs with storage and secure funds on the network. Thus, arbitrary protocols can be implemented on top of Ethereum that are able to enforce various types of contracts. 






% The \ac{IOT} is bridging the digital world and the physical world. In contrast to software agents living on a globally decentralized network, such as a blockchain, connected devices have localized physical instantiation, and are thus in principle subject to the legal system of the respective nation state, and the legal system can at leas interfere by confiscating or destroying the connected device.



% With the design of artificial systems transactions can be regulated. A simple example would be to separate two parts by a wall with just one connecting hole of a specific size, then only transactions involving items of a size smaller or equal to the hole are possible. Computer code is another type of architecture to regulate all kinds of interactions. 




% Also the market, architecture, and norms regulate the actions of the individual and their interactions within the society as a whole. Norms are less important when interactions are anonymous and arguably even less so for software agents and economic devices. Architecture, in form of capabilities and constraints by hardware, software and cryptography, and the market, in form of financial incentives or disincentives, i.e. deposits and penalties, are able to provide a basic regulatory framework for economic devices and their interactions with humans.

 



% The general objectives are to satisfy common contractual conditions (such as payment terms, liens, confidentiality, and even enforcement), minimize exceptions both malicious and accidental, and minimize the need for trusted intermediaries. Related economic goals include lowering fraud loss, arbitrations and enforcement costs, and other transaction costs." 








% The programmability of Bitcoin transactions provided a first means to implement such self-enforcing contracts \parencite{smartcontr}. Payment channels and \ac{HTLC}s are examples of this (see Chapter \ref{sec:goingoffchain}). Ethereum extended the programmability-capabilities of a cryptocurrency by introducing stateful, persistent programs with their own cryptocurrency balance, and are instantiated on the blockchain. These programs can interact with each other, as well as with humans and machines by means of transactions.Although these programs do not necessarily implement an agreement between two or more parties, as a traditional law-enforceable contract does, they are generally called (smart) contracts (c.f. Sec. \ref{sec:ethereum:tech:tx}). Ethereum smart contracts as used in this Chapter do however fulfill the original definition.

% \subsection{Smart Property}

% The term smart property was introduced in \parencite{szabo1997} as a manifestation of the original definition of smart contracts. 
% The idea is "to more fully embed in a property the contractual terms which deal with it" \parencite{szabo1997}. Hence, "these protocols would give control of the cryptographic keys for operating the property to the person who rightfully owns that property, based in the terms of the contract" \parencite{szabo1997}. With the inception of Bitcoin and blockchain technology, the term smart property usually refers to property whose ownership is controlled via a blockchain \parencite{smartproperty2011}. This enables the property to be traded and rented in trust-less atomic transactions. Connected devices provide the basis for instantiating smart property. The typical example of smart property is a car which can only be operated by a certain cryptographic key. For example the key able to redeem a particular Bitcoin \ac{UTXO}. In this case, the car can be traded by creating an atomic transaction that credits the seller with bitcoins and transfers the \emph{ownership} \ac{UTXO} to the buyer. Since the entire trade is handled in a single transaction, the counterparty risk is greatly reduced. The car then acknowledges the new owner by observing the blockchain itself or by accepting proof-of-work-based proof to be provided by the new owner.

% \subsection{Decentralized (Autonomous) Corporations and Organizations}

% \cite{jensen1976theory} articulates the conception that the corporation is essentially a nexus of contracts, i.e a collection of contracts between different entities such as shareholders, employees, suppliers, and customers. In this sense a \ac{DAC} is understood as a collection of interrelated smart contracts that govern all aspects of a corporation. The defining feature of a \ac{DAC} is the ability to have internal capital that can be used to incentivize contractors \parencite{buterin2014dao}. 
% First public experiments with \ac{DAC}s and \ac{DAO}s have been made. Most notably \emph{The DAO}, a set of smart contracts deployed on the Ethereum network in order to provide a decentralized venture-capital fund operated by anonymous shareholders. The DAO provided no intelligence in itself, but rather transparent codified rules for decision making in a democratic process.


% \subsection{Tokens}

% In the context of blockchains, tokens are a general concept to represent \emph{any fungible tradable good}, but also rights with respect to some entity. Fundamentally, bitcoins are the tokens that can be redeemed to write in a global public and immutable database. In this sense the Bitcoin network is a \ac{DAC} that incentives third parties by issuing tokens to provide computing power (mining). However, tokens are of use particularly if they exist in addition of the native cryptocurrency on a blockchain. Then, tokens can be traded in exchange for cryptocurrency without counterparty-risk and without intermediaries. Tokens are issued at some point. Either by an individual, a company, a machine, or by a \ac{DAC}. An important concept thereby is a \emph{crowdsale} or \ac{ICO}. Tokens are issued and sold to fund the development of a specific protocol or decentralized application. These tokens can then later be redeemed to use the protocol or application. Often these tokens are bought by speculators instead of prospective users with the hope that the service will be successful and the value of the tokens appreciates with a growing user base. 

% \section{Economic Devices}

% \subsection{Phase I: Smart Property}



% \subsection{Phase II: Autonomous (Micro) Payments}



% \subsection{Phase III: } 
% Cryptocurrencies are digital bearer assets existing on public blockchains. Decentralization, cryptography and the anonymous and competitive consensus mechanism provide the basis for permissionless access independent of legal status. This allows software agents to act as autonomous economic agents which are able to control monetary value and enter contractual agreements with other economic agents, either humans or machines. These software agents can either be instantiated on a decentralized application platform with built-in cryptocurrency such as Ethereum or on tamper-proof hardware that is able to store and use cryptographic keys without interference. We define connected devices interfacing cryptocurrency blockchains as economic devices due to their ability to contribute directly to an economy by facilitating economic interactions. These economic interactions can either be rigid and predefined in order to automatically execute transactions as programmed by humans, or the result of economic decisions made by intelligent economic devices. Economic devices could be the building blocks of a decentralized \ac{IOT} by interacting with cloud services on the basis of market economics instead of vertical integration.

% In the limit of complete autonomy, economic devices would \emph{own themselves}. However, ownership (in contrast to possession) is a legal concept. Thus, economic devices would need to be recognized as legal entities. Property rights theory assigns the owner of an asset the residual control rights over that asset \parencite{10.2307/2937753}. Thus, in principle, ownership is only of concern if contractual relationships are incomplete, i.e. not all actions and payments for all possible contingencies are specified \parencite{GKlein:2015en}. \cite{lessig2009code} has argued that there are more modes of regulation than law. Also the market, architecture, and norms regulate the actions of the individual and their interactions within the society as a whole. Norms are less important when interactions are anonymous and arguably even less so for software agents and economic devices. Architecture, in form of capabilities and constraints by hardware, software and cryptography, and the market, in form of financial incentives or disincentives, i.e. deposits and penalties, are able to provide a basic regulatory framework for economic devices and their interactions with humans. Cryptocurrencies and blockchains are merging the concepts of architecture and the market and enable the rise of a \emph{Lex Cryptographia} which is "characterized by a set of rules administered through self-executing smart contracts and decentralized organizations" \parencite{wright2015decentralized}.
% These forms of regulation are particularly important in nations with weak law enforcement and when considering global, i.e. supranational, economic interactions via the Internet.

% \subsection{Basic Capabilities}

% In the following, basic capabilities of economic devices that can be implemented with the dominant cryptocurrency platforms Bitcoin and Ethereum are discussed.

% \subsubsection{Autonomous trading of digital goods and services}

% Cryptocurrencies allow a machine to reason about the finality of a payment. Thus, an economic device is able to autonomously \emph{sell services or (digital) goods}. It is worth to stress this point. Only because a machine can be sure that a payment can not be reverted by some authority, it can safely sell anything. A machine itself is not able to enter a legal arbitration process which is also expensive due to the involvement of humans. Sensing as a Service as presented in the last two chapters is one example of this capability to autonomously offer a sensing service or sell data as a digital good. However the most fundamental example is mining which provides security to the cryptocurrency infrastructure itself in return of a probabilistic payment. 

% Once in control of the cryptocurrency, an economic device is able to \emph{buy services or (digital) goods} from other economic devices or from humans (see e.g. Amazon Mechanical Turk\footnote{https://www.mturk.com/mturk/} or participatory mobile crowdsensing \parencite{burke2006participatory}).

% \subsubsection{Contractual relationships based on digital bearer assets}

% In addition to these basic economic capabilities, more complex agreements can be implemented by \emph{issuing tokens representing (transferable) rights} concerning the economic device. These tokens can represent e.g. concrete IOUs, claims on revenue or access/control rights. IOUs are like vouchers or coupons and authorize the bearer to redeem the token for a specified service or good. Claims on revenue allow to implement revenue-sharing schemes between economic devices and humans. We will discuss an application of this in the case study in Section \ref{sec:caseDisplay}.
% Transferable, i.e. tradable, access/control rights allow to own\footnote{Not necessarily in the legal sense.}, as well as to trade, and rent an economic device without having to trust the counterparty. Hence, economic devices entail and extend the notion of smart property.

% \begin{table}
%   \centering
%   \begin{tabularx}{\textwidth}{ l  X  }
%     \toprule
%     Capability  & Example \\
%     \midrule
%     Selling & S\textsuperscript{2}aaS, mining \\ 
%     Buying & Data, energy, Internet access \\ 
%     Token issuance & IOUs, claims on revenue, smart property \\  
%     \bottomrule
%   \end{tabularx}
%   \caption{Basic capabilities of economic devices.}
%   \label{tbl:capabilities}
% \end{table}

% \subsection{Examples and Use Cases of Economic Device}


% \subsection{Requirements}

% In order to implement economic devices a number of requirements have to be met.
% First, a digital bearer asset is needed to enable dis-intermediated and non-repudiable payments between machines. Thus far, cryptocurrencies based on public blockchains are the only practical implementation of this concept. Second, interactions with economic devices should be trust-less. This can either be achieved with financial deposits and penalties or with verifiable open source hardware and software. Deposits and penalties may require arbitration to decide if a counterparty did not fulfill its obligations. Programmable cryptocurrencies enable competitive markets for trust-less arbitration. Open source hardware and software, and trusted computing platforms allow the counterparty to verify the integrity of an economic device and thus reason about its behavior. Third, counterparties should not be able to modify hardware or software for their benefit, i.e. the effort of modification or tampering should outweigh the benefit. In addition, there are the basic requirements of having access to energy and to the Internet. 

% % \begin{table}
% %   \centering
% %   \begin{tabularx}{\textwidth}{ l  X  }
% %     \toprule
% %     Requirement  & Description \\
% %     \midrule
% %     Programmable digital bearer asset & \\ 
% %     Energy & S\textsuperscript{2}aaS, mining \\ 
% %     Internet access & Data, energy, Internet access \\ 
% %     Open Source & IOUs, claims on revenue, smart property \\
% %     Tamper-proof hardware & \\

% %     \bottomrule
% %   \end{tabularx}
% %   \caption{Requirements of economic devices.}
% %   \label{tbl:requirements}
% % \end{table}


\section{Background}


\subsection{Smart Contract}

The term \emph{smart contract} has already been introduced vaguely as a contract enforced by code (c.f. Chapter \ref{sec:crypto}. The thesis provided a few examples of contracts enforced by Bitcoin script. The two most important examples were payment channels (Sec. \ref{sec:paymentchannels}) and \ac{HTLC}s (Sec. \ref{sec:htlc}). These contracts work by posting collateral and programmatically define how the collateral can be spent. In the case of a payment channel, for example, the collateral posted by the payer is frozen for a specified time and can only be redistributed in the meantime collaboratively by the payer and payee. Thereby the blockchain acts a programmable custodian of the collateral. Traditionally, either a trusted third party would have to be employed as a custodian, or a legal contract would have to be created between the parties. Smart contracts, in this sense, are an example of regulation by code. Performance of the contract is enforced automatically. Whereas performance of a traditional contract has to be audited and requires a jurisdictional system for dispute resolution, and an executive system for enforcement. This is a manual and expensive process. In particular, for agreements that cross jurisdictional borders, as is usual for transactions on the Internet. Hence, the long tail of transactions has to rely on trust and reputation. 
Only very few transactional agreements can be encoded completely in Bitcoin contracts. One example are zero knowledge contingency payments (Sec. \ref{subsec:script} and \ac{HTLC}s. In principle, a payment channel itself is only useful if it is used to perform micropayments in exchange for goods or services. However, these exchanges can rarely be made atomic, and thus, either buyer or seller have to advance performance. But because individual payments can be made as tiny as 1 satoshi, the financial loss is vanishingly small for the payer, and the loss of reputation for the payee is arguably larger than her financial gain. 
Ethereum allows, in principle, to deploy arbitrary programs with a sovereign ether balance and memory on a blockchain (c.f. Sec \ref{sec:ethereum:tech:tx} and Sec. \ref{sec:ethereum:tech:tx}). The programs are evaluated within the \ac{EVM}, and there are practical restrictions on the complexity. These programs are called contracts, and thus, programs on a blockchain are often generally identified with smart contracts. However, although Ethereum programs can encode formal agreements between multiple parties, i.e. contracts, this is no necessary condition.
The original notion of a smart contract was developed by Nick Szabo. Szabo defined smart contracts as machine-readable transaction protocols which create a contract with predefined terms. He writes “Many kinds of contractual clauses (such as collateral, bonding, delineation of property rights, etc.) can be embedded in the hardware and software we deal with, in such a way as to make breach of contract expensive (if desired, sometimes prohibitively so) for the breacher.”.  Although certain kinds of smart contracts could already be implemented using trusted hardware, blockchains and cryptocurrencies are important building blocks to advance what is possible. Szabo’s generic example of a smart contract is the vending machine. 
“A canonical real-life example, which we might consider to be the primitive ancestor of smart contracts, is the humble vending machine. Within a limited amount of potential loss (the amount in the till should be less than the cost of breaching the mechanism), the machine takes in coins, and via a simple mechanism, which makes a freshman computer science problem in design with finite automata, dispense change and product according to the displayed price. The vending machine is a contract with bearer: anybody with coins can participate in an exchange with the vendor. The lockbox and other security mechanisms protect the stored coins and contents from attackers, sufficiently to allow profitable deployment of vending machines in a wide variety of areas.” \parencite{szabo1997}. Just as the vending machine is able to autonomously accept cash, machines are now in general able to accept electronic cash using cryptocurrencies.  Furthermore, Ethereum provides an open trusted computing platform which can either provide a programmable trusted third party or provide a trusted partial identity of a machine. This will be discussed later in more detail. Another aspect of smart contracts are smart legal contracts \parencite{1319505,DBLP:journals/corr/ClackBB16}. These focus on the expressibility of legal contracts in code. Although, the goal is also automation and programmatic execution of contract terms, they embrace that most traditional contracts cannot be enforced entirely by code. Hence, they focus on documenting and providing tamper-proof evidence of contract breach for traditional dispute mediation within the judicial system. Cryptographic signatures and tamper-evident append-only logs, as provided by blockchains, are a valuable tool in this regard.

\subsection{Smart Property}
\label{sec:econdev:smartproperty}

The term smart property was also introduced by Nick Szabo. He defined smart property as „Software or physical devices with the desired characteristics of ownership embedded into them; for example devices that can be rendered of far less value to agents who lack possession of a [cryptographic] key“ \parencite{szabo1997}. A simple example of smart property is a car. Modern cars already employ cryptographic protocols and use electronic keys to authenticate the owner. Electronic immobilizers enforce that only the bearer of the key is able to start the engine. Smart property can be seen as a particular type of a smart contract. However, in the case of the car, integration in a legal system is still desirable, since key theft should not imply the transfer of ownership to the thief.  On the other hand, sophisticated methods for key revocation and recovery are being developed which might enable a pure technological implementation of the concept of ownership.
The concept of smart property does also benefit tremendously from cryptocurrencies. One simple example to illustrate this is the concept of an atomic sale (see Appendix \ref{appendix:smartproperty} for a more detailed presentation of concepts related to smart property and their implementation based on Bitcoin and Ethereum). Assume a car is represented on the Bitcoin blockchain as a simple \ac{P2PKH} \ac{UTXO}. In order to open or start the car, the car demands a challenge to be signed with the private key corresponding to the \ac{UTXO}. Spending this \ac{UTXO} and assigning it to another key would allow the bearer of this key to prove the transfer of ownership to the car. Given this setup, it is now possible to construct a single transaction that transfers ownership and bitcoins. Thus, the sale is atomic. Either the trade is successful and the buyer becomes the new owner and the seller receives the coins, or the trade is unsuccessful and no exchange happens at all. Thus, no party has to advance and trust is minimized. In the case of a car sale this might not be so important because buyer and seller would typically meet in person due to the high value that is on stake. However, the model is conceivable for a deployed sensor.

% \subsection{Tokens}

% In the context of blockchains, tokens are a general concept to represent \emph{any fungible tradable good}, but also rights with respect to some entity. Fundamentally, bitcoins are the tokens that can be redeemed to write in a global public and immutable database. In this sense the Bitcoin network is a \ac{DAC} that incentives third parties by issuing tokens to provide computing power (mining). However, tokens are of use particularly if they exist in addition of the native cryptocurrency on a blockchain. Then, tokens can be traded in exchange for cryptocurrency without counterparty-risk and without intermediaries. Tokens are issued at some point. Either by an individual, a company, a machine, or by a \ac{DAC}. An important concept thereby is a \emph{crowdsale} or \ac{ICO}. Tokens are issued and sold to fund the development of a specific protocol or decentralized application. These tokens can then later be redeemed to use the protocol or application. Often these tokens are bought by speculators instead of prospective users with the hope that the service will be successful and the value of the tokens appreciates with a growing user base. 

\section{Economic Devices}

Today, things are passive participants of an economy. Things can be sold and bought. Things can be rented, and in some cases things can be capitalized, e.g. they can be used as collateral for loans. Connected, or smart, things have embedded communication and information technology. They can be digitally upgraded with additional functionality and services, and they can act as a physical point of sales for goods and services. In the latter sense, they instantiate a smart contract in a similar sense as the vending machine. An example is the Amazon Echo, a voice-controlled speaker. It is connected to your Amazon account which has access to your credit card details, and allows you to order items from Amazon with a mere voice command. There is a high degree of automation, but this automation is based on trust. You have given Amazon the ability to debit an arbitrary amount of money from your credit card, because you expect that Amazon will only debit the agreed amount. Furthermore, you expect Amazon to deliver the respective product. If Amazon would debit more money, you would trust that your bank or credit card company is able to refund you, and as a last resort you can go to court. If you'd like to sell an Amazon Echo you have to be permissioned by Amazon. Most of the functionality that the Echo provides is based on cloud services connected to your Amazon identity. Possession of the physical device is only valuable if it is connected to your Amazon identity, and this identity is owned and controlled by Amazon.

Cryptocurrencies and blockchains allow to minimize necessary trust and enable machine-based transactions that would not have been possible before. 
Machine-based transactions thereby can mean transactions between humans or corporations that are mediated by machines, or autonomous transactions between machines. The discussed S\textsuperscript{2}aaS model is an example. Cryptocurrencies allow a machine to reason about the finality of a payment, and micropayments allow the payer to advance payments with little financial risks. Smart contracts allow machines to enforce contractual relationships, and smart property allows the enforcement of property rights on connected devices without permissions of a company or the requirement of a judicial system. 
Hence, cryptocurrencies enable software agents to do autonomous economic interactions. Ideally, these software agents are implemented on trust-less platforms. Ethereum provides such a trust-less platform. However, software agents implemented purely on Ethereum are limited to actions within the platform, and their complexity and performance is limited due to the replicated architecture. Tamper-resistant hardware and trusted computing environments allow the implementation of trustworthy software agents in connected devices. The combination of trustworthy software in connected devices and economic interactions based on cryptocurrencies and smart contracts is the basis for economic devices.

\subsection{Capabilities of Economic Devices}

Capabilities of economic devices can be distinguished between passive and active capabilities. An overview of these capabilities is illustrated in Fig. \ref{fig:econdev} and will be explained in the following.

\begin{figure}
\centering
\includegraphics[width=0.7\textwidth]{./externalized/econdev.pdf}
\caption{Capabilities of economic devices.}
\label{fig:econdev}
\end{figure}

\subsubsection{Passive Capabilities}

Passive capabilities are related to the smart property concept.
Economic devices can be traded atomically. This can be facilitated either by representing the ownership on a public blockchain, or by relying on trustworthy computing on the device itself. The former has been discussed in Sec. \ref{sec:econdev:smartproperty}. The latter relies again on the fact that cryptocurrencies empower machines to reason about the finality of payments. Thus, the device can change its ownership status based on a payment. 

If the ownership of a device can be reassigned based on cryptocurrency transactions, then the device can natively be used as collateral for loans. \cite{smartproperty2011} describes a simple protocol to implement the concept with Bitcoin. If the debtor does not repay the creditor as defined in a smart contract, the device becomes unusable for the debtor. A real example of such a model is that of M-Kopa\footnote{\url{http://www.m-kopa.com/}}. M-Kopa offers solar home systems to the rural population of Western Africa. Solar home systems contain a small solar panel, a battery, and a few loads, such as LED lights and an USB charger. M-Kopa offers these solar home system using pay-as-you-go models in order to attract low-income households. In a pay-as-you-go model, only a small down-payment is made in the beginning, and then the customers pays for usage. The typical model employed by M-Kopa is the lease-to-own model which transfers ownership to the customer after sufficient payments. Typical pay-as-you-go models are based on a centralized service, that can be denied without payment. However, a solar home system is self-sufficient and does only need exposure to the sun. Thus, it is cumbersome to prevent usage without payment. Therefore, M-Kopa embeds connectivity in the device which allows to lock the device remotely - rendering it useless for the customer. After the customer has paid for the device, she is able return to the pay-as-you-go model in order to get a loan to buy selected products. Hence, M-Kopa acts a gatekeeper to provide financing. In contrast, economic devices based on cryptocurrencies allow an open market for loans based on programmable collateral.

\subsubsection{Active Capabilities}

Active capabilities entail the direct exchange of (digital) goods and services for cryptocurrencies. Economic devices can offer goods and services in exchange for cryptocurrency, and they can pay for goods and services. These enables the basic interactions of what IBM calls the \emph{Economy of Things} \parencite{Pureswaran2015}. This capability depends heavily on the possibility to perform micropayments. In particular, this allows to pay for each \ac{API} call directly and enables autonomy and interoperability between web services. 

In addition, economic devices can enter complex contractual relationships. A standard way to implement such relationships are tokens. Tokens are meta coins \ref{sec:eco:altmeta} created on a blockchain. Tokens can be used to grant the bearer defined usage rights. For example a sensor might offer a token that grants access to data for a particular period. Moreover, the sensor could issue a token that grants the bearer a share of its revenue. Tokens can be tradable or ownership transfer can be permissioned. If the tokens are tradable, then the trade against other tokens or cryptocurrency can be atomically, i.e. trust-less.

Economic devices can either be deterministically preprogrammed to determine how they act in these active economic interactions. For example it might be preprogrammed that a sensor offers data for a particular price or how the price should adapt based on various parameters. Tokens can also be used to distribute rights to a number of parties, and a voting mechanism can be used to change parameters defining the behavior of the device. However, with the progress in artificial intelligence it is conceivable that devices are becoming completely autonomous economic agents. 

\section{Applications and Significance} 

Arguably, the greatest potential for economic devices is in developing countries. Developing countries often lack a formal property system. Large parts of the population lack access to formal financial services. Nearly 2.2 bn adults in Africa, Asia, Latin America, and the Middle East are \emph{unbanked}. In sub-saharan Africa this culminates to 80\% of the adult population \parencite{chaia2010counting}. On the other hand, Internet access and smartphone ownership is rising \parencite{poushter2016smartphone} fueled by the diminishing costs of computation and communication technology. This provides access to cryptocurrencies and smart contract platforms, and thus to digital financial and legal systems purely based on open source software.
Productive assets like solar home systems are ideal applicants for economic devices combining active and passive capabilities. The system could offer electricity in exchange for cryptocurrency payments, and issue tokens that collect a share of the revenue. These tokens could be sold to global investors, and thus reduce the upfront cost of the system for the local user. The same model can be applied to other productive assets like small wind turbines or communication infrastructure such as small cells. 
Furthermore, economic devices are ideal building blocks for the sharing economy. The German start-up company Slock.it\footnote{\url{http://wwww.slock.it}} presented the prototype of an economic device called \emph{slock}. A slock is a cryptocurrency and smart contract enabled lock. For example a bike owner could use a slock to lock her bike and allow others to rent the bike. A renter would have to pay a deposit to the smart contract deployed on the Ethereum network. The deposit minus the renting fees is then returned to the renter if evidence of proper return is provided to the smart contract. A similar scheme for all kinds of things is presented in \parencite{Bogner:2016:DSA:2991561.2998465}. 
Continuing this line of thinking, it is conceivable that in some years autonomous cars, drones and robots will be operating autonomously while generating profits for a fluid collection of shareholders.

\section{An Ethereum-enabled Public Display}
\label{sec:caseDisplay}

In order to gain insights into the current possibilities and limits of economic devices, we developed a prototype with similar characteristics as the solar home system described above. However, we decided to develop a prototype of a public display which offers the service to show user-selected content in exchange for cryptocurrency payments. The display is better suited for demonstrations and provides more means for user interaction. The system was demonstrated at the MIT Media Lab Demo days in Fall 2016. The display is able to issue tokens that entitle bearers to a share of the generated revenue. In analogy to a corporation, we call the tokens \emph{shares}, and the bearers \emph{shareholders} or sometimes investors. In contrast to the solar home system, which is typically used by an individual or a household, the public display is inherently a multi-user system comparable to a solar micro utility. However, there needs to be a stakeholder to install and maintain the system. We call this stakeholder \emph{entrepreneur} or \emph{operator}. The \emph{entrepreneur} is also a shareholder in order to have \emph{skin in the game}. However, the entrepreneur does not need to provide the entire financing. The entrepreneur and the manufacturer can initiate a public crowdsale (c.f. \ref{sec:eco:crowdsale}) for the shares in order close the financing gap. Furthermore, share issuance is adapted such that new shares are issued as a function of revenue. These shares are awarded to the entrepreneur. This provides further incentives for the entrepreneur to maximize the utility of the system, and over time shares accumulate with the entrepreneur. 

Shareholders, as well as prospective shareholders, are able to observe the performance of the economic device transparently. Thus, a fair market price of the shares can evolve. 

The prototype is built on the Ethereum blockchain. Financial interactions are implemented as Ethereum smart contracts. The public display is augmented by an embedded Linux computer, and reacts to changes in the smart contract code. The focus of the prototype is on the software implementation. The hardware itself is not tamper-resistant. In addition, a possible implementation based on Bitcoin is briefly described.


% It is argued above that economic devices might be important in developing countries. In particular countries without pervasive financial services in combination with weak law enforcement. These conditions stifle economic development, the procurement of income producing assets and entrepreneurship. A conceptual economic device is presented that allows financial services 

% \subsection{Motivation}

% Outside high-income \ac{OECD} countries large parts of the world population is essentially unserved by traditional financial institutions such as banks. Nearly 2.2 bn adults in Africa, Asia, Latin America, and the Middle East are \emph{unbanked}. In sub-saharan Africa this culminates to 80\% of the adult population \parencite{chaia2010counting}. On the other hand, Internet access and smartphone ownership is rising. Connected devices can serve as an anchor for new forms of financial services. A prominent, and early example is M-Kopa which offers pay-as-you-go models to lease \ac{SHS} to rural customers in Africa, where most customers do not have the capital to purchase such a device. Since 2011, M-Kopa has provided 375 000 households access to solar energy in Kenya, Tanzania, and Uganda \parencite{adabou2016}. Each \ac{SHS}, consisting of a solar panel, a battery and some appliances, is equipped with a SIM card which allows M-Kopa to remotely disable the system if payments have stalled. Furthermore, customers are able to remortgage their \ac{SHS} in exchange for consumer finance loans to procure productive assets such as cooking stoves and smartphones. Thus, M-Kopa is essentially a finance company based on a connected product and mobile payments. The model is tightly vertically integrated. Economic devices allow to \emph{unbundle} this model by creating a transparent and competitive market. Although we were motivated by the prospect of financing rural electrification by instantiating \ac{SHS}s, solar micro utilities and solar micro grids with economic devices, we implemented the public display because it is conceptually similar to a solar micro utility and it can be demonstrated more easily and with more effect.


\section{System Architecture}

In the following we present the system, the stakeholders and their primary roles. Figure \ref{fig:smartproperty} provides a simplified illustration of the stakeholders and their roles. 

\begin{figure}[!t]
    \centering
    \includegraphics[width=0.7\linewidth]{./externalized/publicdisplay.pdf}
    \caption{Stakeholders and their primary roles in the model.}
    \label{fig:smartproperty}
  \end{figure}

\subsection{Stakeholders}

\paragraph{Manufacturer}

The manufacturer produces the economic object, and is the beneficiary of the initial sale of shares.

\paragraph{Entrepreneur / Operator}

The entrepreneur identifies a potentially profitable location for a public display and takes care of proper operation such that maximal revenue is generated. The entrepreneur has to have a double role as investor to have financial skin in the game and is incentivized by being awarded additional shares depending on generated revenue.

\paragraph{Investors}

Investors seek a profitable investment. Hence, they provide capital to close the entrepreneur's financial gap in return of a revenue share.

\paragraph{Customers}

Customers are individuals or software agents that are willing to a pay for the service the economic device provides.

\subsection{Economic Device}

The economic device is represented by a large screen augmented with an embedded Linux computer and a representation on the Ethereum blockchain. The economic device provides the service to screen user-defined content on a pay-per-time basis. Customers pay for the service using cryptocurrency (ether). The revenue is distributed immediately and securely to shareholders. Shares are initially sold in a crowdsale and can be freely traded. 

% A defined part of the initial shares has to be purchased by the entrepreneur. Based on the revenue, additional shares are issued and assigned to the entrepreneur. Thus, the entrepreneur is incentivized to maximize the utility of the economic device. 

\section{Implementation}

The economic device has a physical and a digital instantiation. Although the physical instantiation is important to provide security against physical tamper, the focus is on the digital instantiation. Ethereum allows to implement application logic as stateful programs which are stored and executed on the Ethereum network (c.f. Section \ref{sec:ethereum:tech:tx}). These programs are called (smart) contracts. By linking a connected device with a smart contract, we are able to implement an economic interface. In order to interface with the Ethereum network, a Geth client \parencite{Geth} is running on the device. The Geth client is an implementation of an Ethereum node in Go and provides an \ac{RPC} interface that can be accessed conveniently from Node.js via the web3.js library \parencite{Web3}. Figure \ref{fig:displayImp} provides an illustration of the system as implemented. The aspects of the initial token sale are not shown. 

\begin{figure}
 \centering
 \includegraphics[width=0.7\textwidth]{./externalized/displayImp.pdf}
 \caption{Illustration of the implemented system after deployment by the entrepreneur and the manufacturer.}
 \label{fig:displayImp}
 \end{figure}


\subsection{Economic Device Contract}

The contract is implemented using Solidity, the JavaScript-like higher-level smart contract language that can be compiled to \ac{EVM} byte code. In the following we describe the functions of the economic device contract providing the economic interface.

\paragraph{Shares as Tokens}

Shares are represented as tokens. We base the smart contract on the standard token contract\footnote{https://github.com/ConsenSys/Tokens/blob/master/Token\_Contracts/contracts/StandardToken.sol}. The token contract is essentially a simple accounting system that maps account numbers to values, i.e. the number of \emph{tokens} an account owns. A \emph{transfer function} then allows an account owner to debit her balance and credit another. This provides the basis for investors to trade shares. 

\paragraph{Initial Financing based on Token Sale}

Initial share allocation is achieved with an adapted crowdsale contract\footnote{https://www.ethereum.org/crowdsale}. The amount of initial shares is fixed and corresponds to the price the manufacturer demands (see Table \ref{tbl:parameters}. A certain amount of shares need to be bought by the entrepreneur as a down payment. Revenue of the crowdsale is credited to the manufacturer after a successful crowdsale. The crowdsale has a predefined maximal duration and ends with a deadline. If not enough shares are sold during this period, the contract automatically refunds the investors, and a new contract with different parameters can be deployed.

Instead of having a fixed price per share or a defined goal, an auction-based crowdsale could be implemented.

\paragraph{Payments, dividends and dilution}

The contract exposes a public \emph{pay function} (see Procedure \ref{alg:payforservice}). The function checks if the device is currently rented. If so, the customer is refunded. If not, internal functions to pay dividends and to increase the entrepreneur's share are called. Finally, a \emph{payment event} is created. The payment event carries the data tuple (\emph{payer}, \emph{paidUntil}), where payer denotes the Ethereum account address of the customer invoking the pay function, and paidUntil denotes the timestamp until the customer has paid for renting the display. 

Since share issuance is handled within the contract handling payments, the overhead is low. There is no need for an additional transaction, but only an update of the token balance. Thus, there is no need to aggregate payments until a certain threshold before the entrepreneur gets credited with additional shares. 

Noteworthy, there are two possibilities to pay dividends to shareholders. The obvious approach would be to let the contract actively send dividends to the shareholders' Ethereum accounts. However, sending value from a contract is expensive in comparison to updating an internal balance, and the cost would have to be provided by the customer invoking the pay function. To avoid this, the contract keeps track of the dividend balances and we add a \emph{withdraw function}. The withdraw function allows shareholders to collect their dividends whenever they wish. Thereby, costs for withdrawing have to be borne by the withdrawing shareholder, and shareholders may decide for themselves when collecting dividends is appropriate. 

\begin{algorithm}[!t]
  \floatname{algorithm}{Procedure}
   \caption{Pay for service}
    \begin{algorithmic}[1]
      \If{now $\geq$ paidUntil}
      \State \Call{creditDividends}{$value$}
      \State \Call{dilute}{$value$}
      \State paidUntil = now + value/pricePerTimeUnit
      \State \textbf{Event} Payment(customer,paidUntil)
      \Else 
      \State return money to customer
      \EndIf
      \Statex
      \Function{creditDividends}{$value$}
      \For{sh in shareholders}
      \State balanceEther[sh] += balanceShares[sh]/totalShares*value
      \EndFor
      \EndFunction
      \Statex
      \Function{dilute}{$value$}
      \State newShares = value * supplyIncreaseRate
      \State balanceShares[indexOfEntrepreneur] += newShares
      \State totalShares += newShares
      \EndFunction
\end{algorithmic}
\label{alg:payforservice}
\end{algorithm}

\paragraph{Financial Parameters}

The contract defines several financial parameters. Table \ref{tbl:parameters} provides a listing with a short description of each parameter. In the current implementation all these parameters are fixed in the contract. However, especially the price of the service might need to be adjusted based on competition and demand. These adjustments would have to happen under defined rules since otherwise the entrepreneur could set a low price, rent the display herself, and offer the service . Thus, investors would be circumvented.

 \begin{table}
  \centering
  \begin{tabularx}{\textwidth}{ l  X  }
    \toprule
    Parameter & Description \\
    \midrule
    sharePrice & Price per share in Wei. \\ 
    initialShares & Total supply of shares in sale. initialShares*sharePrice is paid to the manufacturer.\\
    initialSharesEntrepreneur & Number of shares the entrepreneur has to buy (down payment). \\
    issuanceRate & Shares issued to the entrepreneur per revenue \\  
    pricePerTimeUnit & Price of service \\  
    \bottomrule
  \end{tabularx}
  \caption{Financial parameters of the contract.}
  \label{tbl:parameters}
\end{table}


\paragraph{Contract - Device Communication}

The application running locally on the device has essentially only one job: watching for payment events executed by the respective contract instantiation. Solidity events provide an interface to the logging facilities of the Ethereum Virtual Machine, and can be used to trigger client application logic with data payload. The payment event triggers a state transition of the device (c.f. Figure \ref{fig:displayStates}).

Logs are committed to the block header and are thus accessible for light clients (c.f. Sections \ref{sec:eth_blockchain} and \ref{sec:eth_lightclient}). Thus, a future implementation can be based on an Ethereum light client with much lower computation, storage and bandwidth requirements - an important point for possible applications in developing countries.

\begin{figure}
 \centering
 \includegraphics[width=0.7\textwidth]{./externalized/displayStates.pdf}
 \caption{State machine of the device. There are two main states: available and rented. If available, the pay function can be invoked and the payment event notifies the device that customer $C$ has paid to cast content until time $T$.}
 \label{fig:displayStates}
 \end{figure}


\subsection{Interacting with the Device}

In principle, all interactions could be facilitated via the Ethereum network. In particular with the availability of the \ac{P2P} messaging protocol Whisper \parencite{whisper} and the Mist browser\footnote{https://github.com/ethereum/mist}. However, these tools are at a very early stage. Instead, the economic device provides a web server to serve graphical user interfaces in form of (mobile) static single-page web applications. 

The client does even not necessarily need an Ethereum client which is important as long as fully functioning light clients are not available. We use the Lightwallet library\footnote{https://github.com/ConsenSys/eth-lightwallet} to keep the wallet and private keys locally on the client, and interact with the Ethereum network via an Ethereum client with a public \ac{RPC} interface.

In the following we briefly present the steps a customer needs to take in order to rent the display: 

\begin{enumerate}
\item Open \emph{personal} website of the display.
\item Open an existing wallet by providing a seed and a password or create a new wallet. Make sure that the wallet has sufficient funds.
\item Make a payment.
\item Send links to content directly to the device\footnote{This done via a websocket\parencite{rfc6455} connection.}. The messages are authenticated, such that the device can verify if the sender is the same account as the one that paid to the contract. You can change the content as long the period you paid for is not expired.
\end{enumerate} 

An illustration of one view of the customer interface is shown in Figure \ref{fig:payview}.

\begin{figure}
 \centering
 \includegraphics[width=0.7\textwidth]{./externalized/displayUserPay.pdf}
 \caption{An example view of the customer interface. In this view the customer can make the payment.}
 \label{fig:payview}
 \end{figure}
 
\subsection{Implementation with Bitcoin}

The Bitcoin ecosystem is much more mature. The exchange rate to fiat currencies is more stable, there are many more ways to procure bitcoins, and there are already multiple implementations of light clients. Hence, an implementation based on Bitcoin would be desirable. In what follows we briefly the discuss the main points.

Shares could be implemented based on a \emph{colored coins} specification (see Section \ref{sec:coloredcoins}). However, there would be no way to trust-lessly issue shares based on revenue. This role would have to be taken over by an application running on the device with access to the necessary cryptographic key. Since the entrepreneur has physical access to the device, it is important that the device is tamper-proof. Otherwise the entrepreneur could gain access to the keys and start issuing shares.

The crowdsale could be implemented as an assurance contract \parencite{smartcontr} (see also the Lighthouse app, a \ac{P2P} crowdfunding app using Bitcoin assurance contracts\footnote{https://github.com/vinumeris/lighthouse}).

 Distribution of dividends could be implemented in multiple ways: (1) In order to pay the customer (i.e the web application the customer is using to use the service) parses the blockchain for current shareholders, pays the shareholders directly and gives the transaction to the device which can verify the correctness itself. However, if individual payments are small and there are many shareholders, the transaction would have to entail a proportionally large fee, and investors end up with many small \ac{UTXO}s. (2) The device could act as a payment hub (c.f. Section \ref{sec:goingoffchain}) between customers and investors. However, this implies that the device would need the necessary capital to pre-fund the payment channels to the investors. (3) Customers pay to an address controlled by the device itself. The device aggregates payments over a specified time period and distributes the dividends according to this schedule. Since investors have to trust the device to securely store the share issuance key anyways, this would be the most practical implementation. 

In the Ethereum implementation all financial parameters were defined and immutably stored in the smart contract. Since a trusted device is already assumed, the financial parameters and other contract terms can be stored on the device. To guarantee immutability the secure hash of the contract terms can be published to the Bitcoin blockchain. 

In conclusion, an implementation based on Bitcoin is in principle possible. In contrast to Ethereum, where all financial matters and contractual terms are handled by smart contracts deployed on the network, in Bitcoin an application running on the device has to execute all business logic. Therefore, tamper-proof hardware with secure storage of cryptographic keys is indispensable.


\section{Key Findings}

\subsection{Economic devices allow trust-minimization but are not trust-less}
Cryptocurrencies enable machines to autonomously participate in economic interactions with humans and other machines. These economic interactions can be distinguished between active and passive interactions. The basis for this is that machines can reason about the finality of a payment or the state of a contract. This allows for many trust-less interactions. In cases where trust-less interactions are not possible, efficient micropayment schemes enable payments of unprecedented granularity. Thus, necessary trust in the service providing counterparty is minimized because individual payments can be advanced without significant financial risk. In the example of the public display, a customer has to pay before she is able to provide content that is shown on the display. In principle, there is no assurance that the display will show the content. However, the customer is able to probe the device using very small amounts of money. The public payment history to the device can also act as a reputation system. Initial investors, who take part in the crowdsale, need to trust the manufacturer to produce a functional device and need to trust the entrepreneur to operate and maintain the device such that it generates as much revenue as possible. Investors can follow the strategy to invest only tiny amounts in an individual economic device, and to diversify their investment in many economic devices from various manufacturers. We can also conceive the establishment of securitization contracts. These contracts would offer Investors tokens with varying risk profiles and invest their capital in a pool of dividend-paying tokens. 

\subsection{Distributing Logic between the Blockchain and the Device}
Ethereum allows to deploy arbitrary programs on the blockchain itself. Thus, the economic device can be implemented as a thin economic client. In the prototype, the entire business logic is implemented as an Ethereum contract. Execution of contract terms is initiated by transactions from shareholders or customers. In this case, the device does not even need an external Ethereum account, i.e. a local private key with access to an ether balance. However, it is worth stressing that Ethereum functions in Ethereum contracts can only be triggered with transactions originating from other contracts or external accounts. These transactions have to provide the gas required to execute the function. Thus, if the economic device needs to actively initiate an economic interaction (e.g. paying for a service) an external account is required, and the economic device would need to hold the private key. 
The model of thin economic devices allows to keep a lot of functionality on the trustworthy Ethereum platform. However, this comes with a cost and latency for execution. In addition, most economic interactions require some trust in the device anyways. Thus, developers of economic devices have to think carefully about how to split logic between the device and the Ethereum contract.
The brief discussion of an implementation based on Bitcoin and Colored Coins exemplified the \emph{thick economic client model}. Here, software running on the physical device is responsible for executing much of the business logic. Thus, tamper-resistant hardware, and vetted open source software is indispensable.

\subsection{Immutability and Governance}

One of the characteristics of blockchains is immutability\footnote{At least practically and in the absence of hard forks.}. Thus, contracts have to be defined completely at time of deployment. There are obvious issues concerning technical bugs in the contract code, but also more subtle issues concerning game theoretical aspects. A simple example is price setting. In the prototype, the price of the service, i.e. renting the display for a specific time, is predetermined. Assume an entrepreneur set up a display at a crowded location and is generating a lot of revenue. Another entrepreneur is able to observe the blockchain for such business opportunities, and may decide to set up another display with a lower price next to the original one. Now, customers will prefer the new, cheaper service, and the original display would need to adjust the price in order to stay competitive. It is possible to implement a function that is able to set a new price, but who should have the rights to call it. If we would allow the entrepreneur to set the price, she might be tempted to lower the price dramatically and to try to trick customers to pay a higher price on a side channel. Thus, the investors would not get a fair share of the profit. Another possibility would be to implement a voting system. However, coordination might be a problem, and the entrepreneur would get to much importance over time. A third possibility would be to implement an intelligent system to adjust the price. However, this increases the complexity, increasing the chance of bugs and ways to exploit even more. 
Defining and writing smart contracts is challenging and errors might be not fixable. Thus, an open collaboration between open source community, companies and academia is necessary to develop guidelines, tools, and templates that can be ever more defined and tested. For economic devices it is of particular importance that errors can not be exploited remotely, such that the same error can be exploited in many devices simultaneously. 

\subsection{Cryptocurrencies and Developing Countries}

Economic devices might be an approach to mitigate some problems concerning the underdeveloped financial and legal systems in developing countries. In particular, productive assets providing foundational infrastructure for connected devices, such as solar home systems and micro utilities, as well as small cells or other communication access points, provide interesting applications. Therefore, cryptocurrency ecosystems have to evolve such that people of developing countries get access to cryptocurrencies. Although a smartphone is enough to receive and send cryptocurrencies around the globe, the problem is the exchange with traditional fiat currencies. Typical cryptocurrency exchanges depend on the traditional financial infrastructure and require customers to deposit money using credit cards or wire transfers. However via remittance payments and innovative services like Abra\footnote{\url{https://www.goabra.com/}}, cryptocurrency can reach the rural population. Additionally, online services are emerging that allow to earn bitcoins directly. An example is 21 tasks\footnote{\url{https://21.co/tasks/}}, a platform for micro tasks which are compensated with bitcoin payments. Another example is OpenBazaar (see \ref{sec:ecobazaar}). 


\section{Conclusion}

In this chapter, the concept of economic devices was introduced. The concept combines passive economic capabilities, based on the notion of smart property, with active economic capabilities, such as trading of digital goods and services, and indirect contractual agreements encoded as tokens. These capabilities enable economic interactions with connected devices as the object as well as the subject with minimal trust requirements in the respective counterparties. Thus, enabling more automated economic interactions and new types of interactions that were not possible before because trust requirements were inhibiting. Economic devices may contribute a sharing economy that is less dependent on centralized intermediaries and can help to bridge the financing gap for connected productive assets in developing countries. The latter application was illustrated with an Ethereum-enabled public display issuing tokens that function as claims of revenue to be bought by global investors. Fiduciary code is handled by a smart contract deployed on the Ethereum network. Thus, shares of revenue are distributed automatically and transparently, decreasing the necessary trust that investors have to bring up. Even though fiduciary code is executed on a trusted computing platform, limited trust in the manufacturer and the entrepreneur is still necessary. Before economic devices are useful for developing countries, cryptocurrencies itself need to be more prevalent in these countries.