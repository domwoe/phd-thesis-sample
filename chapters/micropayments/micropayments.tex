% \chapter{Economic objects}

% In this short chapter we introduce the notion of an economic objects. 

  
% \section{Building Blocks}

% \subsection{Identity or Representation on the Blockchain}

% Economic objects can have various representations on a blockchain depending on the capabilities of the object and the specifics of the blockchain. A basic requirement for an economic object is to be able to receive and to send cryptocurrency. Thus, an economic object needs at least an \emph{account} on the respective blockchain.

% However, representation of ownership 

% \subsection{Interaction with the Blockchain}

% Smart objects vary significantly in their capabilities. From tiny battery-powered sensors with restricted computation and communication capabilities to smart TVs and refrigerators that are constantly connected to broadband Internet and continuous power supply. How will these objects interact with the blockchain? How do they initiate and verify transactions? 

% \subsection{Micropayments}


% \begin{itemize}
% \item Are able to pay for services
% \item Are able to provide service in exchange for payment
% \item Can be traded / owned non-custodial
% \item Can be shared and rented
% \item Can be securitized
% \end{itemize}




\chapter{Machine-to-Machine Micropayments}
\label{sec:m2m}

\section{Introduction}

Web micropayments have a long history. In the late 1990s, IBM and Compaq had micropayment divisions, and venture capital funded start ups like iPIN offered solutions to aggregate payments. The World Wide Web Consortium (W3C) operated a micropayments markup working group which released a specification to allow interoperability between the different micropayment systems \parencite{w3c1999}. Already the IETF standard for the Hypertext Transfer Protocol (HTTP/1.1) contained the \emph{402 Payment Required} status code \parencite{fielding1999hypertext}.

However, non of the micropayment solutions of the big companies, nor the startup companies focusing on micropayments survived. One reason for this is that most online transactions have involved human decision making. \cite{szabo1999micropayments} argues that mental transaction costs easily surpass the value of micropayments, and similarly \cite{shirky2000} points out the \emph{anxiety of buying}. Thus, consumers refuse to consume services that involve repeated decisions to weigh if the service is worth the cost. In effect, today's prevailing business models on the web are advertisements and subscriptions. However, machine-to-machine communications, driven by microservices, IoT, and artificial intelligence, will become the norm on the Internet, direct human involvement the exception. In case of machine-to-machine interactions, there are no eyeballs involved, and therefore advertisement-based business models are not applicable. On the other hand, mental transaction costs do not apply to machines. Hence, the era of online micropayments may finally arrive.

Before cryptocurrencies, online payments were dependent on various intermediaries. Micropayments usually added even more intermediaries for payment aggregation. These intermediaries have to achieve large scale and market share in order to be profitable. Service providers and consumers have to be persuaded. If large scale is achieved, companies are able to exploit their position to extract more value. These dynamics inhibit the formation of a sustainable ecosystems. Cryptocurrencies, however, are open and permissionless systems, allowing payments between machines without intermediaries.

Bitcoin is by far the most secure, the most stable, and the most pervasive representative. Thus, we will focus on Bitcoin micropayments.

The structure of this chapter is as follows. First, we present the issues with on-chain Bitcoin micropayments. Then we present different approaches for viable micropayments using Bitcoin. Thereafter, we discuss the presented micropayment schemes from an IoT perspective, and present possible applications of machine-to-machine micropayment schemes.

% \section{Motivating Example}

\section{State of the art}

\subsection{PayPal}

The most successful online payment company PayPal considers payments below \$10 as micropayments\footnote{https://www.paypal.com/us/webapps/mpp/merchant-fees}, and the fees for micropayments are 5\% + \$0.05 in June 2016. Hence, for payments below \$1 the fees are at least 10\%, and payments on the order of cents are not viable.

\subsection{Bitcoin} 

In case of Bitcoin, and most other cryptocurrencies, transaction fees can be specified by the creator of a transaction, and miners can individually decide if they are willing to include the respective transaction into a block. Hence, although it is still possible to create zero fee transactions in Bitcoin, most miners expect a non-zero fee. Since blocks currently have a maximal size of 1 Megabyte, block space can be considered a scare resource. This effectively leads to a size-based, instead of the common value-based, market price for a transaction. By tracking for how many blocks a transaction of a particular size and a particular fee resides in the mempool, the local store of unconfirmed transactions, until the transaction is included into a block, the minimal fee for a given priority requirement can be determined. Wallets can either do this on their own or rely on web services such as bitcoinfees.21.co. 
Currently, a transaction with a high chance to be included in the next block has to include a fee of approximately 60 satoshis per byte. A typical P2PKH transaction with one input and two outputs, one for the payment and one for change, has a size of approximately 250 bytes. Hence, the fee is around 15,000 satoshis, and thus almost \$0.1 at the current exchange rate of \$650 per BTC. 
Normal P2PKH transactions are thus not suitable for micropayments. Another obstacle for naive micropayments in Bitcoin is the UTXO-based architecture. UTXOs with values on the order of satoshis would lead to a drastic increase of the UTXO set, and therefore on the resource requirements of network nodes. Furthermore, spending a macroscopic amount by combining hundreds or thousands of microscopic UTXOs would lead to very large and expensive transactions. 

In addition, confirmation times on the order of tens of minutes are too long for many types of  machine-to-machine transactions.

\section{Bitcoin Micropayments}

In order to enable micropayments with Bitcoin we need efficient methods for payment aggregation. In the following, we will present and discuss a variety of methods varying levels of trust in third parties.

\subsection{Off-Chain Micropayments}

Centralized third party services have been emerged that allow zero fee microtransactions between their users. Examples are Coinbase\footnote{https://www.coinbase.com}, 21\footnote{https://www.21.co}, and ChangeTip\footnote{https://www.changetip.com}. Thereby users deposit coins in addresses controlled by the third party service. The service then handles transactions between users internally, i.e. off-chain. However, users lose all control over their funds. The situation looks similar to using PayPal. Why does PayPal charge fees but services based on Bitcoin do not? The main reasons are: (1) PayPal accounts are usually funded by credit cards, and those transactions are susceptible to chargeback fraud. (2) Regulatory overhead. 

Another issue of centralized third party services, independent of Bitcoin-based or traditional payment platforms, is that they work only well when they have large user base. However, with market dominance comes the risk of exploit and unreasonable extraction of value. 

\subsection{Payment Channels}

\subsubsection{Unidirectional Payment Channels}
\label{sec:unipc}

The idea to use Bitcoin contracts to establish payment channels between two parties was introduced by Hearn and Spilman \parencite{Hearn2013,Spilman2013}. An unidirectional payment channel allows a single payer to pay a single payee repeatedly by sending signed transactions\footnote{In practice it is even sufficient to communicate signatures only.} over some direct communication channel instead of broadcasting individual transactions to the Bitcoin network. Individual payments can be as low as 1 satoshi and the payee is able to accept payments instantly without double-spending risk. This is achieved by using the blockchain as a trustless escrow service in form of a timelocked multi-signature output that is funded by the payer. A graphical representation and a description of a non-interactive implementation of the protocol is depicted in Figure \ref{fig:paymentchannels}.

\begin{figure}
\includegraphics[scale=.7]{./externalized/paymentchannel.pdf}
\caption{An unidirectional payment channel between a payer A and a payee B. To open a channel A creates and broadcasts a funding transaction with an output that can be spent only by providing signatures from both parties, i.e. $\sigma(A)$ and $\sigma(B)$ (multisignature). The value of this output is then shared between A and B in off-chain payment transactions. A creates and (partially) signs those payment transactions, and sends them to B via some private or public communication channel. While A has only partially signed payment transactions, B is able sign and broadcast any of those transactions to close the channel. If B is rational he will always broadcast the transaction that allocates the largest share to him. Therefore A can not decrease B's share in a subsequent payment transaction, and the payment channel is thus unidirectional. If B does not broadcast a payment transactions before $T_1$, A is able to recover the funds by broadcasting a refund transaction.}
\label{fig:paymentchannels}
\end{figure}

Unidirectional payment channels between two parties involve two on-chain Bitcoin transactions. One funding transaction and one transaction to close the channel. In the unidirectional setting only one party funds the channel and risks locking her funds (temporarily) in case the receiving party is not cooperative. Simple unidirectional payment channels are useful for repeated, metered payments between a single payer and a single payee. Use cases are for example video streaming\footnote{https://streamium.io/} and WiFi access \cite{Siby2013}.

\subsubsection{Unidirectional Mediated Payment Channels}

Mediated payment channels allow the routing of payments through intermediate payment channels by using a concept called \emph{hashlocks} or rather hash timelocked contracts (HTLC). We can express a HTLC between two parties in common words as follows: I pay you if you can provide the secret within a certain period. Assume Alice (A) wants to send a small payment to Bob (B), but Alice and Bob do not have an open payment channel. However, A has a channel with Charlie (C), and C has a channel with B. In principle A could pay C and tell him to pay B. But A has no way to enforce the payment from B to C if the payment channels are independent. We need a way to condition the individual payments on each other. This can be done by conditioning the payments on a common secret generated by the final recipient B. 

Figure \ref{fig:unimedpc} illustrates the flow of an individual payment over two payment channels. We assume that A and C, as well as C and B have already established payment channels, i.e. there exist shared multisignature outputs with some locktimes $T_0$ and $T_0^{'}$\footnote{The actual locktimes are not important. However, they have to be longer than the HTLC locktimes $T_1$ and $T_2$.}. In order to make it more concrete, we assume that the shared outputs have a value of 1 BTC each, and A wants pay B an amount of 0.1 BTC. Moreover, we assume that the channels are fresh, i.e. no payment transactions have been exchanged. First, the final recipient B creates a payment-specific random secret, computes hash(secret), and communicates hash(secret) to C and A. A creates a payment transaction that consumes the shared multisignature output and creates two outputs: (1) an output assigning 0.9 BTC to her, and (2) a HTLC output with a value of 0.1 BTC\footnote{We neglect necessary fees in order to provide a cleaner explanation.}. Figure \ref{fig:pubScriptHTLC} shows the pubScript of such a HTLC output as used in the payment transactions. A provides her signature for the transaction and sends it to C. C may now sign the transaction as well and broadcast it to the Bitcoin network. However, without the secret, C will not be able to claim the 0.1 BTC. Thus C will store the transaction and create another payment transaction addressing C with a HTLC output requiring the same secret. C signs the transaction and sends it to B. B could sign the transaction and broadcast it to the network. Since he knows the secret, he could claim the 0.1 BTC locked in the HTLC output. But he would have to do it before $T_2$. Otherwise B would be able to claim the output. B would claim the HTLC output by broadcasting a transaction that entails the secret. Thus, the secret would be public and C could use it to claim the HTLC output of the payment transaction from A. To ensure that C is always able to do that before A is able to reclaim the value, $T_1$ has to be sufficiently later than $T_2$. If timelocks are selected appropriately, a payment between A and B mediated by C using HTLCs is atomic, i.e. either both payments succeed or both payments fail. This means also that B can just communicate the secret privately to C and A, and they update their shares accordingly. Hence, A would create a transaction that replaces the HTLC output with an output granting C the 0.1 BTC. After signing and sending the transaction to C, C would do the same concerning the HTLC with B, and the payment is concluded without any on-chain transaction.

Mediated payment channels can be used in a hub and spoke architecture to connect a large number of buyers with a large number of sellers with only one payment channel per buyer/seller. Think of 100 of fitness trackers interested in buying air quality data from 1000 consumer air quality sensors in Bejing in order to suggest a healthy running track. There would have to be 100 000 p2p payment channels, but only 1100 if there were a central mediator (c.f. Figure \ref{fig:nmpaymentchannels} and \ref{fig:hubandspoke}).


A particular payment transaction might entail a number of HTLC outputs corresponding to payments to multiple sellers concurrently (c.f. Fig. \ref{fig:htlcflow}.

\begin{figure}
\centering
\includegraphics[scale=0.7]{./externalized/unimedpc.pdf}
\caption{Protocol for the the k-th payment over two payment channels connected by a hash timelocked contract.}
\label{fig:unimedpc}
\end{figure}

\begin{figure}
\begin{lstlisting}[breaklines,mathescape=true]
OP_IF
  OP_DUP OP_HASH160 <PubKeyHash B (C)> OP_EQUALVERIFY OP_CHECKSIG
  HASH160 <Hash160 (secret)> EQUAL
OP_ELSE
  OP_DUP OP_HASH160 <PubKeyHash A (B)> OP_EQUALVERIFY OP_CHECKSIG
  <$T_1$ ($T_2$)> CHECKLOCKTIMEVERIFY DROP
OP_ENDIF
\end{lstlisting} 
\caption{HTLC pubScript of a payment transaction in an unidirectional mediated payment channel setup. The first branch of the conditional can be used by the recipient to claim the output by providing the secret, the second branch can be used by the sender to reclaim their funds after $T_1$ ($T_2$).}
\label{fig:pubScriptHTLC}
\end{figure}



\begin{figure}[ht]
  \centering
  \begin{subfigure}[t]{0.5\linewidth}
    \centering\includegraphics[width=0.95\textwidth]{./externalized/nmpaymentchannels.pdf}
    \caption{Independent payment channels\label{fig:nmpaymentchannels}}
  \end{subfigure}%
  \begin{subfigure}[t]{0.5\linewidth}
    \centering\includegraphics[width=0.95\textwidth]{./externalized/hubandspoke.pdf}
    \caption{Hub and spoke architecture\label{fig:hubandspoke}}
  \end{subfigure}
  \caption{Comparison between individual payment channels between buyers and sellers, and a hub and spoke architecture based on mediated payment channels. The number of channels can be decreased from N x M to N + M.}
\end{figure}


\begin{figure}
\centering
\includegraphics[width=\linewidth]{./externalized/htlcflow.pdf}
\caption{A series of (off-chain) transactions with two concurrent payments. After the secret is released the corresponding HTLC output can be removed and the balances can be updated. The large byte-wise size of HTLC outputs further disincentive abortion of the protocol by broadcasting a transaction including HTLC outputs.}
\label{fig:htlcflow}
\end{figure}

\subsubsection{Bidirectional Payment Channels}

In the payment channel design presented in Sec. \ref{sec:unipc} it is not possible to revert a payment, because of the inherent asymmetry of the design. The initial payee is able to broadcast \emph{any} of the subsequent payment transactions, and will thus always broadcast the one with the greatest share allocated to him, and not necessarily the latest, if the payer would try to revert a payment. 

There are two designs that allow bidirectional payment channels: (1) Replace by timelock, and (2) Revocable commitment transactions.

The replace by timelock mechanism works by attaching decreasing absolute timelocks to each payment transaction. In contrast to the unidirectional payment channels presented in Sec. \ref{sec:unipc}, in the bidirectional payment channel scenario both parties hold fully-signed, broadcast-ready payment transactions. The decreasing absolute timelock ensures that the latest transaction would always end up in the blockchain first. Since every payment decreases the lifetime of the channel the simple design is not very practical. However, duplex micropayment channels \parencite{decker2015Duplex} are advancing this mechanism by providing trustless cooperative protocols to reset, i.e. balancing values, and refresh channels, i.e. increasing the timelock. 

Bidirectional payment channels based on revocable commitment transactions, as used in Lightning channels \parencite{poonbitcoin},  work as follows:
Assuming two parties, A and B, the protocol begins with the creation of a transaction that attempts to create a shared multi-signature output that is funded by both parties. However, before broadcasting this \emph{funding} transaction, which is not timelocked, both parties create \emph{commitment} transactions. Both commitment transactions attempt to spend the multi-signature output of the funding transaction and create an output with the pubScript depicted in Fig. \ref{fig:pubScriptCommitment}.

\begin{figure}
\begin{lstlisting}[breaklines,mathescape=true]
OP_IF
  OP_DUP OP_HASH160 <PubKeyHash B (A)> OP_EQUALVERIFY OP_CHECKSIG
  HASH160 <Hash160 (revocation secret)> EQUAL
OP_ELSE
  OP_DUP OP_HASH160 <PubKeyHash A (B)> OP_EQUALVERIFY OP_CHECKSIG
  <$\Delta$> CHECKSEQUENCEVERIFY DROP
OP_ENDIF
\end{lstlisting} 
\caption{PubScript of commitment transactions of A (B) for a bidirectional payment channel. The output can either be spent immediately by the counterparty if the counterparty can provide the revocation secret, or by the originator after a relative timelock $\Delta$.}
\label{fig:pubScriptCommitment}
\end{figure}

After the commitment transactions have been exchanged, the funding transaction can be broadcasted. 

\subsubsection{Routable Payment Channel Networks}
\label{sec:paymentchannelnetworks}

Payment channel networks are the generalization of mediated bidirectional payment channels. In principle they allow to interconnect an arbitrary number of payment channels in order to route instant off-chain payments. It is not yet clear what kind of topology payment channel networks will evolve. In order to prevent single points of failure a decentralized topology where regular wallets can act as intermediary nodes is preferred. First developments towards a payment routing protocol are presented in \parencite{BitFuryFlare2016}.

\subsection{Further Micropayment Mechanisms}

\subsubsection{Digital Promissory Note}

In general, a promissory note is a written financial instrument, in which an issuer promises the bearer to pay a specific amount of money under specified terms. Thereby, a payee is able to accept payments by different payers without having to trust them individually, but by only having to the trust the issuer that he will be able to redeem the promissory notes eventually. Micropayments in form of promissory notes can be aggregated, and redeemed at the issuer as soon as a substantial amount is reached. An implementation of this concept is the Stroem protocol \parencite{strawpay}. In this protocol, promissory notes are dedicated to a specific purpose, i.e. a specific purchase. Thus, the payer needs to procure the promissory note at time of the transaction. Procurement of promissory notes is a perfect application for unidirectional payment channels. A payer can open a payment channel with an issuer an purchase low-value promissory notes in order to keep the counterparty risk low. Depending on how many parties accept the promissory notes of the particular issuer the effective reach and usefulness of the payment channel may be increased significantly. An illustration of payments based on promissory notes is shown in Fig. \ref{fig:promissorynotes}.

\begin{figure}
\centering
\includegraphics[width=.9\linewidth]{./externalized/promissorynotes.pdf}
\caption{Possible participants, relationships and transaction flows in a micropayment scheme based on promissory notes.}
\label{fig:promissorynotes}
\end{figure}

\subsubsection{Probabilistic Payments}

Probabilistic payments \parencite{wheeler1996transactions,rivest1996payword,rivest1997electronic} are lottery-based payments. The lottery is biased in such a way that a single draw has an expected value according to the aspired micropayment. Probabilistic payments are only fair, i.e. the actual value paid approaches the value that should be paid, for a large number of payments on the order of tens of thousands. 
\cite{Pass:2015:MDC:2810103.2813713} present such a lottery-based micropayment scheme for ledger-based transaction systems, and discuss implementations in Bitcoin. However an implementation that does not rely on a partially-trusted third party would need a new signature verification primitive in the Bitcoin scripting language. The partially-trusted third party acts as an escrow service and sign a transaction in case the payee is able to provide a winning ticket. In contrast to unidirectional payment channels, a single UTXO could be used to pay more than one party.

Interestingly, probabilistic payments could be combined with payment channels in order to allow for sub-satoshi payments.

% \subsubsection{Pennybank}

% The pennybank protocol allows to implement bidirectional payment channels without chaining partially signed transactions, and is thus not affected by Bitcoin's malleability issue. As for payment channels, the starting point is a shared multi-signature output. Each participant now builds a chain of hashes with length $N$ starting from a random secret $p^i_0$ where the superscript $i$ denotes the participant. 

% \begin{equation*}
% (\text{SHA256}(... \text{SHA256}(\text{SHA256}(\text{SHA256}(p^i_0),p^i_1),p^i_2)...,p^i_N)
% \end{equation*}

% $N$ has to be chosen according to the value the pennybank should have, i.e. the value of the output defined in what follows. Brute force searching for $p^i_0$ should be comparably hard as the difficulty to create the coins.  After that both parties cooperatively create a transaction spending the shared multi-signature output and creating a \emph{conditional} 1-of-2 multi-signature output P2CM with the following pubScript:

% \begin{lstlisting}
% OP_HASH160 <HASH160(p^A_0)> OP_EQUALVERIFY OP_HASH160 <HASH160(p^B_0)> OP_EQUALVERIFY <OP_1> <A pubkey> <B pubkey> <OP_2> <OP_CHECKMULTISIG>
% \end{lstlisting}

% Both parties hold a copy of this transaction and could broadcast the transaction at any time. The P2CM output can be spend by providing only one signature but both secrets have to be provided. Since brute forcing the other secret is expensive they are incentivized to cooperatively close the channel.


\section{Comparison}

All of the micropayment schemes are based on multi-si

Single funded payment channels and unspendable UTXOs.

Lightning and Duplex need to be online 

Unidirectional Payment channels work offline. Mediated only Hub needs to be online 

\section{Applications}

\subsection{Sensing-as-a-Service}
\label{sec:micro:apps:s2aas}

The \emph{as a service} model originates from cloud computing. From an economic perspective of the consumer side it is often equivalent to \emph{pay as you go}. Business models like Software-as-a-Service, Platform-as-a-Service or Infrastructure-as-a-Service have offered an alternative to high initial capital costs for computing infrastructure and skilled employees. However, individuals and companies, in principle, still could decide between having their infrastructure and software on premise or in the cloud, i.e. in a data center operated by another company. Sensing-as-a-Service is fundamentally different in this regard. Sensing-as-a-Service is in most cases local, i.e. fixed to a particular position in space (and time). Deploying company-specific sensors would not only be mediocre from an economic point of view, but in general impossible. 

The Sensing-as-a-Service model is still in its infancy. The data producing, and even more so the data consuming side, is far from being deployed. Private/public sensing and actuating infrastructure in urban environments, in urban environments for agricultural purposes, along the grid, and on streets and in the air due to autonomous vehicles has just started.

Research mainly considers cloud-centric Sensing-as-a-Service models, where sensing tasks and billing are administered by a centralized platform. Machine-to-Machine micropayments allow complementary and alternative architectures, where payments and data exchange can happen directly between devices. We will discuss this in more detail in a later chapter, and in the Filament and 21 case studies.


% \parencite{Perera:2014iz}
% \parencite{Abdelwahab:2014dx}
% \parencite{Sheng2013sensing}
% \parencite{Distefano2012sensing}


\subsection{Machine-payable APIs}

Business models around the web are based on tracking and advertisement. It has been argued that this is the case because there was no way for doing micropayments. While this obviously affects our web experience, the model becomes even more problematic concerning autonomous devices. Usage of APIs is mostly rate limited or only available after (human) subscription. From an API provider perspective, this is reasonable since API usage causes costs. Furthermore, without rate limitation there is a risk of Denial-of-Service attacks \parencite{Needham:1993:DS:168588.168607}. Machine-payable APIs with micropayment support as introduced by 21 Inc. allow autonomous software agents to pay for consumption of individual API access. Thus, in many cases human-mediated sign ups, subscriptions, and credentials can be avoided. This is necessary to enable a market for micro services, where autonomous agents can find prices by real-time bidding . However, it is not sufficient. Semantic web technologies, service descriptions, and in many cases human involvement and curation is still needed. 

\subsection{Coordination of Underutilized Resources}


\subsubsection{Storage}

The price of hard drive storage has fallen even more than the price of computing power over the years\footnote{http://www.jcmit.com/disk2015.htm}. Many devices have an abundance of storage. Still cloud storage like Dropbox\footnote{https://www.dropbox.com/}, Amazon S3\footnote{https://aws.amazon.com/s3/}, or Google Drive\footnote{https://www.google.com/drive/} is used extensively. The reason is availability of data from other devices. However, the same could be achieved by coordination of unused consumer storage capacity. Especially since the connection of end users with optical fiber which provides symmetric high-bandwidth up- and down-stream. Micropayments in combination with storage proofs \parencite{Vorick2014sia} can provide the structure for a reliable service. 

At time of writing, there are multiple projects and start ups working on this application \parencite{Vorick2014sia,filecoin2014,maidsafe2014,storj2014}. 
In contrast to the incumbent, centralized services, these decentralized services use cryptographic techniques like secret sharing \parencite{Shamir:1979:SS:359168.359176} to encrypt and redundantly distribute data across the network. Thus, providing confidentiality in addition to a high level of availability. 

\subsubsection{Computing Power}

Most computing and processing today happens in large data centers operated by companies like Amazon and Google. Economies of scale has led to a consolidation of many services into these specialized data centers, but there are also dis-economies of scale, e.g. the increasing expenses for cooling and networking. Already in the early 2000s, alternative, distributed, models have been envisioned and implemented. Worth mentioning are in particular the large academic Grid computing initiatives Seti@Home \parencite{Anderson:2002:SEP:581571.581573}, Boinc \parencite{Anderson2004boinc}, and Folding@Home \parencite{Beberg2009Folding}. Instead of using a single supercomputer, computations were distributed to scientifically interested volunteers which provided spare computing cycles of their personal computers. Seti@Home,for example, was implemented as a screen saver to ensure that the computer is not in use when performing computations. However, the model could not be emulated for general purpose computation, since there is little incentive for end users to participate. In contrast, participation causes additional costs and risk of exploitation.
Machine-to-Machine micropayments can provide the missing incentives. The usage of incentives, security deposits, and cryptographic mechanism may provide a platform for outsourcing general purpose computing tasks to untrusted machines, and thus enable a distributed general purpose computing grid. Enigma can be seen as one example based on Bitcoin. Golem\footnote{http://golemproject.net/} is another example based on Ethereum.

% \cite{Gray:2008:DCE:1394127.1394131} writes "[Grid computing and web services] are
% designed for computer-to-computer interactions and so have no advertising model—
% because there are no eyeballs involved in the interactions. It is up to the companies to
% invent business models that can leverage the Web-services plumbing".   


\subsubsection{Internet Access and Bandwidth}

Today, our Internet access is based on fixed subscriptions with telecommunication providers. At home our smartphones connect to our wifi which provides access to the Internet through our broadband network provider. On the go, our smartphones connect to wireless mobile telecommunications services and our SIM cards provide identity information that is used for billing. Ad-hoc switching between network providers leads to high roaming costs. With machine-to-machine micropayments a cell towers and smartphones or other connected devices could agree on a time-based or data-transfer-based rate. Overlapping cells operated by different companies could bid for the surcharge of providing Internet access in real time. The consumer would benefit from lower costs and anonymity. Pseudonymous payments, end-to-end encryption, and regular switching between communications provider would impede profiling and de-anonymization.

Besides company-operated cell towers every urban area is covered by a multitude of privately-owned wifi access points representing vastly underutilized resources. Of course, access point owners could provide a public good by deactivating the password protection of their access points. However, there are risks, e.g. access to the home are network with low-security devices and a user might misuse the Internet connection for illicit activities, and there is no personal utility besides altruism. A number of telecommunication provider started programs to share access and bandwidth to provider-operated virtual networks with Internet access. However, usage is, again, based on a subscription or long-term contract with a particular provider.

We have repeatedly discussed that most of the objects that will make up the Internet of Things will be resource-constrained devices. Thus, so called low-power wide area networks (LPWAN) are being deployed all over the world. There are two prevailing models. Either the network is provided similar to traditional mobile networks with fixed subscriptions\footnote{e.g. https://www.sigfox.com}, or the network is provided by enthusiasts as a public good under the terms that transferred data is open\footnote{https://www.thethingsnetwork.org/}.
Machine-to-Machine micropayments enable a third alternative.

Smart objects relying on LPWAN to stream their data might need a faster Internet connection from time to time, e.g. in order to download a firmware update. In principle, mobile phones could act as ad-hoc access points using low-power bluetooth connection. 

\subsection{Energy}

Cars are moving from internal combustion engines and gas tanks to electric engines and battery storage. Since the energy density of gas is much higher than the energy density of a battery, an electric needs to recharge more often than a car has to stop at a gas station. On the other hand electric energy is much cheaper. RWE has presented a prototypical system that allows to charge a car while waiting at a traffic light. Streets and parking spots may become wireless charging stations. The value of an individual transaction between a car and a local charging infrastructure will be rather low, and a car will likely use charging infrastructure provided by various companies, communities, cities and individuals. Thus, direct micropayments between the car and the charging infrastructure are more efficient than many or complex contractual agreements.


\section{Conclusion}






% \subsection{Asset Utilization and the Sharing Economy}

% \subsection{Extending the as-a-Service-Paradigm}

% \subsection{Lease to Own}

% \subsection{Backend-less Operations}

% \subsection{Revenue Sharing}

% \section{Other Opportunities in the Internet of Things}

% \subsection{A Black Box for the Internet of Things}

% \subsection{Enhancing Security through Decentralization}

% \subsection{Bringing the Cloud to the Edge}

% \cite{Loke:2015:MCS:2737797.2656214} show ad-hoc computations with near devices.

% \section{Challenges and Open Questions}


% \cite{Christidis2016} provide a thorough preliminaries section. In particular the sections about different consensus algorithms is interesting. The main contribution is the discussion of applications of blockchains for the Internet of Things, and their deployment considerations. Along the lines of the Device Democracy paper the authors argue that IoT devices lead to high maintenance costs for the manufacturer and give the example of the need to distribute software updates for many years. From the consumer perspective the need for IoT devices to \emph{phone home} raises privacy and security concerns.
% Software update with hash in smart contract and binary on IPFS.
% Billing layer and paves the way for a marketplace of services between devices.
% Sharing of services and property in general -> Slock.it
% Supply chain management with and without IoT.
% Deployment considerations:
% \begin{itemize}
% \item Throughput
% \item Privacy
% \item Who can mine?
% \item Legal enforceability
% \item Expected value of tokenized assets
% \item Complete autonomy is a double-edged sword
% \end{itemize}

