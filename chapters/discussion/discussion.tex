\chapter{Conclusion}
\label{sec:discussion}

\section{Summary and Key Findings}

The Internet of Things has made a leap over the last decades. Smartphones have become ubiquitous in developed countries and developing countries are catching up. The scale of the smartphone supply chain made \ac{SoC}s, combining computing, memory and various communication technologies, cheap and pervasive. Cloud computing, in combination with the \emph{as-a-Service} business model, allows to scale backend infrastructure for connected devices without large capital requirements and expertise. These developments allowed manufacturers of traditional \emph{things}, as well as start-up companies to bring an ever growing number of connected devices to market. Connected devices in combination with artificial intelligence and data analytics have great potential to make life more comfortable and efficient, to save energy, optimize industrial manufacturing, and enable new business models. However, much value can only be unlocked, if connected devices and web services from various vendors are interoperable \parencite{manyika2015unlocking}. Furthermore, connected devices pose a thread to individual as well as national security and privacy. 
The prevailing architectural paradigm of the \ac{IOT} is a tight link between connected devices and corresponding cloud serviced. Data is streamed from the device to a data center without control of the individual, and corporations have to maintain large data centers in order to keep connected devices functional over their lifetime. 
Enabling devices to be more autonomous, i.e. to be less dependent on backend infrastructure, is a promising approach towards a more secure and sustainable \ac{IOT}. In analogy to how money, contracts, and the market enabled decentralization and specialization in the human economy, it can be expected that the same can be true for the machine economy. 

Cryptocurrencies are based on cryptographic primitives, peer-to-peer networks, and a consensus mechanism that employs economic (or game theoretic) considerations. This enables decentralized and global digital money, in combination with a payment network independent of any trusted third party. The technological underpinnings are vaguely conflated to the term \emph{blockchain technology}. Blockchain technology enables digital bearer instruments which can be controlled by cryptographic keys. Thus, machines are able to facilitate autonomous economic interactions. 
The Bitcoin network was initiated in the beginning of 2009. At time of writing, the market capitalization of Bitcoin is on the order of \$10 billion. Due to its nature as open source software, Bitcoin gave rise to a flourishing ecosystem of alternative coins, start-up companies, and industry endeavors. Although, there are hundreds of alternative decentralized currencies Bitcoin's supremacy is unsolicited. Bitcoin transactions provide a built-in scripting language which allows to implement self-enforceable \emph{smart contracts}. One of the most basic, yet powerful, are multi-signature transactions enabling the implementation of trust-less escrow contracts, and are the basis for payment channels.

In order to investigate the impact of cryptocurrencies on the \ac{IOT}, one the most foundational processes, the exchange of sensor data, and its business model manifestation Sensing-as-a-Service was selected. The unique pseudonymous Bitcoin addresses can be used to identify and address a sensor, enabling a global and permissionless sensing platform with a built-in public-key infrastructure for data authentication and encryption. In a first prototypical system, Bitcoin was used as a medium of value \textbf{and} data exchange. Data payload was injected directly in Bitcoin transactions. This limits the maximal payload per transaction to 80 bytes. This is enough space for a typical data point, but already too little to use Bitcoin's native encryption scheme \ac{ECIES} for data encryption. Thus, transmitted sensor data would always be public. A severe problem for the S\textsuperscript{2}aaS scheme on one hand. However, on the other hand, a benefit for certain applications since the data gets authenticated, timestamped, and globally replicated. The broadcast-based peer-to-peer transport protocol and the proof-of-work-based consensus protocol lead to significant latency in the process. Depending on the sensor's trust in the payer, the process may take from seconds (unconfirmed transaction with double spending risk) to a period on the order of tens of minutes. Bitcoin transaction fees are, in principle, market-based. Increasing transaction volume in combination with a limited transaction throughput has lead to an increase in transaction fees. Due to the limitation in the block size, Bitcoin transaction fees are proportional to their size in bytes instead of their value. Thus, micropayments and data transactions are affected in particular, leading to transaction fees on the order of \$ 0.1 USD for the data exchange process. 

The initial prototype illustrated the concept of an autonomous sensor earning money by selling (authenticated) measurement data on an open market, but pointed out the limitations concerning confidentiality, latency, and micropayments. These limitations can be mitigated by introducing an additional communication layer. Bitcoin's cryptographic primitives and the programmability of Bitcoin transactions allow to move most communication \emph{off-chain} without changes in the trust model, i.e without sacrificing security. Bitcoin payment channels allow for instant micropayments between two parties based on a timelocked multi-signature escrow. This means a payer locks funds (capacity of the channel) with a fixed payee. Updates of the balances, with a granularity as low as 1 satoshi (on the order of \$ 0.00001 USD), can then be made securely and instantly via a direct communication channel. S\textsuperscript{2}aaS, and in particular crowdsensing applications, require the collection of data points from a large number of individual sensors. Thus, direct payment channels between a data requester and each sensor are inhibiting in many cases. We introduce hubs which cryptographically interconnect payment channels in order to mediate between data requesters and sensors. Thereby, a protocol based on \ac{HTLC}s is developed that allows for trust-less atomic payments over two interconnected channels. In contrast to the more advanced lightning network protocol \parencite{poonbitcoin}, the presented protocol is not impeded by transaction malleability, and can thus be used securely on the current Bitcoin network. A variant of the protocol has been implemented as part of a mobile crowdsensing system, that enables users of a smartphone application to sell various sensor data to interested parties without having to sign-up, or disclose any additional personal or financial information. Still, there remain two main issues. First, in practice hubs may introduce \emph{Know Your Sensor} requirements, since establishing (and closing) of channels with sensors consumes transaction fees, and locks capital. A viable business models for hubs requires to optimize payment channel lifetimes, capacities, and sensor selection. Second, block size scarcity and the low block generation rate limits the number of possible channel openings/closings. Thus, setting up channels for a million sensors on the current Bitcoin network would take at least a few months.

Scaling issues aside, cryptocurrencies enable connected devices to perform autonomous economic interactions. Direct trading of digital goods and services, such as data, computation or storage, are but one aspect. Drawing from the concepts of smart contracts and smart property, the general concept of economic devices is introduced. Economic devices provide passive and active economic capabilities. These capabilities enable economic interactions with connected devices as the object as well as the subject with minimal trust requirements in the respective counterparties. 
Passive capabilities involve atomic trading and renting of the device, as well as \emph{capitalizing} it in form of collateral for loans, by supporting the enforcement of property rights by code and direct financial incentives or penalties, instead of the legal system and human enforcement. The concept is illustrated with a prototype of an public display that is instantiating active economic capabilities. The display provides the service to screen customer-selected content on a pay-per-time basis in exchange for cryptocurrency. Furthermore, the display is able to issue tradable shares, which provide owners access to a real-time share of revenue. The prototype is implemented based on an Ethereum smart contract. In contrast to Bitcoin, Ethereum allows to deploy autonomous programs with access to cryptocurrency on the network itself. Thus, a trusted representation of the economic device can be instantiated, whereas an implementation based on Bitcoin would involve to executing business logic on the device hardware itself. Thus, tamper-resistant hardware and trusted execution environments would be indispensable. It is argued that economic devices could be beneficial to finance productive assets in developing countries where financial services and legal systems are underdeveloped.


\section{Implications for Research and Practice}

Research on \ac{IOT}, cyber-physical systems, and ubiquitous computing has a long tradition. Cryptocurrency research, on the other hand, has only begun during the time the research for this thesis was conducted. Similar to \ac{IOT}, the science, its technology and its implications touch various disciplines, and are thus inherently interdisciplinary. Until 2013, only few academic papers from the fields of computer security, distributed computing, law, and economics have been published, and the topic was absent from academic conferences. In fact, \cite{Worner:2014:YSE:2638728.2638786} was the first publication on applications of cryptocurrencies in \ac{IOT}, and was covered on CoinDesk, a popular cryptocurrency magazine \parencite{Coindesk2014}. 

Open participatory platforms for S\textsuperscript{2}aaS to enable real-time data markets \parencite{worner2016design} powered by individuals, corporations and research institutions are an emerging tool for researchers to tap novel streams of data. Open global platforms providing privacy and security as well as monetary incentives for their contributors may enable a planetary nervous system \parencite{Giannotti2012}accessible by researchers, entrepreneurs and corporations a like. Although this thesis suggests that Bitcoin is not ready to power a platform with millions or billions of devices, necessary to be \ac{IOT} scale, research and developments on scaling decentralized cryptocurrencies are a growing field (c.f. Sec. \ref{sec:btc_scalability}).

The concept of economic devices enables novel device-centered business and ownership models. This thesis has provided a first investigation of the concept and illustrated the capabilities. Device manufacturers are advised to acknowledge and explore these novel possibilities and opportunities. However, cryptocurrencies are still a niche. These models become only useful if cryptocurrencies are prevalent in the broader population.


\section{Outlook and Future Work}

The state of cryptocurrencies and blockchain technology today is often compared to the state of the Internet in the early 90s\footnote{See e.g. \url{http://www.coindesk.com/marc-andreessen-balaji-srinivasan-discuss-bitcoin/}}. Cryptocurrencies provide a fundamentally new way of establishing and transferring value over the Internet. However, the technological infrastructure is still in a very early stage. Bitcoin has spread globally, but it's still far from being ubiquitous. In order to allow for micropayments between machines to trade digital goods and services, and foster interoperability, on-chain scaling has to improve tremendously. Avenues to towards this have already presented in Sec. \ref{sec:btc_scalability}. A fist step towards higher throughput is provided with segregated witness \parencite{bip141}, but currently miners are blocking the adoption. Segregated witness would also fix Bitcoin's malleability issue which prevents the implementation of lightning payment channels. In contrast to the payment channels used in this thesis, lightning channels are bidirectional and can be open indefinitely. Instead of a single hub, payments could be routed via a multitude of channels. Thus, research on routing algorithms with economic considerations are required.
In general, the future of cryptocurrencies is uncertain. Will there be a large number of currencies and platforms each with specific features and a liquid exchange rate, or will there be a consolidation towards one particular platform? Cryptocurrency platforms with focus on high-frequency machine-to-machine micropayments are being developed, but their security and viability is an open question \parencite{tangle2016}.  As the infrastructure matures, the number of economic device prototypes will increase. Then, the focus will shift towards real-world applications and economic interactions between devices. 




\newpage
\appendix
\chapter{Bitcoin Start-up Ecosystem: Representative Cases}
\label{appendix:cases}

In \parencite{Worner2016ecis}, we investigated six promising start-up companies that are active in the Bitcoin ecosystem with applications beyond financial services. Most of these companies appear repeatedly in this thesis. Therefore, the case studies are presented here for reference.

\section{Filament (Internet of Things)}
\label{sec:ecofilament}

Filament\footnote{\url{https://filament.com/}} provides wireless sensor networks for the (industrial) \ac{IOT}, e.g. 
for smart cities or smart agriculture use case. Most \ac{IOT} platform providers follow 
a centralized approach by connecting all devices to their respective cloud-infrastructure. 
This has the major disadvantage that devices depend on a central infrastructure 
in order to operate. Moreover, it can be argued that this approach cannot keep 
up technically and economically with the increasing number connected devices.

Filament is one of the first companies that develop a fully decentralized IoT infrastructure, 
which encompasses three blockchain-related aspects: (1) Each device is registered 
on the blockchain providing a verifiable and immutable identity. This enables discovering 
of and authenticating with other devices/services without the need of a dedicated 
backend infrastructure. Therefore, devices are technically autonomous and are able 
to operate independently of Filament. (2) Each device is governed by a \emph{smart} 
contract, which manages agreements of device control/ownership, data access and 
financial agreements concerning the device. Ownership can be transferred permanently 
or temporarily by a simple transaction on the blockchain. Filament implements the 
financial agreements as a Product-as-a-Service, which means that the owner gets 
paid directly for the ongoing use of the device. (3) Furthermore, each device is 
able to transfer value in form of bitcoins to other devices in order to get access 
to data or request some service.

As described, devices can be operated and governed by using only the blockchain 
as a backend and therefore without any technical dependence on the platform creator 
(Filament) or other third parties. This might bare great benefits for customers 
needing to deploy large Industrial \ac{IOT} applications with a lifetime of 5-10 years. 
Because they want to minimize the risk of a lock-in with a specific company. Moreover, 
according to Filament, customers prefer paying continuously on a real-time basis 
instead of an upfront investment, which can be solved efficiently by Bitcoin micropayments. 
Ownership is decoupled from usage and both are independent of the manufacturer 
or a platform provider.

Filament itself is a venture capital-backed company formerly known as Pinocc.io. 
They claim to have their first deployments with Fortune 500 companies in 2016. 
They will get paid for the ongoing use of their devices (by owning the smart contract). 
Moreover, they work on a licensing model, i.e. Customers can attach a module version 
to their own devices, which will give them all the described benefits of Filament 
in return to a small fraction of the payments for the ongoing use of the devices. 
However, since all protocols will be open and there is no dependence to Filament 
by design, other companies could use that and build their own hardware without 
Filament.

\section{Ascribe (Intellectual Property)}
\label{sec:ecoascribe}

Ascribe\footnote{\url{https://www.ascribe.io/}} aims to provide provenance of intellectual property. Digital work, 
like art, photos, and music, can be registered publicly on the bitcoin blockchain 
together with its accompanying terms and conditions. The technological basis is 
the spool protocol\footnote{\url{https://github.com/ascribe/spool}}, an overlay protocol that uses 
Bitcoin transactions to represent unforgeable ownership transfers and licensing 
agreements for digital work. Thus, authenticity of ownership and usage rights can 
always be proven. Ascribe focuses on digital art in particular. Although art is 
in principle copyrighted by the time of creation, it is currently cumbersome and 
expensive to officially register work and prove ownership (see e.g. \url{http://copyright.gov/docs/fees.html}). 
Therefore, hardly any digital art gets registered. Ascribe aims to change the status 
quo by providing a virtually free and automatable registration process. Moreover, 
the unique representation of digital work based on cryptography and the blockchain 
is the basis for the creation of a secondary market for digital work. For example 
the spool protocol enables creating limited editions of digital work. So far, this 
has not been possible without relying on some central institution.

The spool protocol is open source and can be used by anybody. In principle, the 
only costs are bitcoin transaction fees. However, Ascribe wraps the protocol in 
convenient web services and provides tools adapted to particular customer groups, 
e.g. individual creators, museums and marketplaces. Ascribe's revenue model is 
then based on a share of the rentals and sales of registered digital work that 
are facilitated by their APIs.

The core innovation is the application of the bitcoin blockchain to provide commoditized 
provenance of intellectual property. Provenance is demonstrated by relying on the 
immutability of the bitcoin blockchain, instead of an authority. 

\section{OpenBazaar (E-Commerce Marketplace)}
\label{sec:ecobazaar}

OpenBazaar\footnote{\url{https://www.openbazaar.org/}} is an open source project consisting of a protocol and a reference 
implementation that enables a decentralized peer-to-peer e-commerce marketplace. 
In comparison to traditional e-commerce market places like eBay and Amazon, there 
is no central server or authority that is running the market place. Thus, there 
is no middleman who is able to charge fees or to restrict offered products and 
services. Everyone with Internet access is able to set up a shop by running a network 
client. Payments are facilitated using Bitcoin transactions. Therefore, no payment 
provider or banking account is needed. This lowers payment transaction fees and 
increases the global reach. Besides sellers and buyers there are notaries and arbiters 
for dispute resolution participating in the marketplace. Those latter participants 
are involved by using bitcoin multi-signature transactions. Thus, OpenBazaar unbundles 
the functions of traditional marketplaces. Trades on the OpenBazaar network are 
based on Ricardian Contracts \parencite{1319505}, i.e. an electronic document that defines 
the terms of a trade such that it is readable by computers and humans, and is cryptographically 
signed. Apart from selling physical and digital products, OpenBazaar can also be 
used to trade speculative contracts, which can be readily represented by Ricardian 
Contracts. 

The main value proposition of OpenBazaar to sellers as well as to buyers is the 
elimination of fees and restrictions. Since marketplaces are subject to network 
effects most users will most probably not switch immediately from traditional marketplaces 
to OpenBazaar. However, OpenBazaar could set foot in under-served niche markets. 
Examples could be digital goods, developing countries with limited access to traditional 
payment services, and prohibited goods. 

OpenBazaar itself is not a company, but its main developers founded the venture 
capital-backed company OB1. Their current focus is on developing the OpenBazaar 
protocol and its reference implementation. As outlined above, OB1 is not able to 
profit directly from transactions on the marketplace in the way traditional revenue 
mechanics on centralized marketplaces work. 

\section{21 (Digital Micro Commerce Marketplace)}
\label{sec:eco21}

At first sight, the categorization of 21 Inc.\footnote{\url{https://www.21.co}} as a challenger seems odd, 
since 21 is an infrastructure and platform provider for the bitcoin ecosystem. 
Indeed, it is often categorized as a mining company. However, we argue that 21 
is better classified as a marketplace for digital micro services, which has the 
potential to challenge traditional Internet business models.

With a funding of \$121M, 21 supersedes every other start up in the bitcoin ecosystem. 
They have developed an embeddable \ac{ASIC} mining chip that they are using in their own mining operations, 
but which is also embeddable into arbitrary connected devices. In November 2015 
they released their first product, the 21 Bitcoin computer. Essentially the 21 
Bitcoin computer is a full-stack development platform to build bitcoin-payable 
digital services, which can be published and discovered on 21's digital marketplace. 
Individual service consumptions, like an \ac{API} call, can be billed at as little as 
1 satoshi\footnote{Satoshi is the smallest denominator of a bitcoin. 100,000,000 
satoshi correspond to 1 bitcoin. Thus, 1 satoshi is currently worth approximately 
USD 0.000003.}. The embedded mining chip, which is currently coupled to a mining 
pool operated by 21, supplies the device with 
a continuous stream of satoshis.

21 aims to embed their chips into any connected device (e.g. smart phones) to establish 
bitcoin as a system resource like CPU, bandwidth or disk space, but for the purpose 
of buying and selling digital goods and services \parencite{Balaji2015}. It is crucial 
to understand that it does not make sense to sell the small amounts of mined bitcoins 
for Fiat currency on an exchange. Instead, the idea is to supply every device with 
a continuous stream of bitcoin from the point of commissioning on so that it can 
directly operate on the marketplace. 

Having such an infrastructure of bitcoin-enabled devices in place at scale, could 
offer compelling new opportunities and even disrupt traditional business models 
of the Internet. For example, it has been difficult for news sites to directly 
monetize their content on the Internet. It is still tedious for users who want 
to read just one article to signup, enter their credit card information and buy 
a subscription. Thus, most news sites still depend on indirect revenue by advertisements, 
which becomes more problematic with the increasing spread of ad-blockers. These 
problems could be eliminated with the diffusion of a bitcoin-enabled infrastructure 
for frictionless micropayments. Similarly, a bitcoin-enabled IoT device, e.g. for 
automatic irrigation of farms, could pay a weather service API in return for accurate 
weather prediction data without the need for signup. Moreover, the device could 
search automatically for the cheapest (and best in terms of reputation) weather 
service API on the marketplace.

The 21 platform is a mixture of a centralized and a decentralized model. As of 
today, the mining power is bound to 21, and returns get allocated to a wallet owned 
by 21. However, it is announced that this will change in the future. The marketplace 
for digital services on the other hand is in principle decentralized and trades 
can be conducted peer-to-peer. Currently, 21 generates revenue by private mining 
operations and by sales of the bitcoin computer. In the future, however, all kinds 
of interesting revenue models are imaginable: selling and licensing of mining chips, 
revenue sharing of embedded mining operations or tiny transaction fees for off-chain 
transactions to name just a few. 

In conclusion, 21 could change how resources in the Internet are paid for, and 
thereby also contribute to making new resources available, which have not been 
available yet because of missing incentives

\section{Factom (Records Management)}
\label{sec:ecofactom}

Factom\footnote{\url{http://factom.org/}} is an open source software project that provides businesses with 
the ability to prove data integrity and to create verifiable and immutable audit 
trails. 

While data integrity could be achieved by directly adding a hash of the data to 
a bitcoin transaction and thereby time-stamp the data on the bitcoin blockchain, 
this method does not scale. On one hand, this is because of inherent scalability 
issues of bitcoin, and on the other hand because of transaction fees. Therefore, 
Factom consists of a peer-to-peer network that is independent of the bitcoin network. 
Customers of Factom generate hashes of their data and send them for recordkeeping 
to the Factom network. There, all hashes are compressed to a single hash by building 
a Merkle tree and taking the root of the tree. This single hash 
value is then stored in the bitcoin blockchain. This provably time-stamps all individual 
records without having to write all records individually into the bitcoin blockchain. 
The network maintains its own cryptocurrency, \emph{factoids}, which is used to incentivize 
participants of the peer-to-peer network to provide their resources. A factoid 
can be transformed into entry credits, which can be used to submit new records. 
The price of a factoid depends on the market value, but the price of an entry credit 
is fixed to 1/10\textsuperscript{th} of a cent. 

In summary, Factom provides a decentralized platform for data provenance with a 
permanent, time-stamped record of an unforgeable reference to the data anchored 
in the blockchain. This offers an efficient and cheap alternative for businesses, 
institutions and governments to have a proof of existence, proof of process or 
proof of audit for their data. Their first publicly announced project is using 
Factom for an official land title registry in partnership with the government of 
Honduras \parencite{Chavez2015}. In fact, Factom is an interesting option for 
governments in developing countries. They often face mismanagement and corruption, 
but cannot afford or enforce infrastructure and processes to guarantee compliance 
of their administration. Moreover, the Factom solution offers also an opportunity 
for small businesses/start-ups to have more auditing and to prove compliance with 
regulations without hiring expensive professional companies for that.

\section{Onename (Identity)}
\label{sec:ecoonename}

Onename\footnote{\url{https://onename.com/}} allows registering identities on the Bitcoin blockchain. This blockchain 
identity can be connected to various online identities like Facebook, Twitter, 
PGP keys and a Bitcoin address. Thus, it provides a probabilistic identity which 
reliability grows with the number of verified connected accounts. Onename is based 
on an open source overlay protocol called Blockchain ID. Everyone is able to register 
his identity without having to rely on Onename services. Blockchain identities 
are independent of the company Onename, are referenced on the Bitcoin blockchain 
and are therefore owned by the holder of the respective private key.

Blockchain IDs will allow signing up on third party websites comparable to Facebook 
Login or Google Sign-In. There is no need for passwords since authentication is 
done using digital signatures. Blockchain ID is also used as an identity provider 
for OpenBazaar. 

Moreover, it is possible to add additional namespaces. In this sense Blockchain 
IDs could represent not only humans, but also machines. Thus, Blockchain IDs might 
be the basis of a decentralized \ac{DNS} system or an IoT registry. 

Essentially, the technology allows individuals to own their online identities, 
rather than being dependent central institutions. Holding, i.e. registering and 
prolonging, a Blockchain ID requires fees that directly support the Bitcoin ecosystem. 
One part of the fee is a typical Bitcoin transaction fee that will be collected 
by a miner. The other part of the fee is particular to the protocol and leads to 
\emph{burning} of bitcoins. Since the number of bitcoins is constrained, the elimination 
of bitcoins theoretically increases the value of existing bitcoins. While it might 
seem that our current identities are free, we actually pay with our personal data. 
Every time we use Facebook to login into a third party website, we give away more 
information about us. Identity is also subject to network effects. Institutions 
will only start to accept Blockchain IDs if there are enough people using them 
and demand acceptance. 

Onename basically follows the same strategy as OB1. The initial focus is on developing 
open source protocols and advocating their adoption, instead of having a revenue 
model in place.

\newpage

\chapter{Implementing Smart Property}
\label{appendix:smartproperty}

The prototype in Section \ref{sec:economicobjects} implemented only active capabilities of economic devices. Thus, this appendix provides supplementary implementations of passive economic capabilities based on the smart property concept. The following concepts are implemented based on Bitcoin and based on Ethereum:

\begin{itemize}
	\item Low-trust atomic trades
	\item Low-trust renting
	\item Liquid property (using smart property as collateral for loans)
\end{itemize}

\section{Low-trust Atomic Trades}

Smart property can be sold via the Internet in an atomic process without third party escrow. Transfer of ownership\footnote{We do not necessarily mean legal rights of ownership, but the ability to control the property} and transfer of money happen at once.

\subsection{Implementation based on Bitcoin}

The simplest way to achieve atomicity in Bitcoin is by executing both parts of the trade in a single transaction. In Fig. \ref{fig:smartproperty} an atomic trade protocol based on \cite{smartproperty2011} is shown. The protocol may either be initiated by the seller (A) or by the buyer (B). We assume A initiates the protocol and sends the price, a Bitcoin address where she would like to receive the payment, and a reference to the \ac{UTXO} that represents the smart property (ownership output) to B. B creates a new ownership key pair and prepares a transaction that sends the price to A and transfers the ownership to the newly created ownership key. B signs the transaction and sends it to A. To accept the offer A signs the ownership input and broadcasts the transaction to the Bitcoin network. 

After a number of confirmations, B can provide a \ac{SPV} proof to the smart property. Therefore, B provides a selection of subsequent block headers entailing the header of the block the transaction is in, plus its merkle proof. The smart property is able to reason about the amount of proof-of-work that was spent to generate the blocks. In case of doubt, the smart property can ask for more block headers.

\begin{figure}[!t]
    \centering
    \includegraphics[width=\linewidth]{./externalized/smartproperty}
    \caption{Protocol of atomic trade of a smart property on the Bitcoin blockchain. The smart property does not need to interact with the Bitcoin network itself.}
    \label{fig:smartproperty}
  \end{figure}

In this protocol the smart property is not part of the Bitcoin network, but is able to verify \ac{SPV} proofs provided by a potentially untrusted party. 

\subsection{Implementation based on Ethereum}

The following contract illustrates a contract representing a tradable smart property in Ethereum. The original issuer of the contract is identified as the owner. In practice this would be the manufacturer. Ethereum does not support collaborative transactions, i.e. transactions with operations authorized by different entities. Therefore, selling has to be implemented in a two-stage process. In the contract below, we do this by implementing a \emph{sell function} which can be only called by the current owner (through the \emph{onlyOwner modifier}). The \emph{sell function} has two parameters: the price, and a buyer address. The \emph{buyer parameter} allows to explicitly state a seller. Otherwise everyone who would pay the price using the \emph{buy function} would be able to buy the smart property.

\newpage
\begin{lstlisting}[breaklines,basicstyle=\tiny]
contract SmartProperty {
    
    address owner;
    address buyer;
    bool onlyBuyerFlag;
    bool isOnSale;
    uint price;
    
    modifier onlyOwner() {
        if (msg.sender != owner) throw;
    }
    
    modifier onlyBuyer(bool flag) {
        if (flag && msg.sender != buyer) throw;
    }
    
    function SmartProperty() {
        owner = msg.sender;
    }
    
    function sell(uint _price, address _buyer) onlyOwner() {
        isOnSale = true;
        if (_buyer != 0) {
            onlyBuyerFlag = true;
            buyer = _buyer;
        } else {
            onlyBuyerFlag = false;
        }
        price = _price;
    }
    
    function stopSale() onlyOwner() {
        isOnSale = false;
    }
        
    function buy() onlyBuyer(onlyBuyerFlag) {
        if (isOnSale) {
            if (msg.value == price) {
                isOnSale = false;
                owner = msg.sender;
                owner.send(msg.value);
                
            } else {
                msg.sender.send(msg.value);
            }
        } else {
            msg.sender.send(msg.value);
        }
    }
}
\end{lstlisting}


\section{Low-trust renting}

Time-restricted transfer of ownership with adjustable counterparty risk. Smart property can be the basis for a peer-to-peer sharing\footnote{In the sense of Uber and AirBnB.} ecosystem. 

\subsection{Implementation based on Bitcoin}

The idea is to combine shared ownership and the unidirectional payment channel. Therefore, the parties create a transaction that creates a multi-signature ownership output for A (owner) and B (renter), and a 2-of-2 multi-signature output where B deposits some amount. Furthermore, both parties create a timelocked refund transaction that allocates the ownership output back to A, and the deposit back to B. The timelock should cover the maximal renting period, and the deposit should cover the renting price for that period. B can now pay in small increments with off-chain payment transactions that entail an additional ownership output that assign the car ownership back to A. Because of Bitcoin's current malleability issue it is important that the order of providing signatures is such that only A is able to broadcast the funding transaction. Furthermore, A needs also a (tiny) payment transaction from B before broadcasting the funding transaction. Otherwise B could just use the smart property without paying for the entire maximal renting period, since A would not have a transaction that is immediately valid. Figure \ref{fig:smartproperty_renting} illustrates the protocol in more detail.

\begin{figure}[!t]
    \centering
    \includegraphics[width=\linewidth]{./externalized/smartproperty_renting}
    \caption{Protocol for trust-minimized renting of smart property using a Bitcoin payment channel.}
    \label{fig:smartproperty_renting}
  \end{figure}

\subsection{Implementation based on Ethereum}


The following contract implements low-trust renting of smart property on Ethereum. Note that the \emph{Rentable Property} contracts inherits properties and functions from the \emph{SmartProperty} contract. When deploying the contract the owner sets a deposit a renter has to provide and a time based renting price. The main functions are a \emph{rent} and a \emph{returnPropery} function. The \emph{rent} function allows an arbitrary account or contract to rent the smart property by providing a deposit. The deposit is held in the contract, and can only be released by the rules of the contract. Neither the owner, nor the renter have control over the deposit. The contract keeps also track of the block in which the rent function was executed. This \emph{startBlock} serves as the begin of the renting period as measured in time of blocks. The Ethereum network has an average block generation rate of approximately 14 s. 
The software running locally on the smart property has now to be notified that it must obey orders signed by the renter for a maximal period defined by the price and the deposit. 
The renter can later call the \emph{returnProperty} function which calculates the price for the renting period and distributes the deposit accordingly. If the renter does not return the property in time her access/control rights expire. 

\begin{lstlisting}[breaklines,basicstyle=\tiny]
contract RentableProperty is SmartProperty {
    
    address renter;
    uint startBlock;
    uint pricePerBlock;
    uint deposit;
    bool isRentable = true;
    
    modifier onlyRenter() {
        if (msg.sender != renter) throw;
    } 
    
    function RentableProperty(uint _pricePerBlock, uint _deposit) {
        owner = msg.sender;
        pricePerBlock = _pricePerBlock;
        deposit = _deposit;
    }
    
    function notRentable() onlyOwner() {
        isRentable = false;
    }
    
    function rentProperty() {
        if (isRentable) {
            if (msg.value >= deposit) {
                renter = msg.sender;
                startBlock = block.number;
                isRentable = false;
                if (msg.value > deposit) {
                    msg.sender.send(msg.value - deposit);
                }
            } else throw;
        } else throw;
    }
    
    function returnProperty() onlyRenter {
        renter = 0;
        if (deposit > price*(block.number-startBlock)) {
            owner.send(price*(block.number-startBlock));
            msg.sender.send(deposit-price*(block.number-startBlock));
        } else {
            owner.send(deposit);
        }
        isRentable = true;
    }
}
\end{lstlisting}

\section{Liquid property}

Smart property can be used as collateral for loans. The underlying principle is that property ownership is transferred to the lender if repayment terms are not met. Thereby the loan gets securitized by the smart property. Because of the global permissionless nature of cryptocurrencies, a global market for loans on individual smart properties can emerge which lowers the cost of loans. Moreover, since transactions are public, a borrower is able to prove timely payments of earlier loans. 

\subsection{Implementation based on Bitcoin}

We will implement the following contract. The owner of a smart property (B) wants a loan of size $L$ and provides the property as security for a creditor (A). If the owner (and debtor) does not repay the loan (plus interest) until time $T$, ownership of the property will be transferred to the creditor.

After A and B agreed on terms, B creates the \emph{loan transaction} spending his ownership output and creating a timelocked 2-of-2 multi-signature ownership output that can be redeemed either collaboratively by A and B, or by A alone after time $T$. Furthermore, B adds an output that credits him with loan $L$, leaving the input, providing the loan, for A to add. 
\begin{figure}
\begin{lstlisting}
OP_IF 
    2 <pubKeyA><pubKeyB> 2
    OP_CHECKMULTISIG
OP_ELSE
    <T> OP_CHECKLOCKTIMEVERIFY OP_DROP
OP_END    
\end{lstlisting}
\caption{PubScript of timelocked 2-of-2 multi-signature ownership output.}
\end{figure}

B sends the partial transaction to A, who prepares the \emph{settlement transaction}. The settlement transaction reassigns the ownership back to B, and credits A with the loan plus interest. A completes the loan transaction and partially signs the settlement transaction, and sends both transactions back to B. B can then broadcast the loan transaction to the Bitcoin network. 
If B can provide an input to the settlement transaction covering $L+\Delta$ before time $T$, B can complete the settlement transaction and regain sole ownership of the property. However, if the settlement transaction does not enter the blockchain before $T$, then A is able to claim sole ownership.

Another Bitcoin-based protocol for smart property as a collateral for loans is described in \cite{smartproperty2011}. However, the protocol has the problem that a creditor has the ability to resell the property immediately without giving the debtor a chance to repay the loan. In the protocol described above this is prevented by use of the multi-signature. 

\subsection{Implementation based on Ethereum}

A simple implementation of property liquidification defines a \texttt{loan} and a latest \texttt{payDay}\footnote{Here represented as a Ethereum block number.}. The owner of the property deploys the \emph{LiquidProperty} contract, and a lender can provide a loan using the \emph{giveLoan} function. The loan is immediately credited to the debtor who can use the \emph{pay} function to pay back the loan. Either lender or debtor can call the \emph{enforce} function to enforce the contract. If the loan is paid, then the contract state return to the initial state, if the loan is not paid and it is later than \texttt{payDay}, then the ownership gets transferred to the lender.

\begin{lstlisting}[breaklines,basicstyle=\tiny]
contract LiquidProperty is SmartProperty {
    
    address lender;
    uint loan;
    uint payDay;
    uint paid = 0;
    
    bool isLiquid = false;
    
    function LiquidProperty(uint _loan, uint _payDay) {
        owner = msg.sender;
        loan = _loan;
        payDay = _payDay;
    }
    
    function giveLoan() {
        if (msg.value < loan || isLiquid) throw;
        isLiquid = true;
        lender = msg.sender;
        owner.send(msg.value);
    }
    
    function pay() {
        if (!isLiquid) throw;
        paid = paid + msg.value;
        lender.send(msg.value);
    }
    
    function enforce() {
        if (!isLiquid) throw;
        if (paid >= loan) {
            isLiquid = false;
        } else if (block.number > payDay) {
            owner = lender;
            isLiquid = false;
        }
    }
}
\end{lstlisting}