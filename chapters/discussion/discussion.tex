\chapter{Discussion and Conclusion}
\label{sec:discussion}

\section{Key Findings and Lessons Learned}


General discussion on evolution of ecosystem, companies, projects, cryptocurrencies.
How is programming with cryptocurrencies


\subsection{Privacy, Aggregation, ...}

Implications on IoT business models. New wave of decentralization. Still need for centralized services

\begin{itemize}
	\item Be sure about payments!
	\item Innovation through experimentation. No governmental monopoly on money
	\item Ethereum bitcoin network effects etc.
\end{itemize}

\section{Discussion}

\section{Implications}

Research Implications


Practical Implications

\section{Outlook and Future Work}

\section{Conclusions}


\newpage
\appendix
\chapter{Bitcoin Ecosystem: Representative Cases} \label{App:AppendixA}
\label{appendix:cases}

In the following, we investigate the challengers beyond financial services in more 
detail. Therefore, we present six comprehensive case studies (marked as bold in Table 
\ref{tbl:ecosystem}).

\section{Filament (Internet of Things)}
\label{sec:ecofilament}

Filament\footnote{https://filament.com/} provides wireless sensor networks for the (industrial) IoT, e.g. 
for smart cities or smart agriculture use case. Most IoT platform providers follow 
a centralized approach by connecting all devices to their respective cloud-infrastructure. 
This has the major disadvantage that devices depend on a central infrastructure 
in order to operate. Moreover, it can be argued that this approach cannot keep 
up technically and economically with the increasing number connected devices.

Filament is one of the first companies that develop a fully decentralized IoT infrastructure, 
which encompasses three blockchain-related aspects: (1) Each device is registered 
on the blockchain providing a verifiable and immutable identity. This enables discovering 
of and authenticating with other devices/services without the need of a dedicated 
backend infrastructure. Therefore, devices are technically autonomous and are able 
to operate independently of Filament. (2) Each device is governed by a \emph{smart} 
contract, which manages agreements of device control/ownership, data access and 
financial agreements concerning the device. Ownership can be transferred permanently 
or temporarily by a simple transaction on the blockchain. Filament implements the 
financial agreements as a Product-as-a-Service, which means that the owner gets 
paid directly for the ongoing use of the device. (3) Furthermore, each device is 
able to transfer value in form of bitcoins to other devices in order to get access 
to data or request some service.

As described, devices can be operated and governed by using only the blockchain 
as a backend and therefore without any technical dependence on the platform creator 
(Filament) or other third parties. This might bare great benefits for customers 
needing to deploy large Industrial IoT applications with a lifetime of 5-10 years. 
Because they want to minimize the risk of a lock-in with a specific company. Moreover, 
according to Filament, customers prefer paying continuously on a real-time basis 
instead of an upfront investment, which can be solved efficiently by Bitcoin micropayments. 
Ownership is decoupled from usage and both are independent of the manufacturer 
or a platform provider.

Filament itself is a venture capital-backed company formerly known as Pinocc.io. 
They claim to have their first deployments with Fortune 500 companies in 2016. 
They will get paid for the ongoing use of their devices (by owning the smart contract). 
Moreover, they work on a licensing model, i.e. Customers can attach a module version 
to their own devices, which will give them all the described benefits of Filament 
in return to a small fraction of the payments for the ongoing use of the devices. 
However, since all protocols will be open and there is no dependence to Filament 
by design, other companies could use that and build their own hardware without 
Filament.

\section{Ascribe (Intellectual Property)}
\label{sec:ecoascribe}

Ascribe\footnote{https://www.ascribe.io/} aims to provide provenance of intellectual property. Digital work, 
like art, photos, and music, can be registered publicly on the bitcoin blockchain 
together with its accompanying terms and conditions. The technological basis is 
the spool protocol\footnote{https://github.com/ascribe/spool}, an overlay protocol that uses 
Bitcoin transactions to represent unforgeable ownership transfers and licensing 
agreements for digital work. Thus, authenticity of ownership and usage rights can 
always be proven. Ascribe focuses on digital art in particular. Although art is 
in principle copyrighted by the time of creation, it is currently cumbersome and 
expensive to officially register work and prove ownership (see e.g. http://copyright.gov/docs/fees.html). 
Therefore, hardly any digital art gets registered. Ascribe aims to change the status 
quo by providing a virtually free and automatable registration process. Moreover, 
the unique representation of digital work based on cryptography and the blockchain 
is the basis for the creation of a secondary market for digital work. For example 
the spool protocol enables creating limited editions of digital work. So far, this 
has not been possible without relying on some central institution.

The spool protocol is open source and can be used by anybody. In principle, the 
only costs are bitcoin transaction fees. However, Ascribe wraps the protocol in 
convenient web services and provides tools adapted to particular customer groups, 
e.g. individual creators, museums and marketplaces. Ascribe's revenue model is 
then based on a share of the rentals and sales of registered digital work that 
are facilitated by their APIs.

The core innovation is the application of the bitcoin blockchain to provide commoditized 
provenance of intellectual property. Provenance is demonstrated by relying on the 
immutability of the bitcoin blockchain, instead of an authority. 

\section{OpenBazaar (E-Commerce Marketplace)}
\label{sec:ecobazaar}

OpenBazaar\footnote{https://www.openbazaar.org/} is an open source project consisting of a protocol and a reference 
implementation that enables a decentralized peer-to-peer e-commerce marketplace. 
In comparison to traditional e-commerce market places like eBay and Amazon, there 
is no central server or authority that is running the market place. Thus, there 
is no middleman who is able to charge fees or to restrict offered products and 
services. Everyone with Internet access is able to set up a shop by running a network 
client. Payments are facilitated using Bitcoin transactions. Therefore, no payment 
provider or banking account is needed. This lowers payment transaction fees and 
increases the global reach. Besides sellers and buyers there are notaries and arbiters 
for dispute resolution participating in the marketplace. Those latter participants 
are involved by using bitcoin multi-signature transactions. Thus, OpenBazaar unbundles 
the functions of traditional marketplaces. Trades on the OpenBazaar network are 
based on Ricardian Contracts \parencite{1319505}, i.e. an electronic document that defines 
the terms of a trade such that it is readable by computers and humans, and is cryptographically 
signed. Apart from selling physical and digital products, OpenBazaar can also be 
used to trade speculative contracts, which can be readily represented by Ricardian 
Contracts. 

The main value proposition of OpenBazaar to sellers as well as to buyers is the 
elimination of fees and restrictions. Since marketplaces are subject to network 
effects most users will most probably not switch immediately from traditional marketplaces 
to OpenBazaar. However, OpenBazaar could set foot in under-served niche markets. 
Examples could be digital goods, developing countries with limited access to traditional 
payment services, and prohibited goods. 

OpenBazaar itself is not a company, but its main developers founded the venture 
capital-backed company OB1. Their current focus is on developing the OpenBazaar 
protocol and its reference implementation. As outlined above, OB1 is not able to 
profit directly from transactions on the marketplace in the way traditional revenue 
mechanics on centralized marketplaces work. 

\section{21 (Digital Micro Commerce Marketplace)}
\label{sec:eco21}

At first sight, the categorization of 21 Inc.\footnote{https://www.21.co} as a challenger seems odd, 
since 21 is an infrastructure and platform provider for the bitcoin ecosystem. 
Indeed, it is often categorized as a mining company. However, we argue that 21 
is better classified as a marketplace for digital micro services, which has the 
potential to challenge traditional Internet business models.

With a funding of \$121M, 21 supersedes every other start up in the bitcoin ecosystem. 
They have developed an embeddable ASIC\footnote{ Application-specific integrated 
circuit. A hardware chip customized for a particular task in contrast to a general 
CPU. In this case task is Bitcoin mining, which involves the extensive execution 
of SHA265 operations.} mining chip that they are using in their own mining operations, 
but which is also embeddable into arbitrary connected devices. In November 2015 
they released their first product, the 21 Bitcoin computer. Essentially the 21 
Bitcoin computer is a full-stack development platform to build bitcoin-payable 
digital services, which can be published and discovered on 21's digital marketplace. 
Individual service consumptions, like an API call, can be billed at as little as 
1 satoshi\footnote{Satoshi is the smallest denominator of a bitcoin. 100,000,000 
satoshi correspond to 1 bitcoin. Thus, 1 satoshi is currently worth approximately 
USD 0.000003.}. The embedded mining chip, which is currently coupled to a mining 
pool operated by 21, supplies the device with 
a continuous stream of satoshis.

21 aims to embed their chips into any connected device (e.g. smart phones) to establish 
bitcoin as a system resource like CPU, bandwidth or disk space, but for the purpose 
of buying and selling digital goods and services \parencite{Balaji2015}. It is crucial 
to understand that it does not make sense to sell the small amounts of mined bitcoins 
for Fiat currency on an exchange. Instead, the idea is to supply every device with 
a continuous stream of bitcoin from the point of commissioning on so that it can 
directly operate on the marketplace. 

Having such an infrastructure of bitcoin-enabled devices in place at scale, could 
offer compelling new opportunities and even disrupt traditional business models 
of the Internet. For example, it has been difficult for news sites to directly 
monetize their content on the Internet. It is still tedious for users who want 
to read just one article to signup, enter their credit card information and buy 
a subscription. Thus, most news sites still depend on indirect revenue by advertisements, 
which becomes more problematic with the increasing spread of ad-blockers. These 
problems could be eliminated with the diffusion of a bitcoin-enabled infrastructure 
for frictionless micropayments. Similarly, a bitcoin-enabled IoT device, e.g. for 
automatic irrigation of farms, could pay a weather service API in return for accurate 
weather prediction data without the need for signup. Moreover, the device could 
search automatically for the cheapest (and best in terms of reputation) weather 
service API on the marketplace.

The 21 platform is a mixture of a centralized and a decentralized model. As of 
today, the mining power is bound to 21, and returns get allocated to a wallet owned 
by 21. However, it is announced that this will change in the future. The marketplace 
for digital services on the other hand is in principle decentralized and trades 
can be conducted peer-to-peer. Currently, 21 generates revenue by private mining 
operations and by sales of the bitcoin computer. In the future, however, all kinds 
of interesting revenue models are imaginable: selling and licensing of mining chips, 
revenue sharing of embedded mining operations or tiny transaction fees for off-chain 
transactions to name just a few. 

In conclusion, 21 could change how resources in the Internet are paid for, and 
thereby also contribute to making new resources available, which have not been 
available yet because of missing incentives

\section{Factom (Records Management)}
\label{sec:ecofactom}

Factom\footnote{http://factom.org/} is an open source software project that provides businesses with 
the ability to prove data integrity and to create verifiable and immutable audit 
trails. 

While data integrity could be achieved by directly adding a hash of the data to 
a bitcoin transaction and thereby time-stamp the data on the bitcoin blockchain, 
this method does not scale. On one hand, this is because of inherent scalability 
issues of bitcoin, and on the other hand because of transaction fees. Therefore, 
Factom consists of a peer-to-peer network that is independent of the bitcoin network. 
Customers of Factom generate hashes of their data and send them for recordkeeping 
to the Factom network. There, all hashes are compressed to a single hash by building 
a Merkle tree and taking the root of the tree. This single hash 
value is then stored in the bitcoin blockchain. This provably time-stamps all individual 
records without having to write all records individually into the bitcoin blockchain. 
The network maintains its own crypto-currency, \emph{factoids}, which is used to incentivize 
participants of the peer-to-peer network to provide their resources. A factoid 
can be transformed into entry credits, which can be used to submit new records. 
The price of a factoid depends on the market value, but the price of an entry credit 
is fixed to 1/10\textsuperscript{th} of a cent. 

In summary, Factom provides a decentralized platform for data provenance with a 
permanent, time-stamped record of an unforgeable reference to the data anchored 
in the blockchain. This offers an efficient and cheap alternative for businesses, 
institutions and governments to have a proof of existence, proof of process or 
proof of audit for their data. Their first publicly announced project is using 
Factom for an official land title registry in partnership with the government of 
Honduras \parencite{Chavez2015}. In fact, Factom is an interesting option for 
governments in developing countries. They often face mismanagement and corruption, 
but cannot afford or enforce infrastructure and processes to guarantee compliance 
of their administration. Moreover, the Factom solution offers also an opportunity 
for small businesses/start-ups to have more auditing and to prove compliance with 
regulations without hiring expensive professional companies for that.

\section{Onename (Identity)}
\label{sec:ecoonename}

Onename\footnote{https://onename.com/} allows registering identities on the Bitcoin blockchain. This blockchain 
identity can be connected to various online identities like Facebook, Twitter, 
PGP keys and a Bitcoin address. Thus, it provides a probabilistic identity which 
reliability grows with the number of verified connected accounts. Onename is based 
on an open source overlay protocol called Blockchain ID. Everyone is able to register 
his identity without having to rely on Onename services. Blockchain identities 
are independent of the company Onename, are referenced on the Bitcoin blockchain 
and are therefore owned by the holder of the respective private key.

Blockchain IDs will allow signing up on third party websites comparable to Facebook 
Login or Google Sign-In. There is no need for passwords since authentication is 
done using digital signatures. Blockchain ID is also used as an identity provider 
for OpenBazaar. 

Moreover, it is possible to add additional namespaces. In this sense Blockchain 
IDs could represent not only humans, but also machines. Thus, Blockchain IDs might 
be the basis of a decentralized DNS system or an IoT registry. 

Essentially, the technology allows individuals to own their online identities, 
rather than being dependent central institutions. Holding, i.e. registering and 
prolonging, a Blockchain ID requires fees that directly support the Bitcoin ecosystem. 
One part of the fee is a typical Bitcoin transaction fee that will be collected 
by a miner. The other part of the fee is particular to the protocol and leads to 
\emph{burning} of bitcoins. Since the number of bitcoins is constrained, the elimination 
of bitcoins theoretically increases the value of existing bitcoins. While it might 
seem that our current identities are free, we actually pay with our personal data. 
Every time we use Facebook to login into a third party website, we give away more 
information about us. Identity is also subject to network effects. Institutions 
will only start to accept Blockchain IDs if there are enough people using them 
and demand acceptance. 

Onename basically follows the same strategy as OB1. The initial focus is on developing 
open source protocols and advocating their adoption, instead of having a revenue 
model in place.

\newpage



\begin{itemize}
	\item Was aus Kapitel 4 runter?
	\item Interactive Refund Transactions
	\item Smart Contract
\end{itemize}
