\chapter{Liquidification of property}



\section{Motivation}

Hundreds of million people globally lack access to energy. Grids are unreliable and last mile connection costs can be inhibitive. Off-grid energy technologies of various sizes are being made available by private companies as well non-profit organizations. In terms of scale, the most important technology are Solar Home Systems (SHS)

A SHS is composed of a small solar panel, a battery, LED lights, and USB charging. The capacity of the system ranges typically between 10 and 250 Watt. Larger systems typically comprise fans and TVs. Small systems can be installed by the user but larger systems need a technician.

World market production of solar panels focuses on a module size of 250 Watt. Hence the price per Watt is lowest for 250 Watt modules. However, 250 Watt exceed the typical demand of a rural off-grid household. Thus micro-utility (c.f. Grameen Shakti) and peer-to-peer microgrids (c.f. uLink) models are appealing. In these models a local entrepreneur shares the generated electricity with her neighbors. Therefore, energy consumption should be able to be metered and the local entrepreneur needs to collect payments.

People that are in need of SHSs are typically at the bottom of the pyramid. They lack the capital to buy a SHS and have no access to traditional financing. Innovative financing schemes that are based on mobile payments and the ability to remote disable the system have been developed to bring those systems to more poor households. These schemes can be divided in two main categories: Pay-for-service and lease-to-own. In the pay-for-service model the SHS manufacturer is the owner of the system and the user pays has to pay fee for its usage. If the user does not pay electricity access is turned off and after some time the system gets taken back. In the lease-to-own model the user also has to pay for the usage but she also pays a principal so that at some point the ownership gets transferred to her and the system gets permanently unlocked. 

A prominent example for the lease-to-own model is M-Kopa which operates in East Africa. M-Kopa offers also that the user may switch back to the lease-to-own model after the user has fully paid the system in order to get financing for some other product. Notably, the user does not get a loan to buy a product somewhere but she gets a product that is distributed by M-Kopa. 

Companies like M-Kopa need large capital in order to finance the distribution of their systems. Besides debt and equity financing some companies have securitized their implicit loans to users. This practice is however only viable for the larger companies that already have reached some scale. 



\cite{gsma2014} gives a good overview of energy access in sub-saharan africa. Discusses solar home systems, mini-grids, and on-grid systems. Stresses the role of cellular connectivity and pay-as-you-go models. According to the report around 40\% of on-grid energy gets stolen. Furthermore, they provide a graph on the evolution of electrification rate and the rate for cellular coverage.

\cite{DBLP:journals/corr/QadirSWC16} discuss the approach of resource pooling for wireless networks in the developing world. This could be used if providing connectivity to villages is implemented as an additional service by the DAO.


\chapter{Smart property as a joint-stock business entity}

\section{Introduction}

Joint-stock companies were invented to share risk and rewards between shareholders in order to finance large-scale commercial endeavors. One of the first joint-stock companies were the Dutch and the British East India Companies. Large amounts of capital were necessary to finance the construction of ships, hire crews, and establish colonies to initiate trading. In short, joint-stock companies enabled corporations on an unprecedented scale. Shares of stock represent  ownership of the business entity. Bearers of these shares are entitled to participate in specific decisions by means of voting, and are entitled to receive dividends. 

However, joint-stock companies are heavily regulated and require complex and expensive auditing processes in order to protect shareholders from fraud and opaque mismanagement. This leads to high initial and operational costs which have to be weight against the benefits

Cryptocurrencies and automated contract enforcement allows to apply similar concepts to endeavors of much smaller scale. In this chapter we discuss the application 

\section{Motivation}

Starting as an Uber driver in Mexico City or as a solar micro-utility provider in rural India does require relatively little capital in comparison to traditional joint-stock companies. 
Nevertheless, financing a car or a solar home system can be challenging. In particular in countries with an underdeveloped financial services infrastructure and limited legal enforceability. This high risk for investors translated into high interest rates for local micro entrepreneurs. 

Innovative start ups like M-Kopa, ... already leverage Internet of Things technology and mobile payments in order to decrease default rates and thus enable lower interest rates. Solar Home Systems are offered in pay-as-you-go and lease-to-own models, requiring low upfront investments. The systems are equipped with SIM cards, in order to allow remote monitoring, and remote disablement. If a customer does not comply to his/her payment plan, the system gets deactivated until the payment is made. 

In this model the risk is centered on the solar home system provider which has to attract debt and equity financing in order to advance funding. Therefore, spreading risk from the top.

Alternatively, we can spread the risk from the bottom up, by interpreting every smart home system as a micro joint-stock business entity.



 
% Notes for further discussion
% Issuance policy for additional revenue-based shares to keep the operator incentivized to keep a high revenue stream.
% Locking operator shares at least  for a specific time. Such that the operator keeps skin in the game.
% On-demand Liquidation
% In principle it is the transformation of a collateral-based loan to a cashflow-based loan, isn't it? What are the implications?
% Trade of between enforcement by law/legal system and enforcement by architecture/code
% What happens in case of defects/maintenance?


\section{General Model}

In the general model we consider a smart property that is able to provide some form of service. The smart object may [verweigern] the service. In the case of the solar home system the service is the provision of energy. In the case of a display or screen it may be the [Darstellung] of content. In the following we will present the model by discussing the [beteiligten] stakeholders, their roles, and the system in which they interact.

\subsection{Stakeholders}

\subsubsection{Manufacturer}

The manufacturer is the producer and initial owner of the smart property.

\subsubsection{Operator}

The operator has the desire to own the smart property, but lacks the necessary capital. The operator will be incentivized to maintain the smart property.

\subsubsection{Investors}

Investors provide the capital to deploy the smart property. In return, they receive shares that entitle them collect dividends. Furthermore, they might be able to interfere witch decisions of the operator in a voting process.

\subsubsection{Customers}

Customers are individuals or software agents that pay for the service provided by the smart property.



\subsection{Tokens as shares}

\subsection{Crowdsale}

\subsection{Real-time dividends}


\section{Implementation}

\subsection{Bitcoin}

\subsection{Ethereum}


\subsubsection{Subleties}

\section{Case: Public Display}

\section{Further Applications}

\subsection{Autonomous cars}

\subsection{Renewable energy generation}

\subsection{}

\section{Discussion}
