
\chapter{Machine-to-Machine Micropayments}

\section{Motivation}

Computing capabilities and connectivity are on the way to finally meet the expectations articulated in the 90s of the last century by Mark Weiser and become ubiquitous. Ever more devices become connected. From cars over thermostats and lightbulbs to literally everything. While we are still aware of the data plan of our smartphone and painstakingly connect our gadgets to our wifi or pair them with our phones, this might well be only a transient appearance. Some devices, like the Kindle eBook reader come already with a data plan valid in more than 100 countries. Free and open Low Power Wide Area Network (LPWAN) networks enable the seamless connectivity of sensors around entire cities. In other words a new distributed computing infrastructure is being deployed. Of course not every device will be a powerful general purpose computing device. Form factor and power consumption are restricting. However, many will. This is because of the unprecedented scale of the smartphone supply chain and the accompanying economies of scale. This computing infrastructure will be mostly dormant and underutilized at first. Just as your tablet, your router and your smart TV today. In the same way as Uber and AirBnB already maximize utilization of macroscopic distributed private physical assets, this trend will continue to assets like data, storage, bandwidth, and computing power. 
In this realm humans do not want to make decisions, because the cognitive load surpasses the value. Therefore machines have to transact autonomously, and in many situations the exchange of value in form of digital currency or game-theoretic protocols with economic incentives can replace a pre-set or user-induced trust relationship between devices. 

\section{Contribution and Outline}

We state necessary requirements for a machine-to-machine micropayment system, and present arguments for a Bitcoin-based system. We survey bitcoin-based micropayment schemes. Thereby we present a scheme for mediated unidirectional payment channels in detail, since the scheme has not been presented before.
Finally, we evaluate the micropayment schemes based on the requirements.


\section{Requirements for M2M Payments}

Exchange of measurement data, temporary file storage or outsourcing computations are services with microscopic, yet non-zero, value. Ascribing value to arbitrary services and requiring payment further inhibits denial service attacks common in open networks. Efficient micropayment schemes are essential. Furthermore, machines have to be able to initiate and verify payments. Payments should not be reversible since delivery might have to be instantaneous. Machines live in the Internet, and while latency might play a role, jurisdiction and local currency should not matter. 


\section{Micropayments with PayPal}

PayPal is one the main traditional online payment providers, and will act as an example why traditional systems are not suitable for M2M micropayments.

PayPal considers payments below \$10 as micropayments\footnote{https://www.paypal.com/us/webapps/mpp/merchant-fees}, and the fees for micropayments are 5\% + \$0.05 in June 2016. Hence, for payments below \$1 the fees are at least 10\%, and payments on the order of cents are not viable. Although PayPal provides payment APIs, those are focused on the merchant side. A merchant can use the API using OAuth to initiate an order. However, the user has accept the order via the browser and username/password-based authentication. Furthermore, \emph{Payments are delayed to help ensure a safe environment for both buyers and sellers.}

\section{Bitcoin transactions are expensive slow}


In case of Bitcoin, and most other cryptocurrencies, transaction fees can be specified by the creator of a transaction, and miners can individually decide if they are willing to include the respective transaction into a block. Hence, although it is still possible to create zero fee transactions in Bitcoin, most miners expect a non-zero fee. Since blocks currently have maximal size of 1 Megabyte, block space can be considered a scare resource. This effectively leads to a size-based, instead of the common value-based, market price for a transaction. By tracking for how many blocks a transaction of a particular size and a particular fee resides in the mempool until the transaction is included into a block, the minimal fee for a given priority requirement can be determined. Wallets can either do this on their own or rely on web services such as bitcoinfees.21.co. 
Currently, a transaction with a high chance to be included in the next block has to include a fee of approximately 60 satoshis per byte. A typical P2PKH transaction with one input and two outputs, one for the payment and one for change, has a size of approximately 250 bytes. Hence, the fee is around 15,000 satoshis, and thus almost \$0.1 at the current exchange rate of \$650 per BTC. 
Normal P2PKH transactions are thus not suitable for micropayments. Another obstacle for naive micropayments in Bitcoin is the UTXO-based architecture. UTXOs with values on the order of satoshis would lead to a drastic increase of the UTXO set, and therefore on the resource requirements of network nodes. Furthermore, spending a macroscopic amount by combining hundreds or thousands of microscopic UTXOs would lead to very large and expensive transactions. 

In addition, confirmation times on the order of tens of minutes are too long for many types of  machine-to-machine transactions.

\section{Bitcoin Micropayments}

\section{Off-Chain Micropayments}

Centralized third party services have been emerged that allow zero fee microtransactions between their users. Examples are Coinbase\footnote{https://www.coinbase.com}, 21\footnote{https://www.21.co}, and ChangeTip\footnote{https://www.changetip.com}. Thereby users deposit coins in addresses controlled by the third party service. The service then handles transactions between users internally, i.e. off-chain. However, users lose all control over their funds. The situation looks similar to using PayPal. Why does PayPal charge fees but services based on Bitcoin do not? The main reasons are: (1) PayPal accounts are usually funded by credit cards, and those transactions are susceptible to chargeback fraud. (2) Regulatory overhead. 

\section{Payment Channels}

\subsection{Unidirectional Payment Channels}
\label{sec:unipc}

The idea to use Bitcoin contracts to establish payment channels between two parties was introduced by Hearn and Spilman \cite{Hearn}. An unidirectional payment channel allows a single payer to pay a single payee repeatedly by sending signed transactions over some communication channel. Individual payments can be as low as 1 satoshi and the payee is able to accept payments instantly without double-spending risk. This is achieved by using the blockchain as a trustless escrow service in form of a timelocked multisignature output that is funded by the payer. A graphical representation and a description of a non-interactive implementation of the protocol is depicted in Figure \ref{fig:paymentchannels}.

\begin{figure}
\includegraphics[scale=.7]{./externalized/paymentchannel.pdf}
\caption{An unidirectional payment channel between a payer A and a payee B. To open a channel A creates and broadcasts a funding transaction with an output that can be spent only by providing signatures from both parties, i.e. $\sigma(A)$ and $\sigma(B)$ (multisignature). The value of this output is then shared between A and B in off-chain payment transactions. A creates and (partially) signs those payment transactions, and sends them to B via some private or public communication channel. While A has only partially signed payment transactions, B is able sign and broadcast any of those transactions to close the channel. If B is rational he will always broadcast the transaction that allocates the largest share to him. Therefore A can not decrease B's share in a subsequent payment transaction, and the payment channel is thus unidirectional. If B does not broadcast a payment transactions before $T_1$, A is able to recover the funds by broadcasting a refund transaction.}
\label{fig:paymentchannels}
\end{figure}

Unidirectional payment channels between two parties involve two on-chain Bitcoin transactions. One funding transaction and one transaction to close the channel. Simple unidirectional payment channels are useful for repeated, metered payments between a single payer and a single payee. Use cases are video streaming\footnote{https://streamium.io/} and WiFi access \cite{Siby2013}. 

\subsection{Unidirectional Mediated Payment Channels}

Mediated payment channels allow the routing of payments through intermediate payment channels by using a concept called \emph{hashlocks} or rather hash timelocked contracts (HTLC). We can express a HTLC between two parties in common words as follows: I pay you if you can provide the secret within a certain period. Assume Alice (A) wants to send a small payment to Bob (B), but Alice and Bob do not have an open payment channel. However, A has a channel with Charlie (C), and C has a channel with B. In principle A could pay C and tell him to pay B. But A has no way to enforce the payment from B to C if the payment channels are independent. We need a way to condition the individual payments on each other. This can be done by conditioning the payments on a common secret generated by the final recipient B. 

Figure \ref{fig:unimedpc} illustrates the flow of an individual payment over two payment channels. We assume that A and C, as well as C and B have already established payment channels, i.e. there exist shared multisignature outputs with some locktimes $T_0$ and $T_0^{'}$\footnote{The actual locktimes are not important. However, they have to be longer than the HTLC locktimes $T_1$ and $T_2$.}. In order to make it more concrete, we assume that the shared outputs have a value of 1 BTC each, and A wants pay B an amount of 0.1 BTC. Moreover, we assume that the channels are fresh, i.e. no payment transactions have been exchanged. First, the final recipient B creates a payment-specific random secret, computes hash(secret), and communicates hash(secret) to C and A. A creates a payment transaction that consumes the shared multisignature output and creates two outputs: (1) an output assigning 0.9 BTC to her, and (2) a HTLC output with a value of 0.1 BTC\footnote{We neglect necessary fees in order to provide a cleaner explanation.}. Figure \ref{fig:pubScriptHTLC} shows the pubScript of such a HTLC output as used in the payment transactions. A provides her signature for the transaction and sends it to C. C may now sign the transaction as well and broadcast it to the Bitcoin network. However, without the secret, C will not be able to claim the 0.1 BTC. Thus C will store the transaction and create another payment transaction addressing C with a HTLC output requiring the same secret. C signs the transaction and sends it to B. B could sign the transaction and broadcast it to the network. Since he knows the secret, he could claim the 0.1 BTC locked in the HTLC output. But he would have to do it before $T_2$. Otherwise B would be able to claim the output. B would claim the HTLC output by broadcasting a transaction that entails the secret. Thus, the secret would be public and C could use it to claim the HTLC output of the payment transaction from A. To ensure that C is always able to do that before A is able to reclaim the value, $T_1$ has to be sufficiently later than $T_2$. If timelocks are selected appropriately, a payment between A and B mediated by C using HTLCs is atomic, i.e. either both payments succeed or both payments fail. This means also that B can just communicate the secret privately to C and A, and they update their shares accordingly. Hence, A would create a transaction that replaces the HTLC output with an output granting C the 0.1 BTC. After signing and sending the transaction to C, C would do the same concerning the HTLC with B, and the payment is concluded without any on-chain transaction.

Mediated payment channels can be used in a hub and spoke architecture to connect a large number of buyers with a large number of sellers with only one payment channel per buyer/seller. Think of 100 of fitness trackers interested in buying air quality data from 1000 consumer air quality sensors in Bejing in order to suggest a healthy running track. There would have to be 100 000 p2p payment channels, but only 1100 if there were a central mediator (c.f. Figure \ref{fig:nmpaymentchannels} and \ref{fig:hubandspoke}). In a later chapter we will see an implementation of this concept in order to pay smartphones for providing sensor data.


%A particular payment transaction might entail a number of HTLC outputs corresponding to payments to multiple sellers concurrently.

\begin{figure}
\centering
\includegraphics[scale=0.7]{./externalized/unimedpc.pdf}
\caption{Protocol for the the k-th payment over two payment channels connected by a hash timelocked contract.}
\label{fig:unimedpc}
\end{figure}

\begin{figure}
\begin{lstlisting}[breaklines,mathescape=true]
OP_IF
  OP_DUP OP_HASH160 <PubKeyHash B (C)> OP_EQUALVERIFY OP_CHECKSIG
  HASH160 <Hash160 (secret)> EQUAL
OP_ELSE
  OP_DUP OP_HASH160 <PubKeyHash A (B)> OP_EQUALVERIFY OP_CHECKSIG
  <$T_1$ ($T_2$)> CHECKLOCKTIMEVERIFY DROP
OP_ENDIF
\end{lstlisting} 
\caption{HTLC pubScript of a payment transaction in an unidirectional mediated payment channel setup. The first branch of the conditional can be used by the recipient to claim the output by providing the secret, the second branch can be used by the sender to reclaim their funds after $T_1$ ($T_2$).}
\label{fig:pubScriptHTLC}
\end{figure}



\begin{figure}[ht]
  \centering
  \begin{subfigure}[t]{0.5\linewidth}
    \centering\includegraphics[width=0.95\textwidth]{./externalized/nmpaymentchannels.pdf}
    \caption{Independent payment channels\label{fig:nmpaymentchannels}}
  \end{subfigure}%
  \begin{subfigure}[t]{0.5\linewidth}
    \centering\includegraphics[width=0.95\textwidth]{./externalized/hubandspoke.pdf}
    \caption{Hub and spoke architecture\label{fig:hubandspoke}}
  \end{subfigure}
  \caption{Comparison between individual payment channels between buyers and sellers, and a hub and spoke architecture based on mediated payment channels. The number of channels can be decreased from N x M to N + M.}
\end{figure}


\begin{figure}
\centering
\includegraphics[width=\linewidth]{./externalized/htlcflow.pdf}
\caption{A series of (off-chain) transactions with two concurrent payments. After the secret is released the corresponding HTLC output can be removed and the balances can be updated. The large byte-wise size of HTLC outputs further disincentive abortion of the protocol by broadcasting a transaction including HTLC outputs.}
\label{fig:htlcflow}
\end{figure}

\subsubsection{Bidirectional Payment Channels}

The simple payment channel design presented in Sec. \ref{sec:unipc} does not allow to invert the payment direction. Although a payment channel could be easily funded by both parties, the first payment locks the direction of the channel, because the receiver of a payment transaction is always able to broadcast the transaction to the network. Therefore, if the payment direction gets inverted, the new payer might just broadcast an old payment transaction.

The obvious way to enable bidirectional payment channels is to form two independent and opposing unidirectional channels. However, in this case, there is no way to rebalance the channels which leads to unnecessary lock-up of bitcoins and unnecessary channel closings and openings.
\cite{decker2015Duplex} introduces a mechanism to rebalance two opposing payment channels, however the construction has two issues: (1) Every rebalancing decreases the lifetime of the channel, and (2) the construction hinges on building a chain of off-chain transactions, i.e. a transaction spends an output of a transaction that is kept off-chain. Bitcoin transactions, however, are malleable, i.e. their transaction ID can be changed without breaking the signature.
A malicious actor can use this fact to keep funds hostage.

Another design for bidirectional payment channels, called lighting channel, is based on revocable commitment transactions which work as follows:
Assuming two parties, A and B, the protocol begins with the creation of a transaction that attempts to create a shared multisignature output that is funded by both parties. However, before broadcasting this \emph{funding} transaction, which is not timelocked, both parties create \emph{commitment} transactions. Both commitment transactions attempt to spend the multisignature output of the funding transaction, and create an output with the pubScript depicted in Fig. \ref{fig:pubScriptCommitment}.

\begin{figure}
\begin{lstlisting}[breaklines,mathescape=true]
OP_IF
  OP_DUP OP_HASH160 <PubKeyHash B (A)> OP_EQUALVERIFY OP_CHECKSIG
  HASH160 <Hash160 (revocation secret)> EQUAL
OP_ELSE
  OP_DUP OP_HASH160 <PubKeyHash A (B)> OP_EQUALVERIFY OP_CHECKSIG
  <$\Delta$> CHECKSEQUENCEVERIFY DROP
OP_ENDIF
\end{lstlisting} 
\caption{PubScript of commitment transactions of A (B) for a bidirectional payment channel. The output can either be spent immediately by the counterparty if the counterparty can provide the revocation secret, or by the originator after a relative timelock $\Delta$.}
\label{fig:pubScriptCommitment}
\end{figure}

After the commitment transactions have been exchanged, the funding transaction can be broadcasted. 
In contrast to duplex micropayment channel, lightning channels can have an infinite lifetime. However, they also suffer from the malleability issue.
Furthermore, lightning channels require that participants periodically observe the blockchain for commitment transactions. The frequency of blockchain checks is dependent on the relative locktime $\Delta$. 


\subsection{Payment Channel Networks}
\label{sec:paymentchannelnetworks}

Payment channel networks are the generalization of mediated bidirectional payment channels. In principle they allow to interconnect an arbitrary number of payment channels using adapted forms of HTLCs in order to route instant off-chain payments. At time of writing, there are multiple groups working on implementations based on the lightning channel design. 


\section{Further Micropayment Mechanisms}

\subsection{Probabilistic Payments}

Probabilistic payments \cite{wheeler1996transactions,rivest1996payword,rivest1997electronic} are lottery-based payments. The lottery is biased in such a way that a single draw has an expected value according to the aspired micropayment.
\cite{Pass:2015:MDC:2810103.2813713} present such a lottery-based micropayment scheme for ledger-based transaction system, and discuss implementations in Bitcoin. However an implementation that does not rely on a semi-trusted third party would need a new signature verification primitive in the Bitcoin scripting language. 

\subsection{Pennybank}

The pennybank protocol allows to implement bidirectional payment channels without chaining partially signed transactions, and is thus not affected by Bitcoin's malleability issue. As for payment channels, the starting point is a shared multi-signature output. Each participant now builds a chain of hashes with length $N$ starting from a random secret $p^i_0$ where the superscript $i$ denotes the participant. 

\begin{equation*}
(\text{SHA256}(... \text{SHA256}(\text{SHA256}(\text{SHA256}(p^i_0),p^i_1),p^i_2)...,p^i_N)
\end{equation*}

$N$ has to be chosen according to the value the pennybank should have, i.e. the value of the output defined in what follows. Brute force searching for $p^i_0$ should be comparably hard as the difficulty to create the coins.  After that both parties cooperatively create a transaction spending the shared multi-signature output and creating a \emph{conditional} 1-of-2 multi-signature output P2CM with the following pubScript:

\begin{lstlisting}
OP_HASH160 <HASH160(p^A_0)> OP_EQUALVERIFY OP_HASH160 <HASH160(p^B_0)> OP_EQUALVERIFY <OP_1> <A pubkey> <B pubkey> <OP_2> <OP_CHECKMULTISIG>
\end{lstlisting}

Both parties hold a copy of this transaction and could broadcast the transaction at any time. The P2CM output can be spend by providing only one signature but both secrets have to be provided. Since brute forcing the other secret is expensive they are incentivized to cooperatively close the channel.


\subsection{Promissory notes}

\cite{strawpay}
\cite{filament}

\begin{figure}
\centering
\includegraphics[width=\linewidth]{./externalized/promissorynotes.pdf}
\caption{Possible participants, relationships and transaction flows in a micropayment scheme based on promissory notes.}
\label{fig:promissorynotes}
\end{figure}


\section{Discussion}

All of the micropayment schemes are based on multi-si

Single funded payment channels and unspendable UTXOs.

Lightning and Duplex need to be online 

Unidirectional Payment channels work offline. Mediated only Hub needs to be online 




\chapter{The Status of the Industry}
\label{sec:economy}

Smart objects are able to compute, to communicate and to sense and act on the physical world. Thus, they represent the building blocks of pervasive systems and the Internet of Things \cite{kortuem2010smart}. But being smart is not enough. Although the importance of interoperability has been highlighted by scholars \cite{zorzi2010} and practitioners \cite{manyika2015unlocking}, most connected devices operate in application-specific silos and communicate not with each other, but solely with the manufacturer's backend services or directly with people via smart phone apps.
Obstacles on the road to interoperability and \textit{the} Internet of Things are not only technical, e.g. heterogeneity (and thus complexity) and security, but also the lock-in business model of companies.
Cryptocurrencies and blockchain technology with their prime representative Bitcoin provide a new design space to tackle these issues by providing smart objects with unprecedented economic capabilities and autonomy from central coordination. This economic empowerment of connected devices is particularly well suited for the emergent paradigms of (autonomous) fog computing \cite{Bonomi:2012:FCR:2342509.2342513} and the web of things . Smart objects can get the ability to offer services for payments, pay other smart object autonomously, and even generate cryptocurrency to stay liquid. Moreover, smart objects that \textit{live on a blockchain} can be managed, shared, sold and rented without the need of backend infrastructure provided by the manufacturer. Furthermore new types of ownership and monetization models are conceivable.


\section{Building Blocks}

\subsection{Identity or Representation on the Blockchain}

Economic objects can have various representations on a blockchain depending on the capabilities of the object and the specifics of the blockchain. A basic requirement for an economic object is to be able to receive and to send cryptocurrency. Thus, an economic object needs at least an \emph{account} on the respective blockchain.

However, representation of ownership 

\subsection{Interaction with the Blockchain}

Smart objects vary significantly in their capabilities. From tiny battery-powered sensors with restricted computation and communication capabilities to smart TVs and refrigerators that are constantly connected to broadband Internet and continuous power supply. How will these objects interact with the blockchain? How do they initiate and verify transactions? 

\subsection{Micropayments}


\section{Case Studies}

\subsection{Data and Method}

The case studies are predominately based on open source code hosted on GitHub. This provides the most objective view on the technology. In addition company websites, white papers, and blog articles have been used. All of the companies, or related projects in the case of Filament, have public Slack channels on which the author was either active or passively monitoring for an extended period of time.

In addition to an technological overview, we present a brief general portrait of the company/project and conclude with remarks concerning the business model. 

\subsection{Filament}

\subsubsection{Portrait}

Filament was already presented in Section \ref{sec:ecofilament}. 

\subsubsection{Technological Overview}

Filaments technological architecture follows the SPADE framework. SPADE is an acronym for secure communications, private interactions, autonomous devices, decentralized network and exchange of value. An illustration of the framework and its more specific implementation is provided in Fig. \ref{fig:filament}. In the following we will investigate the layers of the framework in more detail. 

\begin{figure}
\centering
\includegraphics[width=\textwidth]{./externalized/filament.pdf}
\caption{The SPADE framework and technical implementation as pursued by Filament.}
\label{fig:filament}
\end{figure}


\paragraph{Secure and Private Communications}
The communication between devices is based on telehash\footnote{\url{http://telehash.org/}}, \emph{a light-wight interoperable protocol with strong encryption to enable mesh networking across multiple transports and platforms} independent of a centralized messaging broker.  Telehash endpoints are using the same form of self-authentication as users in Bitcoin and Ethereum, i.e their identity is derived from one or multiple self-selected public key(s). Telehash messages are JSON documents. All messages are encrypted (JSON Web Encryption \cite{rfc7516}) and authenticated (JSON Web Signatures \cite{rfc7515}). Furthermore, Telehash implements cloaking mechanisms that allow adding random noise to messages.

\paragraph{Autonomous Devices}
Filament interprets device autonomy as devices being \emph{first-class citizens} on the network instead of services. Consider the examples of humans in the Internet. Although we think of us as autonomous agents, in the Internet our identity is typically derived from services like Facebook, Google, LinkedIn or Twitter. In order to become independent of central services device autonomy incorporates three parts: (1) Self-authentication, (2) autonomous discovery, and (3) contracts. (1) Self-authentication is described in the last paragraph. (2) Autonomous discovery includes both local, i.e. devices are in the same local physical network, and global discovery.
Local discovery is straight forward using broadcasts. Global discovery is usually achieved by centralized services, e.g. Facebook, or federated services, e.g. the domain name system (DNS). In contrast, Filament uses \emph{blockname}, a simple DNS resolver based on domain name IP address tuples stored in null data outputs of Bitcoin transactions. An example for a telehash entry is shown in Fig. ref{fig:blockname}. In this system domain names do not have an owner and can not be transferred, but if there are multiple null data outputs concerning the same domain, the one with the highest attached value is considered valid by blockname resolvers. (3) Ownership, access control, pricing of resources is defined in a signed machine-readable data structure, a machine-to-machine contract. Fig. \ref{fig:filament} is showing the general structure of such a contract as envisioned by Filament. Contracts are stored on the devices themselves but receipts may be stored on a blockchain in order to provide a verifiable log of contractual obligations.

\begin{figure}
\centering
\includegraphics[width=\textwidth]{./externalized/blockname.pdf}
\caption{An example of a blockname entry for a telehash name resolution entry in the Bitcoin blockchain.}
\label{fig:blockname}
\end{figure}


\begin{figure}
\centering
\includegraphics[width=\textwidth]{./externalized/filamentcontract.pdf}
\caption{Contract structure as envisioned by Filament \cite{filament}.}
\label{fig:filament}
\end{figure}

\paragraph{Decentralized Network}
Filament devices support low-power wireless meshed networking. Devices can communicate over long-range links without immediate access to wifi or cellular connection. Wide area network access can be provided at arbitrary points within a network.


\paragraph{Exchange of Value}

Filament envisions arbitrary value exchange and barter between autonomous devices. However, only two forms are described in more detail: Payments in bitcoin, and some form of proof of payment. For bitcoin micropayments, Filament has developed the \emph{pennybank} protocol. Furthermore, it provides a mediated method that opposes the principle of device autonomy. Someone, or something, who wants to transact with a device pays the \emph{operator} of the device in exchange for a receipt. Then the payer provides the receipt to the device. 


% \subsubsection{Characterization as Economic Object}

% \paragraph{Identity and Representation on the Blockchain}
% Devices can have different representation on

% \paragraph{Interaction with the Blockchain}

% \paragraph{Micropayments}

\subsubsection{Business Model}

Filament advocates a Hardware-as-a-Service business model with an interesting twist. At least in principle,  devices are viewed as autonomous economic agents that provide services in exchange for some form of payment. Machine-readable contracts may specify revenue-sharing agreements between Filament, and various other stakeholders. Instead of the device-centric interpretation, we can also interpret this approach as a shared ownership of devices. This might be an important model for the IoT industry where shared usage of devices is the norm.


\subsection{21 Inc}

\subsubsection{Portrait}
21 Inc was founded May 2013, but remained stealth until May 2015. In June 2016 the company has a total funding of USD 121.05M. Thereby being the most funded start up in the Bitcoin ecosystem. Their first product, the 21 Bitcoin Computer, was released in November 2015. In March 2016 a micropayment marketplace was opened, and in June 2016, the 21 software library which allows to build bitcoin-payable Representational State Transfer (REST) APIs, UNIX command line tools, and access to the 21 marketplace became available on other UNIX-based platforms, free and open source under the free BSD license.

21 aims to make Bitcoin a standard resource for UNIX-based systems and provide the tools and services for developers to build a \emph{machine-payable web} where individual machines can cooperate globally, and potentially autonomously, in exchange for bitcoin micropayments.

A more detailed general introduction of 21 was given in Section \ref{sec:eco21}.

\subsubsection{Technological Overview}

In the following we present the core technology stack provided by 21. Figure \ref{fig:21stack} illustrates the stack together with additional services that distinguish 21.

\begin{figure}
\centering
\includegraphics[width=\textwidth]{./externalized/21stack.pdf}
\caption{The 21 technology stack and accompanying services.}
\label{fig:21stack}
\end{figure}

\paragraph{Embedded Mining Chip}

The 21 BC has an embedded mining chip with a hashing rate of approx. 50 GH/s and an energy efficiency of 0.16 J/GH. The chip contributes its hashing rate to a currently centralized mining pool orchestrated by 21. So far the chip is only deployed in 21 BCs and in data centers operated by 21. 	

In comparison, the total hashing rate of the Bitcoin network is around 1,500,000,000 GH/s in Q2 2016.

\begin{figure}
\centering
\includegraphics[width=\textwidth]{./externalized/21blocks.pdf}
\caption{Daily blocks mined by the 21 pool. The total daily block generation rate of the Bitcoin network is 1440 blocks. After March 2016 21 stopped identifying their blocks.}
\label{fig:21blocks}
\end{figure} 

\paragraph{Machine Wallet}

The 21 machine wallet is a hierarchical deterministic key Bitcoin wallet that can be interfaced either via command line or, more importantly, programmatically. The machine wallet provides access to on-chain funds and off-chain (BitTransfers)funds, and is the basis for interacting with bitcoin-payable APIs. 

BitTransfers are initiated by sending a serialized JSON document to a 21 server. The JSON document is authenticated by a signature corresponding to the wallets main key. 

\begin{figure}
\label{lst:21bittransfer}
\begin{lstlisting}[breaklines]
        bittransfer = json.dumps({
            'payer': self.username,
            'payer_pubkey': compressed_pubkey,
            'payee_address': payee_address,
            'payee_username': payee_username,
            'amount': price,
            'timestamp': time.time(),
            'description': response.url
        })      
        signature = self.wallet.sign_message(bittransfer)
        return {
            'Bitcoin-Transfer': bittransfer,
            'Authorization': signature
        }
\end{lstlisting}
\caption{Authenticated data structure to initiate a BitTransfer%\footnote{c.f. https://github.com/21dotco/two1-python/blob/master/two1/bitrequests/bitrequests.py.
}
\end{figure}


\paragraph{Developer Tools}

The 21 library is a collection of libraries that support a developer to build bitcoin-payable services. The most important parts are libraries to serve and consume/pay REST APIs using on-chain payments, payment channels and BitTransfers. Furthermore, there are libraries to work with various Bitcoin-related primitives.  


\paragraph{Software Defined Networking}

Most machines in the Internet are behind routers that employ network address translation (NAT) and/or disallow incoming connections from machines in other networks. Furthermore, Internet Service Providers (ISPs) often reassign IP addresses on reconnects. Hence, p2p communication and trading between personal machines is inherently hard given the predominant Internet architecture. 21 solves this problem with software defined networking (SDN). In particular, the SDN technology of ZeroTier\footnote{https://www.zerotier.com} is used. SDN allows to virtualize networks such that machines living in different physical networks connected via TCP/IP can communicate with each other as if they were in a single local area network. This entails that all traffic is encrypted and authenticated. 

\paragraph{Marketplace}

The marketplace allows sellers to publish bitcoin-payable APIs, and allows potential buyers to query and discover bitcoin-payable APIs. Although the marketplace is centrally hosted, trades are peer-to-peer. The marketplace is the first marketplace for digital goods and (micro) services. At time of writing the marketplace is small and most services are directly provided by 21. Fig. \ref{fig:21marketplace} shows the number of APIs offered by each individual seller. In July 2016, there are 86 APIs offered. 49 are provided by 21. Independent sellers mostly provide one single API. Table \ref{tbl:21marketplace} gives an overview of the top five services. All of them are offered by 21. 

In order to interact with the marketplace a user has to open an account with 21. However, only an username and password are required. Social connections such as Twitter, LinkedIn or GitHub can be provided. Analogous to existing marketplaces like Amazon and eBay, buyers are able to publicly rate sellers after purchase.  

\begin{figure}
\centering
\includegraphics[scale=0.7]{./externalized/21marketplace_creators.pdf}
\caption{Number of APIs in the 21 marketplace by individual vendors. The first vendor is 21 itself.}
\label{fig:21marketplace}
\end{figure}


\begin{table}
\centering
\begin{tabular}{lr}
    \toprule
    Name & Price [Satoshis] \\
    \midrule
    Zip Code Data & 2500 \\
    Ping21 Aggregator & 3000 \\
    Part of Speech Tagger & 6000 \\
    Twitter Influence Ranking & 5000 \\
    Neural Art & Variable \\
    \bottomrule
    \label{tbl:21marketplace}
  \end{tabular}
  \caption{The five most popular APIs. All are created by 21.}
  \end{table}
 

% \subsubsection{Characterization as Economic Object}

% \paragraph{Identity and Representation on the Blockchain}

% Machines do not have an identity on the blockchain, but represented as IP addresses in a virtual network. A machine can have multiple wallets 

% \paragraph{Interaction with the Blockchain}

% The library provides an abstraction layer for blockchain interaction. 

% \paragraph{Micropayments}

% Two mechanisms for micropayments are implemented. Simple unidirectional payment channels as presented in Sec. \ref{sec:unipc} and off-chain payments (BitTransfer). 

\subsubsection{Business Model}

21 is positioned at the center of a possible new type of digital economy with various ways for monetization. As of now the main revenue is generated by 21 Bitcoin Computer sales. Furthermore, there is potential revenue from mining, digital services on the marketplace, and embedded mining chip sales. Embedded mining could play a more important role in the future, and the mining chips are already prepared for revenue sharing. 21 is ideally positioned to act as a large payment service provider in payment channel networks (c.f. Sec. \ref{sec:paymentchannelnetworks} and \cite{decker2015Duplex}).


\subsection{Slock.it}

\subsubsection{Portrait}
Slock.it was founded in September 2015 by long-term members of the Ethereum community. Slock.it aims to play a significant role in the sharing economy by disintermediating existing sharing economy platform companies like Uber and AirBnB, as well as by providing \emph{sharing capabilities} to many more objects. At first this is envisioned to be enabled by a device termed the Ethereum Computer. The Ethereum Computer is intended to provide a bridge between the physical objects and smart contracts on the Ethereum blockchain.


\subsubsection{Technological Overview}

\begin{figure}
\centering
\includegraphics[width=\textwidth]{./externalized/slockit.pdf}
\caption{High-level overview of the Slock.it architecture. }
\label{fig:21marketplace}
\end{figure}


\paragraph{Ethereum Computer}

The current prototypical implementation of the Ethereum computer which provides the interface between physical objects and the Ethereum network is a light-weight Node.js application. The main part of the application is the eth module which interacts with an Ethereum node using web3.js. The eth module allows to register smart contracts on the Ethereum network which represent physical objects (slocks). These smart contracts are watched. State changes within the smart contracts trigger local state changes and events.
For example the eth module keeps track of the current owner and current user of a slock. Owners and users are represented as Ethereum accounts. Besides on-chain interaction with the slocks which will be discussed in more detail in the next paragraph, the Ethereum computer provides a websocket API. Messages received through the websocket API have to be signed, and are validated within the eth module. Thereby owners and users of slocks are able to authenticate themselves to the Ethereum computer. Hence, access control is in principle provided by the Ethereum network.

\paragraph{Slock Contracts}

The main slock contract defines an owner, a deposit, and a price. The price is time-based in wei per block, i.e. approximately per 13 s. In order to rent an object a user has to call the rent method providing the deposit. Upon return the user has to actively call the returnIt function which calculates the fee and distributes the deposit accordingly. 

\paragraph{Universal Sharing Network}

Each slock is part of the universal sharing network (USV) which provides discoverability and potentially reputation. The participation in the USV was envisioned to be coupled with providing revenue sharing with the DAO. 

\paragraph{The DAO}

The Team behind Slock.it furthermore implemented a set of smart contracts known as the \emph{Standard DAO Framework} and which have been deployed to the Ethereum blockchain as \emph{The DAO}. The DAO was meant as a shareholder-controlled venture fund. Everyone was able to send ether to the DAO contract in exchange for DAO tokens which provide voting rights and potential rewards. Moreover, everyone was able to send a proposal to the DAO, and token holders were able to vote on proposals. Proposals ask for ether payout under specific conditions. Slock.it wanted to fund the development of the Ethereum Computer by means of a DAO proposal, and provide DAO token holders with a share of the revenue for every trade facilitated by an Ethereum computer. THe DAO collected more than USD 130M worth of ether. However, many problems of the DAO contracts became identified which culminated in the exploitation of a re-entry bug.

\subsubsection{Business Model}

% \subsubsection{Characterization as Economic Object}

% \paragraph{Identity and Representation on the Blockchain}

% Each physical slock has a Slock contract representation on the Ethereum blockchain. The Slock contract governs the ownership and usage/access control of the digital representation of the physical slock. 

% \paragraph{Interaction with the Blockchain}

% The Ethereum Computer, and thus the slocks, interact with the Ethereum blockchain by watching the corresponding contract. Currently, the Ethereum Computer needs to run a full Ethereum node. There is no communication from the Ethereum Computer, i.e. the eth module, to the slock contract. 

% \paragraph{Micropayments}

% All payments are handled through regular ether transactions and the contract send function. 

\subsection{IBM ADEPT}

\subsubsection{Portrait}
The Autonomous Decentralized Peer-to-Peer Telemetry (ADEPT) platform is a proof-of-concept implementation by IBM and Samsung Electronics. The aim was to implement the essential functions that typical IoT platforms provide, i.e. message brokering, data sharing, and device coordination, in a decentralized architecture. The motivation is presented in \cite{devicedemocracy}. 
Although it was announced that the code for the ADEPT proof-of-concept would be available on GitHub, at time of writing the only information publicly available is in from of white papers \cite{empoweringtheedge}. 

\subsubsection{Technological Overview}

According to \cite{empoweringtheedge} the three main functions of a decentralized IoT platforms are (1) peer-to-peer messaging, (2) distributed file sharing, and (3) autonomous device coordination (c.f. Fig. \ref{fig:adept}). 

\begin{figure}
\centering
\includegraphics[width=\textwidth]{./externalized/adept.pdf}
\caption{The three main functions of the ADEPT platform, and the technologies they are implemented with.}
\label{fig:adept}
\end{figure}

\paragraph{Peer-to-peer Messaging}

Peer-to-peer messaging allows endpoints to communicate without the need of a central broker. In practice, messages have to be routed since direct links between endpoints do not always exist. ADEPT uses the same encrypted peer-to-peer messaging protocol, telehash, as Filament. 

\paragraph{Distributed File Sharing}

Messaging protocols are designed for small payloads and thus not suitable to distribute larger files such as firmware updates. Hence, ADEPT uses BitTorrent for distributed file transfer.

\paragraph{Autonomous Device Coordination}
Device coordination specifies the transaction properties of devices. Who can transact with whom under what conditions. This involves other machines/devices as well as humans physically interacting with a device. Examples are access control and authentication, and contracts, agreements and payments. Autonomous device coordination is envisioned to be blockchain-based, and the proof of concept implementation is based on a private fork of the Ethereum blockchain. 

\subsubsection{Business Model}

The decentralized architecture based on autonomous devices allows for a reduction of data center costs. This does not mean that data centers and cloud services become obsolete. Could services can still provide optional value-added services while base functionality can be provided without cost on the vendors side. This is almost a necessity for long-living IoT devices.


\subsection{Results}

\begin{table}[\tiny]
\centering
\label{my-label}
\begin{tabular}[\tiny]{@{}lp{1.5 cm}lp{1.5 cm}p{1.5 cm}p{1.5 cm}p{1.5 cm}@{}}
\toprule
Company/Project & Focus                                          & Blockchain & Blockchain Identity  & M2M Payments     & Network Participation       & Status                          \\ \midrule
Filament        & Autonomous devices, for Industrial IoT          & Bitcoin    & Blockname            & Pennybank, notes              & Light client                & Prototype                       \\
21              & Machine-payable web                            & Bitcoin    & -                    & On-chain, payment channels, off-chain              & configurable, typically API & Production, but mostly toy apps \\
Slock.it        & Sharing economy                                & Ethereum   & Slock Contract       & On-chain & Full node (currently)       & Early prototype                 \\
IBM ADEPT       & Autonomous devices/ decentralized IoT platform & Ethereum   & Yes, but not details & On-chain              & Different levels envisioned & Early prototype                 \\ \bottomrule
\end{tabular}
\caption{Blockchain usage of case study companies.}
\end{table}

\subsection{Discussion}

\subsubsection{Slock.it}


Full node
Physical changes and reorganizations
Re-entry bug in returnIt function
Nothing implemented if device is not returned


\section{Use Cases and Applications}

\subsection{Fluid Ownership, as-a-Service and the Sharing Economy}

\subsection{}

\subsection{Product self-servicing}


\section{Business Model Implications}






% \subsection{Asset Utilization and the Sharing Economy}

% \subsection{Extending the as-a-Service-Paradigm}

% \subsection{Lease to Own}

% \subsection{Backend-less Operations}

% \subsection{Revenue Sharing}

% \section{Other Opportunities in the Internet of Things}

% \subsection{A Black Box for the Internet of Things}

% \subsection{Enhancing Security through Decentralization}

% \subsection{Bringing the Cloud to the Edge}

% \cite{Loke:2015:MCS:2737797.2656214} show ad-hoc computations with near devices.

% \section{Challenges and Open Questions}


% \cite{Christidis2016} provide a thorough preliminaries section. In particular the sections about different consensus algorithms is interesting. The main contribution is the discussion of applications of blockchains for the Internet of Things, and their deployment considerations. Along the lines of the Device Democracy paper the authors argue that IoT devices lead to high maintenance costs for the manufacturer and give the example of the need to distribute software updates for many years. From the consumer perspective the need for IoT devices to \emph{phone home} raises privacy and security concerns.
% Software update with hash in smart contract and binary on IPFS.
% Billing layer and paves the way for a marketplace of services between devices.
% Sharing of services and property in general -> Slock.it
% Supply chain management with and without IoT.
% Deployment considerations:
% \begin{itemize}
% \item Throughput
% \item Privacy
% \item Who can mine?
% \item Legal enforceability
% \item Expected value of tokenized assets
% \item Complete autonomy is a double-edged sword
% \end{itemize}

