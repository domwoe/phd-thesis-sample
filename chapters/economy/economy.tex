\chapter{The Economy of Things}
\label{sec:economy}

\cite{Christidis2016} provide a thorough preliminaries section. In particular the sections about different consensus algorithms is interesting. The main contribution is the discussion of applications of blockchains for the Internet of Things, and their deployment considerations. Along the lines of the Device Democracy paper the authors argue that IoT devices lead to high maintenance costs for the manufacturer and give the example of the need to distribute software updates for many years. From the consumer perspective the need for IoT devices to \emph{phone home} raises privacy and security concerns.
Software update with hash in smart contract and binary on IPFS.
Billing layer and paves the way for a marketplace of services between devices.
Sharing of services and property in general -> Slock.it
Supply chain management with and without IoT.
Deployment considerations:
\begin{itemize}
\item Throughput
\item Privacy
\item Who can mine?
\item Legal enforceability
\item Expected value of tokenized assets
\item Complete autonomy is a double-edged sword
\end{itemize}


Although I am constantly changing my mind. I would argue that only public blockchains make sense that have a native token. Sidechains would also work 
\section{From Smart to Economic Objects}

\section{Economic Objects on Bitcoin}

\section{Business Model Opportunities}

\subsection{Asset Utilization and the Sharing Economy}

\subsection{Extending the as-a-Service-Paradigm}

\subsection{Lease to Own}

\subsection{Backend-less Operations}

\subsection{Revenue Sharing}

\section{Other Opportunities in the Internet of Things}

\subsection{A Black Box for the Internet of Things}

\subsection{Enhancing Security through Decentralization}

\subsection{Bringing the Cloud to the Edge}

\cite{Loke:2015:MCS:2737797.2656214} show ad-hoc computations with near devices.

\section{Challenges and Open Questions}

