\chapter{The Internet of Things}
\label{sec:iot}

\epigraph{We're building a world-sized robot, and we don't even realize it.}{--- \textup{Bruce Schneier}}

In this chapter, we introduce the \ac{IOT}. The main take away will be that although \ac{IOT} promises great opportunities for businesses, consumers, and the society as a whole, the prevailing architecture leads to severe issues ranging from the sustainability of business models to privacy and security. We argue that a more decentralized architecture with increased autonomy of edge devices will help tackling these issues. Furthermore, we motivate the need for electronic cash as a basis for incentivization of participation and trust.

\section{What is the Internet of Things?}

Semantically, an \ac{IOT} is a world-wide network network of interconnected objects uniquely addressable, based on standard communication protocols \parencite{infso2008networked}. However, in practice the term is used in very different settings and has been evolved into a suitcase word for various technologies. Organizations have given \emph{vision statements} from different perspectives. \cite{Atzori20102787} classify them as \emph{Things}-oriented, \emph{Internet}-oriented, and \emph{Semantic}-oriented, and argues that the Internet of Things can be seen as the convergence of these perspectives. In this thesis, we view the Internet of Things similar to the vision articulated by the CASAGRAS consortium as \emph{a world where things can automatically communicate to computers and each other providing services to the benefit of the human kind} \parencite[c.f. ]{Atzori20102787}. Arguably, this definition is close to the vision Mark Weiser gave, i.e. when the distinction between thing and computer becomes increasingly blurry \parencite{weiser:13609162}. The Internet of Things extends the digital world into the physical world by providing web services access to physical resources. Sensors are the eyes and ears of the Internet, actuators are the hands. On the other hand, \ac{IOT} provides a digital shadow for physical objects, extending their local capabilities and providing global capabilities.

In the following, we provide a short review of the evolution of the Internet of Things from the concept to auto-identification to today's consumer \ac{IOT}. Thereafter, we discuss the current state and issues from different perspectives. We waive the introduction of the various technologies and their individual application scenarios, advantages and disadvantages. The interested reader may consult e.g. \parencite{Atzori20102787,Mattern2010,Gubbi20131645} for excellent overviews. We approach the \ac{IOT} from a practical perspective and focus on the issues with the prevailing centralized architecture. Finally we discuss approaches towards a more decentralized \ac{IOT}, and the role of digital cash.

\section{Evolution and Status Quo}

\subsection{Auto-Identification}

The term Internet of Things was coined by Kevin Ashton in 1999 \parencite[c.f ][]{ashton2009internet}, then director of the Massachusetts Institute of Technology (MIT) Auto-ID Center which started to develop a global inter-corporate \ac{RFID} infrastructure. \ac{RFID} chips can be passive, i.e. do not need an own power supply, and can be integrated in all kinds of products to automatically identify individual products. Today, \ac{RFID} technology is used in key access cards, to track books in libraries, and in supply chain management in general. \ac{RFID} allows to track and uniquely identify all kinds of objects around us. It may also allow those objects to transmit part of the context they perceive through sensors. Using \ac{RFID}, physical objects get a digital identity. Often this identity is fixed, sometimes it changes with context, but the interactivity is fairly limited. \ac{RFID}-enabled products might be the largest category of today's \ac{IOT}. In this thesis, however, we are concerned with more powerful smart objects and connected devices.  

\subsection{Wireless Sensor (and Actuator) Networks}

\ac{WSN} aim to digitize the physical environment and are based on various kinds of open and proprietary communication technologies, such as LoRaWAN, SigFox or ZigBee. \ac{WSN}s are typically deployed by a single company for a single application. Sensor data is transferred from individual sensing nodes to gateways, connecting local area networks with wide area networks, and finally to corporate databases. Important applications are in agriculture as well as monitoring technical infrastructure such as electricity grids. The term \ac{WSN} is mostly used in the academic community while the industry has embraced terms such as Industrial \ac{IOT}, smart city, smart agriculture etc. Although the term \ac{IOT} is often used, we recognize a big gap to the vision stated above.

\subsection{The iPhone and the Rise of Consumer IOT}

In 2007, the iPhone heralded a new era. Smartphones are the most pervasive wireless sensor (and computing) platform that has ever existed. The ubiquity of smartphones in the developed world, and increasingly in the developing world, came with an unprecedented global supply chain and rapidly decreasing prices for \ac{SoC}, combining microprocessors, storage and different communication technologies. The smartphone has become a programmable interface, and its parts low-cost, easy to procure building blocks for all kinds of smart products and connected devices. This consumer \ac{IOT} has been driven predominantly by startup companies. Crowdfunding platforms such as Kickstarter\footnote{https://www.kickstarter.com/} and Indiegogo\footnote{https://www.indiegogo.com/} allow early adopters to fund new product categories and demonstrate market interest to venture capitalists. Cloud computing, on the other hand, has allowed to access computing, bandwidth, and storage infrastructure on a \emph{pay-as-you-go} basis. Startups need comparably little initial capital to operate backend infrastructure and are able to scale the infrastructure elastically with customer demand. Additionally, \ac{IOT} platforms have emerged, provided \emph{as-a-Service}, and entailing various functions from device provisioning to data analytics. These developments have led to a predominantly cloud-centric architecture of the \ac{IOT}. Figure \ref{fig:smarthome} illustrates the cloud-centric architecture in the smart home context.


\begin{figure}
\centering
\includegraphics[width=0.7\textwidth]{./externalized/smarthome_data_noplatfrom.eps}
\caption{Prevailing cloud-centric architecture in the smart home context.}
\label{fig:smarthome}
\end{figure}

%\subsection{Application Domains}


% \subsubsection{Architectural Design Principles}

% The Internet followed different design principles than its predecessors and competitors. The Internet is fundamentally about open standards and open software that implements those standards. Another important choice is known as the End-to-End (e2e)principle \cite{Saltzer:1984:EAS:357401.357402}. The e2e principle states that \emph{intelligence} should be at the edges, at the top of a layered system. This implies the strict separation of application-specific matter and the actual means of communication and transport. This is what allows the uncountable applications and use cases of the Internet, and low cost for communication.

\section{Opportunities and Challenges}

% \subsection{Research Perspective}



% Cisco gives a definition of fog computing.
% Discussion of cloud and fog. \cite{Yi:2015:SFC:2757384.2757397} briefly discusses the problems of cloud computing. Namely, unreliable latency, lack of mobility support, and location awareness. Issue of fog computing are also discussed. One issue is the missing programming model, but I would argue microservices architectures are step in the right direction. Accounting, billing and monitoring is another issue. So is security and privacy.


% \cite{DBLP:journals/corr/RomanLM16} provide a thorough overview of the security threads and challenges of mobile edge computing and fog computing.


% Cloud Computing - the business perspective\cite{Marston2011176}

% Integration of cloud computing and internet of things a survey \cite{botta2016integration}

% \cite{7123563} surveys IoT research focusing on technology and protocols. It gives a good overview over CoAP, MQTT, XMPP, AMQP etc.

% \cite{Roman20132266} discuss challenges and opportunities for a distributed IoT. Mention the problem of identity and authentication. Authentication is naturally simpler in a centralized model.


\subsection{Industry Perspective}

The \ac{IOT} ecosystem is a lively mixture of established technology companies, startup companies, and traditional hardware and consumer goods companies. The digitization of physical objects is stimulated by the aspiration to get real-time data of products and their life cycle in order to allow for high-resolution management \parencite{fleisch2010internet}, to provide new services to the consumer, and to explore and implement new business models \parencite{Fleisch2015}.	

Off-the-shelf \ac{SoC} and \ac{SiP}, as well as a multitude of \ac{IOT} platforms provided as a service \parencite{ISI:000379632700002}, allow small start-up companies, and non-specialized companies to bring an ever growing number of connected products to market. Analysts predict the \ac{IOT} will generate trillions of dollars per year within the next decade \parencite{manyika2015unlocking}, and there will be tens to hundreds of billions connected devices. Thus, the Internet is determined to undergo a phase transition. Currently the Internet's dominant applications are social, connecting humans, and human-machine communication such as video and music streaming. However, in the future, the dominant application will most likely be based on autonomous machine-to-machine communications. 

Although the marginal cost of connecting a product has become almost negligible, the maintenance and operating costs throughout the product lifetime are of concern. Therefore, vendors aim to use the digital capabilities of a product to reframe the product as a service with the aim to manifest recurring payments in form of subscriptions. Examples are Tado\footnote{\url{https://www.tado.com}}, offering a subscription-based thermostat, Comcast\footnote{\url{https://www.xfinity.com/home-security.html}}, offering a subscription-based security system, and also SolarCity\footnote{\url{http://www.solarcity.com/}}, offering solar panels based on a pay as you go model. We assume that these models will be even more important in developing countries where upfront costs are an obstacle. However, at the same time financial infrastructure such as banking accounts and credit cards necessary to implement subscription-based services are underdeveloped. Successful examples can already be seen in markets where mobile payment systems, such as M-Pesa, have become pervasive \parencite{hughes2007m}. M-Kopa\footnote{\url{http://www.m-kopa.com}} is offering solar home systems, a combination of a small solar panel, a battery, a few LED lights, and a mobile phone charger, based on a pay as you go model \parencite{ISI:000351842100012}. Each solar home system is connected via cellular networks and can be remotely disabled, in case of late payments. 

However, it is rather unlikely that service orientation coupled with subscription-based business models can be applied to all kinds of connected products. Many consumers are reluctant concerning recurrent payments. Tado, for example, started with a subscription-only model but had to offer a traditional one-time payment option soon. Many products that are becoming connected such as LED light bulbs, door locks, washing machines, and refrigerators have a life expectancy of many years. It remains to be seen if enough additional value can be created and extracted to pay for operational costs of the backend infrastructure.

In addition, with increasing distribution of connected devices there will be an increasing demand for interconnectivity of these devices. We see this already in the smart home. A simple example is the communication between the Nest protect smoke alarm\footnote{\url{https://nest.com/smoke-co-alarm/meet-nest-protect/}} and the Lifx\footnote{\url{http://www.lifx.com/}} LED bulbs. If the smoke detector recognizes a fire the light bulbs begin to blink in an alarming red. However, instead of local communication, the communication is facilitated between the backend services of the two vendors (c.f. Figure \ref{fig:smarthome}). With increasing demand for interconnectivity and communication this cloud-centric mode of operation will produce significant costs for the vendor.

Finally, as data is increasingly viewed as an asset, vendors store huge amounts of customer data. However, only few are able to leverage these data to create value. In any case, the collection of sensitive customer data is also a liability due to the risk of cyber attacks. A recent estimate for the average cost of a data breach involving customer data is \$4 million USD\footnote{\url{https://www-03.ibm.com/security/data-breach/}}.

% It is important to note that our discussion here was centered on the consumer IoT. We assume that in the long run the consumer IoT will surpass IoT in the industrial and enterprise context. IoT technology applied to Industrie 4.0 scenarios such as machine monitoring, predictive maintenance, and warehouse management, as well as business-to-business scenarios such as supply chain management 

\subsection{Consumer Perspective}

Simple connected security systems, user-aware thermostats, health and fitness trackers, and many more connected products provide new levels of convenience, security and data-driven knowledge for self-improvement to consumers. Most \ac{IOT} products are stand-alone and proprietary, but \ac{API}s allow at least for restricted interoperability. These \ac{API}s are typically not standardized but hub products and services have emerged, that allow the orchestration of multiple devices from different vendors. Examples of such devices are the Nest learning thermostat\footnote{\url{https://nest.com/thermostat/meet-nest-thermostat/}} and the Amazon Echo\footnote{\url{https://www.amazon.com/echo}} voice control. An example of a popular web service to interconnect devices and various web services is IFTTT\footnote{\url{https://ifttt.com/}}. 

Besides convenience, the current instantiation of the \ac{IOT} has various downsides for the consumer. First, products that are augmented with digital services come with terms of service. Usage of the product, or at least the usage of a significant set of features, requires agreement to these terms. In many cases, these terms are repeatedly subject to change, and the customer has essentially no choice but to agree. These terms of service usually define what data the vendor is allowed to collect and to use.
This practice interferes with the privacy of the customer. Even if the vendor itself is trustworthy, data breaches are common\footnote{See e.g. \url{https://www.privacyrights.org/data-breach}}. We have briefly discussed the role of \ac{IOT} in emergent economies from the vendor perspective. Privacy issues from the consumer perspective are of particularly concern, since privacy and consumer rights are underdeveloped.

Besides data breaches, \ac{IOT} devices often lack necessary security features or are poorly configured. Vulnerabilities of hundreds of thousands of devices have been demonstrated \parencite{7546527,Bodenheim2014114,184449,garcia2016lock}. Vendors are often reluctant to invest in patching older generation devices, and if they do, consumers are often not aware of the need to update. Automatic updates may help, but provide a new opportunity for hackers to centrally attack a large number of devices.

If important features of a product are dependent on services provided solely by the manufacturer's backend, the notion of ownership becomes blurry. Even though a customer is in physical possession of a product, the functionality is owned by the manufacturer. We have seen multiple instances of manufactures discontinuing backend services and hence rendering the connected products useless \parencite{Cox2016}. While this may be expected only for small startup companies, the case of Revolv, owned by Nest, an Alphabet company, proved the contrary \parencite{Gilbert2016}.

% The Twitter handle @internetofshit\footnote{https://twitter.com/internetofshit} provides an entertaining account of the issues with the current status of the consumer IoT.

\subsection{Societal Perspective}

An Internet of Things is the basis of a data-driven society. Collection and fusion of real-time data enables a more efficient usage of resources. Smart Grids allow the integration of decentralized renewable energy sources. Connected buildings and smart cities enable more efficient energy usage. Connected traffic control systems and connected cars allow for a more efficient usage of the physical infrastructure. Smart agriculture allows for more precise usage of pesticides and fertilizers. Health trackers and smart pills will lead to a better understanding of human health and allows for earlier interventions, and thus cost savings. 

However, the Internet of Things does not come without risks. We have already mentioned the impact on personal privacy in the context of connected consumer products. In a connected world we will leave digital traces everywhere. Since smartphones have cameras, the cost of them has rapidly decreased. In many cases a camera in combination with image recognition is cheaper than adding a specialized sensor. Connected cars have cameras, drones have cameras, humans might have cameras in their visual field. If only a subset of these data sources can be accessed and combined, there is a great risk of establishing a surveillance society. 

The interconnectedness, complexity and automated control loops involving connected actuators may allow individual hackers, even without the resources of governmental agencies, to attack physical infrastructure on a large scale. Stuxnet \parencite{5772960} was only an early example. Every year at the hacker conference Defcon\footnote{https://www.defcon.org/} new connected devices are getting hacked. The situation is particularly severe, when large parts of the connected infrastructure are centrally controlled with single points of failure. 


\section{Developments Towards a Decentralized IOT}

Many of the issues encountered in the former sections can be addressed by decentralizing the \ac{IOT} infrastructure and providing more autonomy to individual devices or sub networks. \cite{ISI:000379632700002} provide a comprehensive stop gap analysis of the current \ac{IOT} platform landscape. Out of 39 investigated platforms only 3 are categorized as decentralized. These platforms are research projects and are not ready for productive deployment. LinkSmart and OpenIoT are open source middlewares. Hub of all Things is a personal data store in the early steps of commercialization \parencite{ng2014engineering}. Personal data stores provide means to keep data produced by personal connected devices under control. In particular, if personal data stores allow for selective computation on personal data as proposed by the SafeAnswers system \parencite{10.1371/journal.pone.0098790}. Instead of giving service providers access to raw data, service provider are able to send vetted algorithms to the personal data store. However the idea of personal data stores is not new, still there a no successful implementations. \cite{DBLP:journals/corr/abs-1202-4503} discuss this phenomenon based on technical, economical and usability considerations. Neither manufacturers of connected products push these architectures, nor is there a real consumer pull at the moment. Hub of all Things and Enigma \parencite{7163223} try to establish multi-sided markets and thus provide financial incentives and positive feedback loops to foster participation. In particular in the case of Enigma, are cryptocurrencies and blockchain technology at the core.

An approach towards autonomy of individual devices can be seen with the Web of Things paradigm \parencite{Guinard2011}. Individual devices employ web servers and interact with each other, and with cloud services via \ac{REST}ful \ac{API}s, and machine and human readable \ac{JSON} documents. Thus, proved security and authentication mechanism from the traditional web such as \ac{SSL}/\ac{TLS} and OAuth \parencite{Leiba2012} can be reused. The machine-payable web, as envisioned by 21 Inc. (c.f. Sec. \ref{sec:eco21}, can be seen as an extension to the web in general, and the Web of Things in particular. The machine-payable web reintroduces the \emph{402 payment required code} and allows to pay for individual \ac{HTTP} requests using Bitcoin. Thus, resources can be paid for directly instead of using subscriptions and \ac{API} keys. This allows individuals (i.e. their computers and connected devices) to offer digital goods and services in exchange for payments.

In the Web of Things and machine-payable web approach connected devices become servers in addition to clients as in a cloud-centric architecture. This provides the basis for the concepts of peer-to-peer \parencite{Rodrigues:2010:PS:1831407.1831427} and grid computing \parencite{Anderson2004boinc}. Peer-to-peer computing can be entirely decentralized, where all nodes are equal. However, most peer-to-peer networks are not based on standardized web protocols, but on specialized protocols (c.f. BitTorrent or Bitcoin). In the grid computing paradigm, a network of nodes is considered as one big computer sharing work. However, the work gets typically distributed and aggregated by a central controller. Grid computing has only been successful in the scientific context. People all over the world have contributed CPU cycles to identify extraterrestrial life \parencite{Anderson:2002:SEP:581571.581573} and to simulate the folding of proteins \parencite{Beberg2009Folding}. However, participation based on altruism and scientific interest can not be extended to general application domains.   

Another paradigm is known as edge or fog computing, that aims to extend the cloud paradigm to edge devices \parencite{Bonomi:2012:FCR:2342509.2342513,ISI:000343986300007,Yi:2015:SFC:2757384.2757397}. The focus thereby is in increased utilization of resources at the edge, lower upstream bandwidth requirements, and decreased latency. Data analytics at the edge also enables increased privacy since raw data can be contained at the device level and does need to be transferred and processed in the cloud. \cite{ISI:000343986300007} proposes a definition: "Fog computing is a scenario where a huge number of heterogeneous (wireless and sometimes autonomous) ubiquitous and decentralized devices communicate and potentially cooperate among them and with the network to perform storage and processing tasks without the intervention of third-parties. These tasks can be for supporting basic network functions or new services and applications that run in a sandboxed environment. Users leasing part of their devices to host these services get incentives for doing so". They further highlight the need for accountability and monetization in order to provide incentives for device cooperation. 

On a similar line \cite{Zhang:2015} argues that the cloud-centric architecture does not scale for the \ac{IOT} and suggest a distributed platform, called the Global Data Plane (GDP). It is based on a data-centric design with focus on transport, replication, preservation, and integrity of streams of data while enabling transparent optimization for locality and quality of service. The foundation is a single-writer append-only log, coupled with location-independent routing, overlay multicast and higher level interfaces such as common access \ac{API}s. However, as known from peer-to-peer-based \ac{DHT} like Kademlia, without proper incentivization nodes may leave the network and data may be lost.

Finally, as motivated by \cite{Giannotti2012}, the Nervousnet architecture envisions the \ac{IOT} as a distributed participatory sensing platform "to provide real-time data for all and an AppStore for \ac{IOT} applications" \parencite{7097988}. It aims to provide a resilient \emph{planetary nervous system} to build a digital democracy. Besides aspects of gamification to incentivize user participation, the system aims to embed micropayments as a reward mechanism \parencite{Helbing2015Nature}.

We are still in the early days of the evolution towards an \ac{IOT}. In those days, it is natural that successful commercial applications utilize centralized solutions. Central control and a unified view of data simplifies complexity, and economies of scale decrease costs. However, this may only be true up to a particular scale. Currently, a form of decentralization is achieved by having a large number of vendors and a multitude of platforms, i.e. decentralization by competition at the company level. However consolidation is likely to happen. Furthermore, because standards and interoperability are still issues, individual homes, companies and cities can be expected to settle on a single platform.
We already motivated the need for a digital cash equivalent in order to build a decentralized but cooperative \ac{IOT} -- an \emph{Economy of Things} \parencite{Pureswaran2015}.


% \section{The Need for Digital Cash}

% Money allows humans to cooperate efficiently on a scale beyond mutual trust. 




% Moving from a client-server architecture towards a peer-to-peer architecture has challenges concerning incentivization and trust. Freeriding is a common issue in sharing networks, as well as are Sybil attacks. Peer-to-peer value exchange is a necessary prerequisite in order to incentivize participation. Furthermore, it allows to create security deposits in order to 
% \begin{itemize}
% \item Anonymous payments instead of subscriptions
% \item Ad-hoc payments instead of subscriptions
% \item Global payments without banking infrastructure
% \item Micropayments to incentivize collaboration
% \item Pay-as-you-go and loans with more privacy
% \item MAchines can not assess trustworthyness

% \end{itemize}

\section{Conclusion}

We introduced the Internet of Things as the vision of a world where things can automatically communicate to computers and each other providing services to the benefit of the human kind, and followed the evolution through three major epochs: auto-identification, wireless sensor and actuator networks, and the current consumer \ac{IOT}. We saw that the prevailing architecture of the \ac{IOT} is characterized by cloud computing and the client-server architecture. While this model allowed for simple and effective scaling during the initial stages of the Internet of Things, it becomes problematic with increasing scale. Current business models may not be sustainable. Edge devices provide minimal security. Raw data is routed through, and collected by central servers, acting as single points of failure and providing lucrative targets for attacks. Thus, potentially undermining privacy of the individual. We presented different avenues that strife for decentralization to mitigate these issues. A common prerequisite of these approaches is the ability for incentivizing participation and fair behavior of unknown participants. Thus, a digital analog of cash is needed. 

% \section{Peer-to-peer}
% \parencite{Rodrigues:2010:PS:1831407.1831427}
% Properties:
% \begin{itemize}
% \item High degree of decentralization. 
% \item Self-Organization
% \item Multiple administrative domains
% \item Low barrier to deployment
% \item Organic growth
% \item Resilience to faults and attacks
% \item Abundance and diversity of resources

% \end{itemize}