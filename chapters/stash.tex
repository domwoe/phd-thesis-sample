% It took quite some time until academia noticed Bitcoin. The first academic disciplines gaining interest were computer science, in particular information security \parencite{Karame:2012:DFP:2382196.2382292,Barber2012} and distributed computing \parencite{Babaioff:2012:BRB:2229012.2229022}, as well as law \parencite{Grinberg2011,Ewing2012}. Wide spread interest came with the extraordinary price rally in autumn 2013. The increasing value and media attention of Bitcoin led to an increase of venture capital fueling the ecosystem, and thus to more experimentation and variety of entrepreneurial activities. Bitcoin had the reputation of \emph{funny Internet money} that is used to buy drugs on the dark web \parencite{Christin:2013:TSR:2488388.2488408}. As we will see in this chapter, the first companies that have been founded were exchanges, payment processors, and ASCI miner manufacturers.
% Naturally, entrepreneurs focus initially on the low-hanging fruits and this means Bitcoin's value proposition of peer-to-peer electronic cash was taken literally to apply it for disintermediation in the financial service industry, e.g. payments, peer-to-peer lending, global remittances, and Over-The-Counter securities trading. 

% Arguably, this technology has the potential for disintermediation and disruption, which is not limited to the financial service industry. \cite{Giaglis2014} have highlighted: an interesting area 
% for IS scholars could be \emph{to assist the transition from the first era of applications 
% [...] (i.e. Bitcoin as currency) to more disruptive uses of the Bitcoin protocol 
% as an enabler of decentralized trusted peer-to-peer transaction ledger systems 
% and applications.} (p. 3)

% In this chapter, we categorize the venture-backed Bitcoin start-up 
% ecosystem along two dimensions, i.e. the potential for disruption and the particular 
% sector, and present the evolution of the ecosystem since its inception. (2) Based 
% on the categories, we investigate six companies in case studies with the aim to 
% extract the core innovations and the features of Bitcoin that fuel those innovations, 
% and critically discuss their potential for disruption. Two of these companies, Filament and 21 Inc., can be seen as representatives of the micro economy of things, and will be investigates on a more technical level in Chapter \ref{sec:economy}

% The structure of the chapter is as follows. In Section \ref{sec:eco:primer} we give a slightly technical 
% introduction to Bitcoin as an implementation of a novel information technology. 
% The focus thereby is on explaining the features that are utilized by the representative 
% cases. In Section \ref{sec:eco:method} we present the methodology by which we collected data, how 
% we selected the cases, and how we analyzed them. Section \ref{sec:sec:eco} is dedicated to the 
% introduction and categorization of the Bitcoin start-up ecosystem, and we discuss 
% its evolution over time. Thereafter, we present the case studies in Section \ref{sec:eco:cases}, 
% followed by key findings in Section \ref{sec:eco:findingss} We conclude with a critical discussion of 
% the disruptive potential and avenues for future research.

% \section{Innovation, Disruption and Disruptive Innovation}
% \label{sec:eco:primer}

% In a broader definition, innovation can be described as \emph{the generation, acceptance, 
% and implementation of new ideas, processes, products or services.} \parencite[][2]{10.2307/2391646}. Depending on the context, the term \emph{innovation} incorporates divergent 
% attributes, such as different process stages or underlying constructs \parencite{doi:10.1108/00251740910984578}. Hence, the purpose of an innovation process could be to change or 
% improve products as well as entire business models \parencite{CAIM:CAIM637}. Business 
% models can be described as \emph{the rationale of how an organization creates, delivers 
% and captures value} \parencite{osterwalder2010business}. Describing the logic of a 
% company's business activities \parencite{linder2000so}, the business model comprises 
% four high-level components including value proposition, operational model, financial 
% model and customer relations. \parencite{CAIM:CAIM637}. 

% Depending on whether a company introduces a new service or changes its current 
% business model, innovations can be characterized by their \emph{degree of newness}, 
% a metric, which \cite{JPIM:JPIMJPIM192_0110.XML} describe as \emph{Innovativeness} and 
% \emph{the degree of discontinuity in marketing and/or technological factors} \parencite[][112]{JPIM:JPIMJPIM192_0110.XML}. In recent years the two notions \emph{radical} and \emph{disruptive}
% have received increased attention from scholars and practitioners alike to describe 
% innovations with a high degree of innovativeness \parencite{Latzer01062009,christensen2015disruptive}. 

% Radical innovations refer to highly discontinuous (technological) changes \parencite{Latzer01062009} and are characterized by totally new features, high uncertainty and the necessity 
% for companies to acquire new capabilities to fully exploit emerging opportunities 
% \parencite{Latzer01062009}. Additionally, these often technology-driven innovations can have 
% the potential to shift existing paradigms and might possibly disrupt industries 
% \parencite{Latzer01062009}. Following Christensen's theory, disruptive innovation can be described 
% as a \emph{new product [or service] encroach[ing] on the low end of the existing market} 
% \parencite[][348]{JPIM:JPIM306} with the potential to move upward to satisfy 
% higher customer expectations \parencite{clayton1997innovator,ISI:000234127700008}. This is possible because 
% incumbents focus on improving their offerings along performance dimensions that 
% are valued specifically by the majority of demanding customers, while entrants 
% focus on different performance dimensions that are (initially) valued only by niche 
% segments, which are usually neglected by incumbents. At the same time, \cite{christensen2015disruptive} define \emph{disruption} as \emph{a process whereby a smaller company with 
% fewer resources is able to successfully challenge established incumbent businesses}. 
% Hence, a disruptive innovation does not inevitably leads to the disruption of a 
% market, nor does a disruption of a market necessarily has to be triggered by a 
% disruptive innovation. 

% The literature on disruptive innovation is currently still in an early stage \parencite{JPIM:JPIM177} and it remains controversial, how to define the concept \parencite{ISI:000283729100005}. While some researchers support Christensen's understanding, 
% contributing own thoughts to the theory, others generally \emph{criticize the vagueness 
% of the concept} \parencite[][438]{ISI:000283729100005}. According to its critics Christensen's 
% theorization does not allow for a clear differentiation between underperforming 
% technologies and potentially disruptive technologies with initially inferior performance 
% \parencite{JPIM:JPIM179}. Finally, a main criticism focuses on the lack of measurability of 
% disruptive innovation \parencite{JPIM:JPIM176}. To overcome this gap, \parencite{JPIM:JPIM176}, define the following criteria to identify disruptive innovations. 
% The innovation should (1) possess inferior attributes regarding what mainstream 
% customers would value; (2) attract new customer segments by offering new value 
% propositions; (3) be offered at lower costs; and (4) start from niche markets \parencite{ISI:000283729100005,JPIM:JPIM176}.
\section{Discussion}
\label{sec:eco:discussion}

In the title of this work we stated the question on whether the Bitcoin ecosystem 
may disrupt sectors beyond the financial service industry. We would argue that 
the companies we investigated in this paper are leveraging Bitcoin as global programmable 
money and as an immutable public database to foster radical innovations in their 
respective sectors. Central authorities are eliminated and new markets are created. 
But as shown in Section \ref{sec:eco:primer} radical innovations are not necessarily disruptive 
innovations and disruptive innovations may or may not disrupt a market eventually. 
Thus \emph{identifying disruptive innovation is a complex process} \parencite{Kaltenecker2015234}. The current literature still lacks a thorough theoretical foundation, 
on how to define disruptive innovations \parencite{ISI:000283729100005}. Additionally it remains 
unclear to what degree disruptive innovations can be predicted ex ante \parencite{JPIM:JPIM179}. This article aims to offer an overview on the fast developing 
blockchain ecosystem highlighting mainly the potential inherent to these technologies. 
Most sectors that we considered are underdeveloped. The commoditization of registration 
of digital work and intellectual property in general could be counted as a new-market 
disruptive innovation. Current registration of IP is expensive and cumbersome, 
but is backed directly by jurisdiction. Therefore, in terms of legal certainty, 
the traditional process is clearly superior. Decentralized marketplaces exhibit 
also the characteristics of a disruptive innovation. Today, OpenBazaar has an inferior 
user experience in comparison to Amazon or eBay. Moreover, it obviously does not 
have the same strong user base. However, the OpenBazaar platform is free to use 
and there are no restrictions on what can be sold. These characteristics will most 
probably attract users in niche segments and could grow from there on. In cases 
like Filament and Onename the notion of disruptive innovation does only apply to 
a lesser extend. Interestingly, most of the core innovations that the companies 
provide are in form of open source protocols. Thus, the disruptive potential is 
not entirely dependent on the future of the particular companies. From the point 
of the start-ups, open source protocols and decentralization can help entering 
a market that is dominated by a central institution. However, it becomes much harder 
to capture value and thus to find a sustainable business model.

As shown in Figure \ref{fig:eco:evolution}, most of the ecosystem beyond financial services is younger 
than two years. The companies are in a very early stage and the focus is clearly 
on creating value and on advocating the use of their protocols. Some of the companies 
are experimenting with ways to capture value. But it is too early to investigate 
the business model in detail. Noteworthy, all of the six start-ups under consideration 
build platforms. In fact, these platforms are rather different. For example Filament 
builds a decentralized platform for the Internet of Things, Onename builds a decentralized 
identity layer and so on and so forth. Hence, it will be worthwhile to study the 
emergent value networks as soon as the platforms get populated.

Despite these first insights, more research will be required to gain a better understanding 
of the theory of disruptive innovation in general, as well as in the context of 
blockchain technologies in particular \parencite[cf.]{ISI:000222662900002}. We encourage other 
scholars to take this article as a starting point to investigate more precise measurements 
\parencite[cf.]{JPIM:JPIM176} and clearer definitions \parencite{JPIM:JPIM177} 
of disruptive innovation in a Fintech context as well as to draw clear distinctions 
to related research fields such as radical or open innovation \parencite[cf.]{ISI:000283729100005}. 


\section{Methodology}
\label{sec:eco:method}

\textbf{Mapping the Bitcoin ecosystem:} 

\textbf{Selecting and analyzing the representative companies:}Based on this classification 
and our focus beyond financial services, we identified six application-specific 
subsectors for investigation by representative case studies. The particular companies 
were selected based on funding and attention by the community. The case studies 
are grounded in a rich set of secondary data sources. Besides company websites, 
white papers and press releases, there are detailed founder interviews on Zapchain\footnote{https://www.zapchain.com/} and Epicenter Bitcoin\footnote{https://epicenterbitcoin.com/}. 

Although our research approach was inherently exploratory we had two guiding questions 
in mind: (1) What is the core innovation of the company, and (2) on which Bitcoin 
features is this innovation based.

\section{Case Studies}
\label{sec:eco:cases}


In the following, we investigate the challengers beyond financial services in more 
detail. Therefore, we present six comprehensive case studies (marked as bold in Table 
\ref{tbl:ecosystem}).

\subsection{Filament (Internet of Things)}
\label{sec:ecofilament}

Filament\footnote{https://filament.com/} provides wireless sensor networks for the (industrial) IoT, e.g. 
for smart cities or smart agriculture use case. Most IoT platform providers follow 
a centralized approach by connecting all devices to their respective cloud-infrastructure. 
This has the major disadvantage that devices depend on a central infrastructure 
in order to operate. Moreover, it can be argued that this approach cannot keep 
up technically and economically with the increasing number connected devices.

Filament is one of the first companies that develop a fully decentralized IoT infrastructure, 
which encompasses three blockchain-related aspects: (1) Each device is registered 
on the blockchain providing a verifiable and immutable identity. This enables discovering 
of and authenticating with other devices/services without the need of a dedicated 
backend infrastructure. Therefore, devices are technically autonomous and are able 
to operate independently of Filament. (2) Each device is governed by a \emph{smart} 
contract, which manages agreements of device control/ownership, data access and 
financial agreements concerning the device. Ownership can be transferred permanently 
or temporarily by a simple transaction on the blockchain. Filament implements the 
financial agreements as a Product-as-a-Service, which means that the owner gets 
paid directly for the ongoing use of the device. (3) Furthermore, each device is 
able to transfer value in form of bitcoins to other devices in order to get access 
to data or request some service.

As described, devices can be operated and governed by using only the blockchain 
as a backend and therefore without any technical dependence on the platform creator 
(Filament) or other third parties. This might bare great benefits for customers 
needing to deploy large Industrial IoT applications with a lifetime of 5-10 years. 
Because they want to minimize the risk of a lock-in with a specific company. Moreover, 
according to Filament, customers prefer paying continuously on a real-time basis 
instead of an upfront investment, which can be solved efficiently by Bitcoin micropayments. 
Ownership is decoupled from usage and both are independent of the manufacturer 
or a platform provider.

Filament itself is a venture capital-backed company formerly known as Pinocc.io. 
They claim to have their first deployments with Fortune 500 companies in 2016. 
They will get paid for the ongoing use of their devices (by owning the smart contract). 
Moreover, they work on a licensing model, i.e. Customers can attach a module version 
to their own devices, which will give them all the described benefits of Filament 
in return to a small fraction of the payments for the ongoing use of the devices. 
However, since all protocols will be open and there is no dependence to Filament 
by design, other companies could use that and build their own hardware without 
Filament.

\subsection{Ascribe (Intellectual Property)}
\label{sec:ecoascribe}

Ascribe\footnote{https://www.ascribe.io/} aims to provide provenance of intellectual property. Digital work, 
like art, photos, and music, can be registered publicly on the bitcoin blockchain 
together with its accompanying terms and conditions. The technological basis is 
the spool protocol\footnote{https://github.com/ascribe/spool}, an overlay protocol that uses 
Bitcoin transactions to represent unforgeable ownership transfers and licensing 
agreements for digital work. Thus, authenticity of ownership and usage rights can 
always be proven. Ascribe focuses on digital art in particular. Although art is 
in principle copyrighted by the time of creation, it is currently cumbersome and 
expensive to officially register work and prove ownership (see e.g. http://copyright.gov/docs/fees.html). 
Therefore, hardly any digital art gets registered. Ascribe aims to change the status 
quo by providing a virtually free and automatable registration process. Moreover, 
the unique representation of digital work based on cryptography and the blockchain 
is the basis for the creation of a secondary market for digital work. For example 
the spool protocol enables creating limited editions of digital work. So far, this 
has not been possible without relying on some central institution.

The spool protocol is open source and can be used by anybody. In principle, the 
only costs are bitcoin transaction fees. However, Ascribe wraps the protocol in 
convenient web services and provides tools adapted to particular customer groups, 
e.g. individual creators, museums and marketplaces. Ascribe's revenue model is 
then based on a share of the rentals and sales of registered digital work that 
are facilitated by their APIs.

The core innovation is the application of the bitcoin blockchain to provide commoditized 
provenance of intellectual property. Provenance is demonstrated by relying on the 
immutability of the bitcoin blockchain, instead of an authority. 

\subsection{OpenBazaar (E-Commerce Marketplace)}
\label{sec:ecobazaar}

OpenBazaar\footnote{https://www.openbazaar.org/} is an open source project consisting of a protocol and a reference 
implementation that enables a decentralized peer-to-peer e-commerce marketplace. 
In comparison to traditional e-commerce market places like eBay and Amazon, there 
is no central server or authority that is running the market place. Thus, there 
is no middleman who is able to charge fees or to restrict offered products and 
services. Everyone with Internet access is able to set up a shop by running a network 
client. Payments are facilitated using Bitcoin transactions. Therefore, no payment 
provider or banking account is needed. This lowers payment transaction fees and 
increases the global reach. Besides sellers and buyers there are notaries and arbiters 
for dispute resolution participating in the marketplace. Those latter participants 
are involved by using bitcoin multi-signature transactions. Thus, OpenBazaar unbundles 
the functions of traditional marketplaces. Trades on the OpenBazaar network are 
based on Ricardian Contracts \parencite{1319505}, i.e. an electronic document that defines 
the terms of a trade such that it is readable by computers and humans, and is cryptographically 
signed. Apart from selling physical and digital products, OpenBazaar can also be 
used to trade speculative contracts, which can be readily represented by Ricardian 
Contracts. 

The main value proposition of OpenBazaar to sellers as well as to buyers is the 
elimination of fees and restrictions. Since marketplaces are subject to network 
effects most users will most probably not switch immediately from traditional marketplaces 
to OpenBazaar. However, OpenBazaar could set foot in under-served niche markets. 
Examples could be digital goods, developing countries with limited access to traditional 
payment services, and prohibited goods. 

OpenBazaar itself is not a company, but its main developers founded the venture 
capital-backed company OB1. Their current focus is on developing the OpenBazaar 
protocol and its reference implementation. As outlined above, OB1 is not able to 
profit directly from transactions on the marketplace in the way traditional revenue 
mechanics on centralized marketplaces work. 

\subsection{21 (Digital Micro Commerce Marketplace)}
\label{sec:eco21}

At first sight, the categorization of 21 Inc.\footnote{https://www.21.co} as a challenger seems odd, 
since 21 is an infrastructure and platform provider for the bitcoin ecosystem. 
Indeed, it is often categorized as a mining company. However, we argue that 21 
is better classified as a marketplace for digital micro services, which has the 
potential to challenge traditional Internet business models.

With a funding of \$121M, 21 supersedes every other start up in the bitcoin ecosystem. 
They have developed an embeddable ASIC\footnote{ Application-specific integrated 
circuit. A hardware chip customized for a particular task in contrast to a general 
CPU. In this case task is Bitcoin mining, which involves the extensive execution 
of SHA265 operations.} mining chip that they are using in their own mining operations, 
but which is also embeddable into arbitrary connected devices. In November 2015 
they released their first product, the 21 Bitcoin computer. Essentially the 21 
Bitcoin computer is a full-stack development platform to build bitcoin-payable 
digital services, which can be published and discovered on 21's digital marketplace. 
Individual service consumptions, like an API call, can be billed at as little as 
1 satoshi\footnote{Satoshi is the smallest denominator of a bitcoin. 100,000,000 
satoshi correspond to 1 bitcoin. Thus, 1 satoshi is currently worth approximately 
USD 0.000003.}. The embedded mining chip, which is currently coupled to a mining 
pool operated by 21, supplies the device with 
a continuous stream of satoshis.

21 aims to embed their chips into any connected device (e.g. smart phones) to establish 
bitcoin as a system resource like CPU, bandwidth or disk space, but for the purpose 
of buying and selling digital goods and services \parencite{Balaji2015}. It is crucial 
to understand that it does not make sense to sell the small amounts of mined bitcoins 
for Fiat currency on an exchange. Instead, the idea is to supply every device with 
a continuous stream of bitcoin from the point of commissioning on so that it can 
directly operate on the marketplace. 

Having such an infrastructure of bitcoin-enabled devices in place at scale, could 
offer compelling new opportunities and even disrupt traditional business models 
of the Internet. For example, it has been difficult for news sites to directly 
monetize their content on the Internet. It is still tedious for users who want 
to read just one article to signup, enter their credit card information and buy 
a subscription. Thus, most news sites still depend on indirect revenue by advertisements, 
which becomes more problematic with the increasing spread of ad-blockers. These 
problems could be eliminated with the diffusion of a bitcoin-enabled infrastructure 
for frictionless micropayments. Similarly, a bitcoin-enabled IoT device, e.g. for 
automatic irrigation of farms, could pay a weather service API in return for accurate 
weather prediction data without the need for signup. Moreover, the device could 
search automatically for the cheapest (and best in terms of reputation) weather 
service API on the marketplace.

The 21 platform is a mixture of a centralized and a decentralized model. As of 
today, the mining power is bound to 21, and returns get allocated to a wallet owned 
by 21. However, it is announced that this will change in the future. The marketplace 
for digital services on the other hand is in principle decentralized and trades 
can be conducted peer-to-peer. Currently, 21 generates revenue by private mining 
operations and by sales of the bitcoin computer. In the future, however, all kinds 
of interesting revenue models are imaginable: selling and licensing of mining chips, 
revenue sharing of embedded mining operations or tiny transaction fees for off-chain 
transactions to name just a few. 

In conclusion, 21 could change how resources in the Internet are paid for, and 
thereby also contribute to making new resources available, which have not been 
available yet because of missing incentives

\subsection{Factom (Records Management)}
\label{sec:ecofactom}

Factom\footnote{http://factom.org/} is an open source software project that provides businesses with 
the ability to prove data integrity and to create verifiable and immutable audit 
trails. 

While data integrity could be achieved by directly adding a hash of the data to 
a bitcoin transaction and thereby time-stamp the data on the bitcoin blockchain, 
this method does not scale. On one hand, this is because of inherent scalability 
issues of bitcoin, and on the other hand because of transaction fees. Therefore, 
Factom consists of a peer-to-peer network that is independent of the bitcoin network. 
Customers of Factom generate hashes of their data and send them for recordkeeping 
to the Factom network. There, all hashes are compressed to a single hash by building 
a Merkle tree and taking the root of the tree. This single hash 
value is then stored in the bitcoin blockchain. This provably time-stamps all individual 
records without having to write all records individually into the bitcoin blockchain. 
The network maintains its own crypto-currency, \emph{factoids}, which is used to incentivize 
participants of the peer-to-peer network to provide their resources. A factoid 
can be transformed into entry credits, which can be used to submit new records. 
The price of a factoid depends on the market value, but the price of an entry credit 
is fixed to 1/10\textsuperscript{th} of a cent. 

In summary, Factom provides a decentralized platform for data provenance with a 
permanent, time-stamped record of an unforgeable reference to the data anchored 
in the blockchain. This offers an efficient and cheap alternative for businesses, 
institutions and governments to have a proof of existence, proof of process or 
proof of audit for their data. Their first publicly announced project is using 
Factom for an official land title registry in partnership with the government of 
Honduras \parencite{Chavez2015}. In fact, Factom is an interesting option for 
governments in developing countries. They often face mismanagement and corruption, 
but cannot afford or enforce infrastructure and processes to guarantee compliance 
of their administration. Moreover, the Factom solution offers also an opportunity 
for small businesses/start-ups to have more auditing and to prove compliance with 
regulations without hiring expensive professional companies for that.

\subsection{Onename (Identity)}
\label{sec:ecoonename}

Onename\footnote{https://onename.com/} allows registering identities on the Bitcoin blockchain. This blockchain 
identity can be connected to various online identities like Facebook, Twitter, 
PGP keys and a Bitcoin address. Thus, it provides a probabilistic identity which 
reliability grows with the number of verified connected accounts. Onename is based 
on an open source overlay protocol called Blockchain ID. Everyone is able to register 
his identity without having to rely on Onename services. Blockchain identities 
are independent of the company Onename, are referenced on the Bitcoin blockchain 
and are therefore owned by the holder of the respective private key.

Blockchain IDs will allow signing up on third party websites comparable to Facebook 
Login or Google Sign-In. There is no need for passwords since authentication is 
done using digital signatures. Blockchain ID is also used as an identity provider 
for OpenBazaar. 

Moreover, it is possible to add additional namespaces. In this sense Blockchain 
IDs could represent not only humans, but also machines. Thus, Blockchain IDs might 
be the basis of a decentralized DNS system or an IoT registry. 

Essentially, the technology allows individuals to own their online identities, 
rather than being dependent central institutions. Holding, i.e. registering and 
prolonging, a Blockchain ID requires fees that directly support the Bitcoin ecosystem. 
One part of the fee is a typical Bitcoin transaction fee that will be collected 
by a miner. The other part of the fee is particular to the protocol and leads to 
\emph{burning} of bitcoins. Since the number of bitcoins is constrained, the elimination 
of bitcoins theoretically increases the value of existing bitcoins. While it might 
seem that our current identities are free, we actually pay with our personal data. 
Every time we use Facebook to login into a third party website, we give away more 
information about us. Identity is also subject to network effects. Institutions 
will only start to accept Blockchain IDs if there are enough people using them 
and demand acceptance. 

Onename basically follows the same strategy as OB1. The initial focus is on developing 
open source protocols and advocating their adoption, instead of having a revenue 
model in place.

\section{Key Findings}
\label{sec:eco:findings}

Table 2 presents the key findings concerning our main research questions. We state 
the core innovation of the company in the respective sector and list the features 
of Bitcoin that underlay those innovations. The innovations in the digital asset 
sector and in the notary sector are enabled mainly because of the usage of the 
Bitcoin blockchain as an immutable public database. In contrast, the decentralized 
marketplaces profit from the value transfer features Bitcoin provides, in particular 
multi-signature escrow and micropayments. Both features are ultimately based on 
the scriptability of Bitcoin transactions, i.e. the nature of Bitcoin as programmable 
money. Another important aspect that encompasses all sectors is the inclusivity 
and permissionless nature of Bitcoin. This aspect has two implications. First, 
it allows companies to build protocols on top of Bitcoin and the blockchain without 
having to ask anyone for permission. All of the presented companies are based on 
this. Second, everyone with Internet-connectivity is able to participate in the 
system. In particular, people from developing countries without proper access to 
financial services can be active on the marketplaces. 

Rather surprising is the appearance of mining. The business model of traditional 
mining operations is to generate revenue by selling newly minted bitcoins. In contrast, 
21 aims to leverage mining as a tool to supply users directly with bitcoins circumventing 
the cumbersome process of acquiring those on an exchange. Thus fuelling the emerging 
ecosystem of bitcoin-payable digital micro services. 

\resizebox{\textwidth}{!}{
\tiny
\begin{tabular}{|>{\raggedright}p{12pt}|>{\raggedright}p{44pt}|>{\raggedright}p{37pt}|>{\raggedright}p{80pt}|>{\raggedright}p{38pt}|>{\raggedright}p{47pt}|>{\raggedright}p{20pt}|>{\raggedright}p{30pt}|>{\raggedright}p{23pt}|}
\hline
S\textbf{ector} & S\textbf{ubsector} & C\textbf{ompany/ Project} & C\textbf{ore 
innovation} & \multicolumn{5}{p{159pt}|}{B\textbf{itcoin features}}\tabularnewline
\hline
 &  &  &  & I\textbf{mmutable public database} & I\textbf{nclusive (global and 
permissionless)} & M\textbf{ining} & M\textbf{icropayments} & M\textbf{ultisig}\tabularnewline
\hline
Digital Assets & Internet of Things & Filament & Decentralized IoT infrastructure 
based on autonomous devices & x & x &  & x & \tabularnewline
\hline
 & Intellectual Property & Ascribe & Commoditization of registering IP; trade and 
license digital work & x & x &  &  & \tabularnewline
\hline
Marketplace & E-Commerce Marketplace & OpenBazaar & Decentralized market place 
for physical and digital goods without restrictions and fees &  & x &  &  & x\tabularnewline
\hline
 & Digital Micro Services Marketplace & 21 & Ecosystem for bitcoin-payable digital 
(micro) services &  & x & x & x & \tabularnewline
\hline
Notary & Records Management & Factom & Commoditizing data integrity and auditability 
 & x & x &  &  & \tabularnewline
\hline
 & Identity & Onename & Individually-owned identities & x & x &  &  & \tabularnewline
\hline
\end{tabular}}

% \begin{figure}
% \includegraphics[scale=.7]{./externalized/paymentchannel.pdf}
% \caption{An unidirectional payment channel between a payer A and a payee B. To open a channel A creates and broadcasts a funding transaction with an output that can be spent only by providing signatures from both parties, i.e. $\sigma(A)$ and $\sigma(B)$ (multisignature). The value of this output is then shared between A and B in off-chain payment transactions. A creates and (partially) signs those payment transactions, and sends them to B via some private or public communication channel. While A has only partially signed payment transactions, B is able sign and broadcast any of those transactions to close the channel. If B is rational he will always broadcast the transaction that allocates the largest share to him. Therefore A can not decrease B's share in a subsequent payment transaction, and the payment channel is thus unidirectional. If B does not broadcast a payment transactions before $T_1$, A is able to recover the funds by broadcasting a refund transaction.}
% \label{fig:paymentchannels}
% \end{figure}

% Unidirectional payment channels between two parties involve two on-chain Bitcoin transactions. One funding transaction and one transaction to close the channel. In the unidirectional setting only one party funds the channel and risks locking her funds (temporarily) in case the receiving party is not cooperative. Simple unidirectional payment channels are useful for repeated, metered payments between a single payer and a single payee. Use cases are for example video streaming\footnote{https://streamium.io/} and WiFi access \cite{Siby2013}.

Mobile phones have an unprecedented scale in the diffusion of computing technology. It is the first computing and sensing platform that reaches even the rural and remote areas throughout the developing world. Smartphones are powerful connected mobile general purpose computing devices that can be considered as the largest mobile sensing platform in existence. Essentially the same base technology is now being deployed in all kinds of other devices from cars to lightbulbs, extending the scope of sensing services even further. However thus far, the emerging opportunity of a planetary nervous system \cite{giannotti2012planetary} has not yet come to scale. Mobile crowdsensing, as will be defined in Section \ref{sec:crowdsensing}, has only been used either in small scale scenarios or in application-specific scenarios. A global platform for crowdsensing is missing.



 A fundamental building block of any crowdsensing platform, either centralized or decentralized, is a global, permissionless payment system that reaches as far as the diffusion of smartphones, and is capable of providing micropayments to remunerate ad-hoc sensing services. We argue that Bitcoin is such a global payment network. Although Bitcoin is based on a heavily replicated shared ledger on top of a gossip peer-to-peer network, and an energy and capital intensive distributed consensus mechanism using proof-of-work, we present technologies that leverage the built-in scripting language to enable peer-to-peer and mediated micropayments. In particular, we present an implementation of mediated unidirectional payment channels based on hash time locked contracts.


\section{Mobile Crowdsensing}
\label{sec:crowdsensing}

Crowdsensing is where individuals with sensing and computing devices collectively share data and extract information to measure and map phenomena of common interest \cite{ganti2011mobile}. Predominately, crowdsensing has been engaged in the form of mobile crowdsensing, utilizing the ubiquity of smartphones, in application-specific scenarios. Examples are transit tracking \cite{Thiagarajan:2010:CTT:1869983.1869993}, road and traffic monitoring \cite{Mohan:2008:NRM:1460412.1460444}, site characterization \cite{Chon:2012:ACP:2370216.2370288}, and on-street parking \cite{Chen:2012:COS:2386958.2386960,6569416}. Besides these small-scale academic field studies, notable examples for successful large-scale, application specific mobile crowdsensing application are Google Maps and Waze which was acquired by Google in 2013.   

\cite{guo2015mobile} provide an extensive recent review of the various mobile crowdsensing paradigms, their applications, as well challenges and opportunities. 
The main challenges that have been identified repeatedly (see also \cite{he2015privacy}) are privacy and incentives. \cite{Christin2015} provides an analysis of the privacy implications and threats as well as a survey of available privacy-preserving techniques. 

In the application-specific scenario participants are typically incentivized by receiving an application-specific service. In the case of Google Maps a participant gets routing and traffic information in exchange for providing location and speed information. Especially in the case of participatory sensing where explicit user input is required gamification \cite{Deterding:2011:GDE:2181037.2181040} has been used successfully (e.g. Waze) to incentivize participation. However incentivization by service provision leads to contribution only from users who are in need of that particular service and only during specific times. In contrast monetary incentives are a general purpose incentive mechanism. Thus far monetary incentives have mostly been studied in small-scale, localized field studies, as well as in form of game-theoretic incentive mechanism design \cite{7101300}. 
Related to crowdsensing, there is also sensing as a service model \cite{Sheng:2013cm,ETT:ETT2704}, which emphasizes the ad-hoc access to sensing infrastructure or real-time data without investing in infrastructure itself. The sensing as a service model can be seen as an extension of the crowdsensing model to incorporate commercial sensor networks.

In recent years a multitude of architectures enabling general purpose crowdsensing and sensing as a service applications have been presented \cite{6558754,6525603,giannotti2012planetary,Haderer2015,merlino2016mobile}. However none of them has reached reasonable scale.

What has been mainly neglected is how monetary incentives can be provided efficiently under the conditions of a global mobile crowdsensing platform. 
Individual rewards are rather small and data requesters as well as data providers might be distributed globally. Furthermore traditional online payment mechanisms provide additional means for de-anonymization of participants.  


\subsubsection{Unidirectional Mediated Payment Channels}

Mediated payment channels allow the routing of payments through intermediate payment channels by using a concept called \emph{hashlocks} or rather hash timelocked contracts (HTLC). We can express a HTLC between two parties in common words as follows: I pay you if you can provide the secret within a certain period. Assume Alice (A) wants to send a small payment to Bob (B), but Alice and Bob do not have an open payment channel. However, A has a channel with Charlie (C), and C has a channel with B. In principle A could pay C and tell him to pay B. But A has no way to enforce the payment from B to C if the payment channels are independent. We need a way to condition the individual payments on each other. This can be done by conditioning the payments on a common secret generated by the final recipient B. 

Figure \ref{fig:unimedpc} illustrates the flow of an individual payment over two payment channels. We assume that A and C, as well as C and B have already established payment channels, i.e. there exist shared multisignature outputs with some locktimes $T_0$ and $T_0^{'}$\footnote{The actual locktimes are not important. However, they have to be longer than the HTLC locktimes $T_1$ and $T_2$.}. In order to make it more concrete, we assume that the shared outputs have a value of 1 BTC each, and A wants pay B an amount of 0.1 BTC. Moreover, we assume that the channels are fresh, i.e. no payment transactions have been exchanged. First, the final recipient B creates a payment-specific random secret, computes hash(secret), and communicates hash(secret) to C and A. A creates a payment transaction that consumes the shared multisignature output and creates two outputs: (1) an output assigning 0.9 BTC to her, and (2) a HTLC output with a value of 0.1 BTC\footnote{We neglect necessary fees in order to provide a cleaner explanation.}. Figure \ref{fig:pubScriptHTLC} shows the pubScript of such a HTLC output as used in the payment transactions. A provides her signature for the transaction and sends it to C. C may now sign the transaction as well and broadcast it to the Bitcoin network. However, without the secret, C will not be able to claim the 0.1 BTC. Thus C will store the transaction and create another payment transaction addressing C with a HTLC output requiring the same secret. C signs the transaction and sends it to B. B could sign the transaction and broadcast it to the network. Since he knows the secret, he could claim the 0.1 BTC locked in the HTLC output. But he would have to do it before $T_2$. Otherwise B would be able to claim the output. B would claim the HTLC output by broadcasting a transaction that entails the secret. Thus, the secret would be public and C could use it to claim the HTLC output of the payment transaction from A. To ensure that C is always able to do that before A is able to reclaim the value, $T_1$ has to be sufficiently later than $T_2$. If timelocks are selected appropriately, a payment between A and B mediated by C using HTLCs is atomic, i.e. either both payments succeed or both payments fail. This means also that B can just communicate the secret privately to C and A, and they update their shares accordingly. Hence, A would create a transaction that replaces the HTLC output with an output granting C the 0.1 BTC. After signing and sending the transaction to C, C would do the same concerning the HTLC with B, and the payment is concluded without any on-chain transaction.

Mediated payment channels can be used in a hub and spoke architecture to connect a large number of buyers with a large number of sellers with only one payment channel per buyer/seller. Think of 100 of fitness trackers interested in buying air quality data from 1000 consumer air quality sensors in Bejing in order to suggest a healthy running track. There would have to be 100 000 p2p payment channels, but only 1100 if there were a central mediator (c.f. Figure \ref{fig:nmpaymentchannels} and \ref{fig:hubandspoke}).


A particular payment transaction might entail a number of HTLC outputs corresponding to payments to multiple sellers concurrently (c.f. Fig. \ref{fig:htlcflow}.

\begin{figure}
\centering
\includegraphics[scale=0.7]{./externalized/unimedpc.pdf}
\caption{Protocol for the the k-th payment over two payment channels connected by a hash timelocked contract.}
\label{fig:unimedpc}
\end{figure}

\begin{figure}
\begin{lstlisting}[breaklines,mathescape=true]
OP_IF
  OP_DUP OP_HASH160 <PubKeyHash B (C)> OP_EQUALVERIFY OP_CHECKSIG
  HASH160 <Hash160 (secret)> EQUAL
OP_ELSE
  OP_DUP OP_HASH160 <PubKeyHash A (B)> OP_EQUALVERIFY OP_CHECKSIG
  <$T_1$ ($T_2$)> CHECKLOCKTIMEVERIFY DROP
OP_ENDIF
\end{lstlisting} 
\caption{HTLC pubScript of a payment transaction in an unidirectional mediated payment channel setup. The first branch of the conditional can be used by the recipient to claim the output by providing the secret, the second branch can be used by the sender to reclaim their funds after $T_1$ ($T_2$).}
\label{fig:pubScriptHTLC}
\end{figure}



\begin{figure}[ht]
  \centering
  \begin{subfigure}[t]{0.5\linewidth}
    \centering\includegraphics[width=0.95\textwidth]{./externalized/nmpaymentchannels.pdf}
    \caption{Independent payment channels\label{fig:nmpaymentchannels}}
  \end{subfigure}%
  \begin{subfigure}[t]{0.5\linewidth}
    \centering\includegraphics[width=0.95\textwidth]{./externalized/hubandspoke.pdf}
    \caption{Hub and spoke architecture\label{fig:hubandspoke}}
  \end{subfigure}
  \caption{Comparison between individual payment channels between buyers and sellers, and a hub and spoke architecture based on mediated payment channels. The number of channels can be decreased from N x M to N + M.}
\end{figure}


\begin{figure}
\centering
\includegraphics[width=\linewidth]{./externalized/htlcflow.pdf}
\caption{A series of (off-chain) transactions with two concurrent payments. After the secret is released the corresponding HTLC output can be removed and the balances can be updated. The large byte-wise size of HTLC outputs further disincentive abortion of the protocol by broadcasting a transaction including HTLC outputs.}
\label{fig:htlcflow}
\end{figure}

\section{Prototype}

\subsection{Architecture}
The system consists of three main components. A data requester or data buyer application, a data provider application, and a hub application. In order to fulfill the aforementioned principles the payment of the measurement data is done by means of N-to-M mediated payment channels as presented in Section \ref{sec:mediated} where the hub acts as the mediator over which payments are routed. At the same time the hub provides a query-able data provider registry. Noteworthy, the consolidation of payment routing and registry slightly violates our third design principle. However, both components are logically separated. An illustration of the system's architecture is shown in Figure \ref{fig:architecture}.


 \begin{figure}
 \includegraphics[width=\textwidth]{./externalized/Architecture.pdf}
 \caption{Overview of system's architecture.}
 \label{fig:architecture}
 \end{figure}

\section{Implementation}
\label{sec:implementation}

The implementation of N-to-M mediated payment channels is based on extending the payment channel implementation of the bitcoinj library \cite{Bitcoinj}. Bitcoinj is a Java library for working with the Bitcoin protocol and was the first Bitcoin library that had implemented support for payment channels. The Java implementation was chosen because it allows to reuse the implementation across all of the system's components. JavaScript provides the same benefits but libraries were not as mature at that point. Furthermore, bitcoinj implements simplified payment verification (SPV). It allows to verify on-chain transactions without needing to store and verify the entire Bitcoin blockchain. SPV nodes store only the block headers (80 byte instead of a few hundred kilo byte for a typical block), and rely on the depth of a transaction in the blockchain. Block headers contain the root of a Merkle tree that is built of all transaction hashes in the block. Hence, in order to validate a particular transaction, a SPV node has to retrieve only a subset of the block data, i.e. the corresponding branch of the Merkle tree.
Only due to the SPV is it viable to use Bitcoin on resource constrained devices without trusting third party services.

The data requester and the hub are implemented as native Java applications. The data provider is implemented as an Android application. Before presenting the implementation of the individual components we start with the common building block on which all peer-to-peer payments are based. 

\subsection{Mediated payment channels using HTLCs}

We follow the client-server architecture of the payment channel implementation of bitcoinj \cite{BitcoinjPC}. A client is always on the sending side of a payment while a server is on the receiving side (c.f. Figure \ref{fig:architecture}). Since the hub acts as mediator or router for payments between requesters and providers it has to receive payments from requesters and forward those payments to providers. Hence, the hub incorporates both, server and client.

\begin{table}
  \centering
    \caption{The five layers of the HTLC payment channel implementation.}
  \begin{tabular}{|c|l|}
    \hline
    \tabhead{Layer} &
    \tabhead{Description} \\
    \hline
    Driver Application & \multicolumn{1}{|p{0.5\columnwidth}|}{Provides the API to access lower layers}\\
    \hline
    Channel Connection & \multicolumn{1}{|p{0.5\columnwidth}|}{Provides the interface to the network}\\
    \hline
    Payment Channel & \multicolumn{1}{|p{0.5\columnwidth}|}{Handles messages from/to the network and passes instructions/gets results to/from the lower layer}\\
    \hline
    Channel State Machine & \multicolumn{1}{|p{0.5\columnwidth}|}{Handles the state of the payment channel}\\\\
    \hline
    HTLC State Machines & \multicolumn{1}{|p{0.5\columnwidth}|}{Handles the state of the HTLC}\\\\
    \hline
  \end{tabular}
  \label{tbl:layers}
\end{table}

In order to implement hashed HTLCs we extended the four layers of bitcoinj's payment channel implementation with a fifth layer that is concerned with keeping track of the HTLC flow. The five layers are described briefly in Table \ref{tbl:layers}. Details about the state machines and sequence diagrams can be found in BLINDED FOR REVIEW.
Messages and transactions are serialized using Google protocol buffers\footnote{https://developers.google.com/protocol-buffers/} and are exchanged between the components over TCP connections. The payment channel setup and the HTLC payments in the implementation are slightly more complex and involve even more communication than it is shown in Table \ref{tabl:htlcflow} because we use the interactive approach for channel refunds. 

\subsection{Data provider}

The data provider is implemented as an Android application. The application allows users to offer measurement data from various sensors at an adjustable price denominated in satoshis\footnote{A satoshi is the smallest fraction of a bitcoin and corresponds to $10^{-8}$ bitcoins}. Table \ref{tbl:sensors} provides an overview of possible sensors. With the initial boot of the application, a local wallet is generated. The wallet is responsible for generating and storing key pairs and signing transactions. Furthermore a background service is started that manages the payment channel server and data delivery. If a user decides to offer measurement data from a specific sensor, the application registers the sensor at the global sensor registry. The initial registration then triggers the setup of a payment channel between the hub and the Android application. As soon as the payment channel is established the application is ready to serve requests from buyers.


\begin{table}
  \centering
  \caption{Examples of virtual and physical sensors available in smartphones.}
  \begin{tabular}{|c|l|l|}
    \hline
    \tabhead{Sensor} &
    \tabhead{Type} &
    \tabhead{Application} \\
    \hline
    Barometric pressure & physical & Weather prediction \\
    \hline
    Location & both & People flow  \\
    \hline
    Network strength & physical & Coverage maps \\
    \hline
    Installed apps & virtual & Inter-app correlations    \\
    \hline
    Transportation mode & virtual & City monitoring \\
    \hline
    Steps & virtual & Health monitoring  \\
    \hline
  \end{tabular}
  \label{tbl:sensors}
\end{table}


\subsection{Data requester}
The data requester is implemented as a Java application. The requester application allows to query the sensor registry. In the prototype this is achieved using simple console commands. Table \ref{tbl:commands} shows the available commands.

\begin{table}
  \centering
    \caption{Available commands of the data requester console.}
  \begin{tabular}{|c|l|}
    \hline
    \tabhead{Command} &
    \tabhead{Description} \\
    \hline
    STATS NODES & Returns all connected sensor nodes \\
    \hline
    STATS SENSORS & Returns available sensor types \\
    \hline
    SELECT SENSOR=$<$type$>$ & \multicolumn{1}{|p{0.5\columnwidth}|}{Returns sensors of type $<$type$>$ with pricing information} \\
    \hline
    BUY $<$type$>$ FROM $<$node$>$ & \multicolumn{1}{|p{0.5\columnwidth}|}{Buys the current measurement value of sensor $<$type$>$ from node $<$node$>$} \\
    \hline
   
    \hline
  \end{tabular}
  \label{tbl:commands}
\end{table}



\subsection{Hub}
The hub acts as the central point of coordination. Data providers register their sensors with the sensor registry that the hub provides, and data requesters are able to query the registry. Furthermore, the hub is responsible for mediating the payments between buyers and sellers. The hub instantiates a payment channel client and a payment channel server


\section{Evaluation}

In this section we first present a conceptual evaluation based on the principles presented in Sec. \ref{sec:principles}. Thereafter, we proceed with an empirical evaluation of the performance, and the transaction costs. 

\subsection{Conceptual Evaluation}

\subsubsection{Low barrier for participation}
A potential participant on the data provider side has to download and install the data provider smartphone application. The data provider does not need to own any bitcoins. Furthermore no manual sign-up or registration is required. This means also the system has global reach and could provide a low-income stream to smartphone users in developing countries.

A data requester does not have to register but the wallet of the data requester application needs to have some amount of bitcoin in order to initiate a payment channel with the hub.

\subsubsection{Incentivize participation with micropayments}
Individual payments can be as low as 1 satoshi. In April 2016, the exchange rate is around 450\$/BTC. In regard to this exchange rate individual payments can be as low as 4.5 $\mu$\$. Even smaller payments could be implemented with probabilistic payment schemes \cite{Rivest1997,Pass:2015:MDC:2810103.2813713}. Transaction fees for setting up and closing payment channels will be discussed in Sec. \ref{sec:fees}. 

\subsubsection{Limited trust in hub provider}
The usage of HTLCs to interconnect payment channels of data providers and data requesters allows atomic payments between those parties. This means either the payment happens in both channels or it happens not at all. 

\subsubsection{Anonymity}
Obviously there is a risk of de-anonymization for data providers depending on the data they are selling. For example selling of location data or wifi SSIDs could be used to identify a data provider. We restrict the evaluation of anonymity to the payment aspect. Bitcoin payments themselves are pseudonymous. Transactions only entail cryptographic public keys or hashes thereof. If users . In addition, due to the usage of mediated payment channels, there is no public link between buyers and sellers. Only transactions between buyers and the hub, and sellers and the hub end up in the blockchain. Transactions entailing the linking secret are kept private. 

In addition, there is a risk of de-anonymization based on the IP addresses. However since all communications are based on TCP, it would be possibly to run the system over Tor\footnote{https://www.torproject.org/}.


\subsection{Empirical Evaluation}

\subsection{Performance}

For setting up and closing of payment channels the Bitcoin network itself is dictating the performance. If fees are sufficient a transaction can be assumed to be included in the blockchain in 10 min on average. Depending on the value of the transaction the receiver of the funds might wait until the transaction has six confirmations. This means that there are five additional blocks on top of the block entailing the transaction. 
In the following we assume that payment channels are already in place and evaluate the time is takes to complete a payment. The experiment was done by instantiating 100 data requesters and one data provider. Each requester buys one measurement value from the data provider. Table \ref{tbl:performance} shows a breakdown of the mean times and the standard deviation according to the payment steps. The results request some explanation. First, the HTLC setup process between hub and provider seems to take almost twice as long as the the HTLC setup between requester and hub. The reason for this is that the hub-provider setup is initiated with generating the secret but then has to wait until requester-hub setup is finished. If the hub would initiate the HTLC setup with the provider immediately the hub would risk paying the provider without being able to pull the respective funds from the requester. Second, the teardown process (i.e. updating the setup transacting by assigning the value of HTLC output to the payee.) takes less than half the time of the setup process. The reason for that is mainly that in the actual implementation the interactive refund approach was used. This means that there is an additional transaction which has to be signed by both parties and has to be transferred from payer to payee and back. Thus, we expect a comparable performance between setup and teardown if the non-interactive approach is used. Furthermore, we see that the steps that involve the provider take more time. This is because signing and signature verification are computation intensive tasks and the data provider is a comparably weak smartphone.

\begin{table}
\centering
\caption{Payment duration breakdown}
\begin{tabular}{|l|r|r|}
\hline
{\textbf Payment } & \multicolumn{1}{l|}{{\textbf Mean}} & \multicolumn{1}{l|}{{\textbf St. Dev.}} \\ \hline
Requester-Hub setup             & 479.3 ms                 & 78.8 ms                      \\ \hline
Hub-Provider setup            & 1015.1 ms                & 151.4 ms                     \\ \hline
Hub-Provider teardown         & 199.6 ms                 & 68.3 ms                      \\ \hline
Buyer-Provider teardown          & 146.6 ms                 & 10.1 ms                      \\ \hline
\end{tabular}

\label{tbl:performance}
\end{table} 


\subsection{Transaction costs}
\label{sec:fees}
Transaction fees in Bitcoin are in principle voluntary. The party that creates a transaction specifies the transaction fees as the difference of the sum of the value of all inputs and the sum of the value of all outputs. However, each miner can decide individually if she will accept the transaction for a candidate block. Since there is a limit on the maximum size of a block, and larger blocks need more time to propagate the network, rational miners will select transactions with higher transaction fees. This is also the reason why transaction fees are based on the size of the transaction instead of its value. The transaction size is mainly defined by the number and size of input and output scripts and signatures. Figure \ref{fig:tx_fees} shows the daily averages of transaction fees in USD/kb for the period of one year.

\begin{figure}[!t]
\centering
\includegraphics[width=\textwidth]{./externalized/fees_per_kb}
\caption{Bitcoin transaction fees in USD/kb. Each point represents a daily average.}
\label{fig:tx_fees}
\end{figure}


\begin{table}
  \centering
  \caption{Approximate sizes and costs for Bitcoin transactions}
  \begin{tabular}{|c|l|l|}
    \hline
    \tabhead{Transaction Type} &
    \tabhead{Size} &
    \tabhead{Cost} \\
    \hline
    Standard P2PKH & \multicolumn{1}{|p{0.5\columnwidth}|}{} & \\
    \hline
    Funding & \multicolumn{1}{|p{0.5\columnwidth}|}{} & \\
    \hline
    HTLC setup & \multicolumn{1}{|p{0.5\columnwidth}|}{} & \\
    \hline
    Settlement & \multicolumn{1}{|p{0.5\columnwidth}|}{} & \\
    \hline
    Forfeiture & \multicolumn{1}{|p{0.5\columnwidth}|}{} & \\
    \hline
  \end{tabular}
  \label{tbl:fees}
\end{table}


\section{Discussion}

\subsection{Unidirectional Payment Channels Revisited}

\subsection{Challenges}

\subsubsection{Locking of funds}

The usage of payment channels involves locking of funds. This is a problem in particular for the payment hub. One of our principles was that the onboarding of new data producers should be frictionless. Hence, we assume that data producers do not have any bitcoins to begin with. When a data producer registers a sensor with the registry the payment hub initiates a payment channel with the data producer. As described in Sec. \ref{sec:mediated} this involves broadcasting a funding transaction to the Bitcoin network. Thereby the hub has to lock capital in the channel and has to provide a fee in order for the transaction to be mined. The operator has to balance between locking a large amount and the transaction costs involved in closing and reopening a channel because of depletion. In general the amount of locked bitcoins is proportional to the number of data producers.
Since there is no cost for a data producer to register, a malicious actor might attack the hub by registering a large amount of fake sensors. A strategy to disincentive such a behavior would be to request proof of work from the sensor as part of the registration process.

The situation might change radically when payment channel networks become ubiquitous.  

\subsubsection{Malicious Sensors}

Malicious data producers that offer fake data are not only a problem for the hub but an obstacle to the usefulness of the system as a whole. If the quality and correctness of the offered data is too low data requesters will be reluctant to use the system.
A typical approach to provide an indication about the trustworthiness of a producer is the introduction of a reputation system. Typical reputation system are centrally managed and thus may be subject to censorship. \cite{s16060776} describes a clever reputation system which interprets bitcoin payments from a particular address (or addresses) as reputation, and presents a protocol based on the CoinJoin protocol to transfer ownership. However, the scheme is not perfectly applicable in the present scenario for mainly two reasons: (1) a public reputation score is only visible for a data requester after the data provider closes a channel, and (2) the data requester is not able to see if the reputation originates from a large number of independent data providers or from repeated payments of a small number of requesters. Only repeated payments can indicate satisfactory transactions. In contrast, in \cite{s16060776} there is only one data collection server that verifies all data and pays the data providers accordingly.

\subsubsection{Anonymity}




\section{Current Developments: Routable Bidirectional Payment Networks}

\subsection{Lightning Network}

\subsection{Micropayments with Ethereum}



\section{Conclusion}
The conclusion goes here.


\section{Background}

\subsection{Smart Objects}

\cite{kortuem2010smart} define smart objects as \emph{autonomous physical/digital objects augmented with sensing, processing, and network capabilities}. 

\subsection{Machine-to-Machine (Micro) Payments}

\subsection{Smart Property}




\section{Economic Objects}

\subsection{Basic Characteristics}

\subsubsection{M2M Micropayments}

\subsubsection{Blockchain-mediated Control}

\subsection{Requirements}

\subsubsection{Object}

\subsubsection{Blockchain}

\subsection{Patterns}

\subsubsection{Low-trust Atomic Trades}


Smart property can be sold via the Internet in an atomic process without third party escrow. Transfer of ownership\footnote{We do not necessarily mean legal rights of ownership, but the ability to control the property} and transfer of money happen at once.

\paragraph{Implementation based on Bitcoin}

The simplest way to achieve atomicity in Bitcoin is by executing both parts of the trade in a single transaction. In Fig. \ref{fig:smartproperty} an atomic trade protocol based on \cite{smartproperty2011} is shown. The protocol may either be initiated by the seller (A) or by the buyer (B). We assume A initiates the protocol and sends the price, a Bitcoin address where she would like to receive the payment, and a reference to the UTXO that represents the smart property (ownership output) to B. B creates a new ownership key pair and prepares a transaction that sends the price to A and transfers the ownership to the newly created ownership key. B signs the transaction and sends it to A. To accept the offer A signs the ownership input and broadcasts the transaction to the Bitcoin network. 

After a number of confirmations, B can provide a SPV proof to the smart property. Therefore, B provides a selection of subsequent block headers entailing the header of the block the transaction is in, plus its merkle proof. The smart property is able to reason about the amount of proof-of-work that was spent to generate the blocks. In case of doubt, the smart property can ask for more block headers.

\begin{figure}[!t]
    \centering
    \includegraphics[width=\linewidth]{./externalized/smartproperty}
    \caption{Protocol of atomic trade of a smart property on the Bitcoin blockchain. The smart property does not need to interact with the Bitcoin network itself.}
    \label{fig:smartproperty}
  \end{figure}

In this protocol the smart property is not part of the Bitcoin network, but is able to verify SPV proofs provided by a potentially untrusted party. 

\paragraph{Implementation based on Ethereum}

The following contract illustrates a contract representing a tradable smart property in Ethereum. The original issuer of the contract is identified as the owner. In practice this would be the manufacturer. Ethereum does not support collaborative transactions, i.e. transactions with operations authorized by different entities. Therefore, selling has to be implemented in a two-stage process. In the contract below, we do this by implementing a \emph{sell function} which can be only called by the current owner (through the \emph{onlyOwner modifier}). The \emph{sell function} has two parameters: the price, and a buyer address. The \emph{buyer parameter} allows to explicitly state a seller. Otherwise everyone who would pay the price using the \emph{buy function} would be able to buy the smart property.

\newpage
\begin{lstlisting}[breaklines,basicstyle=\tiny]
contract SmartProperty {
    
    address owner;
    address buyer;
    bool onlyBuyerFlag;
    bool isOnSale;
    uint price;
    
    modifier onlyOwner() {
        if (msg.sender != owner) throw;
    }
    
    modifier onlyBuyer(bool flag) {
        if (flag && msg.sender != buyer) throw;
    }
    
    function SmartProperty() {
        owner = msg.sender;
    }
    
    function sell(uint _price, address _buyer) onlyOwner() {
        isOnSale = true;
        if (_buyer != 0) {
            onlyBuyerFlag = true;
            buyer = _buyer;
        } else {
            onlyBuyerFlag = false;
        }
        price = _price;
    }
    
    function stopSale() onlyOwner() {
        isOnSale = false;
    }
        
    function buy() onlyBuyer(onlyBuyerFlag) {
        if (isOnSale) {
            if (msg.value == price) {
                isOnSale = false;
                owner = msg.sender;
                owner.send(msg.value);
                
            } else {
                msg.sender.send(msg.value);
            }
        } else {
            msg.sender.send(msg.value);
        }
    }
}
\end{lstlisting}


\subsubsection{Low-trust renting}

Time-restricted transfer of ownership with adjustable counterparty risk. Smart property can be the basis for a peer-to-peer sharing\footnote{In the sense of Uber and AirBnB.} ecosystem. 

\paragraph{Implementation based on Bitcoin}

The idea is to combine shared ownership and the unidirectional payment channel. Therefore, the parties create a transaction that creates a multi-signature ownership output for A (owner) and B (renter), and a 2-of-2 multi-signature output where B deposits some amount. Furthermore, both parties create a timelocked refund transaction that allocates the ownership output back to A, and the deposit back to B. The timelock should cover the maximal renting period, and the deposit should cover the renting price for that period. B can now pay in small increments with off-chain payment transactions that entail an additional ownership output that assign the car ownership back to A. Because of Bitcoin's current malleability issue it is important that the order of providing signatures is such that only A is able to broadcast the funding transaction. Furthermore, A needs also a (tiny) payment transaction from B before broadcasting the funding transaction. Otherwise B could just use the smart property without paying for the entire maximal renting period, since A would not have a transaction that is immediately valid. Figure \ref{fig:smartproperty_renting} illustrates the protocol in more detail.

\begin{figure}[!t]
    \centering
    \includegraphics[width=\linewidth]{./externalized/smartproperty_renting}
    \caption{Protocol for trust-minimized renting of smart property using a Bitcoin payment channel.}
    \label{fig:smartproperty_renting}
  \end{figure}

\paragraph{Implementation based on Ethereum}


The following contract implements low-trust renting of smart property on Ethereum. Note that the \emph{Rentable Property} contracts inherits properties and functions from the \emph{SmartProperty} contract. When deploying the contract the owner sets a deposit a renter has to provide and a time based renting price. The main functions are a \emph{rent} and a \emph{returnPropery} function. The \emph{rent} function allows an arbitrary account or contract to rent the smart property by providing a deposit. The deposit is held in the contract, and can only be released by the rules of the contract. Neither the owner, nor the renter have control over the deposit. The contract keeps also track of the block in which the rent function was executed. This \emph{startBlock} serves as the begin of the renting period as measured in time of blocks. The Ethereum network has an average block generation rate of approximately 14s. 
The software running locally on the smart property has now to be notified that it must obey orders signed by the renter for a maximal period defined by the price and the deposit. 
The renter can later call the \emph{returnProperty} function which calculates the price for the renting period and distributes the deposit accordingly. If the renter does not return the property in time her access/control rights expire. 

\begin{lstlisting}[breaklines,basicstyle=\tiny]
contract RentableProperty is SmartProperty {
    
    address renter;
    uint startBlock;
    uint pricePerBlock;
    uint deposit;
    bool isRentable = true;
    
    modifier onlyRenter() {
        if (msg.sender != renter) throw;
    } 
    
    function RentableProperty(uint _pricePerBlock, uint _deposit) {
        owner = msg.sender;
        pricePerBlock = _pricePerBlock;
        deposit = _deposit;
    }
    
    function notRentable() onlyOwner() {
        isRentable = false;
    }
    
    function rentProperty() {
        if (isRentable) {
            if (msg.value >= deposit) {
                renter = msg.sender;
                startBlock = block.number;
                isRentable = false;
                if (msg.value > deposit) {
                    msg.sender.send(msg.value - deposit);
                }
            } else throw;
        } else throw;
    }
    
    function returnProperty() onlyRenter {
        renter = 0;
        if (deposit > price*(block.number-startBlock)) {
            owner.send(price*(block.number-startBlock));
            msg.sender.send(deposit-price*(block.number-startBlock));
        } else {
            owner.send(deposit);
        }
        isRentable = true;
    }
}
\end{lstlisting}

\subsubsection{Liquid property}

Smart property can be used as collateral for loans. The underlying principle is that property ownership is transferred to the lender if repayment terms are not met. Thereby the loan gets securitized by the smart property. Because of the global permissionless nature of cryptocurrencies, a global market for loans on individual smart properties can emerge which lowers the cost of loans. Moreover, since transactions are public, a borrower is able to prove timely payments of earlier loans. 

\paragraph{Implementation based on Bitcoin}

We will implement the following contract. The owner of a smart property (B) wants a loan of size $L$ and provides the property as security for a creditor (A). If the owner (and debtor) does not repay the loan (plus interest) until time $T$, ownership of the property will be transferred to the creditor.

After A and B agreed on terms, B creates the \emph{loan transaction} spending his ownership output and creating a timelocked 2-of-2 multi-signature ownership output that can be redeemed either collaboratively by A and B, or by A alone after time $T$. Furthermore, B adds an output that credits him with loan $L$, leaving the input, providing the loan, for A to add. 
\begin{figure}
\begin{lstlisting}
OP_IF 
    2 <pubKeyA><pubKeyB> 2
    OP_CHECKMULTISIG
OP_ELSE
    <T> OP_CHECKLOCKTIMEVERIFY OP_DROP
OP_END    
\end{lstlisting}
\caption{PubScript of timelocked 2-of-2 multi-signature ownership output.}
\end{figure}

B sends the partial transaction to A, who prepares the \emph{settlement transaction}. The settlement transaction reassigns the ownership back to B, and credits A with the loan plus interest. A completes the loan transaction and partially signs the settlement transaction, and sends both transactions back to B. B can then broadcast the loan transaction to the Bitcoin network. 
If B can provide an input to the settlement transaction covering $L+\Delta$ before time $T$, B can complete the settlement transaction and regain sole ownership of the property. However, if the settlement transaction does not enter the blockchain before $T$, then A is able to claim sole ownership.

Another Bitcoin-based protocol for smart property as a collateral for loans is described in \cite{smartproperty2011}. However, the protocol has the problem that a creditor has the ability to resell the property immediately without giving the debtor a chance to repay the loan. In the protocol described above this is prevented by use of the multi-signature. 

\paragraph{Implementation based on Ethereum}

A simple implementation of property liquidification defines a loan and a latest payDay. The owner  

\begin{lstlisting}[breaklines,basicstyle=\tiny]
contract LiquidProperty is SmartProperty {
    
    address lender;
    uint loan;
    uint payDay;
    uint paid = 0;
    
    bool isLiquid = false;
    
    function LiquidProperty(uint _loan, uint _payDay) {
        owner = msg.sender;
        loan = _loan;
        payDay = _payDay;
    }
    
    function giveLoan() {
        if (msg.value < loan || isLiquid) throw;
        isLiquid = true;
        lender = msg.sender;
        owner.send(msg.value);
    }
    
    function pay() {
        if (!isLiquid) throw;
        paid = paid + msg.value;
        lender.send(msg.value);
    }
    
    function enforce() {
        if (!isLiquid) throw;
        if (paid >= loan) {
            isLiquid = false;
        } else if (block.number > payDay) {
            owner = lender;
            isLiquid = false;
        }
    }
}
\end{lstlisting}

\subsubsection{Issuing tokens}





\subsubsection{Trustworthy hardware and software}

The smart property itself has to be trustworthy. A potential seller has to be sure that the software running the smart property follows the protocol. In particular, after a sale the smart property has to be controlled by the buyers ownership key, whereas the sellers ownership key is disabled. Hard- and software of the smart property must be able to ensure that is has not been tampered with. Ideally, software is open source and independently verifiable. Smart property may prove its integrity to a third party using trusted computing. Thereby the smart property may also prove its current state, i.e mileage and services in the case of a car. A certificate by the manufacturer may be used to ascertain provenance. Alternatively, provenance could be inferred from the signature of the original registrar.

\subsubsection{Kill switch}

Liquidification requires that the smart property can be rendered temporarily unusable. Otherwise the contract between lender and borrower is not remotely enforceable, and the borrower is able to use the property without paying back the loan. Enforcement of the contract would have to happen via the judicature system which is expensive and cumbersome if borrower and lender are in different jurisdictions. 

If the use of a kill switch is possible is dependent of the actual smart property. Cars, for example, are already equipped with immobilizers which are reasonably hard to bypass. Popular solar home systems, as deployed in Africa and South-East Asia, can also be remotely deactivated. Smart property where the value of the object itself or its components is high independent of a function or service that can be switched off, however, is not suitable for this scheme. 

\subsubsection{Native on-chain currency}
Selling, renting, and liquidification all involve currency. Only a fungible native token of value allows these processes without additional counterparty risk. 

\subsubsection{Light-client proofs}

In many cases smart property will not be able to interact with the blockchain network directly, but has to rely on proofs provided by untrusted parties. For example, a buyer has to be able to prove to the car that it now belongs to him. The underlying blockchain and the protocols should account for such a scenario, since e.g. a car should be able to be started even if there is no Internet connectivity.



%\subsection{Why has smart property to live on a blockchain?}

Let us stay with the example of a car. Alice owns the car and the ownership is augmented by the possession of a cryptographic key. Assume Alice wants to transfer the ownership to Bob. Of course, she can not send him the keys, since the key is nothing more than a digital file, and copying a digital file has zero marginal cost. In other words, we face the classic double spending problem. However, we can think of a more complex protocol. Instead of sending a key to Bob, Bob is creating a new private-public ownership key pair. Bob sends the public part to Bob, and Alice uses her current ownership key to initiate an ownership transfer to Bob's key. The car itself acts as a deterministic trusted party. We will later discuss avenues to give Bob more certainty about the trustworthiness of the smart property.
So far, we only used public key cryptography, and there was no need for a blockchain. However, typically someone does not just transfer ownership to someone else. Alice wants something in return. In most cases this something is money. If Alice and Bob meet in person they are able to do the ownership transfer protocol and the cash payment concurrently. However, if the trade happens via the Internet, one party has always be the first and may be defrauded by the counterparty. Traditional payments can be undone. Thus, inhibiting risk for the seller. Simple bitcoin payments mitigate this risk, but expose the buyer to the risk of not getting the car. Traditionally, third party escrow has been used to solve this dilemma.
However, if ownership of the car is represented on a blockchain, Alice and Bob can perform an atomic trade. Ownership and money is exchanged in one single process. If one part fails, the other part will fail as well. If one part succeeds, the other part will succeed as well. A similar logic can be applied to all of the capabilities smart property provides.

\subsubsection{Contract Immutability}

\subsubsection{Permissioned Blockchains}

A permissioned blockchain is a blockchain, in which transaction processing is performed
by a predefined list of subjects with known identities \cite{BitFuryPermissioned2015}. In context, a consortium of device manufacturers could operate a permissioned blockchain collaboratively. Permissioned blockchains do not need a native token of value, and typically do not have one since fair distribution of tokens, such that trust and value can emerge, is non-trivial. However, there are ways to bring external value in and out of permissioned blockchains in order to facilitate economic interactions. 
\paragraph{IOU Issuance}
One approach is to allow specific parties to issue IOUs on the blockchain. These parties could be the device manufacturers themselves but more probably banks and financial service providers. In this case a buyer interested in a smart property would send money off-chain to a third party which then issues tokens on the blockchain. In addition, the seller has to accept those tokens. 

\paragraph{Cross-chain/ledger protocols}
A better approach that requires less trust is the application of a cross-chain/ledger protocol. We discussed sidechains already, which allow the immobilization of a token in one chain and subsequent unlocking in another chain. In addition, there is the Interledger protocol \cite{hope2016interledger} and the atomic cross-chain trading protocol \cite{atomiccrosschaintrading}
