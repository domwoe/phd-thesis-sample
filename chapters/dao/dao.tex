\chapter{Smart Property}
\label{sec:dao}


\section{Introduction}

The term smart property was introduced in \cite{szabo1997} as the manifestation of smart contracts. He gives the example of a car:
\begin{quote}
Consider a hypothetical digital security system for automobiles. The smart contract design strategy suggests that we successively refine security protocols to more fully embed in a property the contractual terms which deal with it. These protocols would give control of the cryptographic keys for operating the property to the person who rightfully owns that property, based on the terms of the contract. In the most straightforward implementation, the car can be rendered inoperable unless the proper challenge-response protocol is completed with its rightful owner, preventing theft.
\end{quote}

And Indeed, modern car keys use smart card technology and cryptographic primitives to authenticate the driver and provide the functionality described by Szabo. However legal ownership is independent of possession of the cryptographic key. Noteworthy, currently the possession of a physical document authenticates the legal ownership. 

With the inception of Bitcoin and blockchain technology, the term smart property usually refers to property whose ownership is controlled by a blockchain.
The Bitcoin wiki notes that smart property may also include non-physical property like shares in a company or access rights to a remote computer. 

However we beg to differ. We understand smart property as a smart object with a digital representation on a blockchain. The digital representation is controlled by cryptographic keys. These cryptographic keys empower the holder to exert immediate and full control over the smart object. Shares of a company, in contrast, have complex counterparty risk, and are mediated by law. Therefore, company shares are a bad candidate for smart property. At least for now, as we will discuss later on. 

In this chapter, we will first discuss the necessity of a blockchain. Thereafter, we list and describe capabilities of smart property. These capabilities lay down certain requirements for the smart property and the associated blockchain. Next, we present conceptual protocols based on Bitcoin and Ethereum, followed by a discussion. 

\section{Why has smart property to live on a blockchain?}

Let us stay with the example of a car. Alice owns the car and the ownership is augmented by the possession of a cryptographic key. Assume Alice wants to transfer the ownership to Bob. Of course, she can not send him the keys, since the key is a nothing more than a digital file, and copying a digital file has zero marginal cost. In other words, we face the classic double spending problem. However, we can think of a more complex protocol. Instead of sending a key to Bob, Bob is creating a new private-public ownership key pair. Bob sends the public part to Bob, and Alice uses her current ownership key to initiate an ownership transfer to Bob's key. The car itself acts as a deterministic trusted party. We will later discuss avenues to give Bob more certainty about the trustworthiness of the smart property.
So far, we only used public key cryptography, and there was no need for a blockchain. However, typically someone does not just transfer ownership to someone else. Alice wants something in return. In most cases this something is money. If Alice and Bob meet in person they are able to do the ownership transfer protocol and the cash payment concurrently. However, if the trade happens via the Internet, one party has always be the first and may be defrauded by the counterparty. Traditional payments can be undone. Thus, inhibiting risk for the seller. Simple bitcoin payments mitigate this risk, but expose the buyer to the risk of not getting the car. Traditionally, third party escrow has been used to solve this dilemma.
However, if ownership of the car is represented on a blockchain, Alice and Bob can perform an atomic trade. Ownership and money is exchanged in one single process. If one part fails, the other part will fail as well. If one part succeeds, the other part will succeed as well. A similar logic can be applied to all of the capabilities smart property is able to provide.


\section{Capabilities of smart property}

\subsection{Trust-minimized atomic trades}

Smart property can be sold via the Internet in an atomic process without third party escrow. Transfer of ownership rights\footnote{We do not necessarily mean legal rights of ownership.} and transfer of money happen at once. 

\subsection{Trust-minimized renting}

Time-restricted transfer of ownership with adjustable counterparty risk. Smart property can be the basis for a peer-to-peer sharing\footnote{In the sense of Uber and AirBnB.} ecosystem. 

\subsection{Shared ownership}

Multi-party ownership of smart property. Complex scenarios are possible. For example selling of a smart property may need approval from all parties and revenue is distributed evenly.

\subsection{Liquidification of property}

Smart property can be used as collateral for loans. Various models are possible. The underlying principle is that ownership is transferred to the lender automatically if repayment terms are not met. Thereby the loan is securitized by the smart property. Because of the global permissionless nature of cryptocurrencies, a global market for loans on individual smart properties can emerge which lowers the cost of the loan. Moreover, since transactions are public, a borrower is able to prove timely payments of earlier loans. 

\section{Requirements}

\subsection{Trustworthy hardware and software}

The smart property itself has to be trustworthy. A potential seller has to be sure that the software running the smart property follows the protocol. In particular, after a sale the smart property has to be controlled by the buyers ownership key, whereas the sellers ownership key is disabled. Hard- and software of the smart property must be able to ensure that is has not been tampered with. Ideally software is open source and independently verifiable. Smart property may prove its integrity to a third party using trusted computing. Thereby the smart property may also prove its current state, i.e mileage and services in the case of a car. A certificate by the manufacturer may be used to ascertain provenance. Alternatively, provenance could be inferred from the signature of the original registrar.

\subsection{Kill switch}

Liquidification requires that the smart property can be rendered unusable temporarily. Otherwise the contract between lender and borrower is not enforceable remotely, and the borrower is able to use the property without paying the back the loan. Enforcement of the contract would have to happen via the judicature system which is cumbersome and expensive if borrower and lender are in different jurisdictions. 

If the use of a kill switch is possible is dependent of the actual smart property. Cars, for example, are already equipped with immobilizers which are reasonably hard to deactivate. Popular Solar Home Systems as deployed in Africa and South-East Asia can also be remotely deactivated. Smart property where the value of the object itself or its components is high independent of a function that can be switched off is not suitable for this scheme. 

Then there is the question who is able to flip the kill switch or does it happen automatically?

\subsection{On-chain currency}
Selling, renting , and liquidification all involves currency. Therefore, a fungible token of value that is easy to procure is necessary. 
Otherwise protocols like atomic cross-chain swaps or Interledger might be used but complicate the processes.

\subsection{Light-client proofs}

In many cases smart property will not be able to interact with the blockchain network directly, but has to rely on proofs. For example a buyer has to be able to prove to the car that it now belongs to him. The underlying blockchain and the protocols should account for such a scenario, since a car should be able to be started even if there is no Internet connectivity. 


\section{Basic Implementation}

\subsection{Bitcoin}

\begin{figure}[!t]
    \centering
    \includegraphics[width=\linewidth]{./externalized/smartproperty}
    \caption{Protocol of atomic trade of a smart property on the Bitcoin blockchain. The smart property does not need to interact with the Bitcoin network itself.}
    \label{fig:smartproperty}
  \end{figure}

\subsection{Ethereum}


\section{Comparison}

\section{Issues}

\subsection{Security properties of the underlying blockchain}

\subsection{Security}


\section{A more complex case: Solar micro utilities}




https://www.kpmg.com/IN/en/IssuesAndInsights/ArticlesPublications/Documents/ENRich2015.pdf
Rooftop solar is already competitive in some areas in India. Storage costs are still high, but falling. Stress the need for private investments and the need to tap global capital markets.

\section{Motivation}

Hundreds of million people globally lack access to energy. Grids are unreliable and last mile connection costs can be inhibitive. Off-grid energy technologies of various sizes are being made available by private companies as well non-profit organizations. In terms of scale, the most important technology are Solar Home Systems (SHS)

A SHS is composed of a small solar panel, a battery, LED lights, and USB charging. The capacity of the system ranges typically between 10 and 250 Watt. Larger systems typically comprise fans and TVs. Small systems can be installed by the user but larger systems need a technician.

World market production of solar panels focuses on a module size of 250 Watt. Hence the price per Watt is lowest for 250 Watt modules. However, 250 Watt exceed the typical demand of a rural off-grid household. Thus micro-utility (c.f. Grameen Shakti) and peer-to-peer microgrids (c.f. uLink) models are appealing. In these models a local entrepreneur shares the generated electricity with her neighbors. Therefore, energy consumption should be able to be metered and the local entrepreneur needs to collect payments.

People that are in need of SHSs are typically at the bottom of the pyramid. They lack the capital to buy a SHS and have no access to traditional financing. Innovative financing schemes that are based on mobile payments and the ability to remote disable the system have been developed to bring those systems to more poor households. These schemes can be divided in two main categories: Pay-for-service and lease-to-own. In the pay-for-service model the SHS manufacturer is the owner of the system and the user pays has to pay fee for its usage. If the user does not pay electricity access is turned off and after some time the system gets taken back. In the lease-to-own model the user also has to pay for the usage but she also pays a principal so that at some point the ownership gets transferred to her and the system gets permanently unlocked. 

A prominent example for the lease-to-own model is M-Kopa which operates in East Africa. M-Kopa offers also that the user may switch back to the lease-to-own model after the user has fully paid the system in order to get financing for some other product. Notably, the user does not get a loan to buy a product somewhere but she gets a product that is distributed by M-Kopa. 

Companies like M-Kopa need large capital in order to finance the distribution of their systems. Besides debt and equity financing some companies have securitized their implicit loans to users. This practice is however only viable for the larger companies that already have reached some scale. 

\section{Problem statement}


\section{General Model}

When transaction costs become negligible and contract enforcement gets automated large-scale principles can be applied on ever smaller scales. Joint-stock companies have been invented to share risk but also rewards between shareholders in order to finance large-scale commercial endeavors. However regulatory burden and hence bureaucratic costs of such a structure are high in order to protect shareholders.   

Starting as a Uber driver in Mexico City or as a solar micro-utility provider in India are much smaller endeavors. Nevertheless, financing a car or a solar home system can be challenging. Can we transfer the idea of a join-stock company on the level of a product itself? Let us think of the car or the solar home system as a business entity on its own. Cryptocurrencies allow a connected product to accept payments and to have the revenue at its command. As a business entity it may issue shares as well. If those shares are issued as tokens on the same network on which the cryptocurrency lives, then the revenue can be distributed autonomously. In this scenario there is only one party that has to be trusted: the connected product itself. In principle, dividends can be distributed transparently, in real-time, and around the globe.

It has to be ensured that the product only operates in a well-defined way. Therefore payment and share issuance terms have to be defined in a machine-readable contract that is notarized on the blockchain, and is enforced in a trusted environment on the connected product. 

It is important that the connected product's main utility is in providing a metered service which is disabled automatically in case sufficient payments are missing. Otherwise investors would face a high risk in not getting their expected return on investment since the product may not generate the expected revenue.

Let us go through an example. Linda wants to buy a car for \$ 25 000 USD in order to start working as a driver. However she has only \$ 5000 USD at her disposal. Together with the car manufacturer (or some intermediary) a machine- and human-readable contract is agreed upon which entails the rules under which shares are issued, the service is priced, and dividends are paid. 
We omit for the moment the complex process of fixing these terms. This is one of the biggest problems I see currently. In addition service prices might be able to be flexible within bounds or governed by some rules. At the same time it should not be possible to defraud investors by somehow creating artificially low prices after selling of shares and then charge customers via side channels.
The contract is notarized on the blockchain such that the product acknowledges the contract. The product then issues shares according to the contract. Assuming we agreed on an issuance of 25 shares. 5 shares would be allocated to Linda while the rest would be allocated to the manufacturer. The manufacturer would then try to find buyers for the shares. Keeping legal requirements aside, the shares could be sold globally with negligible transaction costs. 

Linda is now the operator (kind of a CEO) and a shareholder of the car. In order to use the car she has to pay a fee to the car. The fee is then distributed among all shareholders according to their holdings. 
 
Notes for further discussion
Issuance policy for additional revenue-based shares to keep the operator incentivized to keep a high revenue stream.
Locking operator shares at least  for a specific time. Such that the operator keeps skin in the game.
On-demand Liquidation
In principle it is the transformation of a collateral-based loan to a cashflow-based loan, isn't it? What are the implications?
Trade of between enforcement by law/legal system and enforcement by architecture/code
What happens in case of defects/maintenance?





\subsection{Requirements}

\subsubsection{Remote Disablement}
\section{Bitcoin colored coins}

\section{Ethereum}

\section{Results}

\section{Discussion}

\section{Related Work}

\cite{gsma2014} gives a good overview of energy access in sub-saharan africa. Discusses solar home systems, mini-grids, and on-grid systems. Stresses the role of cellular connectivity and pay-as-you-go models. According to the report around 40\% of on-grid energy gets stolen. Furthermore, they provide a graph on the evolution of electrification rate and the rate for cellular coverage.

\cite{DBLP:journals/corr/QadirSWC16} discuss the approach of resource pooling for wireless networks in the developing world. This could be used if providing connectivity to villages is implemented as an additional service by the DAO.

\section{Conclusion}


