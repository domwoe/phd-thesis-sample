\chapter{The Case of Crowdsensing with Bitcoin}
\section{Introduction}

Mobile phones have an unprecedented scale in the diffusion of computing technology. It is the first computing and sensing platform that reaches even the rural and remote areas throughout the developing world. Smartphones are powerful connected mobile general purpose computing devices that can be considered as the largest mobile sensing platform in existence. Essentially the same base technology is now being deployed in all kinds of other devices from cars to lightbulbs, extending the scope of sensing services even further. However thus far, the emerging opportunity of a planetary nervous system \cite{giannotti2012planetary} has not yet come to scale. Mobile crowdsensing, as will be defined in Section \ref{sec:crowdsensing}, has only been used either in small scale scenarios or in application-specific scenarios. A global platform for crowdsensing is missing.

 A fundamental building block of any crowdsensing platform, either centralized or decentralized, is a global, permissionless payment system that reaches as far as the diffusion of smartphones, and is capable of providing micropayments to remunerate ad-hoc sensing services. We argue that Bitcoin is such a global payment network. Although Bitcoin is based on a heavily replicated shared ledger on top of a gossip peer-to-peer network, and an energy and capital intensive distributed consensus mechanism using proof-of-work, we present technologies that leverage the built-in scripting language to enable peer-to-peer and mediated micropayments. In particular, we present an implementation of mediated unidirectional payment channels based on hash time locked contracts.


\section{Background}

\subsection{Crowdsensing and Sensing-as-Service}
\label{sec:crowdsensing}

Crowdsensing is where individuals with sensing and computing devices collectively share data and extract information to measure and map phenomena of common interest \cite{ganti2011mobile}. Predominately, crowdsensing has been engaged in the form of mobile crowdsensing, utilizing the ubiquity of smartphones, in application-specific scenarios. Examples are transit tracking \cite{Thiagarajan:2010:CTT:1869983.1869993}, road and traffic monitoring \cite{Mohan:2008:NRM:1460412.1460444}, site characterization \cite{Chon:2012:ACP:2370216.2370288}, and on-street parking \cite{Chen:2012:COS:2386958.2386960,6569416}. Besides these small-scale academic field studies, notable examples for successful large-scale, application specific mobile crowdsensing application are Google Maps and Waze which was acquired by Google in 2013.   

\cite{guo2015mobile} provide an extensive recent review of the various mobile crowdsensing paradigms, their applications, as well challenges and opportunities. 
The main challenges that have been identified repeatedly (see also \cite{he2015privacy}) are privacy and incentives. \cite{Christin2015} provides an analysis of the privacy implications and threats as well as a survey of available privacy-preserving techniques. 

In the application-specific scenario participants are typically incentivized by receiving an application-specific service. In the case of Google Maps a participant gets routing and traffic information in exchange for providing location and speed information. Especially in the case of participatory sensing where explicit user input is required gamification \cite{Deterding:2011:GDE:2181037.2181040} has been used successfully (e.g. Waze) to incentivize participation. However incentivization by service provision leads to contribution only from users who are in need of that particular service and only during specific times. In contrast monetary incentives are a general purpose incentive mechanism. Thus far monetary incentives have mostly been studied in small-scale, localized field studies, as well as in form of game-theoretic incentive mechanism design \cite{7101300}. 
Related to crowdsensing, there is also sensing as a service model \cite{Sheng:2013cm,ETT:ETT2704}, which emphasizes the ad-hoc access to sensing infrastructure or real-time data without investing in infrastructure itself. The sensing as a service model can be seen as an extension of the crowdsensing model to incorporate commercial sensor networks.

In recent years a multitude of architectures enabling general purpose crowdsensing and sensing as a service applications have been presented \cite{6558754,6525603,giannotti2012planetary,Haderer2015,merlino2016mobile}. However none of them has reached reasonable scale.

What has been mainly neglected is how monetary incentives can be provided efficiently under the conditions of a global mobile crowdsensing platform. 
Individual rewards are rather small and data requesters as well as data providers might be distributed globally. Furthermore traditional online payment mechanisms provide additional means for de-anonymization of participants. 

\section{Why Bitcoin?}


\section{Bitcoin Micropayments}


\section{Reputation with Bitcoin}


\section{System Design}

In this section we first state the main principles that lead to the system's design. Thereafter, we describe the architecture and explain the components in more detail.

\subsection{Principles}
\label{sec:principles}

\subsubsection{Low barrier for participation}
The usefulness of mobile crowdsensing depends primarily on the participation of a large number of data providers. Therefore, it is essential to keep barriers to participation as low as possible.

\subsubsection{Incentivize participation with micropayments}
Participation in crowdsensing 

\subsubsection{Anonymity}
Neither data producers nor data buyers should need to disclose their identity. 

\subsection{Architecture}
The system consists of three main components. A data requester or data buyer application, a data provider application, and a hub application. In order to fulfill the aforementioned principles the payment of the measurement data is done by means of N-to-M mediated payment channels as presented in Section \ref{sec:mediated} where the hub acts as the mediator over which payments are routed. At the same time the hub provides a query-able data provider registry. Noteworthy, the consolidation of payment routing and registry slightly violates our third design principle. However, both components are logically separated. An illustration of the system's architecture is shown in Figure \ref{fig:architecture}.


 \begin{figure}
 \includegraphics[width=\textwidth]{./externalized/Architecture.pdf}
 \caption{Overview of system's architecture.}
 \label{fig:architecture}
 \end{figure}

\section{Prototype Implementation}
\label{sec:implementation}

The implementation of N-to-M mediated payment channels is based on extending the payment channel implementation of the bitcoinj library \cite{Bitcoinj}. Bitcoinj is a Java library for working with the Bitcoin protocol and was the first Bitcoin library that had implemented support for payment channels. The Java implementation was chosen because it allows to reuse the implementation across all of the system's components. JavaScript provides the same benefits but libraries were not as mature at that point. Furthermore, bitcoinj implements simplified payment verification (SPV). It allows to verify on-chain transactions without needing to store and verify the entire Bitcoin blockchain. SPV nodes store only the block headers (80 byte instead of a few hundred kilo byte for a typical block), and rely on the depth of a transaction in the blockchain. Block headers contain the root of a Merkle tree that is built of all transaction hashes in the block. Hence, in order to validate a particular transaction, a SPV node has to retrieve only a subset of the block data, i.e. the corresponding branch of the Merkle tree.
Only due to the SPV is it viable to use Bitcoin on resource constrained devices without trusting third party services.

The data requester and the hub are implemented as native Java applications. The data provider is implemented as an Android application. Before presenting the implementation of the individual components we start with the common building block on which all peer-to-peer payments are based. 

\subsection{Mediated payment channels using HTLCs}

We follow the client-server architecture of the payment channel implementation of bitcoinj \cite{BitcoinjPC}. A client is always on the sending side of a payment while a server is on the receiving side (c.f. Figure \ref{fig:architecture}). Since the hub acts as mediator or router for payments between requesters and providers it has to receive payments from requesters and forward those payments to providers. Hence, the hub incorporates both, server and client.

\begin{table}
  \centering
    \caption{The five layers of the HTLC payment channel implementation.}
  \begin{tabular}{|c|l|}
    \hline
    \tabhead{Layer} &
    \tabhead{Description} \\
    \hline
    Driver Application & \multicolumn{1}{|p{0.5\columnwidth}|}{Provides the API to access lower layers}\\
    \hline
    Channel Connection & \multicolumn{1}{|p{0.5\columnwidth}|}{Provides the interface to the network}\\
    \hline
    Payment Channel & \multicolumn{1}{|p{0.5\columnwidth}|}{Handles messages from/to the network and passes instructions/gets results to/from the lower layer}\\
    \hline
    Channel State Machine & \multicolumn{1}{|p{0.5\columnwidth}|}{Handles the state of the payment channel}\\\\
    \hline
    HTLC State Machines & \multicolumn{1}{|p{0.5\columnwidth}|}{Handles the state of the HTLC}\\\\
    \hline
  \end{tabular}
  \label{tbl:layers}
\end{table}

In order to implement hashed HTLCs we extended the four layers of bitcoinj's payment channel implementation with a fifth layer that is concerned with keeping track of the HTLC flow. The five layers are described briefly in Table \ref{tbl:layers}. Details about the state machines and sequence diagrams can be found in BLINDED FOR REVIEW.
Messages and transactions are serialized using Google protocol buffers\footnote{https://developers.google.com/protocol-buffers/} and are exchanged between the components over TCP connections. The payment channel setup and the HTLC payments in the implementation are slightly more complex and involve even more communication than it is shown in Table \ref{tabl:htlcflow} because we use the interactive approach for channel refunds. 

\subsection{Data provider}

The data provider is implemented as an Android application. The application allows users to offer measurement data from various sensors at an adjustable price denominated in satoshis\footnote{A satoshi is the smallest fraction of a bitcoin and corresponds to $10^{-8}$ bitcoins}. Table \ref{tbl:sensors} provides an overview of possible sensors. With the initial boot of the application, a local wallet is generated. The wallet is responsible for generating and storing key pairs and signing transactions. Furthermore a background service is started that manages the payment channel server and data delivery. If a user decides to offer measurement data from a specific sensor, the application registers the sensor at the global sensor registry. The initial registration then triggers the setup of a payment channel between the hub and the Android application. As soon as the payment channel is established the application is ready to serve requests from buyers.


\begin{table}
  \centering
  \caption{Examples of virtual and physical sensors available in smartphones.}
  \begin{tabular}{|c|l|l|}
    \hline
    \tabhead{Sensor} &
    \tabhead{Type} &
    \tabhead{Application} \\
    \hline
    Barometric pressure & physical & Weather prediction \\
    \hline
    Location & both & People flow  \\
    \hline
    Network strength & physical & Coverage maps \\
    \hline
    Installed apps & virtual & Inter-app correlations    \\
    \hline
    Transportation mode & virtual & City monitoring \\
    \hline
    Steps & virtual & Health monitoring  \\
    \hline
  \end{tabular}
  \label{tbl:sensors}
\end{table}


\subsection{Data requester}
The data requester is implemented as a Java application. The requester application allows to query the sensor registry. In the prototype this is achieved using simple console commands. Table \ref{tbl:commands} shows the available commands.

\begin{table}
  \centering
    \caption{Available commands of the data requester console.}
  \begin{tabular}{|c|l|}
    \hline
    \tabhead{Command} &
    \tabhead{Description} \\
    \hline
    STATS NODES & Returns all connected sensor nodes \\
    \hline
    STATS SENSORS & Returns available sensor types \\
    \hline
    SELECT SENSOR=$<$type$>$ & \multicolumn{1}{|p{0.5\columnwidth}|}{Returns sensors of type $<$type$>$ with pricing information} \\
    \hline
    BUY $<$type$>$ FROM $<$node$>$ & \multicolumn{1}{|p{0.5\columnwidth}|}{Buys the current measurement value of sensor $<$type$>$ from node $<$node$>$} \\
    \hline
   
    \hline
  \end{tabular}
  \label{tbl:commands}
\end{table}



\subsection{Hub}
The hub acts as the central point of coordination. Data providers register their sensors with the sensor registry that the hub provides, and data requesters are able to query the registry. Furthermore, the hub is responsible for mediating the payments between buyers and sellers. The hub instantiates a payment channel client and a payment channel server


\section{Evaluation}

In this section we first present a conceptual evaluation based on the principles presented in Sec. \ref{sec:principles}. Thereafter, we proceed with an empirical evaluation of the performance, and the transaction costs. 

\subsection{Conceptual Evaluation}

\subsubsection{Low barrier for participation}
A potential participant on the data provider side has to download and install the data provider smartphone application. The data provider does not need to own any bitcoins. Furthermore no manual sign-up or registration is required. This means also the system has global reach and could provide a low-income stream to smartphone users in developing countries.

A data requester does not have to register but the wallet of the data requester application needs to have some amount of bitcoin in order to initiate a payment channel with the hub.

\subsubsection{Incentivize participation with micropayments}
Individual payments can be as low as 1 satoshi. In April 2016, the exchange rate is around 450\$/BTC. In regard to this exchange rate individual payments can be as low as 4.5 $\mu$\$. Even smaller payments could be implemented with probabilistic payment schemes \cite{Rivest1997,Pass:2015:MDC:2810103.2813713}. Transaction fees for setting up and closing payment channels will be discussed in Sec. \ref{sec:fees}. 

\subsubsection{Limited trust in hub provider}
The usage of HTLCs to interconnect payment channels of data providers and data requesters allows atomic payments between those parties. This means either the payment happens in both channels or it happens not at all. 

\subsubsection{Anonymity}
Obviously there is a risk of de-anonymization for data providers depending on the data they are selling. For example selling of location data or wifi SSIDs could be used to identify a data provider. We restrict the evaluation of anonymity to the payment aspect. Bitcoin payments themselves are pseudonymous. Transactions only entail cryptographic public keys or hashes thereof. If users . In addition, due to the usage of mediated payment channels, there is no public link between buyers and sellers. Only transactions between buyers and the hub, and sellers and the hub end up in the blockchain. Transactions entailing the linking secret are kept private. 

In addition, there is a risk of de-anonymization based on the IP addresses. However since all communications are based on TCP, it would be possibly to run the system over Tor\footnote{https://www.torproject.org/}.


\subsection{Empirical Evaluation}

\subsection{Performance}

For setting up and closing of payment channels the Bitcoin network itself is dictating the performance. If fees are sufficient a transaction can be assumed to be included in the blockchain in 10 min on average. Depending on the value of the transaction the receiver of the funds might wait until the transaction has six confirmations. This means that there are five additional blocks on top of the block entailing the transaction. 
In the following we assume that payment channels are already in place and evaluate the time is takes to complete a payment. The experiment was done by instantiating 100 data requesters and one data provider. Each requester buys one measurement value from the data provider. Table \ref{tbl:performance} shows a breakdown of the mean times and the standard deviation according to the payment steps. The results request some explanation. First, the HTLC setup process between hub and provider seems to take almost twice as long as the the HTLC setup between requester and hub. The reason for this is that the hub-provider setup is initiated with generating the secret but then has to wait until requester-hub setup is finished. If the hub would initiate the HTLC setup with the provider immediately the hub would risk paying the provider without being able to pull the respective funds from the requester. Second, the teardown process (i.e. updating the setup transacting by assigning the value of HTLC output to the payee.) takes less than half the time of the setup process. The reason for that is mainly that in the actual implementation the interactive refund approach was used. This means that there is an additional transaction which has to be signed by both parties and has to be transferred from payer to payee and back. Thus, we expect a comparable performance between setup and teardown if the non-interactive approach is used. Furthermore, we see that the steps that involve the provider take more time. This is because signing and signature verification are computation intensive tasks and the data provider is a comparably weak smartphone.

\begin{table}
\centering
\caption{Payment duration breakdown}
\begin{tabular}{|l|r|r|}
\hline
{\textbf Payment } & \multicolumn{1}{l|}{{\textbf Mean}} & \multicolumn{1}{l|}{{\textbf St. Dev.}} \\ \hline
Requester-Hub setup             & 479.3 ms                 & 78.8 ms                      \\ \hline
Hub-Provider setup            & 1015.1 ms                & 151.4 ms                     \\ \hline
Hub-Provider teardown         & 199.6 ms                 & 68.3 ms                      \\ \hline
Buyer-Provider teardown          & 146.6 ms                 & 10.1 ms                      \\ \hline
\end{tabular}

\label{tbl:performance}
\end{table} 


\subsection{Transaction costs}
\label{sec:fees}
Transaction fees in Bitcoin are in principle voluntary. The party that creates a transaction specifies the transaction fees as the difference of the sum of the value of all inputs and the sum of the value of all outputs. However, each miner can decide individually if she will accept the transaction for a candidate block. Since there is a limit on the maximum size of a block, and larger blocks need more time to propagate the network, rational miners will select transactions with higher transaction fees. This is also the reason why transaction fees are based on the size of the transaction instead of its value. The transaction size is mainly defined by the number and size of input and output scripts and signatures. Figure \ref{fig:tx_fees} shows the daily averages of transaction fees in USD/kb for the period of one year.

\begin{figure}[!t]
\centering
\includegraphics[width=\textwidth]{./externalized/fees_per_kb}
\caption{Bitcoin transaction fees in USD/kb. Each point represents a daily average.}
\label{fig:tx_fees}
\end{figure}


\begin{table}
  \centering
  \caption{Approximate sizes and costs for Bitcoin transactions}
  \begin{tabular}{|c|l|l|}
    \hline
    \tabhead{Transaction Type} &
    \tabhead{Size} &
    \tabhead{Cost} \\
    \hline
    Standard P2PKH & \multicolumn{1}{|p{0.5\columnwidth}|}{} & \\
    \hline
    Funding & \multicolumn{1}{|p{0.5\columnwidth}|}{} & \\
    \hline
    HTLC setup & \multicolumn{1}{|p{0.5\columnwidth}|}{} & \\
    \hline
    Settlement & \multicolumn{1}{|p{0.5\columnwidth}|}{} & \\
    \hline
    Forfeiture & \multicolumn{1}{|p{0.5\columnwidth}|}{} & \\
    \hline
  \end{tabular}
  \label{tbl:fees}
\end{table}


\section{Discussion}

\subsection{Challenges}

\subsubsection{Locking of funds}

The usage of payment channels involves locking of funds. This is a problem in particular for the payment hub. One of our principles was that the onboarding of new data producers should be frictionless. Hence, we assume that data producers do not have any bitcoins to begin with. When a data producer registers a sensor with the registry the payment hub initiates a payment channel with the data producer. As described in Sec. \ref{sec:mediated} this involves broadcasting a funding transaction to the Bitcoin network. Thereby the hub has to lock capital in the channel and has to provide a fee in order for the transaction to be mined. The operator has to balance between locking a large amount and the transaction costs involved in closing and reopening a channel because of depletion. In general the amount of locked bitcoins is proportional to the number of data producers.
Since there is no cost for a data producer to register, a malicious actor might attack the hub by registering a large amount of fake sensors. A strategy to disincentive such a behavior would be to request proof of work from the sensor as part of the registration process.

The situation might change radically when payment channel networks become ubiquitous.  

\subsubsection{Malicious Sensors}

Malicious data producers that offer fake data are not only a problem for the hub but an obstacle to the usefulness of the system as a whole. If the quality and correctness of the offered data is too low data requesters will be reluctant to use the system.
A typical approach to provide an indication about the trustworthiness of a producer is the introduction of a reputation system. Typical reputation system are centrally managed and thus may be subject to censorship. \cite{s16060776} describes a clever reputation system which interprets bitcoin payments from a particular address (or addresses) as reputation, and presents a protocol based on the CoinJoin protocol to transfer ownership. However, the scheme is not perfectly applicable in the present scenario for mainly two reasons: (1) a public reputation score is only visible for a data requester after the data provider closes a channel, and (2) the data requester is not able to see if the reputation originates from a large number of independent data providers or from repeated payments of a small number of requesters. Only repeated payments can indicate satisfactory transactions. In contrast, in \cite{s16060776} there is only one data collection server that verifies all data and pays the data providers accordingly.

\subsection{Improving anonymity}


\subsection{From mobile crowdsensing to Sensing-as-Service}

The presented micropayment scheme is also suitable for rewarding 

\subsubsection{Additional challenges}



\section{Conclusion}
The conclusion goes here.
