\chapter{Case: Sensing as a Service and Datamarkets}
\label{sec:datamarket}

\section{Data as New Asset Class}

Well, it's clearly a financial asset—just not for you. That's because the asset is data about you—not your data you own, personally.

There are several approaches. One is to create a market for personal data that you participate in. Statz  for example, calls itself The Data Marketplace, and it gives you this pitch:

Benefit From the Data Economy

You may not even know it, but you're in the data business. Data about your phone calls, prescriptions, home energy use, purchases, investments and more is being sold every day. Without your permission or profit.

Statz lets you gather all your behavior, product usage, and activity data—FREE—and participate in the consumer-data market, anonymously and securely. And make money.

Other companies—Personal.com, Connect.me, Mydex.org, Trustfabric.com, Azigo.com, Singly.com—also work in various ways to protect your personal data (or, in the case of Connect.me, your reputation), though none have Statz' business model. (Not exactly, anyway. They're all different.)

In other words, if all of us actually had full control of data about us, there might not be a market for personal data at all. There would simply be all the other markets we know—for goods and services we buy and sell. We might disclose some data on a permitted-use basis, such as most of us do every day using credit cards. But that's not the same as selling data as an “asset”.

\cite{Searls:2012:EPD:2132860.2132869}

\section{Today's Data Economy}

\section{Architectures for Future Datamarkets}

\section{When your Sensor Earns Money}

\section{Incentivizing Crowdsensing}

\section{A Datamarket based on the 21 Bitcoin Computer}
