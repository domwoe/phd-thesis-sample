\pdfbookmark[1]{Abstract}{Abstract}
\begingroup
\let\clearpage\relax
\let\cleardoublepage\relax
\let\cleardoublepage\relax

\chapter*{Abstract}

The vision of an \ac{IOT} in which every physical object can become part of the Internet is almost 25 years old. Only during the last years, due to the interplay of recent technologies like cloud computing, the unprecedented scale of the smart phone supply chain, and the global expansion of communication infrastructure, computing and networking has become so cheap and convenient that the number of connected objects is increasing rapidly. Novel business and financing models for connected devices need a new a type of value exchange infrastructure that scales with the Internet itself. While information can travel bit by bit with the speed of light between humans, machines and across borders, the transfer of value has always been cumbersome. A new type of digital currencies may be able to change that. Instead of trusted third parties, cryptocurrencies are based on public peer-to-peer networks, cryptography and mechanism design. Within a few years Bitcoin has risen from funny Internet money to a global currency with a market capitalization of more than \$10 billion. But Bitcoin is more than currency. It is programmable money and a platform for permissionless innovation. Thus, a rich start-up ecosystem with a combined investment of more than \$1 billion has emerged. A global community of developers is constantly improving Bitcoin itself, and is building new platforms based on the underlying technology, the blockchain. Hundreds of alternative cryptocurrencies have emerged and large corporations have formed consortia to utilize the technology in a permissioned setting.

Aside Bitcoin, the most prominent cryptocurrency is Ethereum which aims to build a global permissionless trusted computing platform with an integrated economy. On these platforms machines are first class citizens. This allows to rethink the capabilities of connected devices and their role in a global economy.
One of the promising \ac{IOT} business models is Sensing-as-a-Service (S\textsuperscript{2}aaS) in which a local sensing unit becomes a globally accessible resource.  This thesis identifies and discusses a number of characteristics of cryptocurrencies that uniquely suit as a basis for a global S\textsuperscript{2}aaS infrastructure. This thesis further presents a first prototype based on Bitcoin to illustrate the concept. The major issues are scalability, latency and the impracticality of direct micropayments. The second prototype leverages Bitcoin’s programmability to implement self-enforcing contracts, so called smart contracts, enabling mediated unidirectional micropayment channels -- a means for low-latency trust-less micropayments based on a hub and spoke architecture. The sensing client in this prototype is implemented as a smartphone application. Thus, providing the basis for a global mobile crowdsensing application.

Due to cryptocurrencies, connected devices can become economic devices enabling trust-minimizing economic interactions with the devices as subjects or objects. This concept is illustrated with a prototype of an Ethereum-enabled public display, which provides the service of showing user-selected content in exchange for cryptocurrency payments. In addition, the device issues tradable claims on revenue. Revenue distribution is executed autonomously and transparently based on an Ethereum smart contract. Thus, the concept allows for low-risk investments into connected productive assets, which is in particular interesting for developing countries where traditional financial and judicial systems are underdeveloped.


% Furthermore, an application of blockchain technology beyond cryptocurrency is explorer. Motivated by the question how technological infrastructure in developing countries can be financed without reliance on traditional financial and juridical infrastructure, we describe the novel concept of a connected product employing two novel economic patterns simultaneously. First, the product employs a pay-per-use model based on cryptocurrency payments. Second the product is able to issue tradable claims on revenue to offer low-risk investments for global (micro) investors. These claims are implemented as self-enforcing tradable tokens based on Ethereum smart contracts. This thesis illustrates and evaluate the concept based on an artifact comprising of a public display generating revenue by renting out screen time on a pay-per-use basis.



%\vskip 5cm
%
\vfill

\pagebreak

\selectlanguage{ngerman}

\pdfbookmark[1]{Zusammenfassung}{Zusammenfassung}
\chapter*{Zusammenfassung}

Die Vision eines Internets der Dinge, in dem jedes physische Objekt Teil des Internets werden kann, ist beinahe 25 Jahre alt. Allerdings hat erst in den letzten Jahren das Zusammenspiel der Reife neuer Technologien wie Cloud-Computing, die beispiellose Gr{\"o}sse der Smartphone-Lieferkette und die globale Ausbreitung von Kommunikationsinfrastruktur haben dazu geführt, dass Datenverarbeitung und Vernetzung so einfach und g{\"u}nstig wurde, dass die Anzahl vernetzter Objekte drastisch ansteigt. Sch{\"a}tzungen zu Folge gibt es seit dem Jahr 2016 mehr vernetzte Objekte als Menschen auf der Welt. Bis zum Jahr 2020 soll die Anzahl vernetzter Objekte gar auf etwa 25 bis 30 Milliarden ansteigen. Neuartige Gesch{\"a}fts- und Finanzierungsmodelle für vernetzte Ger{\"a}te ben{\"o}tigen eine Werttransferinfrastruktur, die mit dem Internet selbst skaliert. W{\"a}hrend Information Bit f{\"u}r Bit mit Lichtgeschwindigkeit zwischen Menschen, Maschinen und über Grenzen hinweg reisen kann, war der Transfer von Wert immer mit Schwierigkeiten verbunden. Eine neue Art von digitalen W{\"a}hrungen könnte dies endg{\"u}ltig {\"a}ndern. Kryptow{\"a}hrungen basieren auf offenen Peer-to-Peer Netzwerken, Kryptographie und Spieletheorie, und ersetzen damit die Notwendigkeit von Institutionen oder anderen vertrauensw{\"u}rdigen Dritten. Bitcoin, die erste Kryptow{\"a}hrung, ist innerhalb weniger Jahre zu einer globalen W{\"a}hrung mit einer Marktkapitalisierung von mehr als 10 Milliarden USD gewachsen. Aber Bitcoin ist mehr als eine gew{\"o}hnliche W{\"a}hrung. Bitcoin ist programmierbares Geld und eine offene Plattform für Innovationen. Aus diesem Grund ist Bitcoin nicht nur die Basis f{\"u}r ein reichhaltiges {\"O}kosystem an Start-up Unternehmen mit einem Gesamtinvestment von mehr als 1 Milliarde USD, sondern eine globale Gemeinschaft an Entwicklern arbeitet st{\"a}ndig daran Bitcoin zu verbessern und entwickelt neue Plattformen auf Basis der zugrundeliegenden Technologie, die \emph{Blockchain} genannt wird. Dabei sind hunderte alternativer Kryptow{\"a}hrungen entstanden und Konzerne bilden Konsortien, um die Technologie in einem kontrollierten, nicht {\"o}ffentlichen Umfeld einzusetzen. 

Die weitverbreitetste Kryptow{\"a}hrung neben Bitcoin ist Ethereum. Ethereum stellt eine Generalisierung des Blockchainkonzepts dar und hat das Ziele eine globale, offene, Trusted Computing-Plattform mit einer eingebauten {\"O}konomie zu erschaffen. Innerhalb dieser offenen Plattformen, die auf Kryptow{\"a}hrungen basieren, sind Maschinen Teilnehmer erster Klasse. Diese Grundlage erlaubt es das Potential vernetzter Einheiten und deren Rolle in der globalen Wirtschaft zu {\"u}berdenken. Eines der vielversprechenden Gesch{\"a}ftsmodelle im Internet der Dinge ist Sensing-as-a Service (S\textsuperscript{2}aaS), bei dem eine lokale Sensoreinheit zu einer weltweit verf{\"u}gbare Messeinrichtung wird.  Diese Dissertation identifiziert und diskutiert Eigenschaften von Kryptow{\"a}hrungen, die als einzigartige Basis für eine globale S\textsuperscript{2}aaS-Infrastruktur dienen. Dazu wird eine erste prototypische Implementierung dieses Konzepts vorgestellt. Die Hauptschwierigkeiten, die sich bei der Umsetzung dieses Konzepts zeigen, sind Skalierbarkeit, Latenz und die praktische Unm{\"o}glichkeit direkte Mikrozahlungen durchzuf{\"u}hren. Um diese Hauptschwierigkeiten anzugehen, wird die Programmierbarkeit von Bitcoin genutzt, um eine Art von selbst-vollstreckender Vertr{\"a}ge, sogenannte \emph{ Smart Contracts}, zu entwickeln.  Diese erm{\"o}glichen es vermittelbare Einwegzahlungskan{\"a}le zu erschaffen, die sichere Mikrozahlungen mit niedriger Latenz zwischen vielen Parteien mittels einer \emph{Hub and Spoke}-Architektur erlauben. Der Sensor-Client, in der dazugeh{\"o}rigen prototypischen Implementierung, ist eine Smartphone Applikation und damit die Basis für eine globale \emph{Crowdsensing}-Applikation.

Es wird gezeigt, dass Kryptow{\"a}hrungen vernetzten Ger{\"a}ten den autonomen Umgang mit Geld über Distanz erm{\"o}glichen. Smart Contracts und das verwandte Konzept von \emph{Smart Property} erm{\"o}glichen weitere {\"o}konomische Interkationen mit vernetzten Ger{\"a}ten, bei denen das n{\"o}tige Vertrauen in die Gegenpartie geringgehalten werden kann. Bei diesen Interkationen k{\"o}nnen vernetzte Ger{\"a}te sowohl als Subjekt, als auch als Objekt, auftreten.  Diese Konzept, \emph{{\"o}konomischer Ger{\"a}te}, wird mithilfe eines auf Ethereum basierenden Prototyps verdeutlicht. Der Prototyp stellt einen {\"o}ffentlichen Bildschirm dar, der gegen Zahlung Inhalte anzeigt. Die Zahlungen werden von einem Smart Contract verwaltet und automatisch an Investoren als sofortige Mikrodividenden weltweit weiterverteilt. Dieses Beispiel illustriert neue Finanzierungs- und Besitzmodelle für produktive Verm{\"o}genswerte auf Basis vernetzter Geräte und ist insbesondere für Entwicklungsl{\"a}nder von Interesse.






\selectlanguage{american}

\endgroup

\vfill
