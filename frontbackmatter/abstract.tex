\pdfbookmark[1]{Abstract}{Abstract}
\begingroup
\let\clearpage\relax
\let\cleardoublepage\relax
\let\cleardoublepage\relax

\chapter*{Abstract}

The vision of an Internet of Things (IoT) in which every physical object can become part of the Internet is almost 25 years old. Only during the last years, due to the interplay of recent technologies like cloud computing, the unprecedented scale of the smart phone supply chain, and the global expansion of communication infrastructure, computing and networking has become so cheap and convenient that the number of connected objects is increasing rapidly. Novel business and financing models for connected devices need a new a type of value exchange infrastructure that scales with the Internet itself. While information can travel bit by bit with the speed of light between humans, machines and across borders, the transfer of value has always been cumbersome. A new type of digital currencies may be able to change that. Instead of trusted third parties, cryptocurrencies are based on public peer-to-peer networks, cryptography and mechanism design. Within a few years Bitcoin has risen from funny Internet money to a global currency with a market capitalization of more than \$10 billion. But Bitcoin is more than currency. It is programmable money and a platform for permissionless innovation. Thus, a rich start-up ecosystem with a combined investment of more than \$1 billion has emerged. A global community of developers is constantly improving Bitcoin itself, and is building new platforms based on the underlying technology, the blockchain. Hundreds of alternative cryptocurrencies have emerged and large corporations have formed consortia to utilize the technology in a permissioned setting.

Aside Bitcoin, the most prominent cryptocurrency is Ethereum which aims to build a global permissionless trusted computing platform with an integrated economy. On these platforms machines are first class citizens. This allows to rethink the capabilities of connected devices and their role in a global economy.
One of the promising IoT business models is Sensing-as-a-Service (S\textsuperscript{2}aaS) in which a local sensing unit becomes a globally accessible resource.  This thesis identifies and discusses a number of characteristics of cryptocurrencies that uniquely suit as a basis for a global S\textsuperscript{2}aaS infrastructure. This thesis further presents a first artifact based on Bitcoin to illustrate the concept, as well as its strength and weaknesses. The major weaknesses are scalability, latency and the impracticality of direct micropayments. The second artifact leverages Bitcoin’s programmability to implement smart contracts enabling mediated unidirectional micropayment channels, a means for low-latency trustless micropayments based on a hub and spoke architecture. The sensing client in this artifact is implemented as a smartphone application and provides therefore the basis for a mobile crowdsensing application. We further discuss improvements of the scheme and present notable developments in the Bitcoin ecosystem that will enable increased privacy and a path towards real-world application.
Furthermore, an application of blockchain technology beyond cryptocurrency is explorer. Motivated by the question how technological infrastructure in developing countries can be financed without reliance on traditional financial and juridical infrastructure, we describe the novel concept of a connected product employing two novel economic patterns simultaneously. First, the product employs a pay-per-use model based on cryptocurrency payments. Second the product is able to issue tradable claims on revenue to offer low-risk investments for global (micro) investors. These claims are implemented as self-enforcing tradable tokens based on Ethereum smart contracts. This thesis illustrates and evaluate the concept based on an artifact comprising of a public display generating revenue by renting out screen time on a pay-per-use basis.



%\vskip 5cm
%
\vfill

\pagebreak

\selectlanguage{ngerman}

\pdfbookmark[1]{Zusammenfassung}{Zusammenfassung}
\chapter*{Zusammenfassung}



\selectlanguage{american}

\endgroup

\vfill
